%% Generated by Sphinx.
\def\sphinxdocclass{jupyterBook}
\documentclass[letterpaper,10pt,english]{jupyterBook}
\ifdefined\pdfpxdimen
   \let\sphinxpxdimen\pdfpxdimen\else\newdimen\sphinxpxdimen
\fi \sphinxpxdimen=.75bp\relax
\ifdefined\pdfimageresolution
    \pdfimageresolution= \numexpr \dimexpr1in\relax/\sphinxpxdimen\relax
\fi
%% let collapsible pdf bookmarks panel have high depth per default
\PassOptionsToPackage{bookmarksdepth=5}{hyperref}
%% turn off hyperref patch of \index as sphinx.xdy xindy module takes care of
%% suitable \hyperpage mark-up, working around hyperref-xindy incompatibility
\PassOptionsToPackage{hyperindex=false}{hyperref}
%% memoir class requires extra handling
\makeatletter\@ifclassloaded{memoir}
{\ifdefined\memhyperindexfalse\memhyperindexfalse\fi}{}\makeatother

\PassOptionsToPackage{warn}{textcomp}

\catcode`^^^^00a0\active\protected\def^^^^00a0{\leavevmode\nobreak\ }
\usepackage{cmap}
\usepackage{fontspec}
\defaultfontfeatures[\rmfamily,\sffamily,\ttfamily]{}
\usepackage{amsmath,amssymb,amstext}
\usepackage{polyglossia}
\setmainlanguage{english}



\setmainfont{FreeSerif}[
  Extension      = .otf,
  UprightFont    = *,
  ItalicFont     = *Italic,
  BoldFont       = *Bold,
  BoldItalicFont = *BoldItalic
]
\setsansfont{FreeSans}[
  Extension      = .otf,
  UprightFont    = *,
  ItalicFont     = *Oblique,
  BoldFont       = *Bold,
  BoldItalicFont = *BoldOblique,
]
\setmonofont{FreeMono}[
  Extension      = .otf,
  UprightFont    = *,
  ItalicFont     = *Oblique,
  BoldFont       = *Bold,
  BoldItalicFont = *BoldOblique,
]



\usepackage[Bjarne]{fncychap}
\usepackage[,numfigreset=1,mathnumfig]{sphinx}

\fvset{fontsize=\small}
\usepackage{geometry}


% Include hyperref last.
\usepackage{hyperref}
% Fix anchor placement for figures with captions.
\usepackage{hypcap}% it must be loaded after hyperref.
% Set up styles of URL: it should be placed after hyperref.
\urlstyle{same}

\addto\captionsenglish{\renewcommand{\contentsname}{Introduzione alla fisica}}

\usepackage{sphinxmessages}



        % Start of preamble defined in sphinx-jupyterbook-latex %
         \usepackage[Latin,Greek]{ucharclasses}
        \usepackage{unicode-math}
        % fixing title of the toc
        \addto\captionsenglish{\renewcommand{\contentsname}{Contents}}
        \hypersetup{
            pdfencoding=auto,
            psdextra
        }
        % End of preamble defined in sphinx-jupyterbook-latex %
        

\title{Fisica per le superiori}
\date{Nov 06, 2024}
\release{}
\author{basics}
\newcommand{\sphinxlogo}{\vbox{}}
\renewcommand{\releasename}{}
\makeindex
\begin{document}

\pagestyle{empty}
\sphinxmaketitle
\pagestyle{plain}
\sphinxtableofcontents
\pagestyle{normal}
\phantomsection\label{\detokenize{intro::doc}}


\sphinxAtStartPar
Questo libro fa parte del materiale pensato per \sphinxhref{https://basics2022.github.io/bbooks-hs}{le scuole superiori}



\sphinxAtStartPar
\sphinxstylestrong{Discipline.}

\begin{sphinxuseclass}{sd-container-fluid}
\begin{sphinxuseclass}{sd-sphinx-override}
\begin{sphinxuseclass}{sd-mb-4}
\begin{sphinxuseclass}{sd-row}
\begin{sphinxuseclass}{sd-row-cols-1}
\begin{sphinxuseclass}{sd-row-cols-xs-1}
\begin{sphinxuseclass}{sd-row-cols-sm-1}
\begin{sphinxuseclass}{sd-row-cols-md-1}
\begin{sphinxuseclass}{sd-row-cols-lg-1}
\begin{sphinxuseclass}{sd-g-3}
\begin{sphinxuseclass}{sd-g-xs-3}
\begin{sphinxuseclass}{sd-g-sm-3}
\begin{sphinxuseclass}{sd-g-md-3}
\begin{sphinxuseclass}{sd-g-lg-3}
\begin{sphinxuseclass}{sd-col}
\begin{sphinxuseclass}{sd-d-flex-row}
\begin{sphinxuseclass}{sd-card}
\begin{sphinxuseclass}{sd-sphinx-override}
\begin{sphinxuseclass}{sd-w-100}
\begin{sphinxuseclass}{sd-shadow-sm}
\begin{sphinxuseclass}{sd-card-hover}
\begin{sphinxuseclass}{sd-card-header}
\sphinxAtStartPar
\sphinxstylestrong{Introduzione alla fisica}

\end{sphinxuseclass}
\begin{sphinxuseclass}{sd-card-body}
\end{sphinxuseclass}\sphinxhref{ch/intro.html}{}
\end{sphinxuseclass}
\end{sphinxuseclass}
\end{sphinxuseclass}
\end{sphinxuseclass}
\end{sphinxuseclass}
\end{sphinxuseclass}
\end{sphinxuseclass}
\begin{sphinxuseclass}{sd-col}
\begin{sphinxuseclass}{sd-d-flex-row}
\begin{sphinxuseclass}{sd-card}
\begin{sphinxuseclass}{sd-sphinx-override}
\begin{sphinxuseclass}{sd-w-100}
\begin{sphinxuseclass}{sd-shadow-sm}
\begin{sphinxuseclass}{sd-card-hover}
\begin{sphinxuseclass}{sd-card-header}
\sphinxAtStartPar
\sphinxstylestrong{Meccanica classica}

\end{sphinxuseclass}
\begin{sphinxuseclass}{sd-card-body}
\end{sphinxuseclass}\sphinxhref{ch/mechanics.html}{}
\end{sphinxuseclass}
\end{sphinxuseclass}
\end{sphinxuseclass}
\end{sphinxuseclass}
\end{sphinxuseclass}
\end{sphinxuseclass}
\end{sphinxuseclass}
\begin{sphinxuseclass}{sd-col}
\begin{sphinxuseclass}{sd-d-flex-row}
\begin{sphinxuseclass}{sd-card}
\begin{sphinxuseclass}{sd-sphinx-override}
\begin{sphinxuseclass}{sd-w-100}
\begin{sphinxuseclass}{sd-shadow-sm}
\begin{sphinxuseclass}{sd-card-hover}
\begin{sphinxuseclass}{sd-card-header}
\sphinxAtStartPar
\sphinxstylestrong{Termodinamica}

\end{sphinxuseclass}
\begin{sphinxuseclass}{sd-card-body}
\end{sphinxuseclass}\sphinxhref{ch/thermodynamics.html}{}
\end{sphinxuseclass}
\end{sphinxuseclass}
\end{sphinxuseclass}
\end{sphinxuseclass}
\end{sphinxuseclass}
\end{sphinxuseclass}
\end{sphinxuseclass}
\begin{sphinxuseclass}{sd-col}
\begin{sphinxuseclass}{sd-d-flex-row}
\begin{sphinxuseclass}{sd-card}
\begin{sphinxuseclass}{sd-sphinx-override}
\begin{sphinxuseclass}{sd-w-100}
\begin{sphinxuseclass}{sd-shadow-sm}
\begin{sphinxuseclass}{sd-card-hover}
\begin{sphinxuseclass}{sd-card-header}
\sphinxAtStartPar
\sphinxstylestrong{Elettromagnetismo}

\end{sphinxuseclass}
\begin{sphinxuseclass}{sd-card-body}
\end{sphinxuseclass}\sphinxhref{ch/electromagnetism.html}{}
\end{sphinxuseclass}
\end{sphinxuseclass}
\end{sphinxuseclass}
\end{sphinxuseclass}
\end{sphinxuseclass}
\end{sphinxuseclass}
\end{sphinxuseclass}
\begin{sphinxuseclass}{sd-col}
\begin{sphinxuseclass}{sd-d-flex-row}
\begin{sphinxuseclass}{sd-card}
\begin{sphinxuseclass}{sd-sphinx-override}
\begin{sphinxuseclass}{sd-w-100}
\begin{sphinxuseclass}{sd-shadow-sm}
\begin{sphinxuseclass}{sd-card-hover}
\begin{sphinxuseclass}{sd-card-header}
\sphinxAtStartPar
\sphinxstylestrong{Fisica del XX secolo}

\end{sphinxuseclass}
\begin{sphinxuseclass}{sd-card-body}
\end{sphinxuseclass}\sphinxhref{ch/modern.html}{}
\end{sphinxuseclass}
\end{sphinxuseclass}
\end{sphinxuseclass}
\end{sphinxuseclass}
\end{sphinxuseclass}
\end{sphinxuseclass}
\end{sphinxuseclass}
\end{sphinxuseclass}
\end{sphinxuseclass}
\end{sphinxuseclass}
\end{sphinxuseclass}
\end{sphinxuseclass}
\end{sphinxuseclass}
\end{sphinxuseclass}
\end{sphinxuseclass}
\end{sphinxuseclass}
\end{sphinxuseclass}
\end{sphinxuseclass}
\end{sphinxuseclass}
\end{sphinxuseclass}
\end{sphinxuseclass}




\sphinxstepscope


\part{Introduzione alla fisica}

\sphinxstepscope


\chapter{Introduzione alla fisica}
\label{\detokenize{ch/intro:introduzione-alla-fisica}}\label{\detokenize{ch/intro:physics-hs-intro}}\label{\detokenize{ch/intro::doc}}\begin{itemize}
\item {} 
\sphinxAtStartPar
Metodo scientifico:
\begin{itemize}
\item {} 
\sphinxAtStartPar
ricerca di princìpi fisici in accordo con le attività sperimentali

\item {} 
\sphinxAtStartPar
deduzione di una teoria, a partire dai princìpi

\end{itemize}

\item {} 
\sphinxAtStartPar
Grandezze fisiche:
\begin{itemize}
\item {} 
\sphinxAtStartPar
fondamentali e derivate
\begin{itemize}
\item {} 
\sphinxAtStartPar
lunghezza, aree e volumi

\item {} 
\sphinxAtStartPar
massa e densità

\item {} 
\sphinxAtStartPar
tempo

\end{itemize}

\item {} 
\sphinxAtStartPar
processo e strumenti di misura:
\begin{itemize}
\item {} 
\sphinxAtStartPar
logica \sphinxstylestrong{todo} \sphinxstyleemphasis{?}

\item {} 
\sphinxAtStartPar
analisi dati, errori

\end{itemize}

\end{itemize}

\item {} 
\sphinxAtStartPar
\sphinxstylestrong{Astrazione} e costruzione di \sphinxstylestrong{modelli}, con il linguaggio matematico

\item {} 
\sphinxAtStartPar
Rappresentazione dei dati
\begin{itemize}
\item {} 
\sphinxAtStartPar
…

\end{itemize}

\end{itemize}

\sphinxstepscope


\chapter{Grandezze fisiche}
\label{\detokenize{ch/intro/physical_quantities:grandezze-fisiche}}\label{\detokenize{ch/intro/physical_quantities:physics-hs-intro-physical-quantities}}\label{\detokenize{ch/intro/physical_quantities::doc}}\begin{itemize}
\item {} 
\sphinxAtStartPar
Come conosciamo il mondo? Come misuriamo il mondo?

\item {} 
\sphinxAtStartPar
Necessità di avere delle grandezze di riferimento stabili o facilmente riproducibili in maniera precisa, da usare come unità di misura delle grandezze fisiche.

\item {} 
\sphinxAtStartPar
Nell’antichità, dall’esperienza:
\begin{itemize}
\item {} 
\sphinxAtStartPar
spazio:
\begin{itemize}
\item {} 
\sphinxAtStartPar
importanza di misurare le distanze (es. distanze da percorrere), le aree (es. misura dei campi,…), e i volumi

\item {} 
\sphinxAtStartPar
grandezze di riferimento: lunghezze ideali di parti anatomiche umane: cubito, pollice, piede,…

\end{itemize}

\item {} 
\sphinxAtStartPar
tempo:
\begin{itemize}
\item {} 
\sphinxAtStartPar
alternanza di luce e buio, alternanza delle stagioni, alternanza di configurazioni degli astri osservati dalla terra; queste alternanze scandiscono

\item {} 
\sphinxAtStartPar
grandezze di riferimento: intervalli temporali scanditi dalla natura

\end{itemize}

\item {} 
\sphinxAtStartPar
peso:
\begin{itemize}
\item {} 
\sphinxAtStartPar
misura della quantità di merce, quantità di denaro o materiali preziosi, per le prescrizioni mediche (apothecary,…)

\item {} 
\sphinxAtStartPar
grandezze di riferimento: grano (basato su un seme ideale di cereale), libbra (dallo strumento usato per la misura del peso/massa, \sphinxstyleemphasis{libra} = bilancia)

\end{itemize}

\end{itemize}

\item {} 
\sphinxAtStartPar
In epoca moderna:
\begin{itemize}
\item {} 
\sphinxAtStartPar
aggiornamento delle grandezze di riferimento
\begin{itemize}
\item {} 
\sphinxAtStartPar
Parigi tra fine XVIII e XIX secolo:
\begin{itemize}
\item {} 
\sphinxAtStartPar
lunghezza: metro (1791) come \(1/10.000.000\) la distanza tra l’equatore e il polo nord sul meridiano terrestre passante per Parigi

\item {} 
\sphinxAtStartPar
tempo: \sphinxstylestrong{todo}

\item {} 
\sphinxAtStartPar
\sphinxstylestrong{todo}

\end{itemize}

\end{itemize}

\item {} 
\sphinxAtStartPar
nuove grandezze fisiche misurate nelle nuove scienze, chimica, termodinamica ed elettromagnetismo:
\begin{itemize}
\item {} 
\sphinxAtStartPar
quantità di sostanza

\item {} 
\sphinxAtStartPar
temperatura

\item {} 
\sphinxAtStartPar
corrente elettrica

\item {} 
\sphinxAtStartPar
luminosità

\end{itemize}

\end{itemize}

\item {} 
\sphinxAtStartPar
XX\sphinxhyphen{}XXI secolo: continuo aggiornamento delle unità di misura, usando definizioni più precise e replicabili, tramite misure non disponibili solo qualche decennio prima

\end{itemize}

\sphinxstepscope


\section{Spazio}
\label{\detokenize{ch/intro/physical_quantities-space:spazio}}\label{\detokenize{ch/intro/physical_quantities-space:physics-hs-intro-physical-quantities-space}}\label{\detokenize{ch/intro/physical_quantities-space::doc}}
\sphinxstepscope


\section{Quantità di materia}
\label{\detokenize{ch/intro/physical_quantities-mass:quantita-di-materia}}\label{\detokenize{ch/intro/physical_quantities-mass:physics-hs-intro-physical-quantities-mass}}\label{\detokenize{ch/intro/physical_quantities-mass::doc}}
\sphinxstepscope


\section{Tempo}
\label{\detokenize{ch/intro/physical_quantities-time:tempo}}\label{\detokenize{ch/intro/physical_quantities-time:physics-hs-intro-physical-quantities-time}}\label{\detokenize{ch/intro/physical_quantities-time::doc}}
\sphinxAtStartPar
\sphinxstylestrong{Prime esperienze e strumenti.}
\begin{itemize}
\item {} 
\sphinxAtStartPar
tempo scandito dall’astronomia:
\begin{itemize}
\item {} 
\sphinxAtStartPar
osservazioni astronomiche degli astri

\item {} 
\sphinxAtStartPar
alternanza stagioni

\item {} 
\sphinxAtStartPar
alternanza luce/buio e meridiana

\item {} 
\sphinxAtStartPar
astroloabio

\end{itemize}

\item {} 
\sphinxAtStartPar
primi strumenti rudimentali
\begin{itemize}
\item {} 
\sphinxAtStartPar
orologio ad acqua

\item {} 
\sphinxAtStartPar
orologio a candela e clessidra

\end{itemize}

\item {} 
\sphinxAtStartPar
orologi meccanici:
\begin{itemize}
\item {} 
\sphinxAtStartPar
a pendolo

\item {} 
\sphinxAtStartPar
a molla

\end{itemize}

\item {} 
\sphinxAtStartPar
orologi a batteria,
\begin{itemize}
\item {} 
\sphinxAtStartPar
al quarzo, digitali,…

\end{itemize}

\end{itemize}

\sphinxAtStartPar
\sphinxstylestrong{Ma cos’è il tempo?}
\begin{itemize}
\item {} 
\sphinxAtStartPar
Kairos e Chronos, come percezione personale e percezione assoluta

\item {} 
\sphinxAtStartPar
Newton e la formulazione della meccanica con spazio e tempo assoluti, il tempo come misurato dall’orologio meccanico allora disponibile

\item {} 
\sphinxAtStartPar
Relatività di Einstein: la misura del tempo dipende dall’osservatore; spazio e tempo non sono più assoluti; assoluta \sphinxhyphen{} indipendente dall’osservatore \sphinxhyphen{} è la misura della velocità della luce per ogni osservatore

\end{itemize}

\sphinxstepscope


\section{Temperatura}
\label{\detokenize{ch/intro/physical_quantities-temperature:temperatura}}\label{\detokenize{ch/intro/physical_quantities-temperature:physics-hs-intro-physical-quantities-temperature}}\label{\detokenize{ch/intro/physical_quantities-temperature::doc}}
\sphinxstepscope


\section{Carica e corrente elettrica}
\label{\detokenize{ch/intro/physical_quantities-charge:carica-e-corrente-elettrica}}\label{\detokenize{ch/intro/physical_quantities-charge:physics-hs-intro-physical-quantities-charge}}\label{\detokenize{ch/intro/physical_quantities-charge::doc}}
\sphinxstepscope


\section{Quantità di sostanza, la mole}
\label{\detokenize{ch/intro/physical_quantities-mole:quantita-di-sostanza-la-mole}}\label{\detokenize{ch/intro/physical_quantities-mole:physics-hs-intro-physical-quantities-mole}}\label{\detokenize{ch/intro/physical_quantities-mole::doc}}
\sphinxstepscope


\section{Intensità luminosa}
\label{\detokenize{ch/intro/physical_quantities-luminosity:intensita-luminosa}}\label{\detokenize{ch/intro/physical_quantities-luminosity:physics-hs-intro-physical-quantities-luminosity}}\label{\detokenize{ch/intro/physical_quantities-luminosity::doc}}
\sphinxstepscope


\part{Meccanica classica}

\sphinxstepscope


\chapter{Meccanica classica}
\label{\detokenize{ch/mechanics:meccanica-classica}}\label{\detokenize{ch/mechanics:physics-hs-mechanics}}\label{\detokenize{ch/mechanics::doc}}
\sphinxAtStartPar
La meccanica classica fu sviluppata da Newton nel XVII secolo, e presentanta nei \sphinxstyleemphasis{Principi matematici di filosofia naturale} (1687) \sphinxstylestrong{todo} \sphinxstyleemphasis{Dire due parole si Principi matemtatici di filosofia naturale}

\sphinxAtStartPar
L’opera di Newton \sphinxstylestrong{todo} usa i metodi della geometria analitica introdotti da Cartesio, e viene sviluppata in accordo con le osservazioni astronomiche, come le leggi di Keplero, e le esperienze di Galileo sul principio di inerzia, sulla caduta dei gravi (\sphinxstylestrong{todo} anche se probabilmente era lo studio del rotolamento di corpi su piani inclinati) e sull’isocronismo delle piccole oscillazioni libere di un pendolo(\sphinxstylestrong{todo} cosa c’entra? Isocronismo come principio alla base del principio di funzionamento dei primi orologi meccanici, a gravità o a molla, con regolazione tramite scappamento; lo stesso principio, anche se migliorato, degli orologi meccanici contemporanei)

\sphinxAtStartPar
A partire dal principio di conservazione della massa di sistemi chiusi e da tre principi della dinamica, Newton formula una teoria capace di descrivere il moto dei corpi in termini usando concetti definiti nelle prime pagine della sua opera, come:
\begin{itemize}
\item {} 
\sphinxAtStartPar
lo spazio assoluto, inteso come uno spazio euclideo o un sottoinsieme dello spazio euclideo \sphinxstylestrong{todo} ref

\item {} 
\sphinxAtStartPar
il tempo assoluto, in contrapposizione con il tempo percepito dagli individui (che può sembrare scorrere più velocmente o più lentamente a seconda dell’individuo, del divertimento o della noia provata,…) \sphinxstylestrong{todo} ref, e discussione su cos’è il tempo e come si misura

\item {} 
\sphinxAtStartPar
la massa, intesa come la quantità di materia di un sistema \sphinxstylestrong{todo} come la misura Newton?

\item {} 
\sphinxAtStartPar
la quantità di moto, intesa come il prodotto di massa e velocità del centro di massa del sistema \sphinxstylestrong{todo} ok, ma cos’è il centro di massa? E’ proprio questa la definizione?

\item {} 
\sphinxAtStartPar
le forze, intese come le azioni che possono far cambiare la quantità di moto di un sistema.

\end{itemize}

\sphinxstepscope


\chapter{Modelli}
\label{\detokenize{ch/mechanics/models:modelli}}\label{\detokenize{ch/mechanics/models:physics-hs-mechanics-models}}\label{\detokenize{ch/mechanics/models::doc}}
\sphinxAtStartPar
Muovere in una sezione “Introduzione alla meccanica”, per rendere lo schema uniforme con Termodinamica ed Elettromagnetismo: prime esperienze; approccio di Newton (grandezze fisiche e concetti); modelli

\sphinxAtStartPar
Quando si costruisce una teoria scientifica, è spesso necessario compiere uno sforzo di astrazione (\sphinxstylestrong{todo} \sphinxstyleemphasis{come conosciamo? Discorso filosofico…}), di modellazione dei fenomeni di interesse. Un buon modello è in grado di rappresentare con la precisione (\sphinxstylestrong{todo} \sphinxstyleemphasis{o accuratezza?}) richiesta il fenomeno studiato, essere in accordo con attività sperimentali e garantire capacità di previsione che coinvolgono tali fenomeni.

\sphinxAtStartPar
Nello studio della meccanica e della fisica in generale, si è soliti distinguere gli elementi oggetti di studio da tutti gli altri elementi:
\begin{itemize}
\item {} 
\sphinxAtStartPar
\sphinxstylestrong{sistema}, unione degli elementi oggetti di studio

\item {} 
\sphinxAtStartPar
\sphinxstylestrong{ambiente esterno}, tutto quello che non fa parte del sistema

\end{itemize}

\sphinxAtStartPar
In meccanica, è necessario uno sforzo di modellazione per costruire un modello matematico che rappresenti:
\begin{itemize}
\item {} 
\sphinxAtStartPar
i componenti meccanici, che costituiscono il sistema

\item {} 
\sphinxAtStartPar
le connessioni tra componenti meccanici del sistema, o le connessioni con l’ambiente esterno

\item {} 
\sphinxAtStartPar
le azioni che operano sul sistema, dovute alle interazioni del sistema con l’esterno o scambiate tra componenti del sistema

\end{itemize}

\sphinxAtStartPar
A seconda del livello di dettaglio richiesto, si possono definire diversi modelli di componenti meccanici in base a:
\begin{itemize}
\item {} 
\sphinxAtStartPar
dimensioni:
\begin{itemize}
\item {} 
\sphinxAtStartPar
sistemi puntiformi, di dimensioni trascurabili per il problema di interesse

\item {} 
\sphinxAtStartPar
sistemi estesi, di dimensioni non trascurabili per il problema di interesse; a seconda della loro deformabilità e/o del livello di dettaglio dell’analisi:
\begin{itemize}
\item {} 
\sphinxAtStartPar
rigidi

\item {} 
\sphinxAtStartPar
deformabili

\end{itemize}

\end{itemize}

\item {} 
\sphinxAtStartPar
inertia:
\begin{itemize}
\item {} 
\sphinxAtStartPar
massa non trascurabile

\item {} 
\sphinxAtStartPar
massa trascurabile

\end{itemize}

\end{itemize}

\sphinxAtStartPar
\sphinxstylestrong{Esempio.} Analisi di un aereo:
\begin{itemize}
\item {} 
\sphinxAtStartPar
per lo studio di traiettorie e prestazioni, può essere considerato come un sistema puntiforme

\item {} 
\sphinxAtStartPar
per uno studio preliminare di equilibrio e dinamica del velivolo, può essere usato un modello esteso rigido

\item {} 
\sphinxAtStartPar
per lo studio accurato dell’equilibrio, della dinamica del volo e del progetto aero\sphinxhyphen{}servo\sphinxhyphen{}elastico l’aereo viene modellato come un insieme di elementi: viene usato un modello esteso deformabile dotato di massa per molti elementi strutturali, connessi tramite vincoli

\end{itemize}

\sphinxstepscope

\begin{sphinxuseclass}{sd-container-fluid}
\begin{sphinxuseclass}{sd-sphinx-override}
\begin{sphinxuseclass}{sd-p-0}
\begin{sphinxuseclass}{sd-mt-2}
\begin{sphinxuseclass}{sd-mb-4}
\begin{sphinxuseclass}{sd-row}
\begin{sphinxuseclass}{sd-row-cols-2}
\begin{sphinxuseclass}{sd-gx-2}
\begin{sphinxuseclass}{sd-gy-1}
\begin{sphinxuseclass}{sd-col}
\begin{sphinxuseclass}{sd-d-flex-row}
\begin{sphinxuseclass}{sd-align-minor-center}
\begin{sphinxuseclass}{sd-container-fluid}
\begin{sphinxuseclass}{sd-sphinx-override}
\begin{sphinxuseclass}{sd-row}
\begin{sphinxuseclass}{sd-row-cols-2}
\begin{sphinxuseclass}{sd-row-cols-xs-2}
\begin{sphinxuseclass}{sd-row-cols-sm-3}
\begin{sphinxuseclass}{sd-row-cols-md-3}
\begin{sphinxuseclass}{sd-row-cols-lg-3}
\begin{sphinxuseclass}{sd-gx-3}
\begin{sphinxuseclass}{sd-gy-1}
\begin{sphinxuseclass}{sd-col}
\begin{sphinxuseclass}{sd-col-auto}
\begin{sphinxuseclass}{sd-d-flex-row}
\begin{sphinxuseclass}{sd-align-minor-center}
\sphinxAtStartPar
basics

\end{sphinxuseclass}
\end{sphinxuseclass}
\end{sphinxuseclass}
\end{sphinxuseclass}
\begin{sphinxuseclass}{sd-col}
\begin{sphinxuseclass}{sd-col-auto}
\begin{sphinxuseclass}{sd-d-flex-row}
\begin{sphinxuseclass}{sd-align-minor-center}
\sphinxAtStartPar
Nov 06, 2024

\end{sphinxuseclass}
\end{sphinxuseclass}
\end{sphinxuseclass}
\end{sphinxuseclass}
\begin{sphinxuseclass}{sd-col}
\begin{sphinxuseclass}{sd-col-auto}
\begin{sphinxuseclass}{sd-d-flex-row}
\begin{sphinxuseclass}{sd-align-minor-center}
\sphinxAtStartPar
1 min read

\end{sphinxuseclass}
\end{sphinxuseclass}
\end{sphinxuseclass}
\end{sphinxuseclass}
\end{sphinxuseclass}
\end{sphinxuseclass}
\end{sphinxuseclass}
\end{sphinxuseclass}
\end{sphinxuseclass}
\end{sphinxuseclass}
\end{sphinxuseclass}
\end{sphinxuseclass}
\end{sphinxuseclass}
\end{sphinxuseclass}
\end{sphinxuseclass}
\end{sphinxuseclass}
\end{sphinxuseclass}
\end{sphinxuseclass}
\end{sphinxuseclass}
\end{sphinxuseclass}
\end{sphinxuseclass}
\end{sphinxuseclass}
\end{sphinxuseclass}
\end{sphinxuseclass}
\end{sphinxuseclass}
\end{sphinxuseclass}

\chapter{Cinematica}
\label{\detokenize{ch/mechanics/kinematics:cinematica}}\label{\detokenize{ch/mechanics/kinematics:physics-hs-mechanics-kinematics}}\label{\detokenize{ch/mechanics/kinematics::doc}}
\sphinxAtStartPar
La cinematica si occupa della descrizione del moto dei sistemi, senza indagarne le cause. La cinematica si occupa della descrizione dello stato di un sistema, e della sua variazione, nello spazio.

\sphinxAtStartPar
La \sphinxstylestrong{configurazione di un sistema} è definita da un insieme di variabili indipendenti, o coordinate, dette \sphinxstylestrong{gradi di libertà}.  Il numero di gradi di libertà di un sistema dipende dalla dimensione dello spazio nel quale avviene il moto, dal numero e dal tipo degli elementi che lo compongono e dai vincoli che connettono gli elementi del sistema tra di loro o con l’ambiente esterno.
In generale, in meccanica classica lo \sphinxstylestrong{stato di un sistema} è definito dalla sua configurazione e dalla derivata prima nel tempo delle variabili che definiscono i gradi di libertà: questo è sensato per sistemi meccanici la cui dinamica è governata da equazioni differenziali ordinarie del secondo ordine.

\sphinxAtStartPar
La configurazione di un \sphinxstylestrong{punto} libero nello \sphinxhref{https://basics2022.github.io/bbooks-math-miscellanea-hs/ch/analytic\_geometry/euclidean\_space.html}{spazio euclideo \(E^n\)} (\(n=2\) piano, \(n=3\) spazio) è definita dalla sua posizione nello spazio, tramite un insieme di \(n\) coordinate:
\begin{itemize}
\item {} 
\sphinxAtStartPar
un punto libero nel piano ha 2 gradi di libertà (traslazione);

\item {} 
\sphinxAtStartPar
un punto libero nello spazio ha 3 gradi di libertà (traslazione).

\end{itemize}

\sphinxAtStartPar
La configurazione di un \sphinxstylestrong{corpo rigido} è definita dalla posizione di un suo punto nello spazio e dalla sua orientazione:
\begin{itemize}
\item {} 
\sphinxAtStartPar
un corpo rigido nel piano ha 3 gradi di libertà, 2 per definire la posizione di un suo punto nello spazio (traslazione) e 1 per definire la sua orientazione (rotazione) rispetto a un asse ortogonale al piano;

\item {} 
\sphinxAtStartPar
un corpo rigido nello spazio ha 6 gradi di libertà, 3 per definire la posizione di un suo punto nello spazio (traslazione) e 3 per definire la sua orientazione (rotazione)

\end{itemize}

\sphinxstepscope

\begin{sphinxuseclass}{sd-container-fluid}
\begin{sphinxuseclass}{sd-sphinx-override}
\begin{sphinxuseclass}{sd-p-0}
\begin{sphinxuseclass}{sd-mt-2}
\begin{sphinxuseclass}{sd-mb-4}
\begin{sphinxuseclass}{sd-row}
\begin{sphinxuseclass}{sd-row-cols-2}
\begin{sphinxuseclass}{sd-gx-2}
\begin{sphinxuseclass}{sd-gy-1}
\begin{sphinxuseclass}{sd-col}
\begin{sphinxuseclass}{sd-d-flex-row}
\begin{sphinxuseclass}{sd-align-minor-center}
\begin{sphinxuseclass}{sd-container-fluid}
\begin{sphinxuseclass}{sd-sphinx-override}
\begin{sphinxuseclass}{sd-row}
\begin{sphinxuseclass}{sd-row-cols-2}
\begin{sphinxuseclass}{sd-row-cols-xs-2}
\begin{sphinxuseclass}{sd-row-cols-sm-3}
\begin{sphinxuseclass}{sd-row-cols-md-3}
\begin{sphinxuseclass}{sd-row-cols-lg-3}
\begin{sphinxuseclass}{sd-gx-3}
\begin{sphinxuseclass}{sd-gy-1}
\begin{sphinxuseclass}{sd-col}
\begin{sphinxuseclass}{sd-col-auto}
\begin{sphinxuseclass}{sd-d-flex-row}
\begin{sphinxuseclass}{sd-align-minor-center}
\sphinxAtStartPar
basics

\end{sphinxuseclass}
\end{sphinxuseclass}
\end{sphinxuseclass}
\end{sphinxuseclass}
\begin{sphinxuseclass}{sd-col}
\begin{sphinxuseclass}{sd-col-auto}
\begin{sphinxuseclass}{sd-d-flex-row}
\begin{sphinxuseclass}{sd-align-minor-center}
\sphinxAtStartPar
Nov 06, 2024

\end{sphinxuseclass}
\end{sphinxuseclass}
\end{sphinxuseclass}
\end{sphinxuseclass}
\begin{sphinxuseclass}{sd-col}
\begin{sphinxuseclass}{sd-col-auto}
\begin{sphinxuseclass}{sd-d-flex-row}
\begin{sphinxuseclass}{sd-align-minor-center}
\sphinxAtStartPar
2 min read

\end{sphinxuseclass}
\end{sphinxuseclass}
\end{sphinxuseclass}
\end{sphinxuseclass}
\end{sphinxuseclass}
\end{sphinxuseclass}
\end{sphinxuseclass}
\end{sphinxuseclass}
\end{sphinxuseclass}
\end{sphinxuseclass}
\end{sphinxuseclass}
\end{sphinxuseclass}
\end{sphinxuseclass}
\end{sphinxuseclass}
\end{sphinxuseclass}
\end{sphinxuseclass}
\end{sphinxuseclass}
\end{sphinxuseclass}
\end{sphinxuseclass}
\end{sphinxuseclass}
\end{sphinxuseclass}
\end{sphinxuseclass}
\end{sphinxuseclass}
\end{sphinxuseclass}
\end{sphinxuseclass}
\end{sphinxuseclass}

\section{Cinematica del punto}
\label{\detokenize{ch/mechanics/kinematics-point:cinematica-del-punto}}\label{\detokenize{ch/mechanics/kinematics-point:physics-hs-mechanics-kinematics-point}}\label{\detokenize{ch/mechanics/kinematics-point::doc}}
\sphinxAtStartPar
La cinematica di un punto \(P(t)\) è completamente definita dalla sua posizione nello spazio in funzione del tempo.

\sphinxAtStartPar
\sphinxstylestrong{Posizione di un punto.} \(P(t) - O = \vec{r}_P(t)\)

\sphinxAtStartPar
\sphinxstylestrong{Velocità di un punto.} \(\vec{v}_P = \dfrac{d \vec{r}_P}{dt}\)

\sphinxAtStartPar
\sphinxstylestrong{Accelerazione di un punto.} \(\vec{a}_P = \dfrac{d \vec{v}_P}{dt} = \dfrac{d^2 \vec{r}_P}{d t^2}\)


\subsection{Moti particolari}
\label{\detokenize{ch/mechanics/kinematics-point:moti-particolari}}

\subsubsection{Moto non accelerato}
\label{\detokenize{ch/mechanics/kinematics-point:moto-non-accelerato}}
\sphinxAtStartPar
Un moto non accelerato di un punto \(P\) rispetto a un sistema di riferimento con origine in \(O\) può essere definito dalla condizione di accelerazione nulla
\begin{equation*}
\begin{split}\vec{a}_P = \ddot{\vec{r}}_P(t) = \vec{0} \ ,\end{split}
\end{equation*}
\sphinxAtStartPar
la cui integrazione due volte in tempo fornisce le leggi della velocità e dello spazio
\begin{equation*}
\begin{split}\begin{cases}
  \vec{v}_P(t) & = \vec{c}_1 \\
  \vec{r}_P(t) & = \vec{c}_1 \, t + \vec{c}_2 \ .
\end{cases}\end{split}
\end{equation*}
\sphinxAtStartPar
\sphinxstylestrong{todo} dimostrazione come esercizio con procedimento

\sphinxAtStartPar
Si nota che la condizione di accelerazione nulla implica la condizione di velocità costante. \sphinxstylestrong{todo}

\sphinxAtStartPar
accompagnata da condizioni tra di loro compatibili (\sphinxstylestrong{todo} \sphinxstyleemphasis{fare esempio di condizioni non compatibili, es. velocità diverse in due istanti temporali diversi}) che identifichino unicamente il moto, come possono essere ad esempio:
\begin{itemize}
\item {} 
\sphinxAtStartPar
posizione e velocità a un istante temporale
\begin{equation*}
\begin{split}\begin{cases}
    \vec{r}(t_0) = \vec{r}_0 \\
    \vec{v}(t_0) = \vec{v}_0
  \end{cases}\end{split}
\end{equation*}
\sphinxAtStartPar
Le leggi del moto possono diventano
\begin{equation*}
\begin{split}\begin{cases}
    \vec{v}_P    & = \vec{v}_0  \\
    \vec{r}_P(t) & = \vec{v}_0 \, ( t - t_0 ) + \vec{r}_0 \ ,
  \end{cases}\end{split}
\end{equation*}
\sphinxAtStartPar
\sphinxstylestrong{todo} dimostrazione come esercizio con procedimento

\item {} 
\sphinxAtStartPar
posizione in due istanti temporali
\begin{equation*}
\begin{split}\begin{cases}
    \vec{r}(t_0) = \vec{r}_0 \\
    \vec{r}(t_1) = \vec{r}_1
  \end{cases}\end{split}
\end{equation*}
\sphinxAtStartPar
Le leggi del moto possono diventano
\begin{equation*}
\begin{split}\begin{cases}
    \vec{v}_P    & = \frac{\vec{r}_1 - \vec{r}_0}{t_1 - t_0} \\
    \vec{r}_P(t) & = \vec{v}_P \, ( t - t_0 ) + \vec{r}_0 \ ,
  \end{cases}\end{split}
\end{equation*}
\sphinxAtStartPar
\sphinxstylestrong{todo} dimostrazione come esercizio con procedimento

\end{itemize}




\subsubsection{Moto uniformemente accelerato}
\label{\detokenize{ch/mechanics/kinematics-point:moto-uniformemente-accelerato}}
\sphinxAtStartPar
Un moto uniformemente accelerato di un punto \(P\) rispetto a un sistema di riferimento con origine in \(O\) può essere definito dalla condizione di accelerazione costante
\begin{equation*}
\begin{split}\vec{a}_P = \ddot{\vec{r}}_P(t) = \vec{a} \ ,\end{split}
\end{equation*}
\sphinxAtStartPar
la cui integrazione due volte in tempo fornisce le leggi della velocità e dello spazio
\begin{equation*}
\begin{split}\begin{cases}
  \vec{v}_P(t) & = \vec{a} t               + \vec{c}_1 \\
  \vec{r}_P(t) & = \dfrac{1}{2}\vec{a} t^2 + \vec{c}_1 \, t + \vec{c}_2 \ .
\end{cases}\end{split}
\end{equation*}
\sphinxAtStartPar
\sphinxstylestrong{todo} dimostrazione come esercizio con procedimento

\sphinxAtStartPar
accompagnata da condizioni tra di loro compatibili che identifichino unicamente il moto, come possono essere ad esempio:
\begin{itemize}
\item {} 
\sphinxAtStartPar
posizione e velocità a un istante temporale
\begin{equation*}
\begin{split}\begin{cases}
    \vec{r}(t_0) = \vec{r}_0 \\
    \vec{v}(t_0) = \vec{v}_0
  \end{cases}\end{split}
\end{equation*}
\sphinxAtStartPar
Le leggi del moto possono diventano
\begin{equation*}
\begin{split}\begin{cases}
    \vec{v}_P(t) & = \vec{a} t                     + \vec{v}_0  \\
    \vec{r}_P(t) & = \dfrac{1}{2}\vec{a} (t-t_0)^2 + \vec{v}_0 \, ( t - t_0 ) + \vec{r}_0 \ ,
  \end{cases}\end{split}
\end{equation*}
\sphinxAtStartPar
\sphinxstylestrong{todo} dimostrazione come esercizio con procedimento

\item {} 
\sphinxAtStartPar
posizione in due istanti temporali… \sphinxstylestrong{todo}

\sphinxAtStartPar
\sphinxstylestrong{todo} dimostrazione come esercizio con procedimento

\end{itemize}




\subsubsection{Moto circolare}
\label{\detokenize{ch/mechanics/kinematics-point:moto-circolare}}
\sphinxAtStartPar
La cinematica di un punto su una traiettoria circolare (\sphinxstylestrong{todo} è un vincolo!) può essere rappresentato usando un sistema di \sphinxstylestrong{coordinate polari} con origine coincidente con il centro della circonferenza
\begin{equation*}
\begin{split}r = R \ , \quad \theta_P(t)=\text{**todo**}\end{split}
\end{equation*}
\sphinxAtStartPar
o in coordinate cartesiane
\begin{equation*}
\begin{split}\begin{cases}
 x_P(t) = R \, \cos \theta_P(t) \\
 y_P(t) = R \, \sin \theta_P(t) \\
\end{cases}\end{split}
\end{equation*}
\sphinxAtStartPar
che permettono di identificare il punto \(P\) con il raggio vettore rispetto all’origine
\begin{equation*}
\begin{split}\vec{r}_P = R \cos \theta_P(t) \hat{x} + R \sin \theta_P(t) \hat{y} \ .\end{split}
\end{equation*}\begin{itemize}
\item {} 
\sphinxAtStartPar
Definizione vettori \(\hat{r}\), \(\hat{\theta}\) \sphinxstylestrong{todo} *dipendenza di questi versori dalla posizione di \(P\) nello spazio, e quindi in generale dal tempo

\item {} 
\sphinxAtStartPar
La velocità e l’accelerazione del punto \sphinxstylestrong{todo}
\begin{itemize}
\item {} 
\sphinxAtStartPar
direzione e modulo di velocità e accelerazione

\end{itemize}

\end{itemize}
\begin{equation*}
\begin{split}\begin{cases}
  \vec{v}_P(t) & = R \dot{\theta}(t) \left( -\sin \theta_P(t) \hat{x} + \cos \theta_P(t) \hat{y} \right) = R \dot{\theta}(t) \hat{\theta}(t) \\
  \vec{a}_P(t) & = R \ddot{\theta}(t) \left( -\sin \theta_P(t) \hat{x} + \cos \theta_P(t) \hat{y} \right) + \\
               & + R \ddot{\theta}^2(t) \left( -\cos \theta_P(t) \hat{x} - \sin \theta_P(t) \hat{y} \right)     
                 = R \ddot{\theta}(t) \hat{\theta}(t) - R \dot{\theta}^2(t) \hat{r}(t) \ .
\end{cases}\end{split}
\end{equation*}

\paragraph{Moto circolare uniforme}
\label{\detokenize{ch/mechanics/kinematics-point:moto-circolare-uniforme}}
\sphinxAtStartPar
Il moto circolare uniforme ha modulo della velocità costante,
\begin{equation*}
\begin{split}|\vec{v}_P| = |R \dot \theta_P|\end{split}
\end{equation*}
\sphinxAtStartPar
e la derivata nel tempo della coordinata \(\theta_P\) è costante \sphinxstylestrong{todo}
\begin{itemize}
\item {} 
\sphinxAtStartPar
\sphinxstylestrong{todo} pulsazione, periodo, frequenza,…

\end{itemize}


\subsubsection{Moto armonico lungo un segmento}
\label{\detokenize{ch/mechanics/kinematics-point:moto-armonico-lungo-un-segmento}}
\sphinxAtStartPar
Un moto armonico lungo un segmento può essere definito come la proiezione di un punto che compie un moto circolare uniforme su una circonferenza che ha il segmento come diametro \sphinxstylestrong{todo}


\subsection{Problemi}
\label{\detokenize{ch/mechanics/kinematics-point:problemi}}
\sphinxstepscope

\begin{sphinxuseclass}{sd-container-fluid}
\begin{sphinxuseclass}{sd-sphinx-override}
\begin{sphinxuseclass}{sd-p-0}
\begin{sphinxuseclass}{sd-mt-2}
\begin{sphinxuseclass}{sd-mb-4}
\begin{sphinxuseclass}{sd-row}
\begin{sphinxuseclass}{sd-row-cols-2}
\begin{sphinxuseclass}{sd-gx-2}
\begin{sphinxuseclass}{sd-gy-1}
\begin{sphinxuseclass}{sd-col}
\begin{sphinxuseclass}{sd-d-flex-row}
\begin{sphinxuseclass}{sd-align-minor-center}
\begin{sphinxuseclass}{sd-container-fluid}
\begin{sphinxuseclass}{sd-sphinx-override}
\begin{sphinxuseclass}{sd-row}
\begin{sphinxuseclass}{sd-row-cols-2}
\begin{sphinxuseclass}{sd-row-cols-xs-2}
\begin{sphinxuseclass}{sd-row-cols-sm-3}
\begin{sphinxuseclass}{sd-row-cols-md-3}
\begin{sphinxuseclass}{sd-row-cols-lg-3}
\begin{sphinxuseclass}{sd-gx-3}
\begin{sphinxuseclass}{sd-gy-1}
\begin{sphinxuseclass}{sd-col}
\begin{sphinxuseclass}{sd-col-auto}
\begin{sphinxuseclass}{sd-d-flex-row}
\begin{sphinxuseclass}{sd-align-minor-center}
\sphinxAtStartPar
basics

\end{sphinxuseclass}
\end{sphinxuseclass}
\end{sphinxuseclass}
\end{sphinxuseclass}
\begin{sphinxuseclass}{sd-col}
\begin{sphinxuseclass}{sd-col-auto}
\begin{sphinxuseclass}{sd-d-flex-row}
\begin{sphinxuseclass}{sd-align-minor-center}
\sphinxAtStartPar
Nov 06, 2024

\end{sphinxuseclass}
\end{sphinxuseclass}
\end{sphinxuseclass}
\end{sphinxuseclass}
\begin{sphinxuseclass}{sd-col}
\begin{sphinxuseclass}{sd-col-auto}
\begin{sphinxuseclass}{sd-d-flex-row}
\begin{sphinxuseclass}{sd-align-minor-center}
\sphinxAtStartPar
2 min read

\end{sphinxuseclass}
\end{sphinxuseclass}
\end{sphinxuseclass}
\end{sphinxuseclass}
\end{sphinxuseclass}
\end{sphinxuseclass}
\end{sphinxuseclass}
\end{sphinxuseclass}
\end{sphinxuseclass}
\end{sphinxuseclass}
\end{sphinxuseclass}
\end{sphinxuseclass}
\end{sphinxuseclass}
\end{sphinxuseclass}
\end{sphinxuseclass}
\end{sphinxuseclass}
\end{sphinxuseclass}
\end{sphinxuseclass}
\end{sphinxuseclass}
\end{sphinxuseclass}
\end{sphinxuseclass}
\end{sphinxuseclass}
\end{sphinxuseclass}
\end{sphinxuseclass}
\end{sphinxuseclass}
\end{sphinxuseclass}

\section{Cinematica di un corpo rigido}
\label{\detokenize{ch/mechanics/kinematics-rigid:cinematica-di-un-corpo-rigido}}\label{\detokenize{ch/mechanics/kinematics-rigid:physics-hs-mechanics-kinematics-rigid}}\label{\detokenize{ch/mechanics/kinematics-rigid::doc}}
\sphinxAtStartPar
La cinematica di corpo rigido è definita dalla posizione di un suo punto materiale e dalla propria orientazione in funzione del tempo. In generale, per definire la posizione di corpo rigido nello spazio 3\sphinxhyphen{}dimensionale servono 6 parametri: 3 coordinate per definire la posizione di un punto materiale \(Q\) e 3 parametri per definire l’orientazione del corpo nello spazio. Per definire la posizione di un corpo rigido che compie un moto piano servono 3 parametri: 2 coordinate per definire la posizione di un punto e 1 parametro per definirne l’orientazione.

\sphinxAtStartPar
\sphinxstylestrong{todo} definizione di moto piano

\begin{sphinxadmonition}{note}{Note:}
\sphinxAtStartPar
Questo materiale è rivolto a studenti delle scuole superiori, e si limita a discutere il moto 2\sphinxhyphen{}dimensionale di corpi rigidi. Una discussione del moto 3\sphinxhyphen{}dimensionale di corpi rigidi richiede l’uso e la dimestichezza con oggetti matematici che non sono introdotti nei primi anni delle scuole superiori \sphinxhyphen{} e purtroppo troppo spesso nemmeno nei corsi universitari dei primi anni \sphinxhyphen{}, i tensori.

\sphinxAtStartPar
Al prezzo di non poter trattare i problemi meccanici più generali, questa scelta evita di richiedere la conoscenza dell’algebra tensoriale o di introdurre formule in forma quantomeno discutibile. Per una discussione completa del problema, si rimanda al materiale pensato per studenti più maturi: \sphinxstylestrong{todo}
\begin{itemize}
\item {} 
\sphinxAtStartPar
algebra vettoriale e tensoriale \sphinxstylestrong{todo}

\item {} 
\sphinxAtStartPar
meccanica classica \sphinxstylestrong{todo}

\end{itemize}
\end{sphinxadmonition}




\subsection{Posizione dei punti di un corpo rigido}
\label{\detokenize{ch/mechanics/kinematics-rigid:posizione-dei-punti-di-un-corpo-rigido}}\begin{itemize}
\item {} 
\sphinxAtStartPar
\sphinxstylestrong{Posizione del un punto materiale di riferimento, \(Q\).}

\end{itemize}
\begin{equation*}
\begin{split}Q - O = \vec{r}_Q\end{split}
\end{equation*}\begin{itemize}
\item {} 
\sphinxAtStartPar
\sphinxstylestrong{Posizione di tutti i punti materiali \(P\) del corpo rigido, e orientazione del corpo.} Nell’ipotesi di moto 2\sphinxhyphen{}dimensionale, il vettore tra due punti materiali \(\vec{r}_{QP} = P-Q\) può essere scritto in funzione del vettore \(\vec{r}_{QP}^0\) nella configurazione di riferimento del corpo e della rotazione di un angolo \(\theta\) attorno a un asse di direzione \(\hat{n}\) costante e perpendicolare al piano del moto,

\end{itemize}
\begin{equation*}
\begin{split}P - Q = \vec{r}_{QP} = \cos \theta \, \vec{r}_{QP}^0 + \sin \theta \, \hat{n} \times \vec{r}_{QP}^0\end{split}
\end{equation*}
\sphinxAtStartPar
La posizione di un punto materiale \(P\) di un corpo rigido rispetto al sistema di riferimento scelto, può essere quindi scritta come
\begin{equation*}
\begin{split}\begin{aligned}
    P - O & = Q - O + P - Q = \\
          & = \vec{r}_{OQ} + \cos \theta \, \vec{r}_{QP}^0 + \sin \theta \, \hat{n} \times \vec{r}_{QP}^0  \ .
  \end{aligned}\end{split}
\end{equation*}

\subsection{Velocità dei punti di un corpo rigido}
\label{\detokenize{ch/mechanics/kinematics-rigid:velocita-dei-punti-di-un-corpo-rigido}}\begin{itemize}
\item {} 
\sphinxAtStartPar
\sphinxstylestrong{Velocità del punto materiale di riferimento, \(Q\)}

\end{itemize}
\begin{equation*}
\begin{split}\vec{v}_Q = \dfrac{d \vec{r}_Q}{dt}\end{split}
\end{equation*}\begin{itemize}
\item {} 
\sphinxAtStartPar
\sphinxstylestrong{Velocità di tutti i punti materiali \(P\) del corpo rigido, e velocità angolare del corpo}, \(\vec{\omega} = \dot{\theta} \hat{n}\). La velocità relativa di un punto \(P\) rispetto al punto di riferimento \(Q\) viene calcolata con la derivata del vettore \(\vec{r}_{QP}\) rispetto al tempo, ricordando che \(\hat{n}\) è costante e quindi \(\frac{d}{dt} \hat{n} = \vec{0}\),
\begin{equation*}
\begin{split}\begin{aligned}
    \dfrac{d \vec{r}_{QP}}{dt} 
    & = \dfrac{d}{dt} \left(  \cos \theta \, \vec{r}_{QP}^0 + \sin \theta \, \hat{n} \times \vec{r}_{QP}^0 \right) = \\
    & = \dot{\theta} \left( -\sin \theta \, \vec{r}_{QP}^0 + \cos \theta \, \hat{n} \times \vec{r}_{QP}^0 \right) = \\
    & = \dot{\theta} \hat{n} \times \left( \sin \theta \, \hat{n} \times \vec{r}_{QP}^0 + \cos \theta \, \vec{r}_{QP}^0 \right) = \\
    & = \dot{\theta} \hat{n} \times \vec{r}_{QP} = \\
    & = \vec{\omega} \times \vec{r}_{QP} \ ,
  \end{aligned}\end{split}
\end{equation*}
\sphinxAtStartPar
avendo definito la \sphinxstylestrong{velocità angolare}, \(\vec{\omega} = \dot{\theta} \hat{k}\) per un moto 2\sphinxhyphen{}dimensionale, e usato l’identità vettoriale
\begin{equation*}
\begin{split}\vec{n} \times (\vec{n} \times \vec{w}) = \hat{n} \underbrace{(\hat{n} \cdot \vec{w})}_{=0} - \vec{w} \underbrace{(\hat{n} \cdot \hat{n})}_{=1} = - \vec{w} \ .\end{split}
\end{equation*}
\end{itemize}

\begin{sphinxadmonition}{note}{Note:}
\sphinxAtStartPar
La formula
\begin{equation*}
\begin{split}\vec{v}_P - \vec{v}_Q = \vec{\omega} \times (P - Q)\end{split}
\end{equation*}
\sphinxAtStartPar
vale anche per moti 3\sphinxhyphen{}dimensionali. In questo caso però \sphinxstylestrong{non} è possibile scrivere \(\vec{\omega} = \dot{\theta} \hat{n}\).
\end{sphinxadmonition}

\sphinxAtStartPar
La velocità di un punto materiale \(P\) di un corpo rigido rispetto al sistema di riferimento scelto, può essere quindi scritta come
\begin{equation*}
\begin{split}\begin{aligned}
    \vec{v}_P & = \vec{v}_{Q/O} + \vec{v}_{P/Q} = \\
              & = \vec{v}_{Q/O} + \vec{\omega} \times (P - Q) \ .
  \end{aligned}\end{split}
\end{equation*}

\subsection{Accelerazione dei punti di un corpo rigido}
\label{\detokenize{ch/mechanics/kinematics-rigid:accelerazione-dei-punti-di-un-corpo-rigido}}\begin{itemize}
\item {} 
\sphinxAtStartPar
\sphinxstylestrong{Accelerazione del punto materiale di riferimento, \(Q\)}

\end{itemize}
\begin{equation*}
\begin{split}\vec{a}_P = \dfrac{d \vec{v}_P}{dt} = \dfrac{d^2 \vec{r}_P}{d t^2}\end{split}
\end{equation*}\begin{itemize}
\item {} 
\sphinxAtStartPar
\sphinxstylestrong{Accelerazione di tutti i punti materiali \(P\) del corpo rigido, e accelerazione angolare del corpo}, \(\vec{\alpha} = \dot{\vec{\omega}} = \ddot{\theta} \hat{n}\). L’accelerazione relativa di un punto \(P\) rispetto al punto di riferimento \(Q\) viene calcolata con la derivata seconda del vettore \(\vec{r}_{QP}\) rispetto al tempo, ricordando che \(\hat{n}\) è costante e quindi \(\frac{d}{dt} \hat{n} = \vec{0}\),
\begin{equation*}
\begin{split}\begin{aligned}
   \dfrac{d^2 \vec{r}_{QP}}{dt^2}
     & = \dfrac{d}{dt} \left( \vec{\omega} \times \vec{r}_{QP} \right) = \\
     & = \dfrac{d \vec{\omega}}{dt} \times \vec{r}_{QP} + \vec{\omega} \times \dfrac{d \vec{r}_{QP}}{dt}= \\
     & = \alpha \times \vec{r}_{QP} + \vec{\omega} \times \left( \vec{\omega} \times \vec{r}_{QP} \right) \ .
  \end{aligned}\end{split}
\end{equation*}
\end{itemize}

\sphinxAtStartPar
L’accelerazione di un punto materiale \(P\) di un corpo rigido rispetto al sistema di riferimento scelto, può essere quindi scritta come
\begin{equation*}
\begin{split}\begin{aligned}
    \vec{a}_P & = \vec{a}_{Q/O} + \vec{a}_{P/Q} = \\
              & = \vec{a}_{Q/O} + \alpha \times \vec{r}_{QP} + \vec{\omega} \times \left( \vec{\omega} \times \vec{r}_{QP} \right) \ .
  \end{aligned}\end{split}
\end{equation*}
\sphinxstepscope

\begin{sphinxuseclass}{sd-container-fluid}
\begin{sphinxuseclass}{sd-sphinx-override}
\begin{sphinxuseclass}{sd-p-0}
\begin{sphinxuseclass}{sd-mt-2}
\begin{sphinxuseclass}{sd-mb-4}
\begin{sphinxuseclass}{sd-row}
\begin{sphinxuseclass}{sd-row-cols-2}
\begin{sphinxuseclass}{sd-gx-2}
\begin{sphinxuseclass}{sd-gy-1}
\begin{sphinxuseclass}{sd-col}
\begin{sphinxuseclass}{sd-d-flex-row}
\begin{sphinxuseclass}{sd-align-minor-center}
\begin{sphinxuseclass}{sd-container-fluid}
\begin{sphinxuseclass}{sd-sphinx-override}
\begin{sphinxuseclass}{sd-row}
\begin{sphinxuseclass}{sd-row-cols-2}
\begin{sphinxuseclass}{sd-row-cols-xs-2}
\begin{sphinxuseclass}{sd-row-cols-sm-3}
\begin{sphinxuseclass}{sd-row-cols-md-3}
\begin{sphinxuseclass}{sd-row-cols-lg-3}
\begin{sphinxuseclass}{sd-gx-3}
\begin{sphinxuseclass}{sd-gy-1}
\begin{sphinxuseclass}{sd-col}
\begin{sphinxuseclass}{sd-col-auto}
\begin{sphinxuseclass}{sd-d-flex-row}
\begin{sphinxuseclass}{sd-align-minor-center}
\sphinxAtStartPar
basics

\end{sphinxuseclass}
\end{sphinxuseclass}
\end{sphinxuseclass}
\end{sphinxuseclass}
\begin{sphinxuseclass}{sd-col}
\begin{sphinxuseclass}{sd-col-auto}
\begin{sphinxuseclass}{sd-d-flex-row}
\begin{sphinxuseclass}{sd-align-minor-center}
\sphinxAtStartPar
Nov 06, 2024

\end{sphinxuseclass}
\end{sphinxuseclass}
\end{sphinxuseclass}
\end{sphinxuseclass}
\begin{sphinxuseclass}{sd-col}
\begin{sphinxuseclass}{sd-col-auto}
\begin{sphinxuseclass}{sd-d-flex-row}
\begin{sphinxuseclass}{sd-align-minor-center}
\sphinxAtStartPar
0 min read

\end{sphinxuseclass}
\end{sphinxuseclass}
\end{sphinxuseclass}
\end{sphinxuseclass}
\end{sphinxuseclass}
\end{sphinxuseclass}
\end{sphinxuseclass}
\end{sphinxuseclass}
\end{sphinxuseclass}
\end{sphinxuseclass}
\end{sphinxuseclass}
\end{sphinxuseclass}
\end{sphinxuseclass}
\end{sphinxuseclass}
\end{sphinxuseclass}
\end{sphinxuseclass}
\end{sphinxuseclass}
\end{sphinxuseclass}
\end{sphinxuseclass}
\end{sphinxuseclass}
\end{sphinxuseclass}
\end{sphinxuseclass}
\end{sphinxuseclass}
\end{sphinxuseclass}
\end{sphinxuseclass}
\end{sphinxuseclass}

\section{Cinematica dei sistemi deformabili}
\label{\detokenize{ch/mechanics/kinematics-deformable:cinematica-dei-sistemi-deformabili}}\label{\detokenize{ch/mechanics/kinematics-deformable::doc}}
\sphinxstepscope


\section{Cinematica relativa}
\label{\detokenize{ch/mechanics/kinematics-relative:cinematica-relativa}}\label{\detokenize{ch/mechanics/kinematics-relative:physics-hs-mechanics-kinematics-relative}}\label{\detokenize{ch/mechanics/kinematics-relative::doc}}
\sphinxstepscope

\begin{sphinxuseclass}{sd-container-fluid}
\begin{sphinxuseclass}{sd-sphinx-override}
\begin{sphinxuseclass}{sd-p-0}
\begin{sphinxuseclass}{sd-mt-2}
\begin{sphinxuseclass}{sd-mb-4}
\begin{sphinxuseclass}{sd-row}
\begin{sphinxuseclass}{sd-row-cols-2}
\begin{sphinxuseclass}{sd-gx-2}
\begin{sphinxuseclass}{sd-gy-1}
\begin{sphinxuseclass}{sd-col}
\begin{sphinxuseclass}{sd-d-flex-row}
\begin{sphinxuseclass}{sd-align-minor-center}
\begin{sphinxuseclass}{sd-container-fluid}
\begin{sphinxuseclass}{sd-sphinx-override}
\begin{sphinxuseclass}{sd-row}
\begin{sphinxuseclass}{sd-row-cols-2}
\begin{sphinxuseclass}{sd-row-cols-xs-2}
\begin{sphinxuseclass}{sd-row-cols-sm-3}
\begin{sphinxuseclass}{sd-row-cols-md-3}
\begin{sphinxuseclass}{sd-row-cols-lg-3}
\begin{sphinxuseclass}{sd-gx-3}
\begin{sphinxuseclass}{sd-gy-1}
\begin{sphinxuseclass}{sd-col}
\begin{sphinxuseclass}{sd-col-auto}
\begin{sphinxuseclass}{sd-d-flex-row}
\begin{sphinxuseclass}{sd-align-minor-center}
\sphinxAtStartPar
basics

\end{sphinxuseclass}
\end{sphinxuseclass}
\end{sphinxuseclass}
\end{sphinxuseclass}
\begin{sphinxuseclass}{sd-col}
\begin{sphinxuseclass}{sd-col-auto}
\begin{sphinxuseclass}{sd-d-flex-row}
\begin{sphinxuseclass}{sd-align-minor-center}
\sphinxAtStartPar
Nov 06, 2024

\end{sphinxuseclass}
\end{sphinxuseclass}
\end{sphinxuseclass}
\end{sphinxuseclass}
\begin{sphinxuseclass}{sd-col}
\begin{sphinxuseclass}{sd-col-auto}
\begin{sphinxuseclass}{sd-d-flex-row}
\begin{sphinxuseclass}{sd-align-minor-center}
\sphinxAtStartPar
0 min read

\end{sphinxuseclass}
\end{sphinxuseclass}
\end{sphinxuseclass}
\end{sphinxuseclass}
\end{sphinxuseclass}
\end{sphinxuseclass}
\end{sphinxuseclass}
\end{sphinxuseclass}
\end{sphinxuseclass}
\end{sphinxuseclass}
\end{sphinxuseclass}
\end{sphinxuseclass}
\end{sphinxuseclass}
\end{sphinxuseclass}
\end{sphinxuseclass}
\end{sphinxuseclass}
\end{sphinxuseclass}
\end{sphinxuseclass}
\end{sphinxuseclass}
\end{sphinxuseclass}
\end{sphinxuseclass}
\end{sphinxuseclass}
\end{sphinxuseclass}
\end{sphinxuseclass}
\end{sphinxuseclass}
\end{sphinxuseclass}

\chapter{Azioni}
\label{\detokenize{ch/mechanics/actions:azioni}}\label{\detokenize{ch/mechanics/actions:physics-hs-mechanics-actions}}\label{\detokenize{ch/mechanics/actions::doc}}\begin{itemize}
\item {} 
\sphinxAtStartPar
Tipi di azione:
\begin{itemize}
\item {} 
\sphinxAtStartPar
forza concentrata, momento di una forza, coppia di forze; carichi equivalenti

\item {} 
\sphinxAtStartPar
azioni distribuite: azioni di volume, e di superficie (sforzo e pressione)

\end{itemize}

\item {} 
\sphinxAtStartPar
Lavoro e potenza; azioni conservative

\item {} 
\sphinxAtStartPar
Esempi:
\begin{itemize}
\item {} 
\sphinxAtStartPar
gravitazione: gravitazione universale, nei pressi della superficie terrestre; cenni a interazione a distanza

\item {} 
\sphinxAtStartPar
elasticità

\item {} 
\sphinxAtStartPar
reazioni vincolari

\item {} 
\sphinxAtStartPar
contatto: reazione normale e attrito

\item {} 
\sphinxAtStartPar
cenni ad altre azioni (tra cariche elettriche, cariche in campi EM,…; esempi: levmag,…)

\end{itemize}

\end{itemize}





\sphinxstepscope

\begin{sphinxuseclass}{sd-container-fluid}
\begin{sphinxuseclass}{sd-sphinx-override}
\begin{sphinxuseclass}{sd-p-0}
\begin{sphinxuseclass}{sd-mt-2}
\begin{sphinxuseclass}{sd-mb-4}
\begin{sphinxuseclass}{sd-row}
\begin{sphinxuseclass}{sd-row-cols-2}
\begin{sphinxuseclass}{sd-gx-2}
\begin{sphinxuseclass}{sd-gy-1}
\begin{sphinxuseclass}{sd-col}
\begin{sphinxuseclass}{sd-d-flex-row}
\begin{sphinxuseclass}{sd-align-minor-center}
\begin{sphinxuseclass}{sd-container-fluid}
\begin{sphinxuseclass}{sd-sphinx-override}
\begin{sphinxuseclass}{sd-row}
\begin{sphinxuseclass}{sd-row-cols-2}
\begin{sphinxuseclass}{sd-row-cols-xs-2}
\begin{sphinxuseclass}{sd-row-cols-sm-3}
\begin{sphinxuseclass}{sd-row-cols-md-3}
\begin{sphinxuseclass}{sd-row-cols-lg-3}
\begin{sphinxuseclass}{sd-gx-3}
\begin{sphinxuseclass}{sd-gy-1}
\begin{sphinxuseclass}{sd-col}
\begin{sphinxuseclass}{sd-col-auto}
\begin{sphinxuseclass}{sd-d-flex-row}
\begin{sphinxuseclass}{sd-align-minor-center}
\sphinxAtStartPar
basics

\end{sphinxuseclass}
\end{sphinxuseclass}
\end{sphinxuseclass}
\end{sphinxuseclass}
\begin{sphinxuseclass}{sd-col}
\begin{sphinxuseclass}{sd-col-auto}
\begin{sphinxuseclass}{sd-d-flex-row}
\begin{sphinxuseclass}{sd-align-minor-center}
\sphinxAtStartPar
Nov 06, 2024

\end{sphinxuseclass}
\end{sphinxuseclass}
\end{sphinxuseclass}
\end{sphinxuseclass}
\begin{sphinxuseclass}{sd-col}
\begin{sphinxuseclass}{sd-col-auto}
\begin{sphinxuseclass}{sd-d-flex-row}
\begin{sphinxuseclass}{sd-align-minor-center}
\sphinxAtStartPar
1 min read

\end{sphinxuseclass}
\end{sphinxuseclass}
\end{sphinxuseclass}
\end{sphinxuseclass}
\end{sphinxuseclass}
\end{sphinxuseclass}
\end{sphinxuseclass}
\end{sphinxuseclass}
\end{sphinxuseclass}
\end{sphinxuseclass}
\end{sphinxuseclass}
\end{sphinxuseclass}
\end{sphinxuseclass}
\end{sphinxuseclass}
\end{sphinxuseclass}
\end{sphinxuseclass}
\end{sphinxuseclass}
\end{sphinxuseclass}
\end{sphinxuseclass}
\end{sphinxuseclass}
\end{sphinxuseclass}
\end{sphinxuseclass}
\end{sphinxuseclass}
\end{sphinxuseclass}
\end{sphinxuseclass}
\end{sphinxuseclass}

\section{Forza, momento di una forza, azioni distribuite}
\label{\detokenize{ch/mechanics/actions-types:forza-momento-di-una-forza-azioni-distribuite}}\label{\detokenize{ch/mechanics/actions-types:physics-hs-mechanics-actions-def}}\label{\detokenize{ch/mechanics/actions-types::doc}}

\subsection{Forza concentrata}
\label{\detokenize{ch/mechanics/actions-types:forza-concentrata}}
\sphinxAtStartPar
Una forza (concentrata) è una quantità vettoriale di dimensioni fisiche,
\begin{equation*}
\begin{split}[\text{forza}] = \frac{\text{[massa]}\text{[lunghezza]}}{\text{[tempo]}^2}\end{split}
\end{equation*}
\sphinxAtStartPar
che può essere misurata tramite un dinamometro, e il cui effetto può alterare le condizioni di equilibrio o di moto di un sistema fisico.

\sphinxAtStartPar
Oltre alle informazioni tipiche di una quantità vettoriale \sphinxhyphen{} intensità, direzione e verso \sphinxhyphen{} contenute nel vettore forza \(\vec{F}\), è spesso necessario conoscere il \sphinxstylestrong{punto di applicazione}, o la retta di applicazione, della forza.


\subsection{Momento di una forza concentrata}
\label{\detokenize{ch/mechanics/actions-types:momento-di-una-forza-concentrata}}
\sphinxAtStartPar
Il momento di una forza \(\vec{F}\) applicata nel punto \(P\), o con retta di applicazione passante per \(P\), rispetto al punto \(H\) viene definito come il prodotto vettoriale,
\begin{equation*}
\begin{split}\vec{M}_H = (P - H) \times \vec{F}\end{split}
\end{equation*}

\subsection{Sistema di forze, risultante delle azioni e carichi equivalenti}
\label{\detokenize{ch/mechanics/actions-types:sistema-di-forze-risultante-delle-azioni-e-carichi-equivalenti}}
\sphinxAtStartPar
Dato un sistema di \(N\) forze \(\left\{ \vec{F}_n \right\}_{n=1:N}\), applicate nei punti \(P_n\), si definiscono:
\begin{itemize}
\item {} 
\sphinxAtStartPar
\sphinxstylestrong{risultante} del sistema di forze: la somma delle forze,
\begin{equation*}
\begin{split}\vec{R} = \sum_{n=1}^{N} \vec{F}_n \ ,\end{split}
\end{equation*}
\item {} 
\sphinxAtStartPar
risultante dei momenti rispetto a un punto \(H\): la somma dei momenti
\begin{equation*}
\begin{split}\vec{M}_H = \sum_{n=1}^{N} (P_n - H) \times \vec{F}_n \ ,\end{split}
\end{equation*}
\item {} 
\sphinxAtStartPar
un \sphinxstylestrong{carico equivalente}: un sistema di forze che ha la stessa risultante di forze e momenti; per un sistema di forze, è possibile definire un carico equivalente formato da una sola forza, la risultante delle forze \(\vec{R}\) applicata nel punto \(Q\) ricavato dall’equivalenza ai momenti
\begin{equation*}
\begin{split}\begin{aligned}
    \vec{R} & = \sum_{n=1}^{N} \vec{F}_n \\
    (Q - H) \times \vec{R} & = \sum_{n=1}^{N} (P_n - H) \times \vec{F}_n \\
  \end{aligned}\end{split}
\end{equation*}
\end{itemize}


\subsection{Coppia di forze}
\label{\detokenize{ch/mechanics/actions-types:coppia-di-forze}}
\sphinxAtStartPar
Una coppia di forze è un carico equivalente a due forze di uguale intensità e verso opposto, \(\vec{F}_2 = - \vec{F}_1\), applicate in due punti \(P_1\), \(P_2\) non allineati lungo la retta di applicazione delle forze per avere effetti non nulli.

\sphinxAtStartPar
\sphinxstylestrong{todo} \sphinxstyleemphasis{immagine}

\sphinxAtStartPar
La risultante delle forze è nulla,
\begin{equation*}
\begin{split}\vec{R} = \vec{F}_1 + \vec{F}_2 = \vec{F}_1 - \vec{F}_1 = \vec{0} \ ,\end{split}
\end{equation*}
\sphinxAtStartPar
mentre la risultante dei momenti non dipende dal polo dei momenti,
\begin{equation*}
\begin{split}\begin{aligned}
  \vec{M}_H & = (P_1 - H) \times \vec{F}_1 + (P_2 - H) \times \vec{F}_2 = \\
  & = (P_1 - H) \times \vec{F}_1 - (P_2 - H) \times \vec{F}_1 = \\
  & = (P_1 - P_2) \times \vec{F}_1 =: \vec{C} \ .
\end{aligned}\end{split}
\end{equation*}

\subsection{Campi di forze}
\label{\detokenize{ch/mechanics/actions-types:campi-di-forze}}
\sphinxAtStartPar
\sphinxstylestrong{todo}


\subsection{Azioni distribuite}
\label{\detokenize{ch/mechanics/actions-types:azioni-distribuite}}
\sphinxAtStartPar
\sphinxstylestrong{todo}

\sphinxstepscope

\begin{sphinxuseclass}{sd-container-fluid}
\begin{sphinxuseclass}{sd-sphinx-override}
\begin{sphinxuseclass}{sd-p-0}
\begin{sphinxuseclass}{sd-mt-2}
\begin{sphinxuseclass}{sd-mb-4}
\begin{sphinxuseclass}{sd-row}
\begin{sphinxuseclass}{sd-row-cols-2}
\begin{sphinxuseclass}{sd-gx-2}
\begin{sphinxuseclass}{sd-gy-1}
\begin{sphinxuseclass}{sd-col}
\begin{sphinxuseclass}{sd-d-flex-row}
\begin{sphinxuseclass}{sd-align-minor-center}
\begin{sphinxuseclass}{sd-container-fluid}
\begin{sphinxuseclass}{sd-sphinx-override}
\begin{sphinxuseclass}{sd-row}
\begin{sphinxuseclass}{sd-row-cols-2}
\begin{sphinxuseclass}{sd-row-cols-xs-2}
\begin{sphinxuseclass}{sd-row-cols-sm-3}
\begin{sphinxuseclass}{sd-row-cols-md-3}
\begin{sphinxuseclass}{sd-row-cols-lg-3}
\begin{sphinxuseclass}{sd-gx-3}
\begin{sphinxuseclass}{sd-gy-1}
\begin{sphinxuseclass}{sd-col}
\begin{sphinxuseclass}{sd-col-auto}
\begin{sphinxuseclass}{sd-d-flex-row}
\begin{sphinxuseclass}{sd-align-minor-center}
\sphinxAtStartPar
basics

\end{sphinxuseclass}
\end{sphinxuseclass}
\end{sphinxuseclass}
\end{sphinxuseclass}
\begin{sphinxuseclass}{sd-col}
\begin{sphinxuseclass}{sd-col-auto}
\begin{sphinxuseclass}{sd-d-flex-row}
\begin{sphinxuseclass}{sd-align-minor-center}
\sphinxAtStartPar
Nov 06, 2024

\end{sphinxuseclass}
\end{sphinxuseclass}
\end{sphinxuseclass}
\end{sphinxuseclass}
\begin{sphinxuseclass}{sd-col}
\begin{sphinxuseclass}{sd-col-auto}
\begin{sphinxuseclass}{sd-d-flex-row}
\begin{sphinxuseclass}{sd-align-minor-center}
\sphinxAtStartPar
2 min read

\end{sphinxuseclass}
\end{sphinxuseclass}
\end{sphinxuseclass}
\end{sphinxuseclass}
\end{sphinxuseclass}
\end{sphinxuseclass}
\end{sphinxuseclass}
\end{sphinxuseclass}
\end{sphinxuseclass}
\end{sphinxuseclass}
\end{sphinxuseclass}
\end{sphinxuseclass}
\end{sphinxuseclass}
\end{sphinxuseclass}
\end{sphinxuseclass}
\end{sphinxuseclass}
\end{sphinxuseclass}
\end{sphinxuseclass}
\end{sphinxuseclass}
\end{sphinxuseclass}
\end{sphinxuseclass}
\end{sphinxuseclass}
\end{sphinxuseclass}
\end{sphinxuseclass}
\end{sphinxuseclass}
\end{sphinxuseclass}

\section{Lavoro e potenza}
\label{\detokenize{ch/mechanics/actions-work:lavoro-e-potenza}}\label{\detokenize{ch/mechanics/actions-work:physics-hs-mechanics-actions-work}}\label{\detokenize{ch/mechanics/actions-work::doc}}
\sphinxAtStartPar
In meccanica, come sarà più chiaro avanti (\sphinxstylestrong{todo} aggiungere riferimento), il concetto di lavoro è legato al concetto di energia. \sphinxstylestrong{todo}


\subsection{Lavoro e potenza di una forza}
\label{\detokenize{ch/mechanics/actions-work:lavoro-e-potenza-di-una-forza}}
\sphinxAtStartPar
\sphinxstylestrong{Lavoro.} Il lavoro elementare di una forza \(\vec{F}\) applicata nel punto \(P\) che subisce uno spostamento elementare \(d \vec{r}_P\) è definito come il prodotto scalare tra la forza e lo spostamento,
\begin{equation*}
\begin{split}\delta L := \vec{F} \cdot d \vec{r}_P \ .\end{split}
\end{equation*}
\sphinxAtStartPar
Il lavoro compiuto dalla forza \(\vec{F}\) applicata nel punto \(P\) che si muove dal punto \(A\) al punto \(B\) lungo il percorso \(\ell_{AB}\) è la somma di tutti i contributi elementari \sphinxhyphen{} e quindi, al limite per spostamenti elementari \(\rightarrow 0\) per variazioni continue, l’integrale di linea,
\begin{equation*}
\begin{split}L_{\ell_{AB}} = \int_{\ell_{AB}} \delta L = \int_{\ell_{AB}} \vec{F} \cdot d \vec{r}_{P} \ .\end{split}
\end{equation*}
\sphinxAtStartPar
In generale, il lavoro di una forza o di un campo di forze dipende dal percorso \({\ell}_{AB}\). Nei casi in cui il lavoro è indipendente dal percorso, ma dipende solo dagli estremi del percorso, si parla di {\hyperref[\detokenize{ch/mechanics/actions-conservative:physics-hs-mechanics-actions-conservative}]{\sphinxcrossref{\DUrole{std,std-ref}{azioni conservative}}}}.

\sphinxAtStartPar
\sphinxstylestrong{Potenza.} La potenza della forza viene definita come la derivata nel tempo del lavoro,
\begin{equation*}
\begin{split}P := \frac{\delta L}{dt} = \vec{F} \cdot \frac{d \vec{r}_P}{d t} = \vec{F} \cdot \vec{v}_P \ , \end{split}
\end{equation*}
\sphinxAtStartPar
e coincide con il prodotto scalare tra la forza e la velocità del punto di applicazione. Prestare attenzione se una forza è applicata a punti geometrici e non materiali, come ad esempio il caso di una disco che rotola senza strisciare su una superficie: in ogni istante il (nuovo) punto materiale di contatto ha velocità nulla, mentre il punto geometricodi contatto è la proiezione del centro del disco e si muove con la stessa velocità, \(v = R \theta\)


\subsection{Lavoro e potenza di un sistema di forze}
\label{\detokenize{ch/mechanics/actions-work:lavoro-e-potenza-di-un-sistema-di-forze}}
\sphinxAtStartPar
\sphinxstylestrong{Lavoro.} Il lavoro di un sistema di forze è la somma dei lavori delle singole forze,
\begin{equation*}
\begin{split}\delta L = \sum_{n=1}^{N} \delta L_n = \sum_{n=1}^{N} \vec{F}_n \cdot d \vec{r}_n\end{split}
\end{equation*}
\sphinxAtStartPar
\sphinxstylestrong{Potenza.} La potenza di un sistema di forze è la somma delle potenze delle singole forze
\begin{equation*}
\begin{split}P = \sum_{n=1}^{N} P_n = \sum_{n=1}^{N} \vec{F}_n \cdot \vec{v}_n \ .\end{split}
\end{equation*}

\subsection{Lavoro e potenza di una coppia di forze}
\label{\detokenize{ch/mechanics/actions-work:lavoro-e-potenza-di-una-coppia-di-forze}}
\sphinxAtStartPar
\sphinxstylestrong{Lavoro.} Il lavoro elementare di una coppia di forze è la somma dei lavori elementari
\begin{equation*}
\begin{split}\begin{aligned}
  \delta L & = \vec{F}_1 \cdot d \vec{r}_1 + \vec{F}_2 \cdot d \vec{r}_2 = \\
           & = \vec{F}_1 \cdot ( d \vec{r}_1 - d \vec{r}_2 ) = 
\end{aligned}\end{split}
\end{equation*}
\sphinxAtStartPar
\sphinxstylestrong{Potenza.} La potenza di una coppia di forze,
\begin{equation*}
\begin{split}P = \vec{F}_1 \cdot (\vec{v}_1 - \vec{v}_2)\end{split}
\end{equation*}
\sphinxAtStartPar
può essere riscritta se i punti di applicazione compiono un atto di moto rigido (\sphinxstylestrong{todo} verificare la definizione di atto di moto e se è il caso di introdurla),
\begin{equation*}
\begin{split}\vec{v}_1 - \vec{v}_2 = \vec{\omega} \times (P_1 - P_2) \ ,\end{split}
\end{equation*}
\sphinxAtStartPar
come
\begin{equation*}
\begin{split}\begin{aligned}
  P & =  \vec{F}_1 \cdot (\vec{v}_1 - \vec{v}_2) = \\
    & =  \vec{F}_1 \cdot \left[ \vec{\omega} \times (P_1 - P_2) \right] = \\
    & =  \vec{\omega} \cdot \left[ (P_1 - P_2) \times \vec{F}_1\right] = \\
    & =  \vec{\omega} \cdot \vec{C} \ . 
\end{aligned}\end{split}
\end{equation*}
\sphinxstepscope

\begin{sphinxuseclass}{sd-container-fluid}
\begin{sphinxuseclass}{sd-sphinx-override}
\begin{sphinxuseclass}{sd-p-0}
\begin{sphinxuseclass}{sd-mt-2}
\begin{sphinxuseclass}{sd-mb-4}
\begin{sphinxuseclass}{sd-row}
\begin{sphinxuseclass}{sd-row-cols-2}
\begin{sphinxuseclass}{sd-gx-2}
\begin{sphinxuseclass}{sd-gy-1}
\begin{sphinxuseclass}{sd-col}
\begin{sphinxuseclass}{sd-d-flex-row}
\begin{sphinxuseclass}{sd-align-minor-center}
\begin{sphinxuseclass}{sd-container-fluid}
\begin{sphinxuseclass}{sd-sphinx-override}
\begin{sphinxuseclass}{sd-row}
\begin{sphinxuseclass}{sd-row-cols-2}
\begin{sphinxuseclass}{sd-row-cols-xs-2}
\begin{sphinxuseclass}{sd-row-cols-sm-3}
\begin{sphinxuseclass}{sd-row-cols-md-3}
\begin{sphinxuseclass}{sd-row-cols-lg-3}
\begin{sphinxuseclass}{sd-gx-3}
\begin{sphinxuseclass}{sd-gy-1}
\begin{sphinxuseclass}{sd-col}
\begin{sphinxuseclass}{sd-col-auto}
\begin{sphinxuseclass}{sd-d-flex-row}
\begin{sphinxuseclass}{sd-align-minor-center}
\sphinxAtStartPar
basics

\end{sphinxuseclass}
\end{sphinxuseclass}
\end{sphinxuseclass}
\end{sphinxuseclass}
\begin{sphinxuseclass}{sd-col}
\begin{sphinxuseclass}{sd-col-auto}
\begin{sphinxuseclass}{sd-d-flex-row}
\begin{sphinxuseclass}{sd-align-minor-center}
\sphinxAtStartPar
Nov 06, 2024

\end{sphinxuseclass}
\end{sphinxuseclass}
\end{sphinxuseclass}
\end{sphinxuseclass}
\begin{sphinxuseclass}{sd-col}
\begin{sphinxuseclass}{sd-col-auto}
\begin{sphinxuseclass}{sd-d-flex-row}
\begin{sphinxuseclass}{sd-align-minor-center}
\sphinxAtStartPar
1 min read

\end{sphinxuseclass}
\end{sphinxuseclass}
\end{sphinxuseclass}
\end{sphinxuseclass}
\end{sphinxuseclass}
\end{sphinxuseclass}
\end{sphinxuseclass}
\end{sphinxuseclass}
\end{sphinxuseclass}
\end{sphinxuseclass}
\end{sphinxuseclass}
\end{sphinxuseclass}
\end{sphinxuseclass}
\end{sphinxuseclass}
\end{sphinxuseclass}
\end{sphinxuseclass}
\end{sphinxuseclass}
\end{sphinxuseclass}
\end{sphinxuseclass}
\end{sphinxuseclass}
\end{sphinxuseclass}
\end{sphinxuseclass}
\end{sphinxuseclass}
\end{sphinxuseclass}
\end{sphinxuseclass}
\end{sphinxuseclass}

\section{Azioni conservative}
\label{\detokenize{ch/mechanics/actions-conservative:azioni-conservative}}\label{\detokenize{ch/mechanics/actions-conservative:physics-hs-mechanics-actions-conservative}}\label{\detokenize{ch/mechanics/actions-conservative::doc}}
\sphinxAtStartPar
Un campo di forze conservativo viene definito tramite il lavoro compiuto. In generale, il lavoro di un campo di forze agente su un punto \(P\) che si muove nello spazio dal punto \(A\) al punto \(B\) lungo un percorso \(\ell_{AB}\) dipende dal percorso. (\sphinxstylestrong{todo} aggiungere riferimento)

\sphinxAtStartPar
Se il lavoro di un campo di forze non dipende dal percorso \(\ell_{AB}\) ma solo dai punti estremi \(A\), \(B\), per tutte le coppie di punti appartententi a una regione dello spazio \(\Omega\), si dice che il \sphinxstylestrong{campo di forze} è \sphinxstylestrong{conservativo} nella regione \(\Omega\) dello spazio.

\sphinxAtStartPar
In questo caso, il lavoro compiuto può essere scritto come differenza di una campo scalare, \(U(P)\) o il suo opposto \(V(P) := - U(P)\),
\begin{equation*}
\begin{split}\begin{aligned}
  L_{AB} &  = U(B) - U(A) = \Delta_{AB} U  \\
         &  = V(A) - V(B) =-\Delta_{AB} V  \\
\end{aligned}\end{split}
\end{equation*}
\sphinxAtStartPar
Le funzioni \(U\), \(V\) vengono definite rispettivamente \sphinxstylestrong{potenziale} ed \sphinxstylestrong{energia potenziale} del campo di forze.

\sphinxAtStartPar
Il lavoro elementare può quindi essere scritto in termini del differenziale di queste funzioni,
\begin{equation*}
\begin{split}\begin{aligned}
  \delta L & = \ \ \ d U =\ \ \ d \vec{r} \cdot \nabla U  = \\
           & =     - d V =    - d \vec{r} \cdot \nabla V \\
\end{aligned}\end{split}
\end{equation*}
\sphinxAtStartPar
Confrontando questa relazione con la definizone di lavoro \(\delta L = d \vec{r} \cdot \vec{F}\), è possibile identificare il campo di forze con il gradiente della funzione potenziale, e l’opposto del gradiente dell’energia potenziale,
\begin{equation*}
\begin{split}\vec{F} = \nabla U = - \nabla V \ .\end{split}
\end{equation*}
\sphinxstepscope

\begin{sphinxuseclass}{sd-container-fluid}
\begin{sphinxuseclass}{sd-sphinx-override}
\begin{sphinxuseclass}{sd-p-0}
\begin{sphinxuseclass}{sd-mt-2}
\begin{sphinxuseclass}{sd-mb-4}
\begin{sphinxuseclass}{sd-row}
\begin{sphinxuseclass}{sd-row-cols-2}
\begin{sphinxuseclass}{sd-gx-2}
\begin{sphinxuseclass}{sd-gy-1}
\begin{sphinxuseclass}{sd-col}
\begin{sphinxuseclass}{sd-d-flex-row}
\begin{sphinxuseclass}{sd-align-minor-center}
\begin{sphinxuseclass}{sd-container-fluid}
\begin{sphinxuseclass}{sd-sphinx-override}
\begin{sphinxuseclass}{sd-row}
\begin{sphinxuseclass}{sd-row-cols-2}
\begin{sphinxuseclass}{sd-row-cols-xs-2}
\begin{sphinxuseclass}{sd-row-cols-sm-3}
\begin{sphinxuseclass}{sd-row-cols-md-3}
\begin{sphinxuseclass}{sd-row-cols-lg-3}
\begin{sphinxuseclass}{sd-gx-3}
\begin{sphinxuseclass}{sd-gy-1}
\begin{sphinxuseclass}{sd-col}
\begin{sphinxuseclass}{sd-col-auto}
\begin{sphinxuseclass}{sd-d-flex-row}
\begin{sphinxuseclass}{sd-align-minor-center}
\sphinxAtStartPar
basics

\end{sphinxuseclass}
\end{sphinxuseclass}
\end{sphinxuseclass}
\end{sphinxuseclass}
\begin{sphinxuseclass}{sd-col}
\begin{sphinxuseclass}{sd-col-auto}
\begin{sphinxuseclass}{sd-d-flex-row}
\begin{sphinxuseclass}{sd-align-minor-center}
\sphinxAtStartPar
Nov 06, 2024

\end{sphinxuseclass}
\end{sphinxuseclass}
\end{sphinxuseclass}
\end{sphinxuseclass}
\begin{sphinxuseclass}{sd-col}
\begin{sphinxuseclass}{sd-col-auto}
\begin{sphinxuseclass}{sd-d-flex-row}
\begin{sphinxuseclass}{sd-align-minor-center}
\sphinxAtStartPar
3 min read

\end{sphinxuseclass}
\end{sphinxuseclass}
\end{sphinxuseclass}
\end{sphinxuseclass}
\end{sphinxuseclass}
\end{sphinxuseclass}
\end{sphinxuseclass}
\end{sphinxuseclass}
\end{sphinxuseclass}
\end{sphinxuseclass}
\end{sphinxuseclass}
\end{sphinxuseclass}
\end{sphinxuseclass}
\end{sphinxuseclass}
\end{sphinxuseclass}
\end{sphinxuseclass}
\end{sphinxuseclass}
\end{sphinxuseclass}
\end{sphinxuseclass}
\end{sphinxuseclass}
\end{sphinxuseclass}
\end{sphinxuseclass}
\end{sphinxuseclass}
\end{sphinxuseclass}
\end{sphinxuseclass}
\end{sphinxuseclass}

\section{Esempi di forze}
\label{\detokenize{ch/mechanics/actions-examples:esempi-di-forze}}\label{\detokenize{ch/mechanics/actions-examples:physics-hs-mechanics-actions-examples}}\label{\detokenize{ch/mechanics/actions-examples::doc}}

\subsection{Gravità}
\label{\detokenize{ch/mechanics/actions-examples:gravita}}\label{\detokenize{ch/mechanics/actions-examples:physics-hs-mechanics-actions-gravitation}}

\subsubsection{Legge di gravitazione universale}
\label{\detokenize{ch/mechanics/actions-examples:legge-di-gravitazione-universale}}\label{\detokenize{ch/mechanics/actions-examples:physics-hs-mechanics-actions-gravitation-newton}}
\sphinxAtStartPar
La forza \(\vec{F}_{12}\) esercitata da un corpo di massa \(m_2\) in \(P_2\) su un corpo di massa \(m_1\) in \(P_1\) è descritta dalla \sphinxstylestrong{legge di gravitazione universale di Newton},
\begin{equation*}
\begin{split}\vec{F}_{12} = G \dfrac{m_1 m_2}{r_{12}^2} \hat{r}_{12} \ ,\end{split}
\end{equation*}
\sphinxAtStartPar
avendo indicato con \(\vec{r}_{12} = (P_2 - P_1)\) il vettore che punta dal punto \(P_1\) al punto \(P_2\), \(r_{12} = |\vec{r}_{12}|\) il suo modulo, e \(\hat{r}_{12} = \frac{\vec{r}_{12}}{|\vec{r}_{12}|}\) il vettore unitario lungo la stessa direzione, e con
\begin{equation*}
\begin{split}G = 6.67 \cdot 10^{-11} \frac{N \, m^2}{kg^2}\end{split}
\end{equation*}
\sphinxAtStartPar
la \sphinxstylestrong{costante di gravitazione universale}, considerata una costante della natura. \sphinxstylestrong{todo}


\subsubsection{Campo di gravità}
\label{\detokenize{ch/mechanics/actions-examples:campo-di-gravita}}
\sphinxAtStartPar
Il campo di gravità generato da una massa \(m_2\) posta in \(P_2\) è \sphinxstylestrong{todo} una funzione dello spazio che associa a ogni punto \(P\) un vettore,
\begin{equation*}
\begin{split}\vec{g}(\vec{r}_1) = \dfrac{\vec{F}_{12}}{m_1} = G \dfrac{m_2}{r_{12}^2} \hat{r}_{12} \ ,\end{split}
\end{equation*}\begin{itemize}
\item {} 
\sphinxAtStartPar
\sphinxstylestrong{todo} abituarsi al concetto di campo, introdotto a partire dalla definizione operativa con \sphinxstyleemphasis{massa test}

\item {} 
\sphinxAtStartPar
\sphinxstylestrong{todo} PSCE

\item {} 
\sphinxAtStartPar
\sphinxstylestrong{todo} noto il campo di gravità \(\vec{g}(P)\), la forza di gravità percepita da un sistema di massa \(m\) in \(P\) può essere scritta come
\begin{equation*}
\begin{split}\vec{F}_g = m \vec{g}(P)\end{split}
\end{equation*}
\end{itemize}

\sphinxAtStartPar
\sphinxstylestrong{Energia potenziale gravitazionale.} E’ possibile dimostrare che il campo gravitazionale è … \sphinxstylestrong{todo}
\begin{equation*}
\begin{split}V(P) = - G \, m \, m_1 \frac{1}{|P - P_1|}\end{split}
\end{equation*}

\subsubsection{Campo di gravità nei pressi della superficie terrestre}
\label{\detokenize{ch/mechanics/actions-examples:campo-di-gravita-nei-pressi-della-superficie-terrestre}}
\sphinxAtStartPar
All’interno di un dominio limitato nei pressi della superficie terrestre, è comune approssimare il campo di gravitazione terrestre come un campo uniforme, diretto lungo la verticale locale verso il centro della terra e di intensità \(g = G \frac{M_E}{R_E^2}\).

\sphinxAtStartPar
E’ possibile derivare questo modello, approssimando il vettore posizione rispetto al centro della terra \(P - P_E \sim R_E \hat{r}\) e il versore che identifica la direzione da un punto del dominio al centro della Terra con la verticale locale \(\hat{r}_{12} \sim - \hat{z}\)
\begin{equation*}
\begin{split}\vec{g}(\vec{r}) = - G \dfrac{M_E}{R_E^2} \hat{z} = - g \hat{z} \ .\end{split}
\end{equation*}
\sphinxAtStartPar
La forza di gravità percepita da un corpo di massa \(m\) nei pressi della superficie terrestre è quindi
\begin{equation*}
\begin{split}\vec{F}_g = - m g \hat{z} \ ,\end{split}
\end{equation*}
\sphinxAtStartPar
quello che viene comunemente chiamato \sphinxstylestrong{peso}.

\sphinxAtStartPar
\sphinxstylestrong{Energia potenziale gravitazionale.} E’ possibile dimostrare che il potenziale gravitazionale nei pressi della superficie terrestre diventa
\begin{equation*}
\begin{split}V(P) = m \, g \, z_P \ .\end{split}
\end{equation*}\subsubsection*{Dimostrazione.}

\sphinxAtStartPar
Con l’espansione in serie, con \(P - P_E = R_E \hat{r} + \vec{d}\), e \(|\vec{d}| \ll R_E\)
\begin{equation*}
\begin{split}\begin{aligned}
  V(P) & = - G \, m \, M_E \frac{1}{|P - P_E|} = \\
       & \approx G M_E \, m \left[ - \frac{1}{R_E} + \frac{R_E \hat{r} \cdot \vec{d}}{R_E^3}  \right]   = \\
       & = \underbrace{- m \, \frac{ G M_E}{R_E}}_{\text{const}} + m \, \underbrace{\frac{G \, M_E}{R_E^2}}_{= g} \underbrace{\hat{r} \cdot \vec{d}}_{= z}
\end{aligned}\end{split}
\end{equation*}

\subsection{Azioni elastiche: molle lineari}
\label{\detokenize{ch/mechanics/actions-examples:azioni-elastiche-molle-lineari}}
\sphinxAtStartPar
Secondo la legge di Hooke, il comportamento di una molla elastica lineare ideale è descritto dall’equazione costitutiva
\begin{equation*}
\begin{split}F = k (\ell - \ell_0) \ ,\end{split}
\end{equation*}
\sphinxAtStartPar
essendo \(F\) il valore assoluto della forza trasmessa dalla molla, \(k\) la costante elastica della molla, \(\ell_0\) la lunghezza a riposo della molla, e \(\ell\) la lunghezza nella configurazione considerata.

\sphinxAtStartPar
\sphinxstylestrong{Energia potenziale.}
\begin{equation*}
\begin{split}\delta L =  F d \ell = k (\ell - \ell_0) d \ell\end{split}
\end{equation*}\begin{equation*}
\begin{split}L = \int_{\ell_1}^{\ell_2} = \left[ \frac{1}{2} k \ell^2 - k \ell_0 \, \ell \right]\bigg|_{\ell_1}^{\ell_2}\end{split}
\end{equation*}\begin{equation*}
\begin{split}V = \frac{1}{2} k (\ell - \ell_0)^2\end{split}
\end{equation*}

\subsection{Azioni di contatto}
\label{\detokenize{ch/mechanics/actions-examples:azioni-di-contatto}}

\subsubsection{Reazioni vincolari di vincoli ideali}
\label{\detokenize{ch/mechanics/actions-examples:reazioni-vincolari-di-vincoli-ideali}}
\sphinxAtStartPar
I vincoli ideali sono modelli di vincolo che non compiono lavoro netto.
\begin{itemize}
\item {} 
\sphinxAtStartPar
\sphinxstylestrong{Incastro}

\item {} 
\sphinxAtStartPar
\sphinxstylestrong{Pattino}

\item {} 
\sphinxAtStartPar
\sphinxstylestrong{Appoggio}

\item {} 
\sphinxAtStartPar
\sphinxstylestrong{Cerniera}

\item {} 
\sphinxAtStartPar
\sphinxstylestrong{Carrello}

\end{itemize}


\subsubsection{Attrito}
\label{\detokenize{ch/mechanics/actions-examples:attrito}}

\paragraph{Attrito statico}
\label{\detokenize{ch/mechanics/actions-examples:attrito-statico}}
\sphinxAtStartPar
L’attrito statico è la forma di attrito che si manifesta tra due corpi quando non c’è moto relativo tra di essi, come una forza tangenziale alla superficie di contatto. Il più semplice modello di attrito statico prevede che il modulo massimo della forza di attrito \(F^s_{max}\) che si può esercitare tra due corpi è proporzionale alla reazione normale tra di essi, \(N\),
\begin{equation*}
\begin{split}F^s_{max} = \mu^s \, N \ .\end{split}
\end{equation*}
\sphinxAtStartPar
Il coefficiente di proporzionalità \(\mu^s\) viene definito \sphinxstylestrong{coefficiente di attrito statico}. In generale, le forze di attrito statico sono determinate dalle condizioni di equilibrio del corpo, se queste condizioni sono ottenibili ed  è soddisfatta la relazione
\begin{equation*}
\begin{split}|F^s| \ge F^s_{max} \ .\end{split}
\end{equation*}

\paragraph{Attrito dinamico}
\label{\detokenize{ch/mechanics/actions-examples:attrito-dinamico}}
\sphinxAtStartPar
L’attrito dinamico è la forma di attrito che si manifesta tra due corpi a contatto e in moto relativo, come una forza tangenziale alla superficie di contatto. Il più semplice modello di attrito statico prevede che la forza di attrito dinamico sia proporzionale alla reazione normale tra i due corpi e diretta in verso opposto alla velocità relativa,
\begin{equation*}
\begin{split}\vec{F}_{12} = - \mu^d N \frac{\vec{v}_{12}}{|\vec{v}_{12}|} \ ,\end{split}
\end{equation*}
\sphinxAtStartPar
avendo definito \(\vec{F}_{12}\) come la forza agente sul corpo \(1\) a causa del corpo \(2\), e \(\vec{v}_{12} = \vec{v}_1 - \vec{v}_2\) la velocità del corpo \(1\) relativa al corpo \(2\).
\begin{itemize}
\item {} 
\sphinxAtStartPar
lavoro e potenza di una forza

\item {} 
\sphinxAtStartPar
azioni conservative

\end{itemize}

\sphinxstepscope


\chapter{Statica}
\label{\detokenize{ch/mechanics/statics:statica}}\label{\detokenize{ch/mechanics/statics:physics-hs-mechanics-statics}}\label{\detokenize{ch/mechanics/statics::doc}}
\sphinxAtStartPar
La statica si occupa dello studio delle condizioni di equilibrio di un sistema, cioè le condizioni in cui un sistema rimane in quiete anche quando soggetto ad {\hyperref[\detokenize{ch/mechanics/actions:physics-hs-mechanics-actions}]{\sphinxcrossref{\DUrole{std,std-ref}{azioni}}}} esterne. Le condizioni di equilibrio dei sistemi sono un caso particolare delle {\hyperref[\detokenize{ch/mechanics/dynamics:physics-hs-mechanics-dynamics}]{\sphinxcrossref{\DUrole{std,std-ref}{equazioni della dinamica}}}}, nel caso in cui il sistema sia in quiete e le derivate delle quantità dinamiche nulle.

\sphinxAtStartPar
In generale le condizioni di equilibrio dipendono dalla natural del sistema. Come si vedrà meglio nelle sezioni di questo capitolo,
\begin{itemize}
\item {} 
\sphinxAtStartPar
le condizioni di equilibrio di un sistema puntiforme sono garantite dall’equilibrio globale delle forze agenti sul sistema

\item {} 
\sphinxAtStartPar
le condizioni di equilibrio di un corpo rigido sono garantite dall’equilibrio globale delle forze e dall’equilibrio globale dei momenti agenti sul sistema

\item {} 
\sphinxAtStartPar
le condizioni di equilibrio di sistemi composti da corpi puntiformi e corpi rigidi sono garantite dalle condizioni di equilibrio di ognuna delle sue parti

\item {} 
\sphinxAtStartPar
le condizioni di equilibrio di mezzi continui deformabili è garantito dall’equilibrio globale e locale delle forze

\end{itemize}

\sphinxstepscope


\section{Statica del punto}
\label{\detokenize{ch/mechanics/statics-point:statica-del-punto}}\label{\detokenize{ch/mechanics/statics-point:physics-hs-mechanics-statics-point}}\label{\detokenize{ch/mechanics/statics-point::doc}}
\sphinxAtStartPar
La condizione necessaria all’equilibrio di un sistema puntiforme è:
\begin{itemize}
\item {} 
\sphinxAtStartPar
l’equilibrio \sphinxstylestrong{globale} delle \sphinxstylestrong{forze esterne} agenti sul sistema

\end{itemize}
\begin{equation*}
\begin{split}\sum_k \vec{F}^{ext}_k = \vec{0} \ .\end{split}
\end{equation*}
\sphinxstepscope


\section{Statica di un corpo rigido}
\label{\detokenize{ch/mechanics/statics-rigid:statica-di-un-corpo-rigido}}\label{\detokenize{ch/mechanics/statics-rigid:physics-hs-mechanics-statics-rigid}}\label{\detokenize{ch/mechanics/statics-rigid::doc}}
\sphinxAtStartPar
Le condizioni necessarie all’equilibrio di un corpo rigido sono:
\begin{itemize}
\item {} 
\sphinxAtStartPar
l’equilibrio \sphinxstylestrong{globale} delle \sphinxstylestrong{forze esterne} agenti sul sistema

\item {} 
\sphinxAtStartPar
l’equilibrio \sphinxstylestrong{globale} dei \sphinxstylestrong{momenti esterni} agenti sul sistema

\end{itemize}
\begin{equation*}
\begin{split}\begin{cases}
  \sum_k \vec{F}^{ext} = \vec{0} \\
  \sum_k \vec{M}^{ext}_{k,H} = \vec{0} \ .  
\end{cases}\end{split}
\end{equation*}

\subsection{Problemi nel piano}
\label{\detokenize{ch/mechanics/statics-rigid:problemi-nel-piano}}

\subsubsection{Esempi ed esercizi}
\label{\detokenize{ch/mechanics/statics-rigid:esempi-ed-esercizi}}\begin{itemize}
\item {} 
\sphinxAtStartPar
leve

\item {} 
\sphinxAtStartPar
carrucole

\item {} 
\sphinxAtStartPar
ingranaggi e trasmissioni

\end{itemize}

\sphinxstepscope


\section{Statica dei mezzi deformabili}
\label{\detokenize{ch/mechanics/statics-fluid:statica-dei-mezzi-deformabili}}\label{\detokenize{ch/mechanics/statics-fluid:physics-hs-mechanics-statics-fluid}}\label{\detokenize{ch/mechanics/statics-fluid::doc}}
\sphinxAtStartPar
La condizione necessaria all’equilibrio di un sistema puntiforme è:
\begin{itemize}
\item {} 
\sphinxAtStartPar
l’equilibrio \sphinxstylestrong{locale} delle forze agenti sul sistema

\item {} 
\sphinxAtStartPar
l’equilibrio \sphinxstylestrong{locale} dei momenti agenti sul sistema

\end{itemize}

\sphinxAtStartPar
Gli equilibri globali sono una diretta conseguenza degli equilibri locali, e del principio di azione e reazione della dinamica.


\subsection{Strutture}
\label{\detokenize{ch/mechanics/statics-fluid:strutture}}

\subsection{Fluidi}
\label{\detokenize{ch/mechanics/statics-fluid:fluidi}}\begin{itemize}
\item {} 
\sphinxAtStartPar
Stevino e principio di Archimede

\end{itemize}

\sphinxstepscope


\section{Problemi}
\label{\detokenize{ch/mechanics/statics-problems:problemi}}\label{\detokenize{ch/mechanics/statics-problems:physics-hs-mechanics-statics-problems}}\label{\detokenize{ch/mechanics/statics-problems::doc}}

\begin{equation*}
\begin{split}
\begin{minipage}[t]{.55\textwidth}
\vspace{0pt}
\textbf{Problema 1.} Data la massa $m$ della massa puntiforme appeso tramite due fili inestensibili ideali di lunghezza $L_1$ e $L_2$ note, si calcolino le reazioni a terra.
\end{minipage}
\hspace{.05\textwidth}
\begin{minipage}[t]{.40\textwidth}
\vspace{0pt}
  \includegraphics[width=.95\textwidth]{../../media/pb-statics-000-ese-000.png}
\end{minipage}
\end{split}
\end{equation*}\subsubsection*{Soluzione.}

\begin{sphinxuseclass}{sd-container-fluid}
\begin{sphinxuseclass}{sd-sphinx-override}
\begin{sphinxuseclass}{sd-mb-4}
\begin{sphinxuseclass}{sd-row}
\begin{sphinxuseclass}{sd-g-2}
\begin{sphinxuseclass}{sd-g-xs-2}
\begin{sphinxuseclass}{sd-g-sm-2}
\begin{sphinxuseclass}{sd-g-md-2}
\begin{sphinxuseclass}{sd-g-lg-2}
\begin{sphinxuseclass}{sd-col}
\begin{sphinxuseclass}{sd-d-flex-row}
\begin{sphinxuseclass}{sd-col-8}
\begin{sphinxuseclass}{sd-col-xs-8}
\begin{sphinxuseclass}{sd-col-sm-8}
\begin{sphinxuseclass}{sd-col-md-8}
\begin{sphinxuseclass}{sd-col-lg-8}
\begin{sphinxuseclass}{sd-card}
\begin{sphinxuseclass}{sd-sphinx-override}
\begin{sphinxuseclass}{sd-w-100}
\begin{sphinxuseclass}{sd-shadow-sm}
\begin{sphinxuseclass}{sd-card-body}
\begin{sphinxuseclass}{sd-card-title}
\begin{sphinxuseclass}{sd-font-weight-bold}Problema 1.
\end{sphinxuseclass}
\end{sphinxuseclass}
\sphinxAtStartPar
Data la massa \(m\) della massa puntiforme appeso tramite due fili inestensibili ideali di lunghezza \(L_1\) e \(L_2\) note, si calcolino le reazioni a terra.

\end{sphinxuseclass}
\end{sphinxuseclass}
\end{sphinxuseclass}
\end{sphinxuseclass}
\end{sphinxuseclass}
\end{sphinxuseclass}
\end{sphinxuseclass}
\end{sphinxuseclass}
\end{sphinxuseclass}
\end{sphinxuseclass}
\end{sphinxuseclass}
\end{sphinxuseclass}
\begin{sphinxuseclass}{sd-col}
\begin{sphinxuseclass}{sd-d-flex-row}
\begin{sphinxuseclass}{sd-col-4}
\begin{sphinxuseclass}{sd-col-xs-4}
\begin{sphinxuseclass}{sd-col-sm-4}
\begin{sphinxuseclass}{sd-col-md-4}
\begin{sphinxuseclass}{sd-col-lg-4}
\begin{sphinxuseclass}{sd-card}
\begin{sphinxuseclass}{sd-sphinx-override}
\begin{sphinxuseclass}{sd-w-100}
\begin{sphinxuseclass}{sd-shadow-sm}
\begin{sphinxuseclass}{sd-card-body}
\sphinxAtStartPar
\sphinxincludegraphics{{pb-statics-000-ese-000}.png}



\end{sphinxuseclass}
\end{sphinxuseclass}
\end{sphinxuseclass}
\end{sphinxuseclass}
\end{sphinxuseclass}
\end{sphinxuseclass}
\end{sphinxuseclass}
\end{sphinxuseclass}
\end{sphinxuseclass}
\end{sphinxuseclass}
\end{sphinxuseclass}
\end{sphinxuseclass}
\end{sphinxuseclass}
\end{sphinxuseclass}
\end{sphinxuseclass}
\end{sphinxuseclass}
\end{sphinxuseclass}
\end{sphinxuseclass}
\end{sphinxuseclass}
\end{sphinxuseclass}
\end{sphinxuseclass}\subsubsection*{Soluzione.}



\begin{sphinxuseclass}{sd-container-fluid}
\begin{sphinxuseclass}{sd-sphinx-override}
\begin{sphinxuseclass}{sd-mb-4}
\begin{sphinxuseclass}{sd-row}
\begin{sphinxuseclass}{sd-g-2}
\begin{sphinxuseclass}{sd-g-xs-2}
\begin{sphinxuseclass}{sd-g-sm-2}
\begin{sphinxuseclass}{sd-g-md-2}
\begin{sphinxuseclass}{sd-g-lg-2}
\begin{sphinxuseclass}{sd-col}
\begin{sphinxuseclass}{sd-d-flex-row}
\begin{sphinxuseclass}{sd-col-8}
\begin{sphinxuseclass}{sd-col-xs-8}
\begin{sphinxuseclass}{sd-col-sm-8}
\begin{sphinxuseclass}{sd-col-md-8}
\begin{sphinxuseclass}{sd-col-lg-8}
\begin{sphinxuseclass}{sd-card}
\begin{sphinxuseclass}{sd-sphinx-override}
\begin{sphinxuseclass}{sd-w-100}
\begin{sphinxuseclass}{sd-shadow-sm}
\begin{sphinxuseclass}{sd-card-body}
\begin{sphinxuseclass}{sd-card-title}
\begin{sphinxuseclass}{sd-font-weight-bold}Problema 2.
\end{sphinxuseclass}
\end{sphinxuseclass}
\sphinxAtStartPar
Data la massa \(m\) della massa puntiforme appeso tramite due fili inestensibili ideali di lunghezza \(L_1\) nota e \(L_2\) variabile, si calcolino le reazioni a terra in funzione della lunghezza del filo \(2\).

\end{sphinxuseclass}
\end{sphinxuseclass}
\end{sphinxuseclass}
\end{sphinxuseclass}
\end{sphinxuseclass}
\end{sphinxuseclass}
\end{sphinxuseclass}
\end{sphinxuseclass}
\end{sphinxuseclass}
\end{sphinxuseclass}
\end{sphinxuseclass}
\end{sphinxuseclass}
\begin{sphinxuseclass}{sd-col}
\begin{sphinxuseclass}{sd-d-flex-row}
\begin{sphinxuseclass}{sd-col-4}
\begin{sphinxuseclass}{sd-col-xs-4}
\begin{sphinxuseclass}{sd-col-sm-4}
\begin{sphinxuseclass}{sd-col-md-4}
\begin{sphinxuseclass}{sd-col-lg-4}
\begin{sphinxuseclass}{sd-card}
\begin{sphinxuseclass}{sd-sphinx-override}
\begin{sphinxuseclass}{sd-w-100}
\begin{sphinxuseclass}{sd-shadow-sm}
\begin{sphinxuseclass}{sd-card-body}
\sphinxAtStartPar
\sphinxincludegraphics{{pb-statics-000-ese-001}.png}

\end{sphinxuseclass}
\end{sphinxuseclass}
\end{sphinxuseclass}
\end{sphinxuseclass}
\end{sphinxuseclass}
\end{sphinxuseclass}
\end{sphinxuseclass}
\end{sphinxuseclass}
\end{sphinxuseclass}
\end{sphinxuseclass}
\end{sphinxuseclass}
\end{sphinxuseclass}
\end{sphinxuseclass}
\end{sphinxuseclass}
\end{sphinxuseclass}
\end{sphinxuseclass}
\end{sphinxuseclass}
\end{sphinxuseclass}
\end{sphinxuseclass}
\end{sphinxuseclass}
\end{sphinxuseclass}\subsubsection*{Soluzione.}



\begin{sphinxuseclass}{sd-container-fluid}
\begin{sphinxuseclass}{sd-sphinx-override}
\begin{sphinxuseclass}{sd-mb-4}
\begin{sphinxuseclass}{sd-row}
\begin{sphinxuseclass}{sd-g-2}
\begin{sphinxuseclass}{sd-g-xs-2}
\begin{sphinxuseclass}{sd-g-sm-2}
\begin{sphinxuseclass}{sd-g-md-2}
\begin{sphinxuseclass}{sd-g-lg-2}
\begin{sphinxuseclass}{sd-col}
\begin{sphinxuseclass}{sd-d-flex-row}
\begin{sphinxuseclass}{sd-col-8}
\begin{sphinxuseclass}{sd-col-xs-8}
\begin{sphinxuseclass}{sd-col-sm-8}
\begin{sphinxuseclass}{sd-col-md-8}
\begin{sphinxuseclass}{sd-col-lg-8}
\begin{sphinxuseclass}{sd-card}
\begin{sphinxuseclass}{sd-sphinx-override}
\begin{sphinxuseclass}{sd-w-100}
\begin{sphinxuseclass}{sd-shadow-sm}
\begin{sphinxuseclass}{sd-card-body}
\begin{sphinxuseclass}{sd-card-title}
\begin{sphinxuseclass}{sd-font-weight-bold}Problema 3.
\end{sphinxuseclass}
\end{sphinxuseclass}
\sphinxAtStartPar
Data la massa \(m\) della massa puntiforme appeso tramite un filo inestensibile ideale di lunghezza \(L\) e una molla di costante elastica \(k\) e lunghezza a riposo \(x_0\) collegata a terra in un punto distante \(H\) dal punto a terra dove è collegato il filo, si calcoli:
\begin{enumerate}
\sphinxsetlistlabels{\arabic}{enumi}{enumii}{}{.}%
\item {} 
\sphinxAtStartPar
la posizione del punto

\item {} 
\sphinxAtStartPar
la lunghezza della molla

\item {} 
\sphinxAtStartPar
le reazioni vincolari a terra
nella configurazione di equilibrio.

\end{enumerate}

\end{sphinxuseclass}
\end{sphinxuseclass}
\end{sphinxuseclass}
\end{sphinxuseclass}
\end{sphinxuseclass}
\end{sphinxuseclass}
\end{sphinxuseclass}
\end{sphinxuseclass}
\end{sphinxuseclass}
\end{sphinxuseclass}
\end{sphinxuseclass}
\end{sphinxuseclass}
\begin{sphinxuseclass}{sd-col}
\begin{sphinxuseclass}{sd-d-flex-row}
\begin{sphinxuseclass}{sd-col-4}
\begin{sphinxuseclass}{sd-col-xs-4}
\begin{sphinxuseclass}{sd-col-sm-4}
\begin{sphinxuseclass}{sd-col-md-4}
\begin{sphinxuseclass}{sd-col-lg-4}
\begin{sphinxuseclass}{sd-card}
\begin{sphinxuseclass}{sd-sphinx-override}
\begin{sphinxuseclass}{sd-w-100}
\begin{sphinxuseclass}{sd-shadow-sm}
\begin{sphinxuseclass}{sd-card-body}
\sphinxAtStartPar
\sphinxincludegraphics{{pb-statics-000-ese-002}.png}

\end{sphinxuseclass}
\end{sphinxuseclass}
\end{sphinxuseclass}
\end{sphinxuseclass}
\end{sphinxuseclass}
\end{sphinxuseclass}
\end{sphinxuseclass}
\end{sphinxuseclass}
\end{sphinxuseclass}
\end{sphinxuseclass}
\end{sphinxuseclass}
\end{sphinxuseclass}
\end{sphinxuseclass}
\end{sphinxuseclass}
\end{sphinxuseclass}
\end{sphinxuseclass}
\end{sphinxuseclass}
\end{sphinxuseclass}
\end{sphinxuseclass}
\end{sphinxuseclass}
\end{sphinxuseclass}\subsubsection*{Soluzione.}



\begin{sphinxuseclass}{sd-container-fluid}
\begin{sphinxuseclass}{sd-sphinx-override}
\begin{sphinxuseclass}{sd-mb-4}
\begin{sphinxuseclass}{sd-row}
\begin{sphinxuseclass}{sd-g-2}
\begin{sphinxuseclass}{sd-g-xs-2}
\begin{sphinxuseclass}{sd-g-sm-2}
\begin{sphinxuseclass}{sd-g-md-2}
\begin{sphinxuseclass}{sd-g-lg-2}
\begin{sphinxuseclass}{sd-col}
\begin{sphinxuseclass}{sd-d-flex-row}
\begin{sphinxuseclass}{sd-col-8}
\begin{sphinxuseclass}{sd-col-xs-8}
\begin{sphinxuseclass}{sd-col-sm-8}
\begin{sphinxuseclass}{sd-col-md-8}
\begin{sphinxuseclass}{sd-col-lg-8}
\begin{sphinxuseclass}{sd-card}
\begin{sphinxuseclass}{sd-sphinx-override}
\begin{sphinxuseclass}{sd-w-100}
\begin{sphinxuseclass}{sd-shadow-sm}
\begin{sphinxuseclass}{sd-card-body}
\begin{sphinxuseclass}{sd-card-title}
\begin{sphinxuseclass}{sd-font-weight-bold}Problema 4.
\end{sphinxuseclass}
\end{sphinxuseclass}
\sphinxAtStartPar
Data \(m\), \(\mu^s\), trovare l’angolo massimo \(\theta_{\max}\) per il quale esiste una condizione di equilibrio.

\end{sphinxuseclass}
\end{sphinxuseclass}
\end{sphinxuseclass}
\end{sphinxuseclass}
\end{sphinxuseclass}
\end{sphinxuseclass}
\end{sphinxuseclass}
\end{sphinxuseclass}
\end{sphinxuseclass}
\end{sphinxuseclass}
\end{sphinxuseclass}
\end{sphinxuseclass}
\begin{sphinxuseclass}{sd-col}
\begin{sphinxuseclass}{sd-d-flex-row}
\begin{sphinxuseclass}{sd-col-4}
\begin{sphinxuseclass}{sd-col-xs-4}
\begin{sphinxuseclass}{sd-col-sm-4}
\begin{sphinxuseclass}{sd-col-md-4}
\begin{sphinxuseclass}{sd-col-lg-4}
\begin{sphinxuseclass}{sd-card}
\begin{sphinxuseclass}{sd-sphinx-override}
\begin{sphinxuseclass}{sd-w-100}
\begin{sphinxuseclass}{sd-shadow-sm}
\begin{sphinxuseclass}{sd-card-body}
\sphinxAtStartPar
\sphinxincludegraphics{{pb-statics-001-ese-000}.png}

\end{sphinxuseclass}
\end{sphinxuseclass}
\end{sphinxuseclass}
\end{sphinxuseclass}
\end{sphinxuseclass}
\end{sphinxuseclass}
\end{sphinxuseclass}
\end{sphinxuseclass}
\end{sphinxuseclass}
\end{sphinxuseclass}
\end{sphinxuseclass}
\end{sphinxuseclass}
\end{sphinxuseclass}
\end{sphinxuseclass}
\end{sphinxuseclass}
\end{sphinxuseclass}
\end{sphinxuseclass}
\end{sphinxuseclass}
\end{sphinxuseclass}
\end{sphinxuseclass}
\end{sphinxuseclass}\subsubsection*{Soluzione.}



\begin{sphinxuseclass}{sd-container-fluid}
\begin{sphinxuseclass}{sd-sphinx-override}
\begin{sphinxuseclass}{sd-mb-4}
\begin{sphinxuseclass}{sd-row}
\begin{sphinxuseclass}{sd-g-2}
\begin{sphinxuseclass}{sd-g-xs-2}
\begin{sphinxuseclass}{sd-g-sm-2}
\begin{sphinxuseclass}{sd-g-md-2}
\begin{sphinxuseclass}{sd-g-lg-2}
\begin{sphinxuseclass}{sd-col}
\begin{sphinxuseclass}{sd-d-flex-row}
\begin{sphinxuseclass}{sd-col-8}
\begin{sphinxuseclass}{sd-col-xs-8}
\begin{sphinxuseclass}{sd-col-sm-8}
\begin{sphinxuseclass}{sd-col-md-8}
\begin{sphinxuseclass}{sd-col-lg-8}
\begin{sphinxuseclass}{sd-card}
\begin{sphinxuseclass}{sd-sphinx-override}
\begin{sphinxuseclass}{sd-w-100}
\begin{sphinxuseclass}{sd-shadow-sm}
\begin{sphinxuseclass}{sd-card-body}
\begin{sphinxuseclass}{sd-card-title}
\begin{sphinxuseclass}{sd-font-weight-bold}Problema 5.
\end{sphinxuseclass}
\end{sphinxuseclass}
\sphinxAtStartPar
Data \(m\), \(M\), \(\mu^s\) tra i due solidi, si chiede di calcolare:
\begin{itemize}
\item {} 
\sphinxAtStartPar
la risultante delle azioni scambiate tra i due corpi

\item {} 
\sphinxAtStartPar
la risultante delle reazioni vincolari a terra agenti sul solido blu,

\end{itemize}

\sphinxAtStartPar
nella condizione di equilibrio del sistema, nell’ipotesi che l’attrito tra solido blu e terra sia trascurabile. Verificare le condizioni limite tra \(\theta\) e \(\mu^s\) affinché l’equilibrio sia possibile

\end{sphinxuseclass}
\end{sphinxuseclass}
\end{sphinxuseclass}
\end{sphinxuseclass}
\end{sphinxuseclass}
\end{sphinxuseclass}
\end{sphinxuseclass}
\end{sphinxuseclass}
\end{sphinxuseclass}
\end{sphinxuseclass}
\end{sphinxuseclass}
\end{sphinxuseclass}
\begin{sphinxuseclass}{sd-col}
\begin{sphinxuseclass}{sd-d-flex-row}
\begin{sphinxuseclass}{sd-col-4}
\begin{sphinxuseclass}{sd-col-xs-4}
\begin{sphinxuseclass}{sd-col-sm-4}
\begin{sphinxuseclass}{sd-col-md-4}
\begin{sphinxuseclass}{sd-col-lg-4}
\begin{sphinxuseclass}{sd-card}
\begin{sphinxuseclass}{sd-sphinx-override}
\begin{sphinxuseclass}{sd-w-100}
\begin{sphinxuseclass}{sd-shadow-sm}
\begin{sphinxuseclass}{sd-card-body}
\sphinxAtStartPar
\sphinxincludegraphics{{pb-statics-001-ese-001}.png}

\end{sphinxuseclass}
\end{sphinxuseclass}
\end{sphinxuseclass}
\end{sphinxuseclass}
\end{sphinxuseclass}
\end{sphinxuseclass}
\end{sphinxuseclass}
\end{sphinxuseclass}
\end{sphinxuseclass}
\end{sphinxuseclass}
\end{sphinxuseclass}
\end{sphinxuseclass}
\end{sphinxuseclass}
\end{sphinxuseclass}
\end{sphinxuseclass}
\end{sphinxuseclass}
\end{sphinxuseclass}
\end{sphinxuseclass}
\end{sphinxuseclass}
\end{sphinxuseclass}
\end{sphinxuseclass}\subsubsection*{Soluzione.}



\begin{sphinxuseclass}{sd-container-fluid}
\begin{sphinxuseclass}{sd-sphinx-override}
\begin{sphinxuseclass}{sd-mb-4}
\begin{sphinxuseclass}{sd-row}
\begin{sphinxuseclass}{sd-g-2}
\begin{sphinxuseclass}{sd-g-xs-2}
\begin{sphinxuseclass}{sd-g-sm-2}
\begin{sphinxuseclass}{sd-g-md-2}
\begin{sphinxuseclass}{sd-g-lg-2}
\begin{sphinxuseclass}{sd-col}
\begin{sphinxuseclass}{sd-d-flex-row}
\begin{sphinxuseclass}{sd-col-8}
\begin{sphinxuseclass}{sd-col-xs-8}
\begin{sphinxuseclass}{sd-col-sm-8}
\begin{sphinxuseclass}{sd-col-md-8}
\begin{sphinxuseclass}{sd-col-lg-8}
\begin{sphinxuseclass}{sd-card}
\begin{sphinxuseclass}{sd-sphinx-override}
\begin{sphinxuseclass}{sd-w-100}
\begin{sphinxuseclass}{sd-shadow-sm}
\begin{sphinxuseclass}{sd-card-body}
\begin{sphinxuseclass}{sd-card-title}
\begin{sphinxuseclass}{sd-font-weight-bold}Problema 6.
\end{sphinxuseclass}
\end{sphinxuseclass}
\sphinxAtStartPar
Data la massa \(m\) del blocco rosso, la costante elastica \(k\) della molla lineare ideale, con lunghezza a riposo \(\ell_0\), viene chiesto di:
\begin{itemize}
\item {} 
\sphinxAtStartPar
determinare la lunghezza della molla nella condizione di equilibrio, nell’ipotesi che l’attrito tra blocco rosso e piano inclinato sia trascurabile

\item {} 
\sphinxAtStartPar
determinare le possibili condizioni di equilibrio, nell’ipotesi che l’attrito statico tra blocco rosso e piano inclinato sia \(\mu^s\)

\end{itemize}

\end{sphinxuseclass}
\end{sphinxuseclass}
\end{sphinxuseclass}
\end{sphinxuseclass}
\end{sphinxuseclass}
\end{sphinxuseclass}
\end{sphinxuseclass}
\end{sphinxuseclass}
\end{sphinxuseclass}
\end{sphinxuseclass}
\end{sphinxuseclass}
\end{sphinxuseclass}
\begin{sphinxuseclass}{sd-col}
\begin{sphinxuseclass}{sd-d-flex-row}
\begin{sphinxuseclass}{sd-col-4}
\begin{sphinxuseclass}{sd-col-xs-4}
\begin{sphinxuseclass}{sd-col-sm-4}
\begin{sphinxuseclass}{sd-col-md-4}
\begin{sphinxuseclass}{sd-col-lg-4}
\begin{sphinxuseclass}{sd-card}
\begin{sphinxuseclass}{sd-sphinx-override}
\begin{sphinxuseclass}{sd-w-100}
\begin{sphinxuseclass}{sd-shadow-sm}
\begin{sphinxuseclass}{sd-card-body}
\sphinxAtStartPar
\sphinxincludegraphics{{pb-statics-001-ese-002}.png}

\end{sphinxuseclass}
\end{sphinxuseclass}
\end{sphinxuseclass}
\end{sphinxuseclass}
\end{sphinxuseclass}
\end{sphinxuseclass}
\end{sphinxuseclass}
\end{sphinxuseclass}
\end{sphinxuseclass}
\end{sphinxuseclass}
\end{sphinxuseclass}
\end{sphinxuseclass}
\end{sphinxuseclass}
\end{sphinxuseclass}
\end{sphinxuseclass}
\end{sphinxuseclass}
\end{sphinxuseclass}
\end{sphinxuseclass}
\end{sphinxuseclass}
\end{sphinxuseclass}
\end{sphinxuseclass}\subsubsection*{Soluzione.}



\begin{sphinxuseclass}{sd-container-fluid}
\begin{sphinxuseclass}{sd-sphinx-override}
\begin{sphinxuseclass}{sd-mb-4}
\begin{sphinxuseclass}{sd-row}
\begin{sphinxuseclass}{sd-g-2}
\begin{sphinxuseclass}{sd-g-xs-2}
\begin{sphinxuseclass}{sd-g-sm-2}
\begin{sphinxuseclass}{sd-g-md-2}
\begin{sphinxuseclass}{sd-g-lg-2}
\begin{sphinxuseclass}{sd-col}
\begin{sphinxuseclass}{sd-d-flex-row}
\begin{sphinxuseclass}{sd-col-8}
\begin{sphinxuseclass}{sd-col-xs-8}
\begin{sphinxuseclass}{sd-col-sm-8}
\begin{sphinxuseclass}{sd-col-md-8}
\begin{sphinxuseclass}{sd-col-lg-8}
\begin{sphinxuseclass}{sd-card}
\begin{sphinxuseclass}{sd-sphinx-override}
\begin{sphinxuseclass}{sd-w-100}
\begin{sphinxuseclass}{sd-shadow-sm}
\begin{sphinxuseclass}{sd-card-body}
\begin{sphinxuseclass}{sd-card-title}
\begin{sphinxuseclass}{sd-font-weight-bold}Problema 7.
\end{sphinxuseclass}
\end{sphinxuseclass}
\sphinxAtStartPar
Data la massa \(m\) del blocco rosso, il raggio \(R_1\), \(R_2\) delle due carrucole, si chiede di determinare la forza \(\vec{F}\) da applicare nella condizione di equilibrio, nell’ipotesi di fili inestensibili e carrucole ideali e senza massa.

\sphinxAtStartPar
Si chiede poi di ripetere il calcolo nell’ipotesi in cui la massa delle carrucole non sia trascurabile, ma siano \(M_1\) per la carrucola vincolata a terra, e \(M_2\) per la carrucola non vincolata a terra.

\end{sphinxuseclass}
\end{sphinxuseclass}
\end{sphinxuseclass}
\end{sphinxuseclass}
\end{sphinxuseclass}
\end{sphinxuseclass}
\end{sphinxuseclass}
\end{sphinxuseclass}
\end{sphinxuseclass}
\end{sphinxuseclass}
\end{sphinxuseclass}
\end{sphinxuseclass}
\begin{sphinxuseclass}{sd-col}
\begin{sphinxuseclass}{sd-d-flex-row}
\begin{sphinxuseclass}{sd-col-4}
\begin{sphinxuseclass}{sd-col-xs-4}
\begin{sphinxuseclass}{sd-col-sm-4}
\begin{sphinxuseclass}{sd-col-md-4}
\begin{sphinxuseclass}{sd-col-lg-4}
\begin{sphinxuseclass}{sd-card}
\begin{sphinxuseclass}{sd-sphinx-override}
\begin{sphinxuseclass}{sd-w-100}
\begin{sphinxuseclass}{sd-shadow-sm}
\begin{sphinxuseclass}{sd-card-body}
\sphinxAtStartPar
\sphinxincludegraphics{{pb-statics-002-ese-000}.png}

\end{sphinxuseclass}
\end{sphinxuseclass}
\end{sphinxuseclass}
\end{sphinxuseclass}
\end{sphinxuseclass}
\end{sphinxuseclass}
\end{sphinxuseclass}
\end{sphinxuseclass}
\end{sphinxuseclass}
\end{sphinxuseclass}
\end{sphinxuseclass}
\end{sphinxuseclass}
\end{sphinxuseclass}
\end{sphinxuseclass}
\end{sphinxuseclass}
\end{sphinxuseclass}
\end{sphinxuseclass}
\end{sphinxuseclass}
\end{sphinxuseclass}
\end{sphinxuseclass}
\end{sphinxuseclass}\subsubsection*{Soluzione.}



\begin{sphinxuseclass}{sd-container-fluid}
\begin{sphinxuseclass}{sd-sphinx-override}
\begin{sphinxuseclass}{sd-mb-4}
\begin{sphinxuseclass}{sd-row}
\begin{sphinxuseclass}{sd-g-2}
\begin{sphinxuseclass}{sd-g-xs-2}
\begin{sphinxuseclass}{sd-g-sm-2}
\begin{sphinxuseclass}{sd-g-md-2}
\begin{sphinxuseclass}{sd-g-lg-2}
\begin{sphinxuseclass}{sd-col}
\begin{sphinxuseclass}{sd-d-flex-row}
\begin{sphinxuseclass}{sd-col-8}
\begin{sphinxuseclass}{sd-col-xs-8}
\begin{sphinxuseclass}{sd-col-sm-8}
\begin{sphinxuseclass}{sd-col-md-8}
\begin{sphinxuseclass}{sd-col-lg-8}
\begin{sphinxuseclass}{sd-card}
\begin{sphinxuseclass}{sd-sphinx-override}
\begin{sphinxuseclass}{sd-w-100}
\begin{sphinxuseclass}{sd-shadow-sm}
\begin{sphinxuseclass}{sd-card-body}
\begin{sphinxuseclass}{sd-card-title}
\begin{sphinxuseclass}{sd-font-weight-bold}Problema 8.
\end{sphinxuseclass}
\end{sphinxuseclass}
\sphinxAtStartPar
Nel meccanismo di un orologio i 3 componenti che devono guidare il moto delle lancette dei secondi, dei minuti e delle ore, connessi “in cascata” tramite ingranaggi (con rapporto dei raggi \(1:60\) \sphinxstylestrong{todo} scriverlo esplicitamente?). Conoscendo la costante elastica \(k\) e la compressione \(\Delta \theta\) della molla che guida il componente che guida la lancetta delle ore, si chiede di:
\begin{itemize}
\item {} 
\sphinxAtStartPar
determinare la forza necessaria da applicare alla lancetta dei secondi nel punto indicato nell’imagine, necessaria a garantire la posizione di equilibrio

\item {} 
\sphinxAtStartPar
le reazioni vincolari in corrispondenza delle cerniere che collegano a terra i 3 componenti, nell’ipotesi che non si scambino forze in direzione radiale.

\end{itemize}

\end{sphinxuseclass}
\end{sphinxuseclass}
\end{sphinxuseclass}
\end{sphinxuseclass}
\end{sphinxuseclass}
\end{sphinxuseclass}
\end{sphinxuseclass}
\end{sphinxuseclass}
\end{sphinxuseclass}
\end{sphinxuseclass}
\end{sphinxuseclass}
\end{sphinxuseclass}
\begin{sphinxuseclass}{sd-col}
\begin{sphinxuseclass}{sd-d-flex-row}
\begin{sphinxuseclass}{sd-col-4}
\begin{sphinxuseclass}{sd-col-xs-4}
\begin{sphinxuseclass}{sd-col-sm-4}
\begin{sphinxuseclass}{sd-col-md-4}
\begin{sphinxuseclass}{sd-col-lg-4}
\begin{sphinxuseclass}{sd-card}
\begin{sphinxuseclass}{sd-sphinx-override}
\begin{sphinxuseclass}{sd-w-100}
\begin{sphinxuseclass}{sd-shadow-sm}
\begin{sphinxuseclass}{sd-card-body}
\sphinxAtStartPar
\sphinxincludegraphics{{pb-statics-002-ese-001}.png}

\end{sphinxuseclass}
\end{sphinxuseclass}
\end{sphinxuseclass}
\end{sphinxuseclass}
\end{sphinxuseclass}
\end{sphinxuseclass}
\end{sphinxuseclass}
\end{sphinxuseclass}
\end{sphinxuseclass}
\end{sphinxuseclass}
\end{sphinxuseclass}
\end{sphinxuseclass}
\end{sphinxuseclass}
\end{sphinxuseclass}
\end{sphinxuseclass}
\end{sphinxuseclass}
\end{sphinxuseclass}
\end{sphinxuseclass}
\end{sphinxuseclass}
\end{sphinxuseclass}
\end{sphinxuseclass}\subsubsection*{Soluzione.}



\begin{sphinxuseclass}{sd-container-fluid}
\begin{sphinxuseclass}{sd-sphinx-override}
\begin{sphinxuseclass}{sd-mb-4}
\begin{sphinxuseclass}{sd-row}
\begin{sphinxuseclass}{sd-g-2}
\begin{sphinxuseclass}{sd-g-xs-2}
\begin{sphinxuseclass}{sd-g-sm-2}
\begin{sphinxuseclass}{sd-g-md-2}
\begin{sphinxuseclass}{sd-g-lg-2}
\begin{sphinxuseclass}{sd-col}
\begin{sphinxuseclass}{sd-d-flex-row}
\begin{sphinxuseclass}{sd-col-8}
\begin{sphinxuseclass}{sd-col-xs-8}
\begin{sphinxuseclass}{sd-col-sm-8}
\begin{sphinxuseclass}{sd-col-md-8}
\begin{sphinxuseclass}{sd-col-lg-8}
\begin{sphinxuseclass}{sd-card}
\begin{sphinxuseclass}{sd-sphinx-override}
\begin{sphinxuseclass}{sd-w-100}
\begin{sphinxuseclass}{sd-shadow-sm}
\begin{sphinxuseclass}{sd-card-body}
\begin{sphinxuseclass}{sd-card-title}
\begin{sphinxuseclass}{sd-font-weight-bold}Problema 9.
\end{sphinxuseclass}
\end{sphinxuseclass}
\sphinxAtStartPar
Data la lunghezza \(L\) e la massa \(m\) dell’asta rigida con distribuzione di massa uniforme e il coefficiente di attrito stativo \(\mu^s\) tra asta e superficie orizzontale, si chiede di:
\begin{itemize}
\item {} 
\sphinxAtStartPar
determinare la condizione limite dell’equilibrio

\item {} 
\sphinxAtStartPar
determinare le reazioni a terra
nell’ipotesi che l’attrito sulla superficie verticale sia trascurabile

\end{itemize}

\end{sphinxuseclass}
\end{sphinxuseclass}
\end{sphinxuseclass}
\end{sphinxuseclass}
\end{sphinxuseclass}
\end{sphinxuseclass}
\end{sphinxuseclass}
\end{sphinxuseclass}
\end{sphinxuseclass}
\end{sphinxuseclass}
\end{sphinxuseclass}
\end{sphinxuseclass}
\begin{sphinxuseclass}{sd-col}
\begin{sphinxuseclass}{sd-d-flex-row}
\begin{sphinxuseclass}{sd-col-4}
\begin{sphinxuseclass}{sd-col-xs-4}
\begin{sphinxuseclass}{sd-col-sm-4}
\begin{sphinxuseclass}{sd-col-md-4}
\begin{sphinxuseclass}{sd-col-lg-4}
\begin{sphinxuseclass}{sd-card}
\begin{sphinxuseclass}{sd-sphinx-override}
\begin{sphinxuseclass}{sd-w-100}
\begin{sphinxuseclass}{sd-shadow-sm}
\begin{sphinxuseclass}{sd-card-body}
\sphinxAtStartPar
\sphinxincludegraphics{{pb-statics-002-ese-002}.png}

\end{sphinxuseclass}
\end{sphinxuseclass}
\end{sphinxuseclass}
\end{sphinxuseclass}
\end{sphinxuseclass}
\end{sphinxuseclass}
\end{sphinxuseclass}
\end{sphinxuseclass}
\end{sphinxuseclass}
\end{sphinxuseclass}
\end{sphinxuseclass}
\end{sphinxuseclass}
\end{sphinxuseclass}
\end{sphinxuseclass}
\end{sphinxuseclass}
\end{sphinxuseclass}
\end{sphinxuseclass}
\end{sphinxuseclass}
\end{sphinxuseclass}
\end{sphinxuseclass}
\end{sphinxuseclass}\subsubsection*{Soluzione.}



\begin{sphinxuseclass}{sd-container-fluid}
\begin{sphinxuseclass}{sd-sphinx-override}
\begin{sphinxuseclass}{sd-mb-4}
\begin{sphinxuseclass}{sd-row}
\begin{sphinxuseclass}{sd-g-2}
\begin{sphinxuseclass}{sd-g-xs-2}
\begin{sphinxuseclass}{sd-g-sm-2}
\begin{sphinxuseclass}{sd-g-md-2}
\begin{sphinxuseclass}{sd-g-lg-2}
\begin{sphinxuseclass}{sd-col}
\begin{sphinxuseclass}{sd-d-flex-row}
\begin{sphinxuseclass}{sd-col-8}
\begin{sphinxuseclass}{sd-col-xs-8}
\begin{sphinxuseclass}{sd-col-sm-8}
\begin{sphinxuseclass}{sd-col-md-8}
\begin{sphinxuseclass}{sd-col-lg-8}
\begin{sphinxuseclass}{sd-card}
\begin{sphinxuseclass}{sd-sphinx-override}
\begin{sphinxuseclass}{sd-w-100}
\begin{sphinxuseclass}{sd-shadow-sm}
\begin{sphinxuseclass}{sd-card-body}
\begin{sphinxuseclass}{sd-card-title}
\begin{sphinxuseclass}{sd-font-weight-bold}Problema 10.
\end{sphinxuseclass}
\end{sphinxuseclass}
\sphinxAtStartPar
Data la lunghezza \(L\) e la massa \(m\) dell’asta rigida incernierata a terra, e la costante elastica \(k\) della molla rotazionale, si chiede di:
\begin{itemize}
\item {} 
\sphinxAtStartPar
calcolare la condizione di equilibrio

\item {} 
\sphinxAtStartPar
le reazioni vincolari sull’asta
discutendo i due casi determinati dalla condizione di appoggio dell’estremo superiore dell’asta sulla parete verticale.

\end{itemize}

\end{sphinxuseclass}
\end{sphinxuseclass}
\end{sphinxuseclass}
\end{sphinxuseclass}
\end{sphinxuseclass}
\end{sphinxuseclass}
\end{sphinxuseclass}
\end{sphinxuseclass}
\end{sphinxuseclass}
\end{sphinxuseclass}
\end{sphinxuseclass}
\end{sphinxuseclass}
\begin{sphinxuseclass}{sd-col}
\begin{sphinxuseclass}{sd-d-flex-row}
\begin{sphinxuseclass}{sd-col-4}
\begin{sphinxuseclass}{sd-col-xs-4}
\begin{sphinxuseclass}{sd-col-sm-4}
\begin{sphinxuseclass}{sd-col-md-4}
\begin{sphinxuseclass}{sd-col-lg-4}
\begin{sphinxuseclass}{sd-card}
\begin{sphinxuseclass}{sd-sphinx-override}
\begin{sphinxuseclass}{sd-w-100}
\begin{sphinxuseclass}{sd-shadow-sm}
\begin{sphinxuseclass}{sd-card-body}
\sphinxAtStartPar
\sphinxincludegraphics{{pb-statics-002-ese-003}.png}

\end{sphinxuseclass}
\end{sphinxuseclass}
\end{sphinxuseclass}
\end{sphinxuseclass}
\end{sphinxuseclass}
\end{sphinxuseclass}
\end{sphinxuseclass}
\end{sphinxuseclass}
\end{sphinxuseclass}
\end{sphinxuseclass}
\end{sphinxuseclass}
\end{sphinxuseclass}
\end{sphinxuseclass}
\end{sphinxuseclass}
\end{sphinxuseclass}
\end{sphinxuseclass}
\end{sphinxuseclass}
\end{sphinxuseclass}
\end{sphinxuseclass}
\end{sphinxuseclass}
\end{sphinxuseclass}\subsubsection*{Soluzione.}



\begin{sphinxuseclass}{sd-container-fluid}
\begin{sphinxuseclass}{sd-sphinx-override}
\begin{sphinxuseclass}{sd-mb-4}
\begin{sphinxuseclass}{sd-row}
\begin{sphinxuseclass}{sd-g-2}
\begin{sphinxuseclass}{sd-g-xs-2}
\begin{sphinxuseclass}{sd-g-sm-2}
\begin{sphinxuseclass}{sd-g-md-2}
\begin{sphinxuseclass}{sd-g-lg-2}
\begin{sphinxuseclass}{sd-col}
\begin{sphinxuseclass}{sd-d-flex-row}
\begin{sphinxuseclass}{sd-col-8}
\begin{sphinxuseclass}{sd-col-xs-8}
\begin{sphinxuseclass}{sd-col-sm-8}
\begin{sphinxuseclass}{sd-col-md-8}
\begin{sphinxuseclass}{sd-col-lg-8}
\begin{sphinxuseclass}{sd-card}
\begin{sphinxuseclass}{sd-sphinx-override}
\begin{sphinxuseclass}{sd-w-100}
\begin{sphinxuseclass}{sd-shadow-sm}
\begin{sphinxuseclass}{sd-card-body}
\begin{sphinxuseclass}{sd-card-title}
\begin{sphinxuseclass}{sd-font-weight-bold}Problema 11.
\end{sphinxuseclass}
\end{sphinxuseclass}
\sphinxAtStartPar
Testo del problema…

\end{sphinxuseclass}
\end{sphinxuseclass}
\end{sphinxuseclass}
\end{sphinxuseclass}
\end{sphinxuseclass}
\end{sphinxuseclass}
\end{sphinxuseclass}
\end{sphinxuseclass}
\end{sphinxuseclass}
\end{sphinxuseclass}
\end{sphinxuseclass}
\end{sphinxuseclass}
\begin{sphinxuseclass}{sd-col}
\begin{sphinxuseclass}{sd-d-flex-row}
\begin{sphinxuseclass}{sd-col-4}
\begin{sphinxuseclass}{sd-col-xs-4}
\begin{sphinxuseclass}{sd-col-sm-4}
\begin{sphinxuseclass}{sd-col-md-4}
\begin{sphinxuseclass}{sd-col-lg-4}
\begin{sphinxuseclass}{sd-card}
\begin{sphinxuseclass}{sd-sphinx-override}
\begin{sphinxuseclass}{sd-w-100}
\begin{sphinxuseclass}{sd-shadow-sm}
\begin{sphinxuseclass}{sd-card-body}
\sphinxAtStartPar
\sphinxincludegraphics{{pb-statics-002-ese-004}.png}

\end{sphinxuseclass}
\end{sphinxuseclass}
\end{sphinxuseclass}
\end{sphinxuseclass}
\end{sphinxuseclass}
\end{sphinxuseclass}
\end{sphinxuseclass}
\end{sphinxuseclass}
\end{sphinxuseclass}
\end{sphinxuseclass}
\end{sphinxuseclass}
\end{sphinxuseclass}
\end{sphinxuseclass}
\end{sphinxuseclass}
\end{sphinxuseclass}
\end{sphinxuseclass}
\end{sphinxuseclass}
\end{sphinxuseclass}
\end{sphinxuseclass}
\end{sphinxuseclass}
\end{sphinxuseclass}\subsubsection*{Soluzione.}



\begin{sphinxuseclass}{sd-container-fluid}
\begin{sphinxuseclass}{sd-sphinx-override}
\begin{sphinxuseclass}{sd-mb-4}
\begin{sphinxuseclass}{sd-row}
\begin{sphinxuseclass}{sd-g-2}
\begin{sphinxuseclass}{sd-g-xs-2}
\begin{sphinxuseclass}{sd-g-sm-2}
\begin{sphinxuseclass}{sd-g-md-2}
\begin{sphinxuseclass}{sd-g-lg-2}
\begin{sphinxuseclass}{sd-col}
\begin{sphinxuseclass}{sd-d-flex-row}
\begin{sphinxuseclass}{sd-col-8}
\begin{sphinxuseclass}{sd-col-xs-8}
\begin{sphinxuseclass}{sd-col-sm-8}
\begin{sphinxuseclass}{sd-col-md-8}
\begin{sphinxuseclass}{sd-col-lg-8}
\begin{sphinxuseclass}{sd-card}
\begin{sphinxuseclass}{sd-sphinx-override}
\begin{sphinxuseclass}{sd-w-100}
\begin{sphinxuseclass}{sd-shadow-sm}
\begin{sphinxuseclass}{sd-card-body}
\begin{sphinxuseclass}{sd-card-title}
\begin{sphinxuseclass}{sd-font-weight-bold}Problema 12.
\end{sphinxuseclass}
\end{sphinxuseclass}
\sphinxAtStartPar
Testo del problema…

\end{sphinxuseclass}
\end{sphinxuseclass}
\end{sphinxuseclass}
\end{sphinxuseclass}
\end{sphinxuseclass}
\end{sphinxuseclass}
\end{sphinxuseclass}
\end{sphinxuseclass}
\end{sphinxuseclass}
\end{sphinxuseclass}
\end{sphinxuseclass}
\end{sphinxuseclass}
\begin{sphinxuseclass}{sd-col}
\begin{sphinxuseclass}{sd-d-flex-row}
\begin{sphinxuseclass}{sd-col-4}
\begin{sphinxuseclass}{sd-col-xs-4}
\begin{sphinxuseclass}{sd-col-sm-4}
\begin{sphinxuseclass}{sd-col-md-4}
\begin{sphinxuseclass}{sd-col-lg-4}
\begin{sphinxuseclass}{sd-card}
\begin{sphinxuseclass}{sd-sphinx-override}
\begin{sphinxuseclass}{sd-w-100}
\begin{sphinxuseclass}{sd-shadow-sm}
\begin{sphinxuseclass}{sd-card-body}
\sphinxAtStartPar
\sphinxincludegraphics{{pb-statics-002-ese-005}.png}

\end{sphinxuseclass}
\end{sphinxuseclass}
\end{sphinxuseclass}
\end{sphinxuseclass}
\end{sphinxuseclass}
\end{sphinxuseclass}
\end{sphinxuseclass}
\end{sphinxuseclass}
\end{sphinxuseclass}
\end{sphinxuseclass}
\end{sphinxuseclass}
\end{sphinxuseclass}
\end{sphinxuseclass}
\end{sphinxuseclass}
\end{sphinxuseclass}
\end{sphinxuseclass}
\end{sphinxuseclass}
\end{sphinxuseclass}
\end{sphinxuseclass}
\end{sphinxuseclass}
\end{sphinxuseclass}\subsubsection*{Soluzione.}



\begin{sphinxuseclass}{sd-container-fluid}
\begin{sphinxuseclass}{sd-sphinx-override}
\begin{sphinxuseclass}{sd-mb-4}
\begin{sphinxuseclass}{sd-row}
\begin{sphinxuseclass}{sd-g-2}
\begin{sphinxuseclass}{sd-g-xs-2}
\begin{sphinxuseclass}{sd-g-sm-2}
\begin{sphinxuseclass}{sd-g-md-2}
\begin{sphinxuseclass}{sd-g-lg-2}
\begin{sphinxuseclass}{sd-col}
\begin{sphinxuseclass}{sd-d-flex-row}
\begin{sphinxuseclass}{sd-col-8}
\begin{sphinxuseclass}{sd-col-xs-8}
\begin{sphinxuseclass}{sd-col-sm-8}
\begin{sphinxuseclass}{sd-col-md-8}
\begin{sphinxuseclass}{sd-col-lg-8}
\begin{sphinxuseclass}{sd-card}
\begin{sphinxuseclass}{sd-sphinx-override}
\begin{sphinxuseclass}{sd-w-100}
\begin{sphinxuseclass}{sd-shadow-sm}
\begin{sphinxuseclass}{sd-card-body}
\begin{sphinxuseclass}{sd-card-title}
\begin{sphinxuseclass}{sd-font-weight-bold}Problema 13.
\end{sphinxuseclass}
\end{sphinxuseclass}
\sphinxAtStartPar
Testo del problema…

\end{sphinxuseclass}
\end{sphinxuseclass}
\end{sphinxuseclass}
\end{sphinxuseclass}
\end{sphinxuseclass}
\end{sphinxuseclass}
\end{sphinxuseclass}
\end{sphinxuseclass}
\end{sphinxuseclass}
\end{sphinxuseclass}
\end{sphinxuseclass}
\end{sphinxuseclass}
\begin{sphinxuseclass}{sd-col}
\begin{sphinxuseclass}{sd-d-flex-row}
\begin{sphinxuseclass}{sd-col-4}
\begin{sphinxuseclass}{sd-col-xs-4}
\begin{sphinxuseclass}{sd-col-sm-4}
\begin{sphinxuseclass}{sd-col-md-4}
\begin{sphinxuseclass}{sd-col-lg-4}
\begin{sphinxuseclass}{sd-card}
\begin{sphinxuseclass}{sd-sphinx-override}
\begin{sphinxuseclass}{sd-w-100}
\begin{sphinxuseclass}{sd-shadow-sm}
\begin{sphinxuseclass}{sd-card-body}
\sphinxAtStartPar
\sphinxincludegraphics{{pb-statics-002-ese-006}.png}

\end{sphinxuseclass}
\end{sphinxuseclass}
\end{sphinxuseclass}
\end{sphinxuseclass}
\end{sphinxuseclass}
\end{sphinxuseclass}
\end{sphinxuseclass}
\end{sphinxuseclass}
\end{sphinxuseclass}
\end{sphinxuseclass}
\end{sphinxuseclass}
\end{sphinxuseclass}
\end{sphinxuseclass}
\end{sphinxuseclass}
\end{sphinxuseclass}
\end{sphinxuseclass}
\end{sphinxuseclass}
\end{sphinxuseclass}
\end{sphinxuseclass}
\end{sphinxuseclass}
\end{sphinxuseclass}\subsubsection*{Soluzione.}



\begin{sphinxuseclass}{sd-container-fluid}
\begin{sphinxuseclass}{sd-sphinx-override}
\begin{sphinxuseclass}{sd-mb-4}
\begin{sphinxuseclass}{sd-row}
\begin{sphinxuseclass}{sd-g-2}
\begin{sphinxuseclass}{sd-g-xs-2}
\begin{sphinxuseclass}{sd-g-sm-2}
\begin{sphinxuseclass}{sd-g-md-2}
\begin{sphinxuseclass}{sd-g-lg-2}
\begin{sphinxuseclass}{sd-col}
\begin{sphinxuseclass}{sd-d-flex-row}
\begin{sphinxuseclass}{sd-col-8}
\begin{sphinxuseclass}{sd-col-xs-8}
\begin{sphinxuseclass}{sd-col-sm-8}
\begin{sphinxuseclass}{sd-col-md-8}
\begin{sphinxuseclass}{sd-col-lg-8}
\begin{sphinxuseclass}{sd-card}
\begin{sphinxuseclass}{sd-sphinx-override}
\begin{sphinxuseclass}{sd-w-100}
\begin{sphinxuseclass}{sd-shadow-sm}
\begin{sphinxuseclass}{sd-card-body}
\begin{sphinxuseclass}{sd-card-title}
\begin{sphinxuseclass}{sd-font-weight-bold}Problema 14.
\end{sphinxuseclass}
\end{sphinxuseclass}
\sphinxAtStartPar
Testo del problema…

\end{sphinxuseclass}
\end{sphinxuseclass}
\end{sphinxuseclass}
\end{sphinxuseclass}
\end{sphinxuseclass}
\end{sphinxuseclass}
\end{sphinxuseclass}
\end{sphinxuseclass}
\end{sphinxuseclass}
\end{sphinxuseclass}
\end{sphinxuseclass}
\end{sphinxuseclass}
\begin{sphinxuseclass}{sd-col}
\begin{sphinxuseclass}{sd-d-flex-row}
\begin{sphinxuseclass}{sd-col-4}
\begin{sphinxuseclass}{sd-col-xs-4}
\begin{sphinxuseclass}{sd-col-sm-4}
\begin{sphinxuseclass}{sd-col-md-4}
\begin{sphinxuseclass}{sd-col-lg-4}
\begin{sphinxuseclass}{sd-card}
\begin{sphinxuseclass}{sd-sphinx-override}
\begin{sphinxuseclass}{sd-w-100}
\begin{sphinxuseclass}{sd-shadow-sm}
\begin{sphinxuseclass}{sd-card-body}
\sphinxAtStartPar
\sphinxincludegraphics{{pb-statics-002-ese-007}.png}

\end{sphinxuseclass}
\end{sphinxuseclass}
\end{sphinxuseclass}
\end{sphinxuseclass}
\end{sphinxuseclass}
\end{sphinxuseclass}
\end{sphinxuseclass}
\end{sphinxuseclass}
\end{sphinxuseclass}
\end{sphinxuseclass}
\end{sphinxuseclass}
\end{sphinxuseclass}
\end{sphinxuseclass}
\end{sphinxuseclass}
\end{sphinxuseclass}
\end{sphinxuseclass}
\end{sphinxuseclass}
\end{sphinxuseclass}
\end{sphinxuseclass}
\end{sphinxuseclass}
\end{sphinxuseclass}\subsubsection*{Soluzione.}

\sphinxstepscope

\begin{sphinxuseclass}{sd-container-fluid}
\begin{sphinxuseclass}{sd-sphinx-override}
\begin{sphinxuseclass}{sd-p-0}
\begin{sphinxuseclass}{sd-mt-2}
\begin{sphinxuseclass}{sd-mb-4}
\begin{sphinxuseclass}{sd-row}
\begin{sphinxuseclass}{sd-row-cols-2}
\begin{sphinxuseclass}{sd-gx-2}
\begin{sphinxuseclass}{sd-gy-1}
\begin{sphinxuseclass}{sd-col}
\begin{sphinxuseclass}{sd-d-flex-row}
\begin{sphinxuseclass}{sd-align-minor-center}
\begin{sphinxuseclass}{sd-container-fluid}
\begin{sphinxuseclass}{sd-sphinx-override}
\begin{sphinxuseclass}{sd-row}
\begin{sphinxuseclass}{sd-row-cols-2}
\begin{sphinxuseclass}{sd-row-cols-xs-2}
\begin{sphinxuseclass}{sd-row-cols-sm-3}
\begin{sphinxuseclass}{sd-row-cols-md-3}
\begin{sphinxuseclass}{sd-row-cols-lg-3}
\begin{sphinxuseclass}{sd-gx-3}
\begin{sphinxuseclass}{sd-gy-1}
\begin{sphinxuseclass}{sd-col}
\begin{sphinxuseclass}{sd-col-auto}
\begin{sphinxuseclass}{sd-d-flex-row}
\begin{sphinxuseclass}{sd-align-minor-center}
\sphinxAtStartPar
basics

\end{sphinxuseclass}
\end{sphinxuseclass}
\end{sphinxuseclass}
\end{sphinxuseclass}
\begin{sphinxuseclass}{sd-col}
\begin{sphinxuseclass}{sd-col-auto}
\begin{sphinxuseclass}{sd-d-flex-row}
\begin{sphinxuseclass}{sd-align-minor-center}
\sphinxAtStartPar
Nov 06, 2024

\end{sphinxuseclass}
\end{sphinxuseclass}
\end{sphinxuseclass}
\end{sphinxuseclass}
\begin{sphinxuseclass}{sd-col}
\begin{sphinxuseclass}{sd-col-auto}
\begin{sphinxuseclass}{sd-d-flex-row}
\begin{sphinxuseclass}{sd-align-minor-center}
\sphinxAtStartPar
1 min read

\end{sphinxuseclass}
\end{sphinxuseclass}
\end{sphinxuseclass}
\end{sphinxuseclass}
\end{sphinxuseclass}
\end{sphinxuseclass}
\end{sphinxuseclass}
\end{sphinxuseclass}
\end{sphinxuseclass}
\end{sphinxuseclass}
\end{sphinxuseclass}
\end{sphinxuseclass}
\end{sphinxuseclass}
\end{sphinxuseclass}
\end{sphinxuseclass}
\end{sphinxuseclass}
\end{sphinxuseclass}
\end{sphinxuseclass}
\end{sphinxuseclass}
\end{sphinxuseclass}
\end{sphinxuseclass}
\end{sphinxuseclass}
\end{sphinxuseclass}
\end{sphinxuseclass}
\end{sphinxuseclass}
\end{sphinxuseclass}

\chapter{Inerzia}
\label{\detokenize{ch/mechanics/inertia:inerzia}}\label{\detokenize{ch/mechanics/inertia:physics-hs-mechanics-inertia}}\label{\detokenize{ch/mechanics/inertia::doc}}

\section{Massa e distribuzione di massa}
\label{\detokenize{ch/mechanics/inertia:massa-e-distribuzione-di-massa}}
\sphinxAtStartPar
La massa è la grandezza fisica che rappresenta la quantità di materia (\sphinxstylestrong{todo} \sphinxstyleemphasis{Non confonderla con la mole, definita come quantità di sostanza, una volta affermatasi la teoria atomica})

\sphinxAtStartPar
In meccanica classica, la può essere definita in maniera operativa:
\begin{itemize}
\item {} 
\sphinxAtStartPar
tramite la sua {\hyperref[\detokenize{ch/mechanics/actions-examples:physics-hs-mechanics-actions-gravitation}]{\sphinxcrossref{\DUrole{std,std-ref}{interazione gravitazionale}}}} con altri corpi dotati di massa

\item {} 
\sphinxAtStartPar
come una misura della resistenza di un sistema ai cambiamenti del suo stato di moto in risposta a una forza applicata, come sarà chiaro dalle equazioni della {\hyperref[\detokenize{ch/mechanics/dynamics:physics-hs-mechanics-dynamics}]{\sphinxcrossref{\DUrole{std,std-ref}{dinamica}}}}

\end{itemize}


\section{Quantità dinamiche}
\label{\detokenize{ch/mechanics/inertia:quantita-dinamiche}}
\sphinxAtStartPar
Come sarà chiaro nello sviluppo delle {\hyperref[\detokenize{ch/mechanics/dynamics-eom-proof-points:physics-hs-mechanics-dynamics-eom-points}]{\sphinxcrossref{\DUrole{std,std-ref}{equazioni di moto di un sistema}}}}, la definizione di alcune grandezze dinamiche additive risulta naturale, fornendo dei concetti utili e sintetici per la costruzione di un modello e l’interpretazione dei fenomeni fisici.

\sphinxAtStartPar
Queste grandezze dinamiche combinano la massa e la sua distribuzione con le grandezze cinematiche del sistema. In particolare, risulta utile definire tre grandezze:
\begin{itemize}
\item {} 
\sphinxAtStartPar
quantità di moto

\item {} 
\sphinxAtStartPar
momento della quantità di moto

\item {} 
\sphinxAtStartPar
energia cinetica

\end{itemize}

\sphinxAtStartPar
Le {\hyperref[\detokenize{ch/mechanics/dynamics-eom:physics-hs-mechanics-dynamics-eom}]{\sphinxcrossref{\DUrole{std,std-ref}{equazioni del moto}}}} dei sistemi rappresentano delle equazioni differenziali che mettono in relazione la variazione di queste quantità dinamiche con la causa di queste variazioni, in generale riconducibile ad {\hyperref[\detokenize{ch/mechanics/actions:physics-hs-mechanics-actions}]{\sphinxcrossref{\DUrole{std,std-ref}{azioni}}}} agenti sul sistema.

\sphinxAtStartPar
Sotto opportune ipotesi, queste grandezze dinamiche sono costanti del moto, come descritto dalle {\hyperref[\detokenize{ch/mechanics/dynamics-conservation:physics-hs-mechanics-dynamics-conservation}]{\sphinxcrossref{\DUrole{std,std-ref}{leggi di conservazione}}}}.

\sphinxAtStartPar
Le 3 grandezze dinamiche possono avere espressioni diverse, a seconda del sistema di interesse. Nel caso di corpi rigidi, queste possono essere espresse in termini di velocità di un punto materiale e della velocità angolare del corpo.

\sphinxstepscope

\begin{sphinxuseclass}{sd-container-fluid}
\begin{sphinxuseclass}{sd-sphinx-override}
\begin{sphinxuseclass}{sd-p-0}
\begin{sphinxuseclass}{sd-mt-2}
\begin{sphinxuseclass}{sd-mb-4}
\begin{sphinxuseclass}{sd-row}
\begin{sphinxuseclass}{sd-row-cols-2}
\begin{sphinxuseclass}{sd-gx-2}
\begin{sphinxuseclass}{sd-gy-1}
\begin{sphinxuseclass}{sd-col}
\begin{sphinxuseclass}{sd-d-flex-row}
\begin{sphinxuseclass}{sd-align-minor-center}
\begin{sphinxuseclass}{sd-container-fluid}
\begin{sphinxuseclass}{sd-sphinx-override}
\begin{sphinxuseclass}{sd-row}
\begin{sphinxuseclass}{sd-row-cols-2}
\begin{sphinxuseclass}{sd-row-cols-xs-2}
\begin{sphinxuseclass}{sd-row-cols-sm-3}
\begin{sphinxuseclass}{sd-row-cols-md-3}
\begin{sphinxuseclass}{sd-row-cols-lg-3}
\begin{sphinxuseclass}{sd-gx-3}
\begin{sphinxuseclass}{sd-gy-1}
\begin{sphinxuseclass}{sd-col}
\begin{sphinxuseclass}{sd-col-auto}
\begin{sphinxuseclass}{sd-d-flex-row}
\begin{sphinxuseclass}{sd-align-minor-center}
\sphinxAtStartPar
basics

\end{sphinxuseclass}
\end{sphinxuseclass}
\end{sphinxuseclass}
\end{sphinxuseclass}
\begin{sphinxuseclass}{sd-col}
\begin{sphinxuseclass}{sd-col-auto}
\begin{sphinxuseclass}{sd-d-flex-row}
\begin{sphinxuseclass}{sd-align-minor-center}
\sphinxAtStartPar
Nov 06, 2024

\end{sphinxuseclass}
\end{sphinxuseclass}
\end{sphinxuseclass}
\end{sphinxuseclass}
\begin{sphinxuseclass}{sd-col}
\begin{sphinxuseclass}{sd-col-auto}
\begin{sphinxuseclass}{sd-d-flex-row}
\begin{sphinxuseclass}{sd-align-minor-center}
\sphinxAtStartPar
0 min read

\end{sphinxuseclass}
\end{sphinxuseclass}
\end{sphinxuseclass}
\end{sphinxuseclass}
\end{sphinxuseclass}
\end{sphinxuseclass}
\end{sphinxuseclass}
\end{sphinxuseclass}
\end{sphinxuseclass}
\end{sphinxuseclass}
\end{sphinxuseclass}
\end{sphinxuseclass}
\end{sphinxuseclass}
\end{sphinxuseclass}
\end{sphinxuseclass}
\end{sphinxuseclass}
\end{sphinxuseclass}
\end{sphinxuseclass}
\end{sphinxuseclass}
\end{sphinxuseclass}
\end{sphinxuseclass}
\end{sphinxuseclass}
\end{sphinxuseclass}
\end{sphinxuseclass}
\end{sphinxuseclass}
\end{sphinxuseclass}

\section{Inerzia e grandezze dinamiche di un punto}
\label{\detokenize{ch/mechanics/inertia-point:inerzia-e-grandezze-dinamiche-di-un-punto}}\label{\detokenize{ch/mechanics/inertia-point::doc}}\begin{equation*}
\begin{split}\begin{aligned}
  \vec{Q}_P     & = m_P \, \vec{v}_P \\
  \vec{L}_{P,H} & = m_P \, (P - H) \times \vec{v}_P \\
   K_P          & = \frac{1}{2} m_P \left| \vec{v}_P \right|^2
\end{aligned}\end{split}
\end{equation*}
\sphinxstepscope

\begin{sphinxuseclass}{sd-container-fluid}
\begin{sphinxuseclass}{sd-sphinx-override}
\begin{sphinxuseclass}{sd-p-0}
\begin{sphinxuseclass}{sd-mt-2}
\begin{sphinxuseclass}{sd-mb-4}
\begin{sphinxuseclass}{sd-row}
\begin{sphinxuseclass}{sd-row-cols-2}
\begin{sphinxuseclass}{sd-gx-2}
\begin{sphinxuseclass}{sd-gy-1}
\begin{sphinxuseclass}{sd-col}
\begin{sphinxuseclass}{sd-d-flex-row}
\begin{sphinxuseclass}{sd-align-minor-center}
\begin{sphinxuseclass}{sd-container-fluid}
\begin{sphinxuseclass}{sd-sphinx-override}
\begin{sphinxuseclass}{sd-row}
\begin{sphinxuseclass}{sd-row-cols-2}
\begin{sphinxuseclass}{sd-row-cols-xs-2}
\begin{sphinxuseclass}{sd-row-cols-sm-3}
\begin{sphinxuseclass}{sd-row-cols-md-3}
\begin{sphinxuseclass}{sd-row-cols-lg-3}
\begin{sphinxuseclass}{sd-gx-3}
\begin{sphinxuseclass}{sd-gy-1}
\begin{sphinxuseclass}{sd-col}
\begin{sphinxuseclass}{sd-col-auto}
\begin{sphinxuseclass}{sd-d-flex-row}
\begin{sphinxuseclass}{sd-align-minor-center}
\sphinxAtStartPar
basics

\end{sphinxuseclass}
\end{sphinxuseclass}
\end{sphinxuseclass}
\end{sphinxuseclass}
\begin{sphinxuseclass}{sd-col}
\begin{sphinxuseclass}{sd-col-auto}
\begin{sphinxuseclass}{sd-d-flex-row}
\begin{sphinxuseclass}{sd-align-minor-center}
\sphinxAtStartPar
Nov 06, 2024

\end{sphinxuseclass}
\end{sphinxuseclass}
\end{sphinxuseclass}
\end{sphinxuseclass}
\begin{sphinxuseclass}{sd-col}
\begin{sphinxuseclass}{sd-col-auto}
\begin{sphinxuseclass}{sd-d-flex-row}
\begin{sphinxuseclass}{sd-align-minor-center}
\sphinxAtStartPar
0 min read

\end{sphinxuseclass}
\end{sphinxuseclass}
\end{sphinxuseclass}
\end{sphinxuseclass}
\end{sphinxuseclass}
\end{sphinxuseclass}
\end{sphinxuseclass}
\end{sphinxuseclass}
\end{sphinxuseclass}
\end{sphinxuseclass}
\end{sphinxuseclass}
\end{sphinxuseclass}
\end{sphinxuseclass}
\end{sphinxuseclass}
\end{sphinxuseclass}
\end{sphinxuseclass}
\end{sphinxuseclass}
\end{sphinxuseclass}
\end{sphinxuseclass}
\end{sphinxuseclass}
\end{sphinxuseclass}
\end{sphinxuseclass}
\end{sphinxuseclass}
\end{sphinxuseclass}
\end{sphinxuseclass}
\end{sphinxuseclass}

\section{Inerzia e grandezze dinamiche di un sistema esteso con distribuzione discreta di massa}
\label{\detokenize{ch/mechanics/inertia-points:inerzia-e-grandezze-dinamiche-di-un-sistema-esteso-con-distribuzione-discreta-di-massa}}\label{\detokenize{ch/mechanics/inertia-points::doc}}\begin{equation*}
\begin{split}\begin{aligned}
  \vec{Q}       = \sum_i \vec{Q}_i     & = \sum_i  m_i \, \vec{v}_i \\
  \vec{L}_{H}   = \sum_i \vec{L}_{i,H} & = \sum_i  m_i \, (P_i - H) \times \vec{v}_i \\
   K            = \sum_i  K_i          & = \sum_i  \frac{1}{2} m_i \left| \vec{v}_i \right|^2
\end{aligned}\end{split}
\end{equation*}

\subsection{Sistemi rigidi}
\label{\detokenize{ch/mechanics/inertia-points:sistemi-rigidi}}
\sphinxAtStartPar
Usando la definizione di centro di massa
\begin{equation*}
\begin{split}m G = \sum_i m_i P_i\end{split}
\end{equation*}
\sphinxAtStartPar
e legge del moto rigido
\begin{equation*}
\begin{split}\vec{v}_i - \vec{v}_P = \vec{\omega} \times (P_i - P)\end{split}
\end{equation*}
\sphinxAtStartPar
le quantità dinamiche possono essere espresse in funzione della velocità del punto di riferimento \(P\) e della velocità angolare del sistema, tramite la massa e le altre quantità inerziali
\begin{itemize}
\item {} 
\sphinxAtStartPar
la quantità di moto

\end{itemize}
\begin{equation*}
\begin{split}\begin{aligned}
  \vec{Q} = \sum_i m_i \vec{v}_i
        & = \sum_i m_i \left( \vec{v}_P + \vec{\omega} \times (P_i - P) \right) = \\
        & =  m \vec{v}_P + \vec{\omega} \times m (G - P) 
\end{aligned}\end{split}
\end{equation*}\begin{itemize}
\item {} 
\sphinxAtStartPar
momento della quantità di moto

\end{itemize}
\begin{equation*}
\begin{split}\begin{aligned}
  \vec{L}_H = \sum_i m_i (P_i - H) \times \vec{v}_i
        & = \sum_i m_i \left( P_i - P + \vec{r}_P - \vec{r}_H \right) \times \vec{v}_i = \\
        & = \sum_i m_i \left( P_i - P \right) \times \vec{v}_i + \left( P - H \right) \times \vec{Q}  = \\
        & = \sum_i m_i \left( P_i - P \right) \times \left( \vec{v}_P - \left( P_i - P \right) \times \vec{\omega} \right) + \left( P - H \right) \times \vec{Q}  = \\
        & = m (G - P) \times \vec{v}_P - \sum_i m_i \left( P_i - P \right) \times \left( \left( P_i - P \right) \times \vec{\omega} \right) + \left( P - H \right) \times \vec{Q}  = \\
        & = \mathbb{I}_P \cdot \vec{\omega} +  m (G - P) \times \vec{v}_P + \left( P - H \right) \times \vec{Q}
\end{aligned}\end{split}
\end{equation*}
\sphinxAtStartPar
Nel caso di moto 2\sphinxhyphen{}dimensionale e velocità angolare perpendicolare a questo piano, \sphinxstylestrong{todo}
\begin{equation*}
\begin{split}\begin{aligned}
  \vec{r}_{i/P} & := P_i - P = \left( x_i - x_P \right) \hat{x} + \left( y_i - y_P \right) \hat{y} \\
  \vec{\omega} & = \dot{\theta} \, \hat{z}
\end{aligned}\end{split}
\end{equation*}\begin{equation*}
\begin{split}\begin{aligned}
  - \vec{r}_{i/P} \times \left( \vec{r}_{i/P} \times \hat{\omega} \right) 
  & = - ( \Delta x_i \hat{x} + \Delta y_i \hat{y} ) \times \left[ ( \Delta x_i \hat{x} + \Delta y_i \hat{y} ) \times \dot{\theta} \hat{z} \right] = \\ 
  & = - \dot{\theta} ( \Delta x_i \hat{x} + \Delta y_i \hat{y} ) \times \left( - \Delta x_i \hat{y} + \Delta y_i \hat{x} \right) = \\
  & = \left( \Delta x_i^2 + \Delta y_i^2 \right) \dot{\theta} \, \hat{z} \ .
\end{aligned}\end{split}
\end{equation*}
\sphinxAtStartPar
e l’espressione del momento della quantità di moto diventa
\begin{equation*}
\begin{split}\vec{L}_{H} = I_P \, \vec{\omega} +  m (G - P) \times \vec{v}_P + \left( P - H \right) \times \vec{Q}\end{split}
\end{equation*}
\sphinxAtStartPar
con
\begin{equation*}
\begin{split}I_P = \sum_i m_i \left[ \left(x_i - x_P\right)^2 + \left(y_i - y_P\right)^2 \right]\end{split}
\end{equation*}
\sphinxstepscope

\begin{sphinxuseclass}{sd-container-fluid}
\begin{sphinxuseclass}{sd-sphinx-override}
\begin{sphinxuseclass}{sd-p-0}
\begin{sphinxuseclass}{sd-mt-2}
\begin{sphinxuseclass}{sd-mb-4}
\begin{sphinxuseclass}{sd-row}
\begin{sphinxuseclass}{sd-row-cols-2}
\begin{sphinxuseclass}{sd-gx-2}
\begin{sphinxuseclass}{sd-gy-1}
\begin{sphinxuseclass}{sd-col}
\begin{sphinxuseclass}{sd-d-flex-row}
\begin{sphinxuseclass}{sd-align-minor-center}
\begin{sphinxuseclass}{sd-container-fluid}
\begin{sphinxuseclass}{sd-sphinx-override}
\begin{sphinxuseclass}{sd-row}
\begin{sphinxuseclass}{sd-row-cols-2}
\begin{sphinxuseclass}{sd-row-cols-xs-2}
\begin{sphinxuseclass}{sd-row-cols-sm-3}
\begin{sphinxuseclass}{sd-row-cols-md-3}
\begin{sphinxuseclass}{sd-row-cols-lg-3}
\begin{sphinxuseclass}{sd-gx-3}
\begin{sphinxuseclass}{sd-gy-1}
\begin{sphinxuseclass}{sd-col}
\begin{sphinxuseclass}{sd-col-auto}
\begin{sphinxuseclass}{sd-d-flex-row}
\begin{sphinxuseclass}{sd-align-minor-center}
\sphinxAtStartPar
basics

\end{sphinxuseclass}
\end{sphinxuseclass}
\end{sphinxuseclass}
\end{sphinxuseclass}
\begin{sphinxuseclass}{sd-col}
\begin{sphinxuseclass}{sd-col-auto}
\begin{sphinxuseclass}{sd-d-flex-row}
\begin{sphinxuseclass}{sd-align-minor-center}
\sphinxAtStartPar
Nov 06, 2024

\end{sphinxuseclass}
\end{sphinxuseclass}
\end{sphinxuseclass}
\end{sphinxuseclass}
\begin{sphinxuseclass}{sd-col}
\begin{sphinxuseclass}{sd-col-auto}
\begin{sphinxuseclass}{sd-d-flex-row}
\begin{sphinxuseclass}{sd-align-minor-center}
\sphinxAtStartPar
0 min read

\end{sphinxuseclass}
\end{sphinxuseclass}
\end{sphinxuseclass}
\end{sphinxuseclass}
\end{sphinxuseclass}
\end{sphinxuseclass}
\end{sphinxuseclass}
\end{sphinxuseclass}
\end{sphinxuseclass}
\end{sphinxuseclass}
\end{sphinxuseclass}
\end{sphinxuseclass}
\end{sphinxuseclass}
\end{sphinxuseclass}
\end{sphinxuseclass}
\end{sphinxuseclass}
\end{sphinxuseclass}
\end{sphinxuseclass}
\end{sphinxuseclass}
\end{sphinxuseclass}
\end{sphinxuseclass}
\end{sphinxuseclass}
\end{sphinxuseclass}
\end{sphinxuseclass}
\end{sphinxuseclass}
\end{sphinxuseclass}

\section{Inerzia e grandezze dinamiche di un sistema esteso con distribuzione continua di massa}
\label{\detokenize{ch/mechanics/inertia-continuum:inerzia-e-grandezze-dinamiche-di-un-sistema-esteso-con-distribuzione-continua-di-massa}}\label{\detokenize{ch/mechanics/inertia-continuum::doc}}

\subsection{Sistemi rigidi}
\label{\detokenize{ch/mechanics/inertia-continuum:sistemi-rigidi}}
\sphinxstepscope

\begin{sphinxuseclass}{sd-container-fluid}
\begin{sphinxuseclass}{sd-sphinx-override}
\begin{sphinxuseclass}{sd-p-0}
\begin{sphinxuseclass}{sd-mt-2}
\begin{sphinxuseclass}{sd-mb-4}
\begin{sphinxuseclass}{sd-row}
\begin{sphinxuseclass}{sd-row-cols-2}
\begin{sphinxuseclass}{sd-gx-2}
\begin{sphinxuseclass}{sd-gy-1}
\begin{sphinxuseclass}{sd-col}
\begin{sphinxuseclass}{sd-d-flex-row}
\begin{sphinxuseclass}{sd-align-minor-center}
\begin{sphinxuseclass}{sd-container-fluid}
\begin{sphinxuseclass}{sd-sphinx-override}
\begin{sphinxuseclass}{sd-row}
\begin{sphinxuseclass}{sd-row-cols-2}
\begin{sphinxuseclass}{sd-row-cols-xs-2}
\begin{sphinxuseclass}{sd-row-cols-sm-3}
\begin{sphinxuseclass}{sd-row-cols-md-3}
\begin{sphinxuseclass}{sd-row-cols-lg-3}
\begin{sphinxuseclass}{sd-gx-3}
\begin{sphinxuseclass}{sd-gy-1}
\begin{sphinxuseclass}{sd-col}
\begin{sphinxuseclass}{sd-col-auto}
\begin{sphinxuseclass}{sd-d-flex-row}
\begin{sphinxuseclass}{sd-align-minor-center}
\sphinxAtStartPar
basics

\end{sphinxuseclass}
\end{sphinxuseclass}
\end{sphinxuseclass}
\end{sphinxuseclass}
\begin{sphinxuseclass}{sd-col}
\begin{sphinxuseclass}{sd-col-auto}
\begin{sphinxuseclass}{sd-d-flex-row}
\begin{sphinxuseclass}{sd-align-minor-center}
\sphinxAtStartPar
Nov 06, 2024

\end{sphinxuseclass}
\end{sphinxuseclass}
\end{sphinxuseclass}
\end{sphinxuseclass}
\begin{sphinxuseclass}{sd-col}
\begin{sphinxuseclass}{sd-col-auto}
\begin{sphinxuseclass}{sd-d-flex-row}
\begin{sphinxuseclass}{sd-align-minor-center}
\sphinxAtStartPar
1 min read

\end{sphinxuseclass}
\end{sphinxuseclass}
\end{sphinxuseclass}
\end{sphinxuseclass}
\end{sphinxuseclass}
\end{sphinxuseclass}
\end{sphinxuseclass}
\end{sphinxuseclass}
\end{sphinxuseclass}
\end{sphinxuseclass}
\end{sphinxuseclass}
\end{sphinxuseclass}
\end{sphinxuseclass}
\end{sphinxuseclass}
\end{sphinxuseclass}
\end{sphinxuseclass}
\end{sphinxuseclass}
\end{sphinxuseclass}
\end{sphinxuseclass}
\end{sphinxuseclass}
\end{sphinxuseclass}
\end{sphinxuseclass}
\end{sphinxuseclass}
\end{sphinxuseclass}
\end{sphinxuseclass}
\end{sphinxuseclass}

\chapter{Dinamica}
\label{\detokenize{ch/mechanics/dynamics:dinamica}}\label{\detokenize{ch/mechanics/dynamics:physics-hs-mechanics-dynamics}}\label{\detokenize{ch/mechanics/dynamics::doc}}
\sphinxAtStartPar
La dinamica si occupa del moto dei sistemi e delle cause del moto, mettendo insieme la descrizione cinematica, l’inerzia dei sistemi a perseverare nel moto, e le cause di una variazione del moto.

\sphinxAtStartPar
\sphinxstylestrong{Princìpi della dinamica.} Vengono discussi i tre principi della dinamica di Newton e il significato della relatività galileiana.

\sphinxAtStartPar
\sphinxstylestrong{Equazioni cardinali della dinamica.} Vengono presentate le tre equazioni cardinali della dinamica per sistemi chiusi, che mettono in relazione la variazione delle grandezze dinamiche alle azioni, e che nel caso di moti regolari possono essere scritte in forma differenziale
\begin{equation*}
\begin{split}\begin{aligned}
 \dot{\vec{Q}} & = \vec{R}^{ext} & \text{(bilancio quantità di moto)} \\
 \dot{\vec{L}}_H + \dot{\vec{x}}_H \times \vec{Q} & = \vec{M}_H^{ext} & \text{(bilancio momento della quantità di moto)} \\
 \dot{K} & = P^{tot} & \text{(bilancio energia cinetica)} \ .
\end{aligned}\end{split}
\end{equation*}
\sphinxAtStartPar
Viene dimostrato che le equazioni di bilancio hanno la stessa forma per ogni sistema chiuso se scritti in termini di variazione di quantità di moto, momento della quantità di moto ed energia cinetica, senza esplicitare la forma particolare di queste grandezze dinamiche per i sistemi particolari presi in considerazione. Vengono forniti alcuni esempi ed esercizi svolti.

\sphinxAtStartPar
\sphinxstylestrong{Leggi di conservazione.} Sotto opportune ipotesi immediatamente riconoscibili dalle equazioni cardinali, vengono ricavate le leggi di conservazione validi per i sistemi meccanici,
\begin{equation*}
\begin{split}\begin{aligned}
  \vec{R}^{ext} & = \vec{0} \qquad  & \rightarrow \qquad \ \ \vec{Q} = \text{const.} \\
  \vec{M}_H^{ext} & = \vec{0}, \dot{\vec{x}}_H \times \vec{Q} = \vec{0} \qquad  & \rightarrow \qquad \vec{L}_H = \text{const.} \\
  P^{tot} & = \vec{0} \qquad  & \rightarrow \qquad \ \  K = \text{const.} \\
\end{aligned}\end{split}
\end{equation*}
\sphinxAtStartPar
Nel caso in cui le azioni agenti sul sistema non abbiano potenza nulla, ma che siano forze conservative, si riconosce la legge di conservazione dell’energia meccanica \(E^{mec}\), definita come somma dell’energia cinetica, \(K\), e dell’energia potenziale, \(V\),
\begin{equation*}
\begin{split}P^{tot} = -\dot{V} \ , \quad E^{mec} = K + V \qquad \rightarrow \qquad E^{mec} = \text{const.}\end{split}
\end{equation*}
\sphinxAtStartPar
\sphinxstylestrong{Urti.}
Viene presentato un modello di urto tra sistemi fondato unicamente sul coefficiente di restituzione, \(\varepsilon\), per rappresentare la frazione di energia meccanica persa dal sistema durante l’urto. Vengono presentati dei problemi risolti grazie ai princìpi di conservazione e alle equazioni cardinali in forma incrementale.

\sphinxAtStartPar
\sphinxstylestrong{Moti particolari \sphinxhyphen{} gravitazione.} Vengono infine analizzati alcuni sistemi particolare, di interesse pratico, storico, e/o didattico \sphinxstylestrong{todo}



\sphinxstepscope


\section{Princìpi della dinamica di Newton}
\label{\detokenize{ch/mechanics/dynamics-principles:principi-della-dinamica-di-newton}}\label{\detokenize{ch/mechanics/dynamics-principles:physics-hs-mechanics-dynamics-principles}}\label{\detokenize{ch/mechanics/dynamics-principles::doc}}
\sphinxAtStartPar
La meccanica classica di Newton viene costruita assumendo valido il \sphinxstylestrong{principio di conservazione della massa} e i \sphinxstylestrong{tre principi della dinamica}.

\sphinxAtStartPar
\sphinxstylestrong{Principio di conservazione della massa.} In meccanica classica, il principio di Lavoisier di conservazione della massa può essere riassunto con la formula “niente si crea, niente si distrugge”. Per essere più precisi, il principio di conservazione della massa postula che la massa di un sistema chiuso è costante.

\sphinxAtStartPar
\sphinxstylestrong{Primo principio \sphinxhyphen{} principio di inerzia.} Un sistema (o meglio, il baricentro di un sistema) sul quale agisce una forza esterna netta nulla, persevera nel suo stato di quiete o di moto rettilineo uniforme rispetto a un sistema di riferimento inerziale.

\sphinxAtStartPar
\sphinxstylestrong{Secondo principio \sphinxhyphen{} bilancio della quantità di moto per sistemi chiusi.} Rispetto a un sistema di riferimento inerziale, la variazione della quantità di moto \(\vec{Q}\) di un sistema chiuso è uguale all’impulso delle forze esterne \(\vec{I}^{ext}\) agenti su di esso,
\begin{equation*}
\begin{split}\Delta \vec{Q} = \vec{I}^{ext} \ .\end{split}
\end{equation*}
\sphinxAtStartPar
Nel caso di moto regolare, in cui la quantità di moto del sistema è una grandezza continua e differenziabile rispetto al tempo, il secondo principio può essere scritto in forma differenziale, facendo tendere a zero l’intervallo di tempo considerato
\begin{equation*}
\begin{split}\dot{\vec{Q}} = \vec{R}^{ext} \ ,\end{split}
\end{equation*}
\sphinxAtStartPar
avendo indicato con \(\vec{R}^{ext}\) la risultante delle forze esterne agenti sul sistema.

\sphinxAtStartPar
\sphinxstylestrong{Terzo principio \sphinxhyphen{} principio di azione\sphinxhyphen{}reazione.} Se un sistema \(i\) esercita una forza \(\vec{F}_{ji}\) sul sistema \(j\), allora il sistema \(j\) esercita sul sistema \(i\) una forza \(\vec{F}_{ij}\) “uguale e contraria” \sphinxhyphen{} stesso valore assoluto e verso opposto,
\begin{equation*}
\begin{split}\vec{F}_{ij} = - \vec{F}_{ji} \ .\end{split}
\end{equation*}
\sphinxAtStartPar
\sphinxstylestrong{todo} \sphinxstylestrong{Osservazioni}
\begin{itemize}
\item {} 
\sphinxAtStartPar
sistema di riferimento inerziale e invarianza galileiana

\item {} 
\sphinxAtStartPar
sistemi aperti e sistemi chiusi: sottolineare la validità di \(\Delta \vec{Q} = \vec{I}^{ext}\) solo per sistemi chiusi, mentre per sistemi aperti è necessario un termine di flusso della quantità meccanica. Riferimento alla meccanica dei fluidi

\end{itemize}


\subsection{Sistemi di riferimento inerziali e invarianza galileiana.}
\label{\detokenize{ch/mechanics/dynamics-principles:sistemi-di-riferimento-inerziali-e-invarianza-galileiana}}
\sphinxAtStartPar
La formulazione dei princìpi della dinamica si basa sul concetto di sistema di riferimento inerziale, di cui non è stato ancora detto nulla.
E’ possibile dare una definizione operativa di osservatore inerziale (o sistema di riferimento inerziale? \sphinxstylestrong{todo}), supponendo che:
\begin{itemize}
\item {} 
\sphinxAtStartPar
l’osservatore sia dotato di uno strumento in grado di misurare le forze e i momenti ai quali è soggetto (\sphinxstylestrong{todo} ad esempio una bilancia o una combinazione di dinamometri)

\item {} 
\sphinxAtStartPar
sia possibile conoscere le azioni “vere” (\sphinxstylestrong{todo} fare riferimento alle forze “vere” note: gravitazione \sphinxhyphen{} che in meccanica classica è una forza \sphinxhyphen{}, elettromagnetica, nucleare forte e debole; o le loro manifestazioni macroscopiche come ad esempio forze di contatto) agenti sul sistema.

\end{itemize}

\sphinxAtStartPar
\sphinxstylestrong{Definizione.}
Un osservatore è inerziale se la lettura degli strumenti di misura in suo possesso corrisponde alle azioni “vere” agenti sul sistema. In particolare, in assenza di azioni nette gli strumenti restituiscono una misura nulla.

\sphinxAtStartPar
\sphinxstylestrong{Definizione quantità cinematiche.}
Sia \(O\) l’origine di un sistema di riferimento coincidente con un’osservatore inerziale, la velocità di un punto \(P\) rispetto a \(O\) è a derivata del vettore posizione \(P - O\) rispetto al tempo (assoluto in meccanica classica di Newton)
\begin{equation*}
\begin{split}\vec{v}_P = \dfrac{d}{dt} (P - O) \ .\end{split}
\end{equation*}
\sphinxAtStartPar
La quantità di moto di un sistema rispetto al sistema di riferimento inerziale con origine in \(O\) è data dal prodotto della massa del sistema per la velocità del centro di massa \(G\),
\begin{equation*}
\begin{split}\vec{Q} = m \, \vec{v}_G \ .\end{split}
\end{equation*}
\sphinxAtStartPar
\sphinxstylestrong{Equivalenza di sistemi inerziali e invarianza galileiana.}
Dato un sistema inerziale, ogni altro sistema in moto relativo con un moto di traslazione a velocità costante è un sistema inerziale.

\sphinxAtStartPar
\sphinxstylestrong{todo} \sphinxstyleemphasis{Prova.}

\sphinxAtStartPar
\sphinxstylestrong{Invarianza galileiana.}
\begin{itemize}
\item {} 
\sphinxAtStartPar
Posizione
\begin{equation*}
\begin{split}P - O_0 = P - O_1 + O_1 - O_0\end{split}
\end{equation*}
\item {} 
\sphinxAtStartPar
Velocità e quantità di moto
\begin{equation*}
\begin{split}\vec{v}_{P/0} = \vec{v}_{P/1} + \vec{v}_{O_1/0}\end{split}
\end{equation*}\begin{equation*}
\begin{split}\begin{aligned}
    m \vec{v}_{G/0} & = m \vec{v}_{G/1} + m \vec{v}_{O_1/0} \\
       \vec{Q}_{/0} & = \vec{Q}_{/1} + m \vec{v}_{O_1/0}
  \end{aligned}\end{split}
\end{equation*}
\sphinxAtStartPar
con \(\frac{d}{dt} \vec{v}_{O_1/0} = \vec{a}_{O_1/0} = \vec{0}\).

\item {} 
\sphinxAtStartPar
Accelerazione e secondo principio della dinamica
\begin{equation*}
\begin{split}\vec{a}_{P/0} = \vec{a}_{P/1}\end{split}
\end{equation*}\begin{equation*}
\begin{split}\begin{aligned}
    \frac{d}{dt} \vec{Q}_{/0} & = \frac{d}{dt} \vec{Q}_{/1} + \frac{d}{dt} \left( m \vec{v}_{O_1/0}\right) \\
    \dot{\vec{Q}}_{/0} & = \dot{\vec{Q}}_{/1}
  \end{aligned}\end{split}
\end{equation*}
\sphinxAtStartPar
essendo \(\frac{d}{dt} \vec{v}_{O_1/0} = \vec{a}_{O_1/0} = \vec{0}\).

\end{itemize}

\sphinxAtStartPar
Di conseguenza, il secondo principio della dinamica assume la stessa forma quando è riferito a un sistema di riferimento inerziale qualsiasi,
\begin{equation*}
\begin{split}\dot{\vec{Q}} = \vec{R}^{ext} \ ,\end{split}
\end{equation*}
\sphinxAtStartPar
e mentre la regola di trasformazione delle veloctià e delle posizioni rispetto ai diversi sistemi di riferimento inerziali è data dalle leggi
\begin{equation*}
\begin{split}\begin{cases}
  \vec{v}_{P/0} = \vec{v}_{P/1} + \vec{v}_{O_1/0} \\
  \vec{r}_{P/0} = \vec{r}_{P/1} + \vec{v}_{O_1/0} t + \vec{r}_{O_1/0} \\
\end{cases}\end{split}
\end{equation*}
\sphinxAtStartPar
che costituiscono le leggi della \sphinxstylestrong{relatività galileiana}, che legano due sistemi inerziali.

\sphinxstepscope


\section{Equazioni cardinali della dinamica}
\label{\detokenize{ch/mechanics/dynamics-eom:equazioni-cardinali-della-dinamica}}\label{\detokenize{ch/mechanics/dynamics-eom:physics-hs-mechanics-dynamics-eom}}\label{\detokenize{ch/mechanics/dynamics-eom::doc}}
\sphinxAtStartPar
Le equazioni cardinali della dinamica mettono in relazione le variazioni delle grandezze inerziali con le azioni agenti sul sistema.

\sphinxAtStartPar
Usando i princìpi della meccanica di Newton e la conservazione della massa per sistemi chiusi, è possibile ricavare le equazioni cardinali della dinamica, che governano il moto di un sistema meccanico.

\sphinxAtStartPar
Per ogni sistema chiuso le equazioni cardinali assumono la stessa forma, quando vengono espresse in termini di quantità di moto, quantità del momento angolare ed energia cinetica del sistema. Questo viene qui dimostrato per un {\hyperref[\detokenize{ch/mechanics/dynamics-eom-proof-points:physics-hs-mechanics-dynamics-eom-points}]{\sphinxcrossref{\DUrole{std,std-ref}{punto materiale}}}} per un {\hyperref[\detokenize{ch/mechanics/dynamics-eom-proof-points:physics-hs-mechanics-dynamics-eom-points}]{\sphinxcrossref{\DUrole{std,std-ref}{sistema di punti materiali}}}}, e per {\hyperref[\detokenize{ch/mechanics/dynamics-eom-proof-rigid-2d:physics-hs-mechanics-dynamics-eom-rigid-2d}]{\sphinxcrossref{\DUrole{std,std-ref}{un corpo rigido con distribuzione di massa continua in un moto piano}}}} \sphinxstylestrong{todo}, ma è valido per un sistema meccanico qualsiasi.

\sphinxAtStartPar
In particolare, per moti regolari e derivabili (e quindi senza urti impulsivi) le 3 equazioni cardinali del moto sono:
\begin{itemize}
\item {} 
\sphinxAtStartPar
\sphinxstylestrong{bilancio della quantità di moto}: la derivata nel tempo della quantità di moto di un sistema chiuso è uguale alla risultante delle forze esterne agenti sul sistema,
\begin{equation*}
\begin{split}\dot{\vec{Q}} = \vec{R}^{ext} \ ;\end{split}
\end{equation*}
\item {} 
\sphinxAtStartPar
\sphinxstylestrong{bilancio del momento della quantità di moto}: la derivata nel tempo del momento della quantità di moto di un sistema chiuso rispetto a un punto \(H\), a meno di un “termine di trasporto della quantità di moto”, è uguale alla risultante dei momenti esterni rispetto al polo \(H\)
\begin{equation*}
\begin{split}\dot{\vec{L}}_H + \dot{\vec{x}}_H \times \vec{Q} = \vec{M}^{ext}_H \ ;\end{split}
\end{equation*}
\item {} 
\sphinxAtStartPar
\sphinxstylestrong{bilancio dell’energia cinetica}: la derivata nel tempo dell’energia cinetica di un sistema chiuso è uguale all potenza totale agente sul sistema, uguale alla somma della potenza delle azioni interne e delle azioni interne al sistema,
\begin{equation*}
\begin{split}\dot{K} = P^{tot} = P^{ext} + P^{int} \ .\end{split}
\end{equation*}
\end{itemize}

\sphinxstepscope

\begin{sphinxuseclass}{sd-container-fluid}
\begin{sphinxuseclass}{sd-sphinx-override}
\begin{sphinxuseclass}{sd-p-0}
\begin{sphinxuseclass}{sd-mt-2}
\begin{sphinxuseclass}{sd-mb-4}
\begin{sphinxuseclass}{sd-row}
\begin{sphinxuseclass}{sd-row-cols-2}
\begin{sphinxuseclass}{sd-gx-2}
\begin{sphinxuseclass}{sd-gy-1}
\begin{sphinxuseclass}{sd-col}
\begin{sphinxuseclass}{sd-d-flex-row}
\begin{sphinxuseclass}{sd-align-minor-center}
\begin{sphinxuseclass}{sd-container-fluid}
\begin{sphinxuseclass}{sd-sphinx-override}
\begin{sphinxuseclass}{sd-row}
\begin{sphinxuseclass}{sd-row-cols-2}
\begin{sphinxuseclass}{sd-row-cols-xs-2}
\begin{sphinxuseclass}{sd-row-cols-sm-3}
\begin{sphinxuseclass}{sd-row-cols-md-3}
\begin{sphinxuseclass}{sd-row-cols-lg-3}
\begin{sphinxuseclass}{sd-gx-3}
\begin{sphinxuseclass}{sd-gy-1}
\begin{sphinxuseclass}{sd-col}
\begin{sphinxuseclass}{sd-col-auto}
\begin{sphinxuseclass}{sd-d-flex-row}
\begin{sphinxuseclass}{sd-align-minor-center}
\sphinxAtStartPar
basics

\end{sphinxuseclass}
\end{sphinxuseclass}
\end{sphinxuseclass}
\end{sphinxuseclass}
\begin{sphinxuseclass}{sd-col}
\begin{sphinxuseclass}{sd-col-auto}
\begin{sphinxuseclass}{sd-d-flex-row}
\begin{sphinxuseclass}{sd-align-minor-center}
\sphinxAtStartPar
Nov 06, 2024

\end{sphinxuseclass}
\end{sphinxuseclass}
\end{sphinxuseclass}
\end{sphinxuseclass}
\begin{sphinxuseclass}{sd-col}
\begin{sphinxuseclass}{sd-col-auto}
\begin{sphinxuseclass}{sd-d-flex-row}
\begin{sphinxuseclass}{sd-align-minor-center}
\sphinxAtStartPar
0 min read

\end{sphinxuseclass}
\end{sphinxuseclass}
\end{sphinxuseclass}
\end{sphinxuseclass}
\end{sphinxuseclass}
\end{sphinxuseclass}
\end{sphinxuseclass}
\end{sphinxuseclass}
\end{sphinxuseclass}
\end{sphinxuseclass}
\end{sphinxuseclass}
\end{sphinxuseclass}
\end{sphinxuseclass}
\end{sphinxuseclass}
\end{sphinxuseclass}
\end{sphinxuseclass}
\end{sphinxuseclass}
\end{sphinxuseclass}
\end{sphinxuseclass}
\end{sphinxuseclass}
\end{sphinxuseclass}
\end{sphinxuseclass}
\end{sphinxuseclass}
\end{sphinxuseclass}
\end{sphinxuseclass}
\end{sphinxuseclass}

\subsection{Equazioni cardinali della dinamica per un punto}
\label{\detokenize{ch/mechanics/dynamics-eom-proof-point:equazioni-cardinali-della-dinamica-per-un-punto}}\label{\detokenize{ch/mechanics/dynamics-eom-proof-point:physics-hs-mechanics-dynamics-eom-point}}\label{\detokenize{ch/mechanics/dynamics-eom-proof-point::doc}}
\sphinxAtStartPar
Le equazioni cardinali della dinamica in forma differenziale,
\begin{equation*}
\begin{split}\begin{aligned}
 \dot{\vec{Q}} & = \vec{R}^{ext} & \text{(bilancio quantità di moto)} \\
 \dot{\vec{L}}_H + \dot{\vec{x}}_H \times \vec{Q} & = \vec{M}_H^{ext} & \text{(bilancio momento della quantità di moto)} \\
 \dot{K} & = P^{tot} & \text{(bilancio energia cinetica)} \ .
\end{aligned}\end{split}
\end{equation*}
\sphinxAtStartPar
vengono ricavate per un sistema puntiforme calcolando la derivata nel tempo delle grandezze dinamiche di un punto,
\begin{equation*}
\begin{split}\begin{aligned}
  \vec{Q}_P & := m_P \vec{v}_P  & \text{(quantità di moto)} \\
  \vec{L}_{P,H} & := (\vec{r}_P - \vec{r}_H) \times \vec{Q} = m_P (\vec{r}_P - \vec{r}_H) \times \vec{v}_P & \text{(momento della quantità di moto)} \\
  K & := \frac{1}{2} m_P \vec{v}_P \cdot \vec{v}_P = \frac{1}{2} m_P |\vec{v}_P|^2 & \text{(energia cinetica)}
\end{aligned}\end{split}
\end{equation*}
\sphinxAtStartPar
utilizzando i princìpi della dinamica.
\subsubsection*{Bilancio della quantità di moto}

\sphinxAtStartPar
Il bilancio della quantità di moto di un punto materiale \(P\), \(\vec{Q}_P = m \vec{v}_P\) segue direttamente dal secondo principio della dinamica di Newton,
\begin{equation*}
\begin{split}\dot{\vec{Q}}_P = \vec{R}^{ext}_P\end{split}
\end{equation*}\subsubsection*{Bilancio del momento della quantità di moto}

\sphinxAtStartPar
La derivata nel tempo del momento della quantità di moto viene calcolata usando la regola del prodotto,
\begin{equation*}
\begin{split}\begin{aligned}
\dot{\vec{L}}_{P,H} & = \dfrac{d}{dt} \left[ m_P (\vec{r}_P - \vec{r}_H) \times \vec{v}_P \right] = \\
& = m \left[ ( \dot{\vec{r}}_P - \dot{\vec{r}}_H ) \times \vec{v}_P + m_P (\vec{r}_P - \vec{r}_H) \times \dot{\vec{v}}_P \right] = \\
& = - m_P \dot{\vec{r}}_H \times \vec{v}_P + m_P (\vec{r}_P - \vec{r}_H) \times \dot{\vec{v}}_P = \\
& = - \dot{\vec{r}}_H \times \vec{Q} + \vec{M}_H^{ext} \ .
\end{aligned}\end{split}
\end{equation*}\subsubsection*{Bilancio dell’energia cinetica.}
\begin{equation*}
\begin{split}\begin{aligned}
\dot{K}_{P} & = \dfrac{d}{dt} \left( \frac{1}{2} m_P \vec{v}_P \cdot \vec{v}_P \right) = \\
            & = m_P \dot{\vec{v}}_P \cdot \vec{v}_P = \\
            & = \vec{R}^{ext} \cdot \vec{v}_P = \\
            & = \vec{R}^{tot} \cdot \vec{v}_P = P^{tot} \ .
\end{aligned}\end{split}
\end{equation*}
\sphinxstepscope

\begin{sphinxuseclass}{sd-container-fluid}
\begin{sphinxuseclass}{sd-sphinx-override}
\begin{sphinxuseclass}{sd-p-0}
\begin{sphinxuseclass}{sd-mt-2}
\begin{sphinxuseclass}{sd-mb-4}
\begin{sphinxuseclass}{sd-row}
\begin{sphinxuseclass}{sd-row-cols-2}
\begin{sphinxuseclass}{sd-gx-2}
\begin{sphinxuseclass}{sd-gy-1}
\begin{sphinxuseclass}{sd-col}
\begin{sphinxuseclass}{sd-d-flex-row}
\begin{sphinxuseclass}{sd-align-minor-center}
\begin{sphinxuseclass}{sd-container-fluid}
\begin{sphinxuseclass}{sd-sphinx-override}
\begin{sphinxuseclass}{sd-row}
\begin{sphinxuseclass}{sd-row-cols-2}
\begin{sphinxuseclass}{sd-row-cols-xs-2}
\begin{sphinxuseclass}{sd-row-cols-sm-3}
\begin{sphinxuseclass}{sd-row-cols-md-3}
\begin{sphinxuseclass}{sd-row-cols-lg-3}
\begin{sphinxuseclass}{sd-gx-3}
\begin{sphinxuseclass}{sd-gy-1}
\begin{sphinxuseclass}{sd-col}
\begin{sphinxuseclass}{sd-col-auto}
\begin{sphinxuseclass}{sd-d-flex-row}
\begin{sphinxuseclass}{sd-align-minor-center}
\sphinxAtStartPar
basics

\end{sphinxuseclass}
\end{sphinxuseclass}
\end{sphinxuseclass}
\end{sphinxuseclass}
\begin{sphinxuseclass}{sd-col}
\begin{sphinxuseclass}{sd-col-auto}
\begin{sphinxuseclass}{sd-d-flex-row}
\begin{sphinxuseclass}{sd-align-minor-center}
\sphinxAtStartPar
Nov 06, 2024

\end{sphinxuseclass}
\end{sphinxuseclass}
\end{sphinxuseclass}
\end{sphinxuseclass}
\begin{sphinxuseclass}{sd-col}
\begin{sphinxuseclass}{sd-col-auto}
\begin{sphinxuseclass}{sd-d-flex-row}
\begin{sphinxuseclass}{sd-align-minor-center}
\sphinxAtStartPar
2 min read

\end{sphinxuseclass}
\end{sphinxuseclass}
\end{sphinxuseclass}
\end{sphinxuseclass}
\end{sphinxuseclass}
\end{sphinxuseclass}
\end{sphinxuseclass}
\end{sphinxuseclass}
\end{sphinxuseclass}
\end{sphinxuseclass}
\end{sphinxuseclass}
\end{sphinxuseclass}
\end{sphinxuseclass}
\end{sphinxuseclass}
\end{sphinxuseclass}
\end{sphinxuseclass}
\end{sphinxuseclass}
\end{sphinxuseclass}
\end{sphinxuseclass}
\end{sphinxuseclass}
\end{sphinxuseclass}
\end{sphinxuseclass}
\end{sphinxuseclass}
\end{sphinxuseclass}
\end{sphinxuseclass}
\end{sphinxuseclass}

\subsection{Equazioni cardinali della dinamica per sistemi di punti}
\label{\detokenize{ch/mechanics/dynamics-eom-proof-points:equazioni-cardinali-della-dinamica-per-sistemi-di-punti}}\label{\detokenize{ch/mechanics/dynamics-eom-proof-points:physics-hs-mechanics-dynamics-eom-points}}\label{\detokenize{ch/mechanics/dynamics-eom-proof-points::doc}}
\sphinxAtStartPar
Partendo dalle equazioni dinamiche per un punto, si ricavano le equazioni dinamiche per un sistema di punti,
\begin{equation*}
\begin{split}\begin{aligned}
 \dot{\vec{Q}} & = \vec{R}^{ext} & \text{(bilancio quantità di moto)} \\
 \dot{\vec{L}}_H + \dot{\vec{x}}_H \times \vec{Q} & = \vec{M}_H^{ext} & \text{(bilancio momento della quantità di moto)} \\
 \dot{K} & = P^{tot} & \text{(bilancio energia cinetica)} \ .
\end{aligned}\end{split}
\end{equation*}
\sphinxAtStartPar
sfruttando il terzo principio della dinamica di azione/reazione. Lo sviluppo delle equazioni permette di comprendere l’origine della natura additiva delle grandezze dinamiche di sistemi composti da più componenti,
\begin{equation*}
\begin{split}\begin{aligned}
\vec{Q}     & = \sum_i \vec{Q}_i     & \text{(quantità di moto)}\\
\vec{L}_{H} & = \sum_i \vec{L}_{H,i} & \text{(momento della quantità di moto)}\\
 K          & = \sum_i K_i           & \text{(energia cinetica)} \ .
\end{aligned}\end{split}
\end{equation*}
\sphinxAtStartPar
(quantità di moto, momento della quantità di moto, energia cinetica),
\subsubsection*{Bilancio della quantità di moto.}

\sphinxAtStartPar
E’ possibile scrivere il bilancio della quantità di moto per ogni punto \(i\) del sistema, scrivendo la risultante delle forze esterne agente sul punto come la somma delle forze esterne all’intero sistema agenti sul punto e le forze interne scambiate con gli altri punti del sistema,
\begin{equation*}
\begin{split}\vec{R}_i^{ext,i} = \vec{F}_i^{ext} + \sum_{j \ne i} \vec{F}_{ij} \ .\end{split}
\end{equation*}
\sphinxAtStartPar
L’equazione di bilancio per la \(i\)\sphinxhyphen{}esima massa diventa quindi
\begin{equation*}
\begin{split}\dot{\vec{Q}}_i = \vec{R}_i^{ext,i} = \vec{F}_i^{ext} + \sum_{j \ne i} \vec{F}_{ij} \ .\end{split}
\end{equation*}
\sphinxAtStartPar
Sommando le equazioni di bilancio di tutte le masse, si ottiene
\begin{equation*}
\begin{split}\begin{aligned}
\sum_{i} \dot{\vec{Q}}_i & = \sum_i \vec{F}_{i}^{ext} + \sum_i \sum_{j \ne i} \vec{F}_{ij} = \\
                            & = \sum_i \vec{F}_{i}^{ext} + \sum_{\{i,j\}} \underbrace{\left( \vec{F}_{ij} + \vec{F}_{ji} \right)}_{=\vec{0}} 
\end{aligned}\end{split}
\end{equation*}
\sphinxAtStartPar
e definendo la quantità di moto di un sistema come la somma delle quantità di moto delle sue parti e la risultante delle forze esterne come somma delle forze esterne agenti sulle parti del sistema,
\begin{equation*}
\begin{split}\vec{Q} := \sum_i \vec{Q}\end{split}
\end{equation*}\begin{equation*}
\begin{split}\vec{R}^{ext} := \sum_i \vec{F}_i^{ext}\end{split}
\end{equation*}
\sphinxAtStartPar
si ritrova la forma generale del bilancio della quantità di moto,
\begin{equation*}
\begin{split}\dot{\vec{Q}} = \vec{R}^{ext} \ .\end{split}
\end{equation*}\subsubsection*{Bilancio del momento della quantità di moto}

\sphinxAtStartPar
E’ possibile scrivere il bilancio del momento della quantità di moto per ogni punto \(i\) del sistema, scrivendo la risultante dei momenti esterni agente sul punto come la somma dei momenti esterni all’intero sistema agenti sul punto e i momenti interni scambiati con gli altri punti del sistema,
\begin{equation*}
\begin{split}\vec{M}_{H,i}^{ext,i} = \vec{M}_{H,i}^{ext} + \sum_{j \ne i} \vec{M}_{H,ij} \ .\end{split}
\end{equation*}
\sphinxAtStartPar
Nel caso le parti del sistema interagiscano tramite forze, il momento rispetto al polo \(H\) generato dalla massa \(j\) sulla massa \(i\) vale
\begin{equation*}
\begin{split}\vec{M}_{H,ij} = (\vec{r}_i - \vec{r}_H) \times \vec{F}_{ij} \ .\end{split}
\end{equation*}
\sphinxAtStartPar
L’equazione di bilancio per la \(i\)\sphinxhyphen{}esima massa diventa quindi
\begin{equation*}
\begin{split}\dot{\vec{L}}_{H,i} + \dot{\vec{r}}_H \times \vec{Q}_i = \vec{M}_{H,i}^{ext,i} = \vec{M}_{H,i}^{ext} + \sum_{j \ne i} \vec{M}_{H,ij} \ .\end{split}
\end{equation*}
\sphinxAtStartPar
Sommando le equazioni di bilancio di tutte le masse, si ottiene
\begin{equation*}
\begin{split}\begin{aligned}
\sum_{i} \left( \dot{\vec{L}}_i + \dot{\vec{r}}_H \times \vec{Q}_i \right) & = \sum_i \vec{M}_{H,i}^{ext} + \sum_i \sum_{j \ne i} \vec{M}_{H,ij} = \\
                            & = \sum_i \vec{M}_{H,i}^{ext} + \sum_{\{i,j\}} \underbrace{\left( \vec{M}_{H,ij} + \vec{M}_{H,ji} \right)}_{=\vec{0}} 
\end{aligned}\end{split}
\end{equation*}
\sphinxAtStartPar
e riconoscendo la quantità di moto del sistema e definendo il momento della quantità di moto di un sistema come la somma del momento della quantità di moto delle sue parti e la risultante dei momenti esterni come somma dei momenti esterni agenti sulle parti del sistema,
\begin{equation*}
\begin{split}\vec{L}_H := \sum_i \vec{L}_{H,i}\end{split}
\end{equation*}\begin{equation*}
\begin{split}\vec{M}_H^e := \sum_i \vec{M}_{H,i}^{ext}\end{split}
\end{equation*}
\sphinxAtStartPar
si ritrova la forma generale del bilancio del momento della quantità di moto,
\begin{equation*}
\begin{split}\dot{\vec{L}}_{H} + \dot{\vec{r}}_H \times \vec{Q} = \vec{M}_H^{ext} \ .\end{split}
\end{equation*}\subsubsection*{Bilancio dell’energia cinetica.}

\sphinxAtStartPar
E’ possibile ricavare il bilancio dell’energia cinetica del sistema, moltiplicando scalarmente il bilancio della quantità di moto di ogni punto,
\begin{equation*}
\begin{split}\vec{v}_i \cdot m_i \dot{\vec{v}}_i = \vec{v}_i \cdot \left( \vec{F}_i^{e} + \sum_{j \ne i} \vec{F}_{ij} \right) \ ,\end{split}
\end{equation*}
\sphinxAtStartPar
riconoscendo nel primo termine la derivata nel tempo dell’energia cinetica dell’\(i\)\sphinxhyphen{}esimo punto,
\begin{equation*}
\begin{split}\dot{K}_i = \dfrac{d}{dt} \left( \frac{1}{2} m_i \vec{v}_i \cdot \vec{v}_i \right) = m_i \vec{v}_i \cdot \dot{\vec{v}}_i \ ,\end{split}
\end{equation*}
\sphinxAtStartPar
e sommando queste equazioni di bilancio per ottenere
\begin{equation*}
\begin{split}\begin{aligned}
  \sum_i \dot{K}_i = \sum_i \vec{v}_i \cdot  \vec{F}_i^{e} + \sum_i \vec{v}_i \cdot \sum_{j \ne i} \vec{F}_{ij} \ . 
\end{aligned}\end{split}
\end{equation*}
\sphinxAtStartPar
Definendo l’energia cinetica di un sistema come la somma dell’energia cinetica delle sue parti, e definendo la potenza delle forze esterne/interne agenti sul sistema come la somma della potenza di tutte le forze esterne/interni al sistema,
\begin{equation*}
\begin{split}K :=  \sum_i K_i\end{split}
\end{equation*}\begin{equation*}
\begin{split}P^e := \sum_i P^{ext}_i = \sum_i \vec{v}_i \cdot  \vec{F}_i^{ext} \end{split}
\end{equation*}\begin{equation*}
\begin{split}P^i := \sum_i P^{int}_i = \sum_i \vec{v}_i \cdot \sum_{j \ne i} \vec{F}_{ij}\end{split}
\end{equation*}
\sphinxAtStartPar
si ritrova la forma generale del bilancio dell’energia cinetica,
\begin{equation*}
\begin{split}\dot{K} = P^{ext} + P^{int} = P^{tot} \ .\end{split}
\end{equation*}
\sphinxstepscope


\subsection{Equazioni cardinali della dinamica per un corpo rigido in moto piano}
\label{\detokenize{ch/mechanics/dynamics-eom-proof-rigid-2d:equazioni-cardinali-della-dinamica-per-un-corpo-rigido-in-moto-piano}}\label{\detokenize{ch/mechanics/dynamics-eom-proof-rigid-2d:physics-hs-mechanics-dynamics-eom-rigid-2d}}\label{\detokenize{ch/mechanics/dynamics-eom-proof-rigid-2d::doc}}
\sphinxAtStartPar
\sphinxstylestrong{todo}

\sphinxstepscope

\begin{sphinxuseclass}{sd-container-fluid}
\begin{sphinxuseclass}{sd-sphinx-override}
\begin{sphinxuseclass}{sd-p-0}
\begin{sphinxuseclass}{sd-mt-2}
\begin{sphinxuseclass}{sd-mb-4}
\begin{sphinxuseclass}{sd-row}
\begin{sphinxuseclass}{sd-row-cols-2}
\begin{sphinxuseclass}{sd-gx-2}
\begin{sphinxuseclass}{sd-gy-1}
\begin{sphinxuseclass}{sd-col}
\begin{sphinxuseclass}{sd-d-flex-row}
\begin{sphinxuseclass}{sd-align-minor-center}
\begin{sphinxuseclass}{sd-container-fluid}
\begin{sphinxuseclass}{sd-sphinx-override}
\begin{sphinxuseclass}{sd-row}
\begin{sphinxuseclass}{sd-row-cols-2}
\begin{sphinxuseclass}{sd-row-cols-xs-2}
\begin{sphinxuseclass}{sd-row-cols-sm-3}
\begin{sphinxuseclass}{sd-row-cols-md-3}
\begin{sphinxuseclass}{sd-row-cols-lg-3}
\begin{sphinxuseclass}{sd-gx-3}
\begin{sphinxuseclass}{sd-gy-1}
\begin{sphinxuseclass}{sd-col}
\begin{sphinxuseclass}{sd-col-auto}
\begin{sphinxuseclass}{sd-d-flex-row}
\begin{sphinxuseclass}{sd-align-minor-center}
\sphinxAtStartPar
basics

\end{sphinxuseclass}
\end{sphinxuseclass}
\end{sphinxuseclass}
\end{sphinxuseclass}
\begin{sphinxuseclass}{sd-col}
\begin{sphinxuseclass}{sd-col-auto}
\begin{sphinxuseclass}{sd-d-flex-row}
\begin{sphinxuseclass}{sd-align-minor-center}
\sphinxAtStartPar
Nov 06, 2024

\end{sphinxuseclass}
\end{sphinxuseclass}
\end{sphinxuseclass}
\end{sphinxuseclass}
\begin{sphinxuseclass}{sd-col}
\begin{sphinxuseclass}{sd-col-auto}
\begin{sphinxuseclass}{sd-d-flex-row}
\begin{sphinxuseclass}{sd-align-minor-center}
\sphinxAtStartPar
1 min read

\end{sphinxuseclass}
\end{sphinxuseclass}
\end{sphinxuseclass}
\end{sphinxuseclass}
\end{sphinxuseclass}
\end{sphinxuseclass}
\end{sphinxuseclass}
\end{sphinxuseclass}
\end{sphinxuseclass}
\end{sphinxuseclass}
\end{sphinxuseclass}
\end{sphinxuseclass}
\end{sphinxuseclass}
\end{sphinxuseclass}
\end{sphinxuseclass}
\end{sphinxuseclass}
\end{sphinxuseclass}
\end{sphinxuseclass}
\end{sphinxuseclass}
\end{sphinxuseclass}
\end{sphinxuseclass}
\end{sphinxuseclass}
\end{sphinxuseclass}
\end{sphinxuseclass}
\end{sphinxuseclass}
\end{sphinxuseclass}

\section{Leggi di conservazione}
\label{\detokenize{ch/mechanics/dynamics-conservation:leggi-di-conservazione}}\label{\detokenize{ch/mechanics/dynamics-conservation:physics-hs-mechanics-dynamics-conservation}}\label{\detokenize{ch/mechanics/dynamics-conservation::doc}}
\sphinxAtStartPar
Partendo dalle equazioni di bilancio,
\begin{equation*}
\begin{split}\begin{aligned}
 \dot{\vec{Q}} & = \vec{R}^{ext} & \text{(bilancio quantità di moto)} \\
 \dot{\vec{L}}_H + \dot{\vec{x}}_H \times \vec{Q} & = \vec{M}_H^{ext} & \text{(bilancio momento della quantità di moto)} \\
 \dot{K} & = P^{tot} & \text{(bilancio energia cinetica)} \ .
\end{aligned}\end{split}
\end{equation*}
\sphinxAtStartPar
sotto opportune ipotesi, si ottengono alcune leggi di conservazione di quantità meccaniche.

\sphinxAtStartPar
\sphinxstylestrong{Conservazione della quantità di moto.}
L’equazione di bilancio della quantità di moto di un sistema chiuso garantisce che la quantità di moto di un sistema chiuso è costante se la risultante delle forze esterne sul sistema è nulla,
\begin{equation*}
\begin{split}
  \vec{R}^{ext} = \vec{0} \qquad  \rightarrow \qquad \ \ \vec{Q} = \text{const.} 
\end{split}
\end{equation*}
\sphinxAtStartPar
\sphinxstylestrong{Conservazione del momento della quantità di moto.}
L’equazione di bilancio del momento della quantità di moto di un sistema chiuso garantisce che il momento della quantità di moto di un sistema chiuso è costante se la risultante dei momenti esterni sul sistema è nulla, ed è nullo il termine di trasporto,
\begin{equation*}
\begin{split}
  \vec{M}_H^{ext} = \vec{0}, \dot{\vec{x}}_H \times \vec{Q} = \vec{0} \qquad  \rightarrow \qquad \vec{L}_H = \text{const.}
\end{split}
\end{equation*}
\sphinxAtStartPar
\sphinxstylestrong{Conservazione del momento dell’energia cinetica.}
L’equazione di bilancio dell’energia cinetica di un sistema chiuso garantisce che il momento della quantità di moto di un sistema chiuso è costante se la risultante della potenza di tutte le azioni agenti sul sistema è nulla,
\begin{equation*}
\begin{split}
  P^{tot} = \vec{0} \qquad  \rightarrow \qquad \ \  K = \text{const.}
\end{split}
\end{equation*}
\sphinxAtStartPar
\sphinxstylestrong{Conservazione del momento dell’energia meccanica.} Se le azioni agenti su un sistema sono conservative, la loro potenza può essere scritta come derivata nel tempo di un’energia potenziale, \(P^{tot} = - \dot{V}\). Se si definisce \sphinxstylestrong{energia meccanica} la somma dell’energia cinetica del sistema e dell’energia potenziale delle azioni agenti sul sistema, \(E^{mec} := K + V\), segue immediatamente che, in assenza di azioni non\sphinxhyphen{}conservative l’energia meccanica di un sistema è costante,
\begin{equation*}
\begin{split}P^{tot} = - \dot{V} \qquad \rightarrow \qquad E^{mec} = \text{const.}\end{split}
\end{equation*}
\sphinxstepscope


\section{Collisioni}
\label{\detokenize{ch/mechanics/dynamics-collisions:collisioni}}\label{\detokenize{ch/mechanics/dynamics-collisions:physics-hs-mechanics-dynamics-collisions}}\label{\detokenize{ch/mechanics/dynamics-collisions::doc}}
\sphinxAtStartPar
Una descrizione dettagliata delle collisioni tra sistemi qualsiasi va ben al di là dello scopo di un primo approccio alla meccanica.

\sphinxAtStartPar
Qui, ci si limiterà allo studio di collisioni che:
\begin{itemize}
\item {} 
\sphinxAtStartPar
possono essere caratterizzate unicamente da un \sphinxstyleemphasis{coefficiente di ritorno}, \(\varepsilon\) \sphinxstylestrong{todo}

\item {} 
\sphinxAtStartPar
avvengono in intervalli di tempo ridotti, al limite nulli

\end{itemize}

\sphinxAtStartPar
Questi urti comportano delle variazioni finite delle quantità dinamiche in intervalli di tempo finiti, vengono definiti \sphinxstylestrong{urti impulsivi} (\sphinxstylestrong{todo} \sphinxstyleemphasis{verificare}) e  rappresentano un esempio di moto “non regolare”, per il quale le equazioni cardinali della dinamica devono essere scritte in forma incrementale.

\sphinxAtStartPar
\sphinxstylestrong{todo} \sphinxstyleemphasis{approfondimento su forze impulsive e delta di Dirac?}

\sphinxAtStartPar
Tra due istanti temporali immediatamente precedente e immediatamente successivo all’urto tra due sistemi possono essere trascurate tutte le azioni agenti sul sistema complessivo tranne quelle \sphinxstylestrong{impulsive} dovute all’\sphinxstylestrong{urto}, e ad eventuali \sphinxstylestrong{reazioni vincolari} (vedi esercizi),
\begin{equation*}
\begin{split}\begin{aligned}
  \vec{I}^{ext}   & = \Delta \vec{Q} \\
  \vec{J}_H^{ext} & = \Delta \vec{\Gamma}_H + \Delta \vec{x}_H \times \vec{Q} = \Delta \vec{\Gamma}_{H} \\
  L^{ext} + L^{int} & = \Delta K \ ,
\end{aligned}\end{split}
\end{equation*}
\sphinxAtStartPar
con \(\vec{I}^{ext}\) l’impulso delle forze esterne durante l’urto, \(\vec{J}^{ext}\) l’impulso dei momenti esterni durante l’urto, \(L^{ext}\), \(L^{int}\) il lavoro delle forze esterne e interne durante l’urto.

\sphinxAtStartPar
E’ bene osservare che in assenza di forze e momenti impulsivi esterni \sphinxhyphen{} anche dovuti a eventuali vincoli \sphinxhyphen{} ai due sistemi che collidono, la quantità di moto e il momento della quantità di moto del sistema complessivo si conservano in un urto.
Al contrario, in generale, l’\sphinxstylestrong{energia cinetica non si conserva} poiché dipende anche dal lavoro delle azioni interne che includono quelle impulsive scambiate durante l’urto.

\sphinxAtStartPar
Il \sphinxstylestrong{coefficiente di restituzione} \(\varepsilon \in [0, 1]\) caratterizza il tipo di urto e ha una facile interpretazione se l’urto viene studiato usando un sistema di riferimento con orgine il centro di massa del sistema, \(Q\). Le quantità riferite a questo sistema vengono indicate qui con l’apice.

\sphinxAtStartPar
Poiché si è scelto come riferimento il centro di massa, in assenza di forze implusive esterne,
\begin{equation*}
\begin{split}\vec{0} = {\vec{p}^-}  = {\vec{p}^+} \end{split}
\end{equation*}\begin{equation*}
\begin{split}\vec{0} = {\vec{p}^-}  = {\vec{p}_1^-}  + {\vec{p}_2^-} \end{split}
\end{equation*}\begin{equation*}
\begin{split}\vec{0} = {\vec{p}^+}  = {\vec{p}_1^+}  + {\vec{p}_2^+} \end{split}
\end{equation*}
\sphinxAtStartPar
\sphinxstylestrong{todo} \sphinxstyleemphasis{distinguere tra componente normale e tangenziale}

\sphinxAtStartPar
Il coefficiente di restituzione viene definito come l’opposto del rapporto tra il valore assoluto (\sphinxstylestrong{todo} dovrebbe essere la componente normale, assumento che la componente tangenziale si conservi \sphinxhyphen{} oppure trovare anche un modello per la componente tangenziale, dovuta ad attrito) della quantità di moto di uno dei due corpi dopo e prima dell’urto,
\begin{equation*}
\begin{split}\varepsilon := - \frac{|{\vec{p}_1^{+ '}}   |}{|{\vec{p}_1^{- '}}  |} = - \frac{{|\vec{p}_2^{+ '}} |}{|{\vec{p}_2^{- '}} |}\end{split}
\end{equation*}
\sphinxAtStartPar
In termini di energia cinetica, nel sistema di riferimento del centro di massa
\begin{equation*}
\begin{split}\begin{aligned}
{K^{+ '}} & = \frac{1}{2 m_1} {\vec{p}_1^{+ '}}  \cdot {\vec{p}_1^{+ '}}  + \frac{1}{2 m_2} {\vec{p}_2^{+ '}}  \cdot {\vec{p}_2^{+ '}}  = \\
       & = \varepsilon^2 \left[ \frac{1}{2 m_1} {\vec{p}_1^{- '}}  \cdot {\vec{p}_1^{- '}}  + \frac{1}{2 m_2} {\vec{p}_2^{- '}}  \cdot {\vec{p}_2^{- '}}  \right] = \varepsilon^2 {K^{- '}} 
\end{aligned}\end{split}
\end{equation*}

\subsection{Problemi}
\label{\detokenize{ch/mechanics/dynamics-collisions:problemi}}


\begin{sphinxuseclass}{sd-container-fluid}
\begin{sphinxuseclass}{sd-sphinx-override}
\begin{sphinxuseclass}{sd-mb-4}
\begin{sphinxuseclass}{sd-row}
\begin{sphinxuseclass}{sd-g-2}
\begin{sphinxuseclass}{sd-g-xs-2}
\begin{sphinxuseclass}{sd-g-sm-2}
\begin{sphinxuseclass}{sd-g-md-2}
\begin{sphinxuseclass}{sd-g-lg-2}
\begin{sphinxuseclass}{sd-col}
\begin{sphinxuseclass}{sd-d-flex-row}
\begin{sphinxuseclass}{sd-col-8}
\begin{sphinxuseclass}{sd-col-xs-8}
\begin{sphinxuseclass}{sd-col-sm-8}
\begin{sphinxuseclass}{sd-col-md-8}
\begin{sphinxuseclass}{sd-col-lg-8}
\begin{sphinxuseclass}{sd-card}
\begin{sphinxuseclass}{sd-sphinx-override}
\begin{sphinxuseclass}{sd-w-100}
\begin{sphinxuseclass}{sd-shadow-sm}
\begin{sphinxuseclass}{sd-card-body}
\begin{sphinxuseclass}{sd-card-title}
\begin{sphinxuseclass}{sd-font-weight-bold}Collisione tra blocchi su piano orizzontale liscio
\end{sphinxuseclass}
\end{sphinxuseclass}
\sphinxAtStartPar
Date le masse di due blocchi che scivolano su un piano orizzontale liscio, e le velocità iniziali dei due blocchi, e il coefficiente di restituzione dell’urto, viene chiesto di determinare le velocità dei due blocchi dopo l’urto.

\end{sphinxuseclass}
\end{sphinxuseclass}
\end{sphinxuseclass}
\end{sphinxuseclass}
\end{sphinxuseclass}
\end{sphinxuseclass}
\end{sphinxuseclass}
\end{sphinxuseclass}
\end{sphinxuseclass}
\end{sphinxuseclass}
\end{sphinxuseclass}
\end{sphinxuseclass}
\begin{sphinxuseclass}{sd-col}
\begin{sphinxuseclass}{sd-d-flex-row}
\begin{sphinxuseclass}{sd-col-4}
\begin{sphinxuseclass}{sd-col-xs-4}
\begin{sphinxuseclass}{sd-col-sm-4}
\begin{sphinxuseclass}{sd-col-md-4}
\begin{sphinxuseclass}{sd-col-lg-4}
\begin{sphinxuseclass}{sd-card}
\begin{sphinxuseclass}{sd-sphinx-override}
\begin{sphinxuseclass}{sd-w-100}
\begin{sphinxuseclass}{sd-shadow-sm}
\begin{sphinxuseclass}{sd-card-body}
\sphinxAtStartPar
\sphinxincludegraphics{{collisions-1d}.png}



\end{sphinxuseclass}
\end{sphinxuseclass}
\end{sphinxuseclass}
\end{sphinxuseclass}
\end{sphinxuseclass}
\end{sphinxuseclass}
\end{sphinxuseclass}
\end{sphinxuseclass}
\end{sphinxuseclass}
\end{sphinxuseclass}
\end{sphinxuseclass}
\end{sphinxuseclass}
\end{sphinxuseclass}
\end{sphinxuseclass}
\end{sphinxuseclass}
\end{sphinxuseclass}
\end{sphinxuseclass}
\end{sphinxuseclass}
\end{sphinxuseclass}
\end{sphinxuseclass}
\end{sphinxuseclass}\subsubsection*{Soluzione.}

\sphinxAtStartPar
\sphinxstylestrong{todo}



\begin{sphinxuseclass}{sd-container-fluid}
\begin{sphinxuseclass}{sd-sphinx-override}
\begin{sphinxuseclass}{sd-mb-4}
\begin{sphinxuseclass}{sd-row}
\begin{sphinxuseclass}{sd-g-2}
\begin{sphinxuseclass}{sd-g-xs-2}
\begin{sphinxuseclass}{sd-g-sm-2}
\begin{sphinxuseclass}{sd-g-md-2}
\begin{sphinxuseclass}{sd-g-lg-2}
\begin{sphinxuseclass}{sd-col}
\begin{sphinxuseclass}{sd-d-flex-row}
\begin{sphinxuseclass}{sd-col-8}
\begin{sphinxuseclass}{sd-col-xs-8}
\begin{sphinxuseclass}{sd-col-sm-8}
\begin{sphinxuseclass}{sd-col-md-8}
\begin{sphinxuseclass}{sd-col-lg-8}
\begin{sphinxuseclass}{sd-card}
\begin{sphinxuseclass}{sd-sphinx-override}
\begin{sphinxuseclass}{sd-w-100}
\begin{sphinxuseclass}{sd-shadow-sm}
\begin{sphinxuseclass}{sd-card-body}
\begin{sphinxuseclass}{sd-card-title}
\begin{sphinxuseclass}{sd-font-weight-bold}Collisione tra blocchi su piano orizzontale scabro
\end{sphinxuseclass}
\end{sphinxuseclass}
\sphinxAtStartPar
Date le masse di due blocchi che scivolano su un piano orizzontale scabro, le velocità e la distanza iniziale tra i due blocchi, il coefficiente di restituzione dell’urto, il coefficiente di attrito dinamico \(\mu^d\) tra i due blocchi e il piano orizzontale, viene chiesto di determinare:
\begin{itemize}
\item {} 
\sphinxAtStartPar
le condizioni affinché avvenga l’urto

\item {} 
\sphinxAtStartPar
in caso di urto:
\begin{itemize}
\item {} 
\sphinxAtStartPar
le velocità immediatamente dopo l’urto

\item {} 
\sphinxAtStartPar
la posizione finale delle due masse

\end{itemize}

\end{itemize}

\end{sphinxuseclass}
\end{sphinxuseclass}
\end{sphinxuseclass}
\end{sphinxuseclass}
\end{sphinxuseclass}
\end{sphinxuseclass}
\end{sphinxuseclass}
\end{sphinxuseclass}
\end{sphinxuseclass}
\end{sphinxuseclass}
\end{sphinxuseclass}
\end{sphinxuseclass}
\begin{sphinxuseclass}{sd-col}
\begin{sphinxuseclass}{sd-d-flex-row}
\begin{sphinxuseclass}{sd-col-4}
\begin{sphinxuseclass}{sd-col-xs-4}
\begin{sphinxuseclass}{sd-col-sm-4}
\begin{sphinxuseclass}{sd-col-md-4}
\begin{sphinxuseclass}{sd-col-lg-4}
\begin{sphinxuseclass}{sd-card}
\begin{sphinxuseclass}{sd-sphinx-override}
\begin{sphinxuseclass}{sd-w-100}
\begin{sphinxuseclass}{sd-shadow-sm}
\begin{sphinxuseclass}{sd-card-body}
\sphinxAtStartPar
\sphinxincludegraphics{{collisions-1d-friction}.png}



\end{sphinxuseclass}
\end{sphinxuseclass}
\end{sphinxuseclass}
\end{sphinxuseclass}
\end{sphinxuseclass}
\end{sphinxuseclass}
\end{sphinxuseclass}
\end{sphinxuseclass}
\end{sphinxuseclass}
\end{sphinxuseclass}
\end{sphinxuseclass}
\end{sphinxuseclass}
\end{sphinxuseclass}
\end{sphinxuseclass}
\end{sphinxuseclass}
\end{sphinxuseclass}
\end{sphinxuseclass}
\end{sphinxuseclass}
\end{sphinxuseclass}
\end{sphinxuseclass}
\end{sphinxuseclass}\subsubsection*{Soluzione.}

\sphinxAtStartPar
\sphinxstylestrong{todo}



\begin{sphinxuseclass}{sd-container-fluid}
\begin{sphinxuseclass}{sd-sphinx-override}
\begin{sphinxuseclass}{sd-mb-4}
\begin{sphinxuseclass}{sd-row}
\begin{sphinxuseclass}{sd-g-2}
\begin{sphinxuseclass}{sd-g-xs-2}
\begin{sphinxuseclass}{sd-g-sm-2}
\begin{sphinxuseclass}{sd-g-md-2}
\begin{sphinxuseclass}{sd-g-lg-2}
\begin{sphinxuseclass}{sd-col}
\begin{sphinxuseclass}{sd-d-flex-row}
\begin{sphinxuseclass}{sd-col-8}
\begin{sphinxuseclass}{sd-col-xs-8}
\begin{sphinxuseclass}{sd-col-sm-8}
\begin{sphinxuseclass}{sd-col-md-8}
\begin{sphinxuseclass}{sd-col-lg-8}
\begin{sphinxuseclass}{sd-card}
\begin{sphinxuseclass}{sd-sphinx-override}
\begin{sphinxuseclass}{sd-w-100}
\begin{sphinxuseclass}{sd-shadow-sm}
\begin{sphinxuseclass}{sd-card-body}
\begin{sphinxuseclass}{sd-card-title}
\begin{sphinxuseclass}{sd-font-weight-bold}Rimbalzo di una palla
\end{sphinxuseclass}
\end{sphinxuseclass}
\sphinxAtStartPar
Dato il coefficiente di restituzione degli urti tra la palla di massa \(m_1\) nota e ilpiano orizzontale, viene chiesto di determinare la distanza verticale percorsa dalla palla durante i rimbalzi.

\sphinxAtStartPar
\sphinxstylestrong{Oss.} Il numero di rimbalzi è infinito, ma il risultato si ottiene da una serie infinita convergente.

\end{sphinxuseclass}
\end{sphinxuseclass}
\end{sphinxuseclass}
\end{sphinxuseclass}
\end{sphinxuseclass}
\end{sphinxuseclass}
\end{sphinxuseclass}
\end{sphinxuseclass}
\end{sphinxuseclass}
\end{sphinxuseclass}
\end{sphinxuseclass}
\end{sphinxuseclass}
\begin{sphinxuseclass}{sd-col}
\begin{sphinxuseclass}{sd-d-flex-row}
\begin{sphinxuseclass}{sd-col-4}
\begin{sphinxuseclass}{sd-col-xs-4}
\begin{sphinxuseclass}{sd-col-sm-4}
\begin{sphinxuseclass}{sd-col-md-4}
\begin{sphinxuseclass}{sd-col-lg-4}
\begin{sphinxuseclass}{sd-card}
\begin{sphinxuseclass}{sd-sphinx-override}
\begin{sphinxuseclass}{sd-w-100}
\begin{sphinxuseclass}{sd-shadow-sm}
\begin{sphinxuseclass}{sd-card-body}
\sphinxAtStartPar
\sphinxincludegraphics{{collisions-bouncing-ball}.png}



\end{sphinxuseclass}
\end{sphinxuseclass}
\end{sphinxuseclass}
\end{sphinxuseclass}
\end{sphinxuseclass}
\end{sphinxuseclass}
\end{sphinxuseclass}
\end{sphinxuseclass}
\end{sphinxuseclass}
\end{sphinxuseclass}
\end{sphinxuseclass}
\end{sphinxuseclass}
\end{sphinxuseclass}
\end{sphinxuseclass}
\end{sphinxuseclass}
\end{sphinxuseclass}
\end{sphinxuseclass}
\end{sphinxuseclass}
\end{sphinxuseclass}
\end{sphinxuseclass}
\end{sphinxuseclass}\subsubsection*{Soluzione.}

\sphinxAtStartPar
\sphinxstylestrong{todo}



\begin{sphinxuseclass}{sd-container-fluid}
\begin{sphinxuseclass}{sd-sphinx-override}
\begin{sphinxuseclass}{sd-mb-4}
\begin{sphinxuseclass}{sd-row}
\begin{sphinxuseclass}{sd-g-2}
\begin{sphinxuseclass}{sd-g-xs-2}
\begin{sphinxuseclass}{sd-g-sm-2}
\begin{sphinxuseclass}{sd-g-md-2}
\begin{sphinxuseclass}{sd-g-lg-2}
\begin{sphinxuseclass}{sd-col}
\begin{sphinxuseclass}{sd-d-flex-row}
\begin{sphinxuseclass}{sd-col-8}
\begin{sphinxuseclass}{sd-col-xs-8}
\begin{sphinxuseclass}{sd-col-sm-8}
\begin{sphinxuseclass}{sd-col-md-8}
\begin{sphinxuseclass}{sd-col-lg-8}
\begin{sphinxuseclass}{sd-card}
\begin{sphinxuseclass}{sd-sphinx-override}
\begin{sphinxuseclass}{sd-w-100}
\begin{sphinxuseclass}{sd-shadow-sm}
\begin{sphinxuseclass}{sd-card-body}
\begin{sphinxuseclass}{sd-card-title}
\begin{sphinxuseclass}{sd-font-weight-bold}Collisione di un sistema massa\sphinxhyphen{}molla con una parete
\end{sphinxuseclass}
\end{sphinxuseclass}
\sphinxAtStartPar
Data la configurazione iniziale del sistema massa\sphinxhyphen{}molla, con lunghezza a riposo nulla \(\ell_0\) e allungamento iniziale \(x_0\), viene chiesto di descrivere l’evoluzione del sistema in funzione del coefficiente di restituzione \(\varepsilon\) degli urti tra la massa e la parete rigida verticale. In particolare, si chiede di distinguere il caso di urto elastico dai casi di urto parzialmente elastico.

\end{sphinxuseclass}
\end{sphinxuseclass}
\end{sphinxuseclass}
\end{sphinxuseclass}
\end{sphinxuseclass}
\end{sphinxuseclass}
\end{sphinxuseclass}
\end{sphinxuseclass}
\end{sphinxuseclass}
\end{sphinxuseclass}
\end{sphinxuseclass}
\end{sphinxuseclass}
\begin{sphinxuseclass}{sd-col}
\begin{sphinxuseclass}{sd-d-flex-row}
\begin{sphinxuseclass}{sd-col-4}
\begin{sphinxuseclass}{sd-col-xs-4}
\begin{sphinxuseclass}{sd-col-sm-4}
\begin{sphinxuseclass}{sd-col-md-4}
\begin{sphinxuseclass}{sd-col-lg-4}
\begin{sphinxuseclass}{sd-card}
\begin{sphinxuseclass}{sd-sphinx-override}
\begin{sphinxuseclass}{sd-w-100}
\begin{sphinxuseclass}{sd-shadow-sm}
\begin{sphinxuseclass}{sd-card-body}
\sphinxAtStartPar
\sphinxincludegraphics{{collisions-bouncing-block-spring}.png}



\end{sphinxuseclass}
\end{sphinxuseclass}
\end{sphinxuseclass}
\end{sphinxuseclass}
\end{sphinxuseclass}
\end{sphinxuseclass}
\end{sphinxuseclass}
\end{sphinxuseclass}
\end{sphinxuseclass}
\end{sphinxuseclass}
\end{sphinxuseclass}
\end{sphinxuseclass}
\end{sphinxuseclass}
\end{sphinxuseclass}
\end{sphinxuseclass}
\end{sphinxuseclass}
\end{sphinxuseclass}
\end{sphinxuseclass}
\end{sphinxuseclass}
\end{sphinxuseclass}
\end{sphinxuseclass}\subsubsection*{Soluzione.}

\sphinxAtStartPar
\sphinxstylestrong{todo}



\begin{sphinxuseclass}{sd-container-fluid}
\begin{sphinxuseclass}{sd-sphinx-override}
\begin{sphinxuseclass}{sd-mb-4}
\begin{sphinxuseclass}{sd-row}
\begin{sphinxuseclass}{sd-g-2}
\begin{sphinxuseclass}{sd-g-xs-2}
\begin{sphinxuseclass}{sd-g-sm-2}
\begin{sphinxuseclass}{sd-g-md-2}
\begin{sphinxuseclass}{sd-g-lg-2}
\begin{sphinxuseclass}{sd-col}
\begin{sphinxuseclass}{sd-d-flex-row}
\begin{sphinxuseclass}{sd-col-8}
\begin{sphinxuseclass}{sd-col-xs-8}
\begin{sphinxuseclass}{sd-col-sm-8}
\begin{sphinxuseclass}{sd-col-md-8}
\begin{sphinxuseclass}{sd-col-lg-8}
\begin{sphinxuseclass}{sd-card}
\begin{sphinxuseclass}{sd-sphinx-override}
\begin{sphinxuseclass}{sd-w-100}
\begin{sphinxuseclass}{sd-shadow-sm}
\begin{sphinxuseclass}{sd-card-body}
\begin{sphinxuseclass}{sd-card-title}
\begin{sphinxuseclass}{sd-font-weight-bold}Collisioni tra due blocchi e una parete rigida
\end{sphinxuseclass}
\end{sphinxuseclass}
\sphinxAtStartPar
Nel caso di urti perfettamente elastici tra i due blocchi e con la parete, viene chiesto di determinare il numero di urti tra i due blocchi.

\end{sphinxuseclass}
\end{sphinxuseclass}
\end{sphinxuseclass}
\end{sphinxuseclass}
\end{sphinxuseclass}
\end{sphinxuseclass}
\end{sphinxuseclass}
\end{sphinxuseclass}
\end{sphinxuseclass}
\end{sphinxuseclass}
\end{sphinxuseclass}
\end{sphinxuseclass}
\begin{sphinxuseclass}{sd-col}
\begin{sphinxuseclass}{sd-d-flex-row}
\begin{sphinxuseclass}{sd-col-4}
\begin{sphinxuseclass}{sd-col-xs-4}
\begin{sphinxuseclass}{sd-col-sm-4}
\begin{sphinxuseclass}{sd-col-md-4}
\begin{sphinxuseclass}{sd-col-lg-4}
\begin{sphinxuseclass}{sd-card}
\begin{sphinxuseclass}{sd-sphinx-override}
\begin{sphinxuseclass}{sd-w-100}
\begin{sphinxuseclass}{sd-shadow-sm}
\begin{sphinxuseclass}{sd-card-body}
\sphinxAtStartPar
\sphinxincludegraphics{{collisions-bouncing-blocks}.png}



\end{sphinxuseclass}
\end{sphinxuseclass}
\end{sphinxuseclass}
\end{sphinxuseclass}
\end{sphinxuseclass}
\end{sphinxuseclass}
\end{sphinxuseclass}
\end{sphinxuseclass}
\end{sphinxuseclass}
\end{sphinxuseclass}
\end{sphinxuseclass}
\end{sphinxuseclass}
\end{sphinxuseclass}
\end{sphinxuseclass}
\end{sphinxuseclass}
\end{sphinxuseclass}
\end{sphinxuseclass}
\end{sphinxuseclass}
\end{sphinxuseclass}
\end{sphinxuseclass}
\end{sphinxuseclass}\subsubsection*{Soluzione.}

\sphinxAtStartPar
\sphinxstylestrong{todo}



\begin{sphinxuseclass}{sd-container-fluid}
\begin{sphinxuseclass}{sd-sphinx-override}
\begin{sphinxuseclass}{sd-mb-4}
\begin{sphinxuseclass}{sd-row}
\begin{sphinxuseclass}{sd-g-2}
\begin{sphinxuseclass}{sd-g-xs-2}
\begin{sphinxuseclass}{sd-g-sm-2}
\begin{sphinxuseclass}{sd-g-md-2}
\begin{sphinxuseclass}{sd-g-lg-2}
\begin{sphinxuseclass}{sd-col}
\begin{sphinxuseclass}{sd-d-flex-row}
\begin{sphinxuseclass}{sd-col-8}
\begin{sphinxuseclass}{sd-col-xs-8}
\begin{sphinxuseclass}{sd-col-sm-8}
\begin{sphinxuseclass}{sd-col-md-8}
\begin{sphinxuseclass}{sd-col-lg-8}
\begin{sphinxuseclass}{sd-card}
\begin{sphinxuseclass}{sd-sphinx-override}
\begin{sphinxuseclass}{sd-w-100}
\begin{sphinxuseclass}{sd-shadow-sm}
\begin{sphinxuseclass}{sd-card-body}
\begin{sphinxuseclass}{sd-card-title}
\begin{sphinxuseclass}{sd-font-weight-bold}Proiettile su pendolo con massa concentrata
\end{sphinxuseclass}
\end{sphinxuseclass}
\sphinxAtStartPar
Un proiettile colpisce un pendolo. In funzione del coefficiente di restituzione \(\varepsilon\), viene chiesto di determinare:
\begin{itemize}
\item {} 
\sphinxAtStartPar
le condizioni immediatamente successive all’urto

\item {} 
\sphinxAtStartPar
l’angolo massimo raggiunto dal pendolo

\end{itemize}

\sphinxAtStartPar
Si calcolino poi le reazioni vincolari a terra, prima, durante e dopo l’urto.

\end{sphinxuseclass}
\end{sphinxuseclass}
\end{sphinxuseclass}
\end{sphinxuseclass}
\end{sphinxuseclass}
\end{sphinxuseclass}
\end{sphinxuseclass}
\end{sphinxuseclass}
\end{sphinxuseclass}
\end{sphinxuseclass}
\end{sphinxuseclass}
\end{sphinxuseclass}
\begin{sphinxuseclass}{sd-col}
\begin{sphinxuseclass}{sd-d-flex-row}
\begin{sphinxuseclass}{sd-col-4}
\begin{sphinxuseclass}{sd-col-xs-4}
\begin{sphinxuseclass}{sd-col-sm-4}
\begin{sphinxuseclass}{sd-col-md-4}
\begin{sphinxuseclass}{sd-col-lg-4}
\begin{sphinxuseclass}{sd-card}
\begin{sphinxuseclass}{sd-sphinx-override}
\begin{sphinxuseclass}{sd-w-100}
\begin{sphinxuseclass}{sd-shadow-sm}
\begin{sphinxuseclass}{sd-card-body}
\sphinxAtStartPar
\sphinxincludegraphics{{collisions-pendulum-0}.png}



\end{sphinxuseclass}
\end{sphinxuseclass}
\end{sphinxuseclass}
\end{sphinxuseclass}
\end{sphinxuseclass}
\end{sphinxuseclass}
\end{sphinxuseclass}
\end{sphinxuseclass}
\end{sphinxuseclass}
\end{sphinxuseclass}
\end{sphinxuseclass}
\end{sphinxuseclass}
\end{sphinxuseclass}
\end{sphinxuseclass}
\end{sphinxuseclass}
\end{sphinxuseclass}
\end{sphinxuseclass}
\end{sphinxuseclass}
\end{sphinxuseclass}
\end{sphinxuseclass}
\end{sphinxuseclass}\subsubsection*{Soluzione.}

\sphinxAtStartPar
\sphinxstylestrong{todo}



\begin{sphinxuseclass}{sd-container-fluid}
\begin{sphinxuseclass}{sd-sphinx-override}
\begin{sphinxuseclass}{sd-mb-4}
\begin{sphinxuseclass}{sd-row}
\begin{sphinxuseclass}{sd-g-2}
\begin{sphinxuseclass}{sd-g-xs-2}
\begin{sphinxuseclass}{sd-g-sm-2}
\begin{sphinxuseclass}{sd-g-md-2}
\begin{sphinxuseclass}{sd-g-lg-2}
\begin{sphinxuseclass}{sd-col}
\begin{sphinxuseclass}{sd-d-flex-row}
\begin{sphinxuseclass}{sd-col-8}
\begin{sphinxuseclass}{sd-col-xs-8}
\begin{sphinxuseclass}{sd-col-sm-8}
\begin{sphinxuseclass}{sd-col-md-8}
\begin{sphinxuseclass}{sd-col-lg-8}
\begin{sphinxuseclass}{sd-card}
\begin{sphinxuseclass}{sd-sphinx-override}
\begin{sphinxuseclass}{sd-w-100}
\begin{sphinxuseclass}{sd-shadow-sm}
\begin{sphinxuseclass}{sd-card-body}
\begin{sphinxuseclass}{sd-card-title}
\begin{sphinxuseclass}{sd-font-weight-bold}Proiettile su pendolo con massa distribuita
\end{sphinxuseclass}
\end{sphinxuseclass}
\sphinxAtStartPar
Un proiettile colpisce un pendolo. In funzione del coefficiente di restituzione \(\varepsilon\), viene chiesto di determinare:
\begin{itemize}
\item {} 
\sphinxAtStartPar
le condizioni immediatamente successive all’urto

\item {} 
\sphinxAtStartPar
l’angolo massimo raggiunto dal pendolo.

\end{itemize}

\sphinxAtStartPar
Si calcolino poi le reazioni vincolari a terra, prima, durante e dopo l’urto.

\end{sphinxuseclass}
\end{sphinxuseclass}
\end{sphinxuseclass}
\end{sphinxuseclass}
\end{sphinxuseclass}
\end{sphinxuseclass}
\end{sphinxuseclass}
\end{sphinxuseclass}
\end{sphinxuseclass}
\end{sphinxuseclass}
\end{sphinxuseclass}
\end{sphinxuseclass}
\begin{sphinxuseclass}{sd-col}
\begin{sphinxuseclass}{sd-d-flex-row}
\begin{sphinxuseclass}{sd-col-4}
\begin{sphinxuseclass}{sd-col-xs-4}
\begin{sphinxuseclass}{sd-col-sm-4}
\begin{sphinxuseclass}{sd-col-md-4}
\begin{sphinxuseclass}{sd-col-lg-4}
\begin{sphinxuseclass}{sd-card}
\begin{sphinxuseclass}{sd-sphinx-override}
\begin{sphinxuseclass}{sd-w-100}
\begin{sphinxuseclass}{sd-shadow-sm}
\begin{sphinxuseclass}{sd-card-body}
\sphinxAtStartPar
\sphinxincludegraphics{{collisions-pendulum-1}.png}



\end{sphinxuseclass}
\end{sphinxuseclass}
\end{sphinxuseclass}
\end{sphinxuseclass}
\end{sphinxuseclass}
\end{sphinxuseclass}
\end{sphinxuseclass}
\end{sphinxuseclass}
\end{sphinxuseclass}
\end{sphinxuseclass}
\end{sphinxuseclass}
\end{sphinxuseclass}
\end{sphinxuseclass}
\end{sphinxuseclass}
\end{sphinxuseclass}
\end{sphinxuseclass}
\end{sphinxuseclass}
\end{sphinxuseclass}
\end{sphinxuseclass}
\end{sphinxuseclass}
\end{sphinxuseclass}\subsubsection*{Soluzione.}

\sphinxAtStartPar
\sphinxstylestrong{todo}



\begin{sphinxuseclass}{sd-container-fluid}
\begin{sphinxuseclass}{sd-sphinx-override}
\begin{sphinxuseclass}{sd-mb-4}
\begin{sphinxuseclass}{sd-row}
\begin{sphinxuseclass}{sd-g-2}
\begin{sphinxuseclass}{sd-g-xs-2}
\begin{sphinxuseclass}{sd-g-sm-2}
\begin{sphinxuseclass}{sd-g-md-2}
\begin{sphinxuseclass}{sd-g-lg-2}
\begin{sphinxuseclass}{sd-col}
\begin{sphinxuseclass}{sd-d-flex-row}
\begin{sphinxuseclass}{sd-col-8}
\begin{sphinxuseclass}{sd-col-xs-8}
\begin{sphinxuseclass}{sd-col-sm-8}
\begin{sphinxuseclass}{sd-col-md-8}
\begin{sphinxuseclass}{sd-col-lg-8}
\begin{sphinxuseclass}{sd-card}
\begin{sphinxuseclass}{sd-sphinx-override}
\begin{sphinxuseclass}{sd-w-100}
\begin{sphinxuseclass}{sd-shadow-sm}
\begin{sphinxuseclass}{sd-card-body}
\begin{sphinxuseclass}{sd-card-title}
\begin{sphinxuseclass}{sd-font-weight-bold}Proiettile su bersaglio di poligono di tiro
\end{sphinxuseclass}
\end{sphinxuseclass}
\sphinxAtStartPar
Un proiettile colpisce il bersaglio di un poligono, inizialmente appoggiato alla parete verticale. In funzione del coefficiente di restituzione \(\varepsilon\), viene chiesto di determinare:
\begin{itemize}
\item {} 
\sphinxAtStartPar
le condizioni immediatamente successive all’urto

\item {} 
\sphinxAtStartPar
la velocità minima del proiettile prima dell’urto che garantisce di abbattere il bersaglio.

\end{itemize}

\sphinxAtStartPar
Si calcolino poi le reazioni vincolari a terra, prima, durante e dopo l’urto.

\end{sphinxuseclass}
\end{sphinxuseclass}
\end{sphinxuseclass}
\end{sphinxuseclass}
\end{sphinxuseclass}
\end{sphinxuseclass}
\end{sphinxuseclass}
\end{sphinxuseclass}
\end{sphinxuseclass}
\end{sphinxuseclass}
\end{sphinxuseclass}
\end{sphinxuseclass}
\begin{sphinxuseclass}{sd-col}
\begin{sphinxuseclass}{sd-d-flex-row}
\begin{sphinxuseclass}{sd-col-4}
\begin{sphinxuseclass}{sd-col-xs-4}
\begin{sphinxuseclass}{sd-col-sm-4}
\begin{sphinxuseclass}{sd-col-md-4}
\begin{sphinxuseclass}{sd-col-lg-4}
\begin{sphinxuseclass}{sd-card}
\begin{sphinxuseclass}{sd-sphinx-override}
\begin{sphinxuseclass}{sd-w-100}
\begin{sphinxuseclass}{sd-shadow-sm}
\begin{sphinxuseclass}{sd-card-body}
\sphinxAtStartPar
\sphinxincludegraphics{{collisions-pendulum-2}.png}



\end{sphinxuseclass}
\end{sphinxuseclass}
\end{sphinxuseclass}
\end{sphinxuseclass}
\end{sphinxuseclass}
\end{sphinxuseclass}
\end{sphinxuseclass}
\end{sphinxuseclass}
\end{sphinxuseclass}
\end{sphinxuseclass}
\end{sphinxuseclass}
\end{sphinxuseclass}
\end{sphinxuseclass}
\end{sphinxuseclass}
\end{sphinxuseclass}
\end{sphinxuseclass}
\end{sphinxuseclass}
\end{sphinxuseclass}
\end{sphinxuseclass}
\end{sphinxuseclass}
\end{sphinxuseclass}\subsubsection*{Soluzione.}

\sphinxAtStartPar
\sphinxstylestrong{todo}



\begin{sphinxuseclass}{sd-container-fluid}
\begin{sphinxuseclass}{sd-sphinx-override}
\begin{sphinxuseclass}{sd-mb-4}
\begin{sphinxuseclass}{sd-row}
\begin{sphinxuseclass}{sd-g-2}
\begin{sphinxuseclass}{sd-g-xs-2}
\begin{sphinxuseclass}{sd-g-sm-2}
\begin{sphinxuseclass}{sd-g-md-2}
\begin{sphinxuseclass}{sd-g-lg-2}
\begin{sphinxuseclass}{sd-col}
\begin{sphinxuseclass}{sd-d-flex-row}
\begin{sphinxuseclass}{sd-col-8}
\begin{sphinxuseclass}{sd-col-xs-8}
\begin{sphinxuseclass}{sd-col-sm-8}
\begin{sphinxuseclass}{sd-col-md-8}
\begin{sphinxuseclass}{sd-col-lg-8}
\begin{sphinxuseclass}{sd-card}
\begin{sphinxuseclass}{sd-sphinx-override}
\begin{sphinxuseclass}{sd-w-100}
\begin{sphinxuseclass}{sd-shadow-sm}
\begin{sphinxuseclass}{sd-card-body}
\begin{sphinxuseclass}{sd-card-title}
\begin{sphinxuseclass}{sd-font-weight-bold}Collisione su sistema libero rigido di masse concentrate
\end{sphinxuseclass}
\end{sphinxuseclass}
\sphinxAtStartPar
Un proiettile colpisce un sistema rigido di due masse concentrate, libero e inizialmente in quiete. Si chiede di determinare il moto dei sistemi dopo l’urto, in funzione del coefficiente di restituzione.

\end{sphinxuseclass}
\end{sphinxuseclass}
\end{sphinxuseclass}
\end{sphinxuseclass}
\end{sphinxuseclass}
\end{sphinxuseclass}
\end{sphinxuseclass}
\end{sphinxuseclass}
\end{sphinxuseclass}
\end{sphinxuseclass}
\end{sphinxuseclass}
\end{sphinxuseclass}
\begin{sphinxuseclass}{sd-col}
\begin{sphinxuseclass}{sd-d-flex-row}
\begin{sphinxuseclass}{sd-col-4}
\begin{sphinxuseclass}{sd-col-xs-4}
\begin{sphinxuseclass}{sd-col-sm-4}
\begin{sphinxuseclass}{sd-col-md-4}
\begin{sphinxuseclass}{sd-col-lg-4}
\begin{sphinxuseclass}{sd-card}
\begin{sphinxuseclass}{sd-sphinx-override}
\begin{sphinxuseclass}{sd-w-100}
\begin{sphinxuseclass}{sd-shadow-sm}
\begin{sphinxuseclass}{sd-card-body}
\sphinxAtStartPar
\sphinxincludegraphics{{collisions-rod-1}.png}



\end{sphinxuseclass}
\end{sphinxuseclass}
\end{sphinxuseclass}
\end{sphinxuseclass}
\end{sphinxuseclass}
\end{sphinxuseclass}
\end{sphinxuseclass}
\end{sphinxuseclass}
\end{sphinxuseclass}
\end{sphinxuseclass}
\end{sphinxuseclass}
\end{sphinxuseclass}
\end{sphinxuseclass}
\end{sphinxuseclass}
\end{sphinxuseclass}
\end{sphinxuseclass}
\end{sphinxuseclass}
\end{sphinxuseclass}
\end{sphinxuseclass}
\end{sphinxuseclass}
\end{sphinxuseclass}\subsubsection*{Soluzione.}

\sphinxAtStartPar
\sphinxstylestrong{todo}



\begin{sphinxuseclass}{sd-container-fluid}
\begin{sphinxuseclass}{sd-sphinx-override}
\begin{sphinxuseclass}{sd-mb-4}
\begin{sphinxuseclass}{sd-row}
\begin{sphinxuseclass}{sd-g-2}
\begin{sphinxuseclass}{sd-g-xs-2}
\begin{sphinxuseclass}{sd-g-sm-2}
\begin{sphinxuseclass}{sd-g-md-2}
\begin{sphinxuseclass}{sd-g-lg-2}
\begin{sphinxuseclass}{sd-col}
\begin{sphinxuseclass}{sd-d-flex-row}
\begin{sphinxuseclass}{sd-col-8}
\begin{sphinxuseclass}{sd-col-xs-8}
\begin{sphinxuseclass}{sd-col-sm-8}
\begin{sphinxuseclass}{sd-col-md-8}
\begin{sphinxuseclass}{sd-col-lg-8}
\begin{sphinxuseclass}{sd-card}
\begin{sphinxuseclass}{sd-sphinx-override}
\begin{sphinxuseclass}{sd-w-100}
\begin{sphinxuseclass}{sd-shadow-sm}
\begin{sphinxuseclass}{sd-card-body}
\begin{sphinxuseclass}{sd-card-title}
\begin{sphinxuseclass}{sd-font-weight-bold}Collisione su sistema libero rigido a massa distribuita
\end{sphinxuseclass}
\end{sphinxuseclass}
\sphinxAtStartPar
Un proiettile colpisce un sistema rigido di due masse concentrate, libero e inizialmente in quiete. Si chiede di determinare il moto dei sistemi dopo l’urto, in funzione del coefficiente di restituzione.

\end{sphinxuseclass}
\end{sphinxuseclass}
\end{sphinxuseclass}
\end{sphinxuseclass}
\end{sphinxuseclass}
\end{sphinxuseclass}
\end{sphinxuseclass}
\end{sphinxuseclass}
\end{sphinxuseclass}
\end{sphinxuseclass}
\end{sphinxuseclass}
\end{sphinxuseclass}
\begin{sphinxuseclass}{sd-col}
\begin{sphinxuseclass}{sd-d-flex-row}
\begin{sphinxuseclass}{sd-col-4}
\begin{sphinxuseclass}{sd-col-xs-4}
\begin{sphinxuseclass}{sd-col-sm-4}
\begin{sphinxuseclass}{sd-col-md-4}
\begin{sphinxuseclass}{sd-col-lg-4}
\begin{sphinxuseclass}{sd-card}
\begin{sphinxuseclass}{sd-sphinx-override}
\begin{sphinxuseclass}{sd-w-100}
\begin{sphinxuseclass}{sd-shadow-sm}
\begin{sphinxuseclass}{sd-card-body}
\sphinxAtStartPar
\sphinxincludegraphics{{collisions-rod-2}.png}



\end{sphinxuseclass}
\end{sphinxuseclass}
\end{sphinxuseclass}
\end{sphinxuseclass}
\end{sphinxuseclass}
\end{sphinxuseclass}
\end{sphinxuseclass}
\end{sphinxuseclass}
\end{sphinxuseclass}
\end{sphinxuseclass}
\end{sphinxuseclass}
\end{sphinxuseclass}
\end{sphinxuseclass}
\end{sphinxuseclass}
\end{sphinxuseclass}
\end{sphinxuseclass}
\end{sphinxuseclass}
\end{sphinxuseclass}
\end{sphinxuseclass}
\end{sphinxuseclass}
\end{sphinxuseclass}\subsubsection*{Soluzione.}

\sphinxAtStartPar
\sphinxstylestrong{todo}

\sphinxstepscope

\begin{sphinxuseclass}{sd-container-fluid}
\begin{sphinxuseclass}{sd-sphinx-override}
\begin{sphinxuseclass}{sd-p-0}
\begin{sphinxuseclass}{sd-mt-2}
\begin{sphinxuseclass}{sd-mb-4}
\begin{sphinxuseclass}{sd-row}
\begin{sphinxuseclass}{sd-row-cols-2}
\begin{sphinxuseclass}{sd-gx-2}
\begin{sphinxuseclass}{sd-gy-1}
\begin{sphinxuseclass}{sd-col}
\begin{sphinxuseclass}{sd-d-flex-row}
\begin{sphinxuseclass}{sd-align-minor-center}
\begin{sphinxuseclass}{sd-container-fluid}
\begin{sphinxuseclass}{sd-sphinx-override}
\begin{sphinxuseclass}{sd-row}
\begin{sphinxuseclass}{sd-row-cols-2}
\begin{sphinxuseclass}{sd-row-cols-xs-2}
\begin{sphinxuseclass}{sd-row-cols-sm-3}
\begin{sphinxuseclass}{sd-row-cols-md-3}
\begin{sphinxuseclass}{sd-row-cols-lg-3}
\begin{sphinxuseclass}{sd-gx-3}
\begin{sphinxuseclass}{sd-gy-1}
\begin{sphinxuseclass}{sd-col}
\begin{sphinxuseclass}{sd-col-auto}
\begin{sphinxuseclass}{sd-d-flex-row}
\begin{sphinxuseclass}{sd-align-minor-center}
\sphinxAtStartPar
basics

\end{sphinxuseclass}
\end{sphinxuseclass}
\end{sphinxuseclass}
\end{sphinxuseclass}
\begin{sphinxuseclass}{sd-col}
\begin{sphinxuseclass}{sd-col-auto}
\begin{sphinxuseclass}{sd-d-flex-row}
\begin{sphinxuseclass}{sd-align-minor-center}
\sphinxAtStartPar
Nov 06, 2024

\end{sphinxuseclass}
\end{sphinxuseclass}
\end{sphinxuseclass}
\end{sphinxuseclass}
\begin{sphinxuseclass}{sd-col}
\begin{sphinxuseclass}{sd-col-auto}
\begin{sphinxuseclass}{sd-d-flex-row}
\begin{sphinxuseclass}{sd-align-minor-center}
\sphinxAtStartPar
3 min read

\end{sphinxuseclass}
\end{sphinxuseclass}
\end{sphinxuseclass}
\end{sphinxuseclass}
\end{sphinxuseclass}
\end{sphinxuseclass}
\end{sphinxuseclass}
\end{sphinxuseclass}
\end{sphinxuseclass}
\end{sphinxuseclass}
\end{sphinxuseclass}
\end{sphinxuseclass}
\end{sphinxuseclass}
\end{sphinxuseclass}
\end{sphinxuseclass}
\end{sphinxuseclass}
\end{sphinxuseclass}
\end{sphinxuseclass}
\end{sphinxuseclass}
\end{sphinxuseclass}
\end{sphinxuseclass}
\end{sphinxuseclass}
\end{sphinxuseclass}
\end{sphinxuseclass}
\end{sphinxuseclass}
\end{sphinxuseclass}

\section{Gravitazione}
\label{\detokenize{ch/mechanics/dynamics-motion-gravitation:gravitazione}}\label{\detokenize{ch/mechanics/dynamics-motion-gravitation:physics-hs-mechanics-dynamics-motion-gravitation}}\label{\detokenize{ch/mechanics/dynamics-motion-gravitation::doc}}



\subsection{Legge di gravitazione universale}
\label{\detokenize{ch/mechanics/dynamics-motion-gravitation:legge-di-gravitazione-universale}}\begin{equation*}
\begin{split}\vec{F}_{10} = G \, m_0 \, m_1 \frac{\vec{r}_{01}}{\left|\vec{r}_{01}\right|^3}\end{split}
\end{equation*}

\subsection{Problema dei due corpi}
\label{\detokenize{ch/mechanics/dynamics-motion-gravitation:problema-dei-due-corpi}}
\sphinxAtStartPar
In meccanica classica, il problema dei due corpi si riferisce alla dinamica di un sistema formato da due corpi puntiformi soggetti unicamente alla mutua interazione gravitazionale, descritta dalla legge di gravitazione universale di Newton.

\sphinxAtStartPar
Il sistema formato dai due punti è un sistema chiuso e isolato, sul quale non agiscono azioni esterne. La quantità di moto rispetto a un sistema di riferimento inerziale rimane quindi costante. Rimane quindi costante la velocità del centro di massa \(G\),
\begin{equation*}
\begin{split}G = \frac{m_0 \, P_0 + m_1 \, P_1}{m_0 + m_1} \ ,\end{split}
\end{equation*}
\sphinxAtStartPar
ed è possibile definire un sistema di riferimento inerziale con origine nel centro di massa del sistema. Il raggio vettore tra i due corpi può quindi essere riscritto,
\begin{equation*}
\begin{split}P_1 - G = P_1 - \frac{m_0 \, P_0 + m_1 \, P_1}{m_0 + m_1} = \frac{ - m_0 \, P_0 + m_0 \, P_1}{m_0 + m_1} = \frac{m_0}{m_0 + m_1}(P_1 - P_0) \ .\end{split}
\end{equation*}
\sphinxAtStartPar
L’equazione del moto per il corpo \(1\) nel sistema di riferimento inerziale con origine in \(G\) segue il secondo principio della dinamica. L’equazione del moto può essere scritto in termini del raggio vettore tra corpo \(1\) e centro di massa,
\begin{equation*}
\begin{split}\begin{aligned}
  m_1 \dfrac{d^2}{dt^2}(P_1-G) & = - G m_0 m_1 \frac{P_1 - P_0}{|P_1 - P_0|^3} = \\
                               & = - G \dfrac{(m_0 + m_1)^2}{m_0} m_1 \frac{P_1 - G}{|P_1 - G|^3} 
\end{aligned}\end{split}
\end{equation*}
\sphinxAtStartPar
o in termini del raggio vettore tra i due corpi \(P_1 - P_0\)
\begin{equation*}
\begin{split}
  \frac{m_0 m_1}{m_0 + m_1} \dfrac{d^2}{dt^2}(P_1-P_0) = - G m_0 m_1 \frac{P_1 - P_0}{|P_1 - P_0|^3} 
\end{split}
\end{equation*}\begin{equation*}
\begin{split}
  m_1 \dfrac{d^2}{dt^2}(P_1-P_0) = - G ( m_0 + m_1 ) m_1 \frac{P_1 - P_0}{|P_1 - P_0|^3} 
\end{split}
\end{equation*}
\sphinxAtStartPar
Le equazioni del moto in questi due sistemi di riferimento possono essere scritte nella forma
\begin{equation*}
\begin{split}m_1 \, \ddot{\vec{r}} = - G M m_1 \frac{\vec{r}}{r^3} \ .\end{split}
\end{equation*}

\subsubsection{Traiettorie, coniche, ed energia}
\label{\detokenize{ch/mechanics/dynamics-motion-gravitation:traiettorie-coniche-ed-energia}}
\sphinxAtStartPar
E’ possibile dimostrare che il moto di ognuno dei due corpi è un moto piano, e che la traiettoria avviene descrive una conica.
\begin{itemize}
\item {} 
\sphinxAtStartPar
\sphinxstylestrong{todo} Dimostrare che il moto è piano

\item {} 
\sphinxAtStartPar
\sphinxstylestrong{todo} Dimostrare che la traiettoria è una conica

\end{itemize}

\sphinxAtStartPar
Il tipo di curva conica dipende da una grandezza scalare che può essere ricondotta a un’energia. Il prodotto scalare della velocità \(\dot{\vec{r}}\) con l’equazione del moto, permette di ricavare un principio di conservazione dell’energia,
\begin{equation*}
\begin{split}\begin{aligned}
  0 & = \dot{\vec{r}} \cdot \left( m \ddot{\vec{r}} + G M m \frac{\vec{r}}{r^3} \right) = \\
    & = \dfrac{d}{dt} \left( \frac{1}{2} m \left|\dot{\vec{r}}\right|^2 - G M m \frac{1}{r}\right) = \dfrac{d E^{mec}}{dt}
\end{aligned}\end{split}
\end{equation*}
\sphinxAtStartPar
Usando il sistema di coordinate polari, e la costanza della velocità angolare \(\Omega = \frac{1}{2}{r^2}{\dot{\theta}}\), si può scrivere
\begin{equation*}
\begin{split}\begin{aligned}
  \frac{E^{mec}}{m} & = \frac{1}{2} \dot{r}^2 + \frac{1}{2} r^2 \dot{\theta}^2 - \frac{G M}{r} = \\
                    & = \frac{1}{2} \dot{r}^2 + 2 \frac{\Omega^2}{r^2} - \frac{G M}{r} = \\
                    & = \frac{1}{2} \dot{r}^2 + v_r(r) \ .
\end{aligned}\end{split}
\end{equation*}
\sphinxAtStartPar
Poiché \(\frac{1}{2}\dot{r}^2 \ge 0\), il moto è possibile per tutti i valori di \(r\) tali che \(\frac{E}{m} \ge v_r(r)\). Il valore di \(E\) identifica le traiettorie. \sphinxstylestrong{todo} \sphinxstyleemphasis{aggiungere grafici}
\begin{itemize}
\item {} 
\sphinxAtStartPar
esiste un valore minimo di \(E\): questo valore è associato a un’orbita circolare

\item {} 
\sphinxAtStartPar
per \(E_{min} \le E \le 0\) esistono due soluzioni dell’equazione \(\frac{E}{m} - v_r(r) = 0\): orbite chiuse, ellittiche o circolari (per \(E = E_{min}\))

\item {} 
\sphinxAtStartPar
\(E = 0\) è un caso limite che separa le orbite chiuse e le orbite aperte: a \(E = 0\) è associata un’orbita parabolica

\item {} 
\sphinxAtStartPar
per \(E > 0\) le orbite aperte sono iperboliche

\end{itemize}


\subsubsection{Traiettorie chiuse e leggi di Keplero}
\label{\detokenize{ch/mechanics/dynamics-motion-gravitation:traiettorie-chiuse-e-leggi-di-keplero}}
\sphinxAtStartPar
\sphinxstylestrong{Prima legge.} Un pianeta descrive un’orbita ellittica attorno al Sole, che si trova in uno dei due fuochi.

\sphinxAtStartPar
\sphinxstylestrong{Seconda legge legge.} Considerando l’area descritta dal moto del pianeta attorno al Sole, la velocità angolare è costante lungo la traiettoria.

\sphinxAtStartPar
\sphinxstylestrong{Terza legge.} In un sistema di pianeti, il quadrato del periodo delle orbite descritte dai pianeti è proporzionale al cubo del semiasse maggiore della traiettoria, \(T^2 \propto a^3\).

\sphinxAtStartPar
\sphinxstylestrong{todo} rispetto a quale sistema di riferimento? Serve l’approssimazione che la massa del Sole sia \(>>\) delle masse dei pianeti, se si considera inerziale un sistema di coordinate con origine nel Sole? O bisogna/si può usare un sistema inerziale con origine nel centro di massa del sistema (considerato isolato)

\sphinxAtStartPar
\sphinxstylestrong{Moto piano.} Siano \(\vec{r}\), \(\vec{v}\) la posizione e la velocità del pianeta rispetto al Sole. La forza di gravità agente sul pianeta è
\begin{equation*}
\begin{split}\vec{F} = - G M m \frac{\vec{r}}{r^3} \ .\end{split}
\end{equation*}
\sphinxAtStartPar
E’ facile dimostrare che il moto è piano, cioè che la posizione e la velocità del pianeta sono sempre ortogonali a una direzione costante.
\begin{equation*}
\begin{split}\frac{d}{dt} \left( \vec{r} \times \vec{v} \right) = \underbrace{\vec{v} \times \vec{v}}_{=\vec{0}} + \vec{r} \times \vec{a}  = - G M m \underbrace{\vec{r} \times \frac{\vec{r}}{r^3}}_{=\vec{0}} = \vec{0} \ .\end{split}
\end{equation*}
\sphinxAtStartPar
Poiché il vettore \(\vec{r} \times \vec{v} =: \frac{L}{m} \hat{k}\) è costante, è costante sia il suo valore assoluto sia la sua direzione: affinché \(\vec{r} \times \vec{v}\) sia allineato con \(\hat{k}\), i vettori \(\vec{r}\), \(\vec{v}\) devono essere ortogonali a \(\hat{k}\).

\sphinxAtStartPar
\sphinxstylestrong{Coordinate polari.} Per descrivere il moto piano di un punto, si può usare un sistema di coordinate 2\sphinxhyphen{}dimensionale. Si sceglie un sistema di coordinate polari con origine coincidente con il Sole. La posizione del pianeta è identificata dal raggio vettore
\begin{equation*}
\begin{split}\vec{r} = r \, \hat{r} \ ,\end{split}
\end{equation*}
\sphinxAtStartPar
e la derivate dei versori radiale e azimuthale valgono
\begin{equation*}
\begin{split}\begin{aligned}
\dot{\hat{r}}      & =   \dot{\theta} \hat{\theta} \\
\dot{\hat{\theta}} & = - \dot{\theta} \hat{r} \\
\end{aligned}\end{split}
\end{equation*}
\sphinxAtStartPar
La posizione, la velocità e l’accelerazione del pianeta possono essere scritte come
\begin{equation*}
\begin{split}\begin{aligned}
\vec{r} & = r \, \hat{r} \\
\vec{v} & = \dot{r} \, \hat{r} + r \dot{\theta} \, \hat{\theta} \\
\vec{a} & = \left[ \ddot{r} - r \dot{\theta}^2 \right] \, \hat{r} +  \left[ 2 \dot{r} \dot{\theta} + r \ddot{\theta} \right] \, \hat{\theta}  \\
\end{aligned}\end{split}
\end{equation*}
\sphinxAtStartPar
La \sphinxstylestrong{velocità areolare}, \(\vec{\Omega} = \frac{1}{2} \vec{r} \times \vec{v} \) è costante e uguale a
\begin{equation*}
\begin{split}\vec{\Omega} = \frac{1}{2} \frac{L}{m} \hat{k} = \frac{1}{2} r^2 \dot{\theta} \, \hat{k} \ .\end{split}
\end{equation*}


\sphinxAtStartPar
Dall’espressione della velocità angolare costante, si può ricavare il legame tra \(\dot{\theta}\) ed \(r\),
\begin{equation*}
\begin{split}\dot{\theta} = \frac{\Omega}{r^2} \ .\end{split}
\end{equation*}
\sphinxAtStartPar
Usando le coordinate polari, l’equazione del moto \(m \ddot{\vec{r}} = -G M m \frac{\vec{r}}{r^3}\) viene scritta in componenti,
\begin{equation*}
\begin{split}\begin{aligned}
r      & : \ m (\ddot{r} - r\dot{\theta}^2) = - G M m \frac{1}{r^2} \\
\theta & : \ m ( 2 \dot{r} \dot{\theta} + r \ddot{\theta}^2 ) = 0
\end{aligned}\end{split}
\end{equation*}
\sphinxAtStartPar
\sphinxstylestrong{Traiettoria, \(r(\theta)\).}
Inserendo l’espressione \(\dot{\theta} = \frac{\Omega}{r^2}\) nella componente radiale, e definendo la funzione \(z = \frac{1}{r}\), le derivate nel tempo della coordinata radiale possono essere riscritte come
\begin{equation*}
\begin{split}\dot{r} = -\frac{1}{z^2}\frac{d z}{d \theta} \dot{\theta} = -\Omega \frac{dz}{d\theta} \end{split}
\end{equation*}\begin{equation*}
\begin{split}\ddot{r} = \dot{\theta} \frac{d}{d \theta} \left( - \Omega \frac{dz}{d \theta} \right) = - z^2 \Omega^2 z''(\theta)\end{split}
\end{equation*}
\sphinxAtStartPar
e la componente radiale dell’equazione di moto,
\begin{equation*}
\begin{split}-z^2 \Omega^2 z'' - z^3 \Omega^2 = - G M z^2\end{split}
\end{equation*}\begin{equation*}
\begin{split} z'' + z  = \frac{G M}{\Omega^2}\end{split}
\end{equation*}\begin{equation*}
\begin{split}z(\theta) = \frac{G M}{\Omega^2} + A \cos(\theta) + B \sin(\theta) \ .\end{split}
\end{equation*}
\sphinxAtStartPar
e quindi
\begin{equation*}
\begin{split}r(\theta) = \frac{\Omega^2}{G M}\frac{1}{1 + A \dfrac{\, \Omega^2}{GM} \cos \theta + B \dfrac{\, \Omega^2}{GM} \sin \theta}\end{split}
\end{equation*}
\sphinxAtStartPar
Scelta della direzione di riferimento: direzione del perielio: \(r(\theta=0) = \min r\), \(B = 0\),

\sphinxAtStartPar
Scelte diverse si ottengono da una trasformazione di coordinate con una rotazione dell’asse di riferimento: \(\theta_1 = \theta - \theta_0\), e quindi
\begin{equation*}
\begin{split}r(\theta) = \frac{\Omega^2}{GM}\frac{1}{1 + \frac{A\Omega^2}{GM} \cos \theta} = \frac{\Omega^2}{GM}\frac{1}{1 + \frac{A\Omega^2}{GM} \cos (\theta_1 + \theta_0 )} = \frac{\Omega^2}{GM}\frac{1}{1 + \underbrace{\frac{A\Omega^2}{GM} \cos \theta_0}_{= A_1} \cos \theta_1 \underbrace{- \frac{A \Omega^2}{GM} \sin \theta_0}_{= B_1} \sin \theta_1 }\end{split}
\end{equation*}
\sphinxAtStartPar
Il confronto con l’equazione delle coniche in coordinate polari, permette di riconoscere l’eccentricità, \(e\) e il prodotto \(e \, D\) dell’eccentricità per la distanza \(D\) tra fuoco e direttrice,
\begin{equation*}
\begin{split}e = \frac{A \Omega^2}{GM} \qquad , \qquad e \, D = \frac{\Omega^2}{GM}\end{split}
\end{equation*}\begin{equation*}
\begin{split}r(\theta) = \frac{\Omega^2}{GM}\frac{1}{1 + \frac{A\Omega^2}{GM} \cos \theta}\end{split}
\end{equation*}\begin{equation*}
\begin{split}r(\theta) = \frac{e \, D}{1 + e \, \cos \theta}\end{split}
\end{equation*}


\sphinxAtStartPar
Poiché la velocità areolare è costante, il periodo dell’orbita è uguale al raggporto tra l’area dell’ellisse e la velocità areaolare,
\begin{equation*}
\begin{split}T = \frac{\pi a b}{\Omega} = \pi \frac{a^2 \sqrt{1-e^2}}{\Omega} = \end{split}
\end{equation*}\begin{equation*}
\begin{split}1-e^2 = 1 - \left(\frac{A \Omega^2}{GM} \right)^2 = \frac{\Omega^2}{GM \, a} \end{split}
\end{equation*}\begin{equation*}
\begin{split}\rightarrow \qquad \frac{\sqrt{1-e^2}}{\Omega} = \frac{1}{\sqrt{GM} \sqrt{a}}\end{split}
\end{equation*}\begin{equation*}
\begin{split}\rightarrow \qquad T = \pi \frac{a^{\frac{3}{2}}}{\sqrt{GM}}\end{split}
\end{equation*}\begin{equation*}
\begin{split}
 2a = \frac{\Omega^2}{GM+A \Omega^2} + \frac{\Omega^2}{GM-A \Omega^2} 
    = \Omega^2 \frac{2 GM }{(GM)^2 - A^2 \Omega^4}
\end{split}
\end{equation*}\begin{equation*}
\begin{split}A^2 \Omega^4 = (GM)^2 - \frac{GM \, \Omega^2}{a}\end{split}
\end{equation*}\begin{equation*}
\begin{split}\frac{\Omega^2}{GM}\frac{1}{a} = 1 - \left(\frac{A \Omega^2}{GM}\right)^2\end{split}
\end{equation*}\begin{equation*}
\begin{split}\frac{1}{a} = \left( 1 - \left(\frac{A \Omega^2}{GM}\right)^2 \right) \frac{GM}{\Omega^2}\end{split}
\end{equation*}


\sphinxstepscope


\part{Termodinamica}

\sphinxstepscope


\chapter{Termodinamica}
\label{\detokenize{ch/thermodynamics:termodinamica}}\label{\detokenize{ch/thermodynamics:physics-hs-thermodynamics}}\label{\detokenize{ch/thermodynamics::doc}}
\sphinxAtStartPar
La termodinamica si occupa dell’\sphinxstylestrong{energia} di un sistema, definita come \sphinxstyleemphasis{capacità di un sistema di svolgere lavoro}, e la sua variazione tramite scambi di lavoro o calore con l’ambiente esterno.
\begin{itemize}
\item {} 
\sphinxAtStartPar
…

\end{itemize}

\sphinxAtStartPar
La termodinamica classica fornisce \sphinxstylestrong{modelli macroscopici} di sistemi composti da un numero elevato di componenti elementari.
\begin{itemize}
\item {} 
\sphinxAtStartPar
indagini sulla struttura della materia, indagini di chimica:
\begin{itemize}
\item {} 
\sphinxAtStartPar
primi strumenti: barometro, termometro, calorimetro

\item {} 
\sphinxAtStartPar
studi sui gas, reazioni chimiche (Boyle, Lavoisier, Dalton, Charles, Gay\sphinxhyphen{}Lussac, Avogadro): legge dei gas perfetti, teoria atomica

\item {} 
\sphinxAtStartPar
temperatura e calore

\item {} 
\sphinxAtStartPar
transizione di fase

\end{itemize}

\item {} 
\sphinxAtStartPar
macchine termiche
\begin{itemize}
\item {} 
\sphinxAtStartPar
prima: innovazione tecnologica: macchine termiche nella rivoluzione industirale. Trasformazione di calore in lavoro

\item {} 
\sphinxAtStartPar
poi: indagini teoriche sulle macchine termiche, relazione tra calore e lavoro
\begin{itemize}
\item {} 
\sphinxAtStartPar
Carnot

\item {} 
\sphinxAtStartPar
Clausis

\end{itemize}

\item {} 
\sphinxAtStartPar
formalizzazione (Clausius, Rankine, Kelvin, Gibbs):
\begin{itemize}
\item {} 
\sphinxAtStartPar
modello:
\begin{itemize}
\item {} 
\sphinxAtStartPar
variabili di stato \sphinxstyleemphasis{intensive} macroscopiche (come temperatura, pressione, energia interna) per descrivere lo stato di un sistema

\item {} 
\sphinxAtStartPar
regola delle fasi di Gibbs

\item {} 
\sphinxAtStartPar
…

\end{itemize}

\item {} 
\sphinxAtStartPar
princìpi che traducano l’esperienza:
\begin{itemize}
\item {} 
\sphinxAtStartPar
primo principio della TD

\item {} 
\sphinxAtStartPar
secondo principio della TD

\end{itemize}

\end{itemize}

\end{itemize}

\end{itemize}

\sphinxAtStartPar
\sphinxhyphen{} Stati di equilibrio, e come questo equilibrio viene raggiunto
\sphinxhyphen{} Reazioni chimiche
\sphinxhyphen{} Rankine, Kelvin, \sphinxstylestrong{Gibbs}: variabili di stato

\sphinxAtStartPar
\sphinxstylestrong{todo} in meccanica, la presenza di azioni dissipative riduce l’energia meccanica (\sphinxstylestrong{todo} riferimento al teorema dell’energia cinetica). Dove finisce questa energia? \sphinxstylestrong{Equivalenza lavoro\sphinxhyphen{}calore di Joule}

\sphinxAtStartPar
\sphinxstylestrong{todo} approccio macroscopico a sistemi complessi

\sphinxAtStartPar
\sphinxstylestrong{todo} formulazione dei princìpi a partire dall’esperienza:
\begin{itemize}
\item {} 
\sphinxAtStartPar
non si crea energia meccanica dal nulla, ossia la dissipazione dell’energia meccanica è non\sphinxhyphen{}negativa

\item {} 
\sphinxAtStartPar
temperatura, grandezza fisica associata alla sensazione di caldo\sphinxhyphen{}freddo; evidenza della dilatazione delle sostanze al variare della temperatura (principio fisico alla base del principio di funzionamento dei primi termometri)

\item {} 
\sphinxAtStartPar
tendenza di due sistemi chiusi e in quiete inizialemente a temperature diverse a raggiungere l’equilibrio termico: trasferimento di calore “dal corpo a temperatura maggiore a quello a temperatura minore”

\end{itemize}

\sphinxstepscope


\chapter{Introduzione alla termodinamica}
\label{\detokenize{ch/thermodynamics/foundation:introduzione-alla-termodinamica}}\label{\detokenize{ch/thermodynamics/foundation:physics-hs-thermodynamics-foundation}}\label{\detokenize{ch/thermodynamics/foundation::doc}}
\sphinxAtStartPar
\sphinxstylestrong{Concetti e primi strumenti.}
\begin{itemize}
\item {} 
\sphinxAtStartPar
Temperatura, pressione

\item {} 
\sphinxAtStartPar
Primi strumenti, e princìpi fisici utilizzati:
\begin{itemize}
\item {} 
\sphinxAtStartPar
manometro di Torricelli e peso di una colonna di acqua o Hg

\item {} 
\sphinxAtStartPar
termometri e dilatazione termica delle sostanze

\end{itemize}

\end{itemize}

\sphinxAtStartPar
\sphinxstylestrong{Esperienza.}
\begin{itemize}
\item {} 
\sphinxAtStartPar
Tendenze naturali:
\begin{itemize}
\item {} 
\sphinxAtStartPar
Equilibrio termico: calore dal corpo più caldo a quello più freddo

\item {} 
\sphinxAtStartPar
Dissipazione dell’energia meccanica

\item {} 
\sphinxAtStartPar
Conservazione della massa

\end{itemize}

\item {} 
\sphinxAtStartPar
Esperimenti in chimica, sull’indagine della natura della materia:
\begin{itemize}
\item {} 
\sphinxAtStartPar
misura di: massa, pressione, volume, temperatura,

\item {} 
\sphinxAtStartPar
esperimenti su: reazioni chimiche, gas

\end{itemize}

\end{itemize}

\sphinxAtStartPar
\sphinxstylestrong{Macchine termiche.}
\begin{itemize}
\item {} 
\sphinxAtStartPar
Applicazioni che guidano la rivoluzione industriale in Inghilterra,

\item {} 
\sphinxAtStartPar
Approfondimenti e studi teorici sul funzionamento delle macchine termiche, le sostanze e i fenomeni fisici coinvolti

\item {} 
\sphinxAtStartPar
Equivalenza calore\sphinxhyphen{}lavoro di Joule

\end{itemize}

\sphinxAtStartPar
\sphinxstylestrong{Formalizzazione dei principi della termodinamica.}

\sphinxAtStartPar
Le prime esperienze, lo studio delle sostanze, delle trasformazioni, e delle macchine termiche verranno ri\sphinxhyphen{}analizzate nei capitoli successivi, dopo aver formalizzato i princìpi della termodinamica e \sphinxstylestrong{todo}  un modello/approccio allo studio della materia.

\sphinxstepscope


\section{Breve storia della termodinamica}
\label{\detokenize{ch/thermodynamics/foundation-history:breve-storia-della-termodinamica}}\label{\detokenize{ch/thermodynamics/foundation-history:physics-hs-thermodynamics-foundation-history}}\label{\detokenize{ch/thermodynamics/foundation-history::doc}}
\sphinxAtStartPar
Sviluppo non lineare, dovuto ai contributi di molti studiosi, nel corso di più di un paio di secoli.

\sphinxAtStartPar
\sphinxstylestrong{Esperienze.}
\begin{itemize}
\item {} 
\sphinxAtStartPar
Sensazione di caldo e freddo

\item {} 
\sphinxAtStartPar
Dissipazione dell’energia meccanica “macroscopica”

\end{itemize}

\sphinxAtStartPar
\sphinxstylestrong{Domande aperte.}
\begin{itemize}
\item {} 
\sphinxAtStartPar
Come è fatta la materia?
\begin{itemize}
\item {} 
\sphinxAtStartPar
affermazione della \sphinxstylestrong{teoria atomistica}: particelle elementari, che si possono combinare

\item {} 
\sphinxAtStartPar
esiste il vuoto? “la natura ha orrore del vuoto” (Galileo)?

\end{itemize}

\item {} 
\sphinxAtStartPar
Come descrivere le sensazioni di caldo\sphinxhyphen{}freddo: temperatura e calore
\begin{itemize}
\item {} 
\sphinxAtStartPar
teoria \sphinxstylestrong{calorica}? Fluido invisibile

\item {} 
\sphinxAtStartPar
…

\end{itemize}

\end{itemize}

\sphinxAtStartPar
\sphinxstylestrong{Indagine scientifica: natura materia.} Rivoluzione scientifica del XVI\sphinxhyphen{}XVII secolo. Costruzione di strumenti
\begin{itemize}
\item {} 
\sphinxAtStartPar
Torricelli, disceplo di Galileo:
\begin{itemize}
\item {} 
\sphinxAtStartPar
il barometro;

\item {} 
\sphinxAtStartPar
la misura del peso dell’aria: \sphinxstylestrong{pressione} atmosferica;

\item {} 
\sphinxAtStartPar
“vuoto” al di sopra della colonnina di Hg: argomento che rilancia la tesi atomistica

\end{itemize}

\item {} 
\sphinxAtStartPar
Boyle, “primo chimico”; tra i fondatori della Royal Society; preciso sperimentatore (descrizione dettagliata per permettere replica), grazie agli strumenti progettati e realizzati da Robert Hooke:
\begin{itemize}
\item {} 
\sphinxAtStartPar
contributi alla chimica

\item {} 
\sphinxAtStartPar
legge di Boyle sui gas: l’aria si comporta come una molla, \(PV = \text{cost}\) a \(T\) cost. Comportamento elastico, come i solidi studiati da Hooke (legge costitutiva lineare elastica): modello dei gas come costituiti da particelle elementari, collegati da molle

\end{itemize}

\item {} 
\sphinxAtStartPar
D.Bernoulli, \sphinxstyleemphasis{Hydrodynamica}, 1738:
\begin{itemize}
\item {} 
\sphinxAtStartPar
primo modello matematico nella teoria cinetica dei gas: gas costituiti da particelle libere di muoversi: la pressione è il risultato degli urti delle particelle sulle pareti del contenitore.

\end{itemize}

\item {} 
\sphinxAtStartPar
A.Lavoisier, fine ‘700, uno dei più influenti chimici della storia:
\begin{itemize}
\item {} 
\sphinxAtStartPar
misura del peso nelle indagini di chimica: \sphinxstylestrong{conservazione della massa} in fisica classica

\item {} 
\sphinxAtStartPar
altro valido argomento a sostegno della teoria atomistica: le sostanze sono formate da particelle elementari che si combinano a formare diverse sostanze; nelle reazioni chimiche, reagenti e prodotti hanno la stessa massa (\(Hg \, O \rightarrow Hg \, + \, \frac{1}{2} O_2\))

\end{itemize}

\item {} 
\sphinxAtStartPar
Composizione sostanze è ben definita?
\begin{itemize}
\item {} 
\sphinxAtStartPar
Berthollet: no, contrario alla teoria atomica, es. bronzo (lega!): la composizione di una sostanza dipende dal processo con il quale viene prodotto;

\item {} 
\sphinxAtStartPar
Proust: carbonato basico di \(Cu\). I campioni provenienti da diverse parti, trovate sia in natura sia sintetizzate in laboratorio, hanno esattamente la stessa composizione in massa;

\item {} 
\sphinxAtStartPar
Dalton: sostenitore teoria atomica, dopo aver formulato la legge delle proporzioni multiple; pessimo sperimentatore; gli atomi sono indivisibili, ma non pensa che le sostanze possano avere molecole con più atomi; le sue conclusioni sulla composizione dell’acqua saranno causa di grande confusione negli anni successivi

\end{itemize}

\item {} 
\sphinxAtStartPar
Gay\sphinxhyphen{}Lussac, 1808, discepolo di Berthollet
\begin{itemize}
\item {} 
\sphinxAtStartPar
leggi dei gas

\item {} 
\sphinxAtStartPar
studi con controllo del volume: osserva che \(V\), \(n\) sono proporzionali a parti pressione e temperatura; non formula una spiegazione fondata sulla teoria atomica, forse per timore del giudizio di Berthollet, più probabilmente per il disaccordo con le conclusioni sbagliate di Dalton sulla composizione dell’acqua

\end{itemize}

\item {} 
\sphinxAtStartPar
Avogadro, 1811:
\begin{itemize}
\item {} 
\sphinxAtStartPar
volumi di gas uguali nelle stesse condizioni di \(T\), \(P\) contengono lo stesso numero di molecole, anche tipi di gas diverso

\end{itemize}

\item {} 
\sphinxAtStartPar
Berzelius, 1813

\item {} 
\sphinxAtStartPar
Cannizzaro, 1860 \sphinxstyleemphasis{Sunto di un corso di filosofia chimica}

\end{itemize}

\sphinxAtStartPar
\sphinxstylestrong{Indagine scientifica \sphinxhyphen{} Calore e temperatura.} Muovere sopra, prima dell’indagine dei chimici? Fare un paragrafo introduttivo su pressione/temperatura, strumenti per la misura, e scale di misura? Non rispetta un ordine cronologico, ma permette di non spezzettare troppo il racconto”
\begin{itemize}
\item {} 
\sphinxAtStartPar
strumenti e scale di temperatura

\item {} 
\sphinxAtStartPar
equilibrio termico, e tendenza naturale nell’evoluzione della temperatura

\item {} 
\sphinxAtStartPar
calore latente, J.Black

\item {} 
\sphinxAtStartPar
Fourier: equazione per la conduzione

\item {} 
\sphinxAtStartPar
…

\end{itemize}

\sphinxAtStartPar
\sphinxstylestrong{Indagine scientifica \sphinxhyphen{} Macchine termiche: energia, lavoro e calore.}
\begin{itemize}
\item {} 
\sphinxAtStartPar
L’invenzione della macchina a vapore e i motori termici dà il via alla rivoluzione industriale

\item {} 
\sphinxAtStartPar
Indagini teoriche sul funzionamento delle macchine termiche, sulla trasmissione di calore e la generazione di lavoro
\begin{itemize}
\item {} 
\sphinxAtStartPar
1824, \sphinxstylestrong{S.Carnot} \sphinxstyleemphasis{riflessioni sulla forza motrice del fuoco}:
\begin{itemize}
\item {} 
\sphinxAtStartPar
analisi teorica delle macchine termiche, macchina ideale e rendimento massimo

\item {} 
\sphinxAtStartPar
critica della \sphinxstyleemphasis{teoria calorica}: se il calore fosse materia, questo dovrebbe essere creato dal movimento…

\end{itemize}

\item {} 
\sphinxAtStartPar
Joule: equivalenza lavoro\sphinxhyphen{}calore (porterà al I principio)

\item {} 
\sphinxAtStartPar
\sphinxstylestrong{Clausius}:
\begin{itemize}
\item {} 
\sphinxAtStartPar
irreversibilità, in termini di etnropia (II principio)

\end{itemize}

\item {} 
\sphinxAtStartPar
\sphinxstylestrong{Gibbs}: formalizzazione di una teoria termodinamica “macroscopica”, con un approccio geometrico:
\begin{itemize}
\item {} 
\sphinxAtStartPar
variabili di stato, spazio delle fasi, regola delle fasi

\item {} 
\sphinxAtStartPar
energia libera

\end{itemize}

\end{itemize}

\end{itemize}

\sphinxAtStartPar
\sphinxstylestrong{Indagine scientifica \sphinxhyphen{} Meccanica statistica: il microscopico.}
\begin{itemize}
\item {} 
\sphinxAtStartPar
Clausius

\item {} 
\sphinxAtStartPar
Maxwell:
\begin{itemize}
\item {} 
\sphinxAtStartPar
…

\end{itemize}

\item {} 
\sphinxAtStartPar
Gibbs

\item {} 
\sphinxAtStartPar
\sphinxstylestrong{Boltzmann}
\begin{itemize}
\item {} 
\sphinxAtStartPar
…

\end{itemize}

\end{itemize}



\sphinxstepscope


\section{Esperienze ed esperimenti}
\label{\detokenize{ch/thermodynamics/foundation-experiments:esperienze-ed-esperimenti}}\label{\detokenize{ch/thermodynamics/foundation-experiments:physics-hs-thermodynamics-foundation-experiments}}\label{\detokenize{ch/thermodynamics/foundation-experiments::doc}}

\subsection{Dilatazione sostanze}
\label{\detokenize{ch/thermodynamics/foundation-experiments:dilatazione-sostanze}}

\subsection{Esperienza di Torricelli}
\label{\detokenize{ch/thermodynamics/foundation-experiments:esperienza-di-torricelli}}

\subsection{Prime esperienze sui gas}
\label{\detokenize{ch/thermodynamics/foundation-experiments:prime-esperienze-sui-gas}}
\sphinxAtStartPar
Boyle


\subsection{Equilibrio termico}
\label{\detokenize{ch/thermodynamics/foundation-experiments:equilibrio-termico}}
\sphinxAtStartPar
…


\subsection{Scale di temperatura}
\label{\detokenize{ch/thermodynamics/foundation-experiments:scale-di-temperatura}}\begin{itemize}
\item {} 
\sphinxAtStartPar
Scale empiriche: costruite con scelte arbitrarie senza nessun significato fisico profondo

\item {} 
\sphinxAtStartPar
Scala termodinamica: la temperatura assoluta è direttamente legata all’agitazione dei componenti elementari della materia

\end{itemize}


\subsubsection{Scale empiriche}
\label{\detokenize{ch/thermodynamics/foundation-experiments:scale-empiriche}}
\sphinxAtStartPar
Metodo generale per la definizione delle scale di temperatura: scelta di due temperature di riferimento, facilmente riproducibili nei limiti di errori tollerati; suddivisione in parti uguali dell’intervallo ed estensione oltre questi limiti, tipicamente in 100 o 60 (o suoi multipli) parti.
\begin{itemize}
\item {} 
\sphinxAtStartPar
1702, Romer:
\begin{itemize}
\item {} 
\sphinxAtStartPar
estremo inferiore,  \(0 \, \text{°Ro}\): temperatura eutettica del cloruro di ammonio;

\item {} 
\sphinxAtStartPar
estremo superiore, \(60 \, \text{°Ro}\): temperatura di ebollizione dell’acqua
successivamente si accorse che la solidificazione dell’acqua avveniva circa a \(7.5 \, \text{°Ro}\) e decise di usare questa condizione per definire l’estremo inferiore, in modo tale da rendere più facile la taratura dello strumento

\end{itemize}

\item {} 
\sphinxAtStartPar
1709\sphinxhyphen{}15, Fahrenheit:
\begin{itemize}
\item {} 
\sphinxAtStartPar
definizione originale della scala:
\begin{itemize}
\item {} 
\sphinxAtStartPar
estremo inferiore,   \(0 \, \text{°F}\): temperatura eutettica del cloruro di ammonio; le malelingue sostengono la temperatura più bassa registrata negli inverni di Danzica, città allora prussiana in cui viveva mentre metteva a punto gli strumenti

\item {} 
\sphinxAtStartPar
estremo superiore,   \(96 \, \text{°F}\): temperatura media del corpo umano

\end{itemize}

\item {} 
\sphinxAtStartPar
le scelte rocambolesche e definite in maniera imprecisa non costituivano delle condizioni facilmente replicabili per la costruzione e/o taratura di nuovi strumenti. Vennero scelte quindi le condizioni di solidificazione (\(32 \, \text{°F}\)) e di evaporazione (\(212 \, \text{°F}\)) dell’acqua a pressione ambiente al livello del mare, in modo tale da suddividere tale intervallo in 180 sotto\sphinxhyphen{}intervalli

\end{itemize}

\item {} 
\sphinxAtStartPar
1731, de Réaumur:
\begin{itemize}
\item {} 
\sphinxAtStartPar
estremi: solidificazione (\(0 \, \text{°Re}\)) ed evaporazione (\(80 \, \text{°Re}\)) dell’acqua a temperatura ambiente. Perché 80 intervalli tra queste due condizioni? Perché il termometro costruito da Reaumur usava come principio fisico l’espansione dell’etanolo, e il volume dell’etanolo varia dell’8\% tra le due condizioni di riferimento scelte.

\end{itemize}

\item {} 
\sphinxAtStartPar
1742, Celsius:
\begin{itemize}
\item {} 
\sphinxAtStartPar
estremi: solidificazione (\(100 \, \text{°C}\)) ed evaporazione (\(0 \, \text{°C}\)) dell’acqua a temperatura ambiente. Perché questa definizione “invertita” rispetto alle altre? Perché no, si potrebbe rispondere. Per rendere più pratica la misura e adeguarsi al verso delle altre scale, un anno dopo la morte di Celsius, la scala fu invertita da Linneo (\sphinxstylestrong{todo} lo stesso Linneo, biologo, che si dilettava con la classificazione di piante e animali, padre della classificazione scientifica degli organismi viventi, usata tuttora)

\end{itemize}

\end{itemize}


\subsubsection{Scala termodinamica}
\label{\detokenize{ch/thermodynamics/foundation-experiments:scala-termodinamica}}
\sphinxAtStartPar
Scala di temperatura assoluta
\begin{itemize}
\item {} 
\sphinxAtStartPar
Esperimenti sui gas, estrapolando i dati sperimentali delle {\hyperref[\detokenize{ch/thermodynamics/ideal-gas-experiments:physics-hs-thermodynamics-matter-gases-ideal-experiments-charles}]{\sphinxcrossref{\DUrole{std,std-ref}{leggi di Charles}}}} e di {\hyperref[\detokenize{ch/thermodynamics/ideal-gas-experiments:physics-hs-thermodynamics-matter-gases-ideal-experiments-gay-lussac}]{\sphinxcrossref{\DUrole{std,std-ref}{Gay\sphinxhyphen{}Lussac}}}}

\item {} 
\sphinxAtStartPar
1848, Kelvin \sphinxstyleemphasis{On an Absolute Thermometric Scale}.

\end{itemize}


\subsection{Teoria cinetica dei gas}
\label{\detokenize{ch/thermodynamics/foundation-experiments:teoria-cinetica-dei-gas}}
\sphinxAtStartPar
1738, D.Bernoulli \sphinxstyleemphasis{Hydrodynamica}


\subsection{Calore latente e calore specifico}
\label{\detokenize{ch/thermodynamics/foundation-experiments:calore-latente-e-calore-specifico}}
\sphinxAtStartPar
1750\sphinxhyphen{}60, J.Black. I suoi studi sul calore aiutano a distinguere i concetti di temperatura e di calore \sphinxstylestrong{todo}
\begin{itemize}
\item {} 
\sphinxAtStartPar
sistemi fisici sul quale non viene compiuto lavoro, scambiano tra di loro calore per raggiungere l’equilibrio termico
\begin{itemize}
\item {} 
\sphinxAtStartPar
la quantità di calore “entrante” in un sistema, ne fa variare la temperatura. La variazione di temperatura nel sistema è inversamente proporzionale alla sua massa,
\begin{equation*}
\begin{split}m \, c_x \, d T = \delta Q \ ,\end{split}
\end{equation*}
\sphinxAtStartPar
la costante di proporzionalità è definita \sphinxstylestrong{calore specifico}. \sphinxstylestrong{todo} \sphinxstyleemphasis{controllare commenti su stato termodinamico \({\cdot}_x\) del sistema}

\item {} 
\sphinxAtStartPar
la quantità di calore scambiata tra due sistemi è uguale e opposta: \(d Q_{ij} = - d Q_{ji}\).
Mettendo a contatto due sistemi che non manifestano cambiamenti di fase, isolati dall’ambiente, si ottiene quindi
\begin{equation*}
\begin{split}\begin{cases}
      d E_i = m_i \, c_i \, d T_i = \delta Q_{ij} \\
      d E_j = m_j \, c_j \, d T_j = \delta Q_{ji} = - \delta Q_{ij}
    \end{cases}
    \end{split}
\end{equation*}\begin{equation*}
\begin{split}\rightarrow \qquad 0 = d E_i + d E_j = m_i \, c_i \, d T_i + m_j \, c_j \, d T_j\end{split}
\end{equation*}
\sphinxAtStartPar
\sphinxstylestrong{todo} \sphinxstyleemphasis{definire energia interna e aggiungere riferimento alla sezione “Princìpi della termodinamica”}

\end{itemize}

\item {} 
\sphinxAtStartPar
i cambiamenti di fase avvengono a temperatura costante. Ad esempio, l’apporto di calore a un sistema in equilibrio contenente ghiaccio alla temperatura di solidificazione non ne fa aumentare la temperatura, ma la massa liquida. L’aumento della temperatura. Una volta completata la trasformazione di fase, l’apporto di calore causa una variazione di temperatura,
\begin{equation*}
\begin{split}\delta Q = \begin{cases}
      d m_{l} \, L_{sl}                 \quad & , \quad {\delta m_l < m} \\
      d m_{l} \, L_{sl} + m \, c \, d T \quad & , \quad {\delta m_l = m} \ .
    \end{cases}\end{split}
\end{equation*}
\sphinxAtStartPar
Viene definito \sphinxstylestrong{calore latente di fusione} il coefficiente \(L_{sl}\) di proporzionalità tra il calore entrante nel sistema durante la trasformazione di fase e la quantità di massa liquefatta \(\delta m_l\).

\end{itemize}


\subsection{Esperienze sui gas, ed equazione di stato dei gas perfetti}
\label{\detokenize{ch/thermodynamics/foundation-experiments:esperienze-sui-gas-ed-equazione-di-stato-dei-gas-perfetti}}\begin{itemize}
\item {} 
\sphinxAtStartPar
Boyle: \(PV = \text{const.}\)

\item {} 
\sphinxAtStartPar
Charles: \(V \propto T\)

\item {} 
\sphinxAtStartPar
Gay\sphinxhyphen{}Lussac: \(P \propto T\)

\item {} 
\sphinxAtStartPar
Avogadro: \(V \propto n\)

\end{itemize}

\sphinxAtStartPar
L’equazione di stato dei gas perfetti riassume questi risultati
\begin{equation*}
\begin{split}\dfrac{P V}{T n} = R = \text{const.}\end{split}
\end{equation*}

\subsection{Lavoro, … \sphinxstylestrong{todo}}
\label{\detokenize{ch/thermodynamics/foundation-experiments:lavoro-todo}}
\begin{sphinxShadowBox}
\sphinxstylesidebartitle{}

\sphinxAtStartPar
\sphinxstylestrong{todo} Diatriba sulla priorità sulla formulazione del principio di conservazione dell’energia: von Meyer; Joule; successivamente Hemlholtz; Tyndall \sphinxhyphen{} scienziato e uno dei primi alpinisti \sphinxhyphen{} uno dei pochi a riconoscere il contributo di von Meyer
\end{sphinxShadowBox}
\begin{itemize}
\item {} 
\sphinxAtStartPar
Rivoluzione industriale

\item {} 
\sphinxAtStartPar
1798, B.Thompson, \sphinxstyleemphasis{An Inquiry Concerning the Source of the Heat Which is Excited by Friction}, oggi può essere interpretato come un lavoro che identificava l’attrito come fenomeno di dissipazione dell’energia meccanica “utile”/”macroscopica” e della sua conversione in calore;

\item {} 
\sphinxAtStartPar
1824, S.Carnot, \sphinxstyleemphasis{Riflessioni sulla forza motrice del fuoco}

\item {} 
\sphinxAtStartPar
1840\sphinxhyphen{}42, J.von Meyer, medico, chimico e fisico, intuisce il principio di conservazione dell’energia, \sphinxstyleemphasis{“che non può essere né creata né distrutta”} \sphinxstylestrong{ref} \sphinxstyleemphasis{Remarks on the Forces of Nature}, 1841

\item {} 
\sphinxAtStartPar
1842\sphinxhyphen{}45, J.P.Joule \sphinxstyleemphasis{The Mechanical Equivalent of Heat}

\item {} 
\sphinxAtStartPar
1850, R.Clausius

\end{itemize}


\subsection{Formalismo e prìncipi della termodinamica classica}
\label{\detokenize{ch/thermodynamics/foundation-experiments:formalismo-e-principi-della-termodinamica-classica}}
\sphinxAtStartPar
\sphinxstylestrong{todo}
\begin{itemize}
\item {} 
\sphinxAtStartPar
usando il formalismo di Gibbs:
\begin{itemize}
\item {} 
\sphinxAtStartPar
funzioni di stato (energia interna,…), regola delle fasi, spazio di fase,…

\end{itemize}

\item {} 
\sphinxAtStartPar
si possono formulare i prìncipi della termodinamica

\end{itemize}


\subsection{Meccanica statistica}
\label{\detokenize{ch/thermodynamics/foundation-experiments:meccanica-statistica}}\begin{itemize}
\item {} 
\sphinxAtStartPar
Maxwell

\item {} 
\sphinxAtStartPar
Gibbs

\item {} 
\sphinxAtStartPar
\sphinxstylestrong{Boltzmann}

\end{itemize}

\sphinxstepscope


\section{Termodinamica e teoria atomica}
\label{\detokenize{ch/thermodynamics/foundation-atomic-theory:termodinamica-e-teoria-atomica}}\label{\detokenize{ch/thermodynamics/foundation-atomic-theory:physics-hs-thermodynamics-foundation-atomic-theory}}\label{\detokenize{ch/thermodynamics/foundation-atomic-theory::doc}}

\subsection{Stati della materia}
\label{\detokenize{ch/thermodynamics/foundation-atomic-theory:stati-della-materia}}

\subsection{Cambiamenti di stato}
\label{\detokenize{ch/thermodynamics/foundation-atomic-theory:cambiamenti-di-stato}}

\subsection{Variabili di stato}
\label{\detokenize{ch/thermodynamics/foundation-atomic-theory:variabili-di-stato}}
\sphinxstepscope


\section{Problemi}
\label{\detokenize{ch/thermodynamics/foundation-problems:problemi}}\label{\detokenize{ch/thermodynamics/foundation-problems:physics-hs-thermodynamics-foundation-problems}}\label{\detokenize{ch/thermodynamics/foundation-problems::doc}}

\subsection{\sphinxstylestrong{todo} …}
\label{\detokenize{ch/thermodynamics/foundation-problems:todo}}


\begin{sphinxuseclass}{sd-container-fluid}
\begin{sphinxuseclass}{sd-sphinx-override}
\begin{sphinxuseclass}{sd-mb-4}
\begin{sphinxuseclass}{sd-row}
\begin{sphinxuseclass}{sd-g-2}
\begin{sphinxuseclass}{sd-g-xs-2}
\begin{sphinxuseclass}{sd-g-sm-2}
\begin{sphinxuseclass}{sd-g-md-2}
\begin{sphinxuseclass}{sd-g-lg-2}
\begin{sphinxuseclass}{sd-col}
\begin{sphinxuseclass}{sd-d-flex-row}
\begin{sphinxuseclass}{sd-col-8}
\begin{sphinxuseclass}{sd-col-xs-8}
\begin{sphinxuseclass}{sd-col-sm-8}
\begin{sphinxuseclass}{sd-col-md-8}
\begin{sphinxuseclass}{sd-col-lg-8}
\begin{sphinxuseclass}{sd-card}
\begin{sphinxuseclass}{sd-sphinx-override}
\begin{sphinxuseclass}{sd-w-100}
\begin{sphinxuseclass}{sd-shadow-sm}
\begin{sphinxuseclass}{sd-card-body}
\begin{sphinxuseclass}{sd-card-title}
\begin{sphinxuseclass}{sd-font-weight-bold}Problema … Manometro di Torricelli
\end{sphinxuseclass}
\end{sphinxuseclass}
\end{sphinxuseclass}
\end{sphinxuseclass}
\end{sphinxuseclass}
\end{sphinxuseclass}
\end{sphinxuseclass}
\end{sphinxuseclass}
\end{sphinxuseclass}
\end{sphinxuseclass}
\end{sphinxuseclass}
\end{sphinxuseclass}
\end{sphinxuseclass}
\end{sphinxuseclass}
\begin{sphinxuseclass}{sd-col}
\begin{sphinxuseclass}{sd-d-flex-row}
\begin{sphinxuseclass}{sd-col-4}
\begin{sphinxuseclass}{sd-col-xs-4}
\begin{sphinxuseclass}{sd-col-sm-4}
\begin{sphinxuseclass}{sd-col-md-4}
\begin{sphinxuseclass}{sd-col-lg-4}
\begin{sphinxuseclass}{sd-card}
\begin{sphinxuseclass}{sd-sphinx-override}
\begin{sphinxuseclass}{sd-w-100}
\begin{sphinxuseclass}{sd-shadow-sm}
\begin{sphinxuseclass}{sd-card-body}




\end{sphinxuseclass}
\end{sphinxuseclass}
\end{sphinxuseclass}
\end{sphinxuseclass}
\end{sphinxuseclass}
\end{sphinxuseclass}
\end{sphinxuseclass}
\end{sphinxuseclass}
\end{sphinxuseclass}
\end{sphinxuseclass}
\end{sphinxuseclass}
\end{sphinxuseclass}
\end{sphinxuseclass}
\end{sphinxuseclass}
\end{sphinxuseclass}
\end{sphinxuseclass}
\end{sphinxuseclass}
\end{sphinxuseclass}
\end{sphinxuseclass}
\end{sphinxuseclass}
\end{sphinxuseclass}\subsubsection*{Soluzione.}


\subsection{Calorimetria}
\label{\detokenize{ch/thermodynamics/foundation-problems:calorimetria}}


\begin{sphinxuseclass}{sd-container-fluid}
\begin{sphinxuseclass}{sd-sphinx-override}
\begin{sphinxuseclass}{sd-mb-4}
\begin{sphinxuseclass}{sd-row}
\begin{sphinxuseclass}{sd-g-2}
\begin{sphinxuseclass}{sd-g-xs-2}
\begin{sphinxuseclass}{sd-g-sm-2}
\begin{sphinxuseclass}{sd-g-md-2}
\begin{sphinxuseclass}{sd-g-lg-2}
\begin{sphinxuseclass}{sd-col}
\begin{sphinxuseclass}{sd-d-flex-row}
\begin{sphinxuseclass}{sd-col-8}
\begin{sphinxuseclass}{sd-col-xs-8}
\begin{sphinxuseclass}{sd-col-sm-8}
\begin{sphinxuseclass}{sd-col-md-8}
\begin{sphinxuseclass}{sd-col-lg-8}
\begin{sphinxuseclass}{sd-card}
\begin{sphinxuseclass}{sd-sphinx-override}
\begin{sphinxuseclass}{sd-w-100}
\begin{sphinxuseclass}{sd-shadow-sm}
\begin{sphinxuseclass}{sd-card-body}
\begin{sphinxuseclass}{sd-card-title}
\begin{sphinxuseclass}{sd-font-weight-bold}Problema … Calore latente \sphinxhyphen{} B.Franklin \sphinxstyleemphasis{Cooling by Evaporation}
\end{sphinxuseclass}
\end{sphinxuseclass}
\end{sphinxuseclass}
\end{sphinxuseclass}
\end{sphinxuseclass}
\end{sphinxuseclass}
\end{sphinxuseclass}
\end{sphinxuseclass}
\end{sphinxuseclass}
\end{sphinxuseclass}
\end{sphinxuseclass}
\end{sphinxuseclass}
\end{sphinxuseclass}
\end{sphinxuseclass}
\begin{sphinxuseclass}{sd-col}
\begin{sphinxuseclass}{sd-d-flex-row}
\begin{sphinxuseclass}{sd-col-4}
\begin{sphinxuseclass}{sd-col-xs-4}
\begin{sphinxuseclass}{sd-col-sm-4}
\begin{sphinxuseclass}{sd-col-md-4}
\begin{sphinxuseclass}{sd-col-lg-4}
\begin{sphinxuseclass}{sd-card}
\begin{sphinxuseclass}{sd-sphinx-override}
\begin{sphinxuseclass}{sd-w-100}
\begin{sphinxuseclass}{sd-shadow-sm}
\begin{sphinxuseclass}{sd-card-body}




\end{sphinxuseclass}
\end{sphinxuseclass}
\end{sphinxuseclass}
\end{sphinxuseclass}
\end{sphinxuseclass}
\end{sphinxuseclass}
\end{sphinxuseclass}
\end{sphinxuseclass}
\end{sphinxuseclass}
\end{sphinxuseclass}
\end{sphinxuseclass}
\end{sphinxuseclass}
\end{sphinxuseclass}
\end{sphinxuseclass}
\end{sphinxuseclass}
\end{sphinxuseclass}
\end{sphinxuseclass}
\end{sphinxuseclass}
\end{sphinxuseclass}
\end{sphinxuseclass}
\end{sphinxuseclass}\subsubsection*{Soluzione.}



\begin{sphinxuseclass}{sd-container-fluid}
\begin{sphinxuseclass}{sd-sphinx-override}
\begin{sphinxuseclass}{sd-mb-4}
\begin{sphinxuseclass}{sd-row}
\begin{sphinxuseclass}{sd-g-2}
\begin{sphinxuseclass}{sd-g-xs-2}
\begin{sphinxuseclass}{sd-g-sm-2}
\begin{sphinxuseclass}{sd-g-md-2}
\begin{sphinxuseclass}{sd-g-lg-2}
\begin{sphinxuseclass}{sd-col}
\begin{sphinxuseclass}{sd-d-flex-row}
\begin{sphinxuseclass}{sd-col-8}
\begin{sphinxuseclass}{sd-col-xs-8}
\begin{sphinxuseclass}{sd-col-sm-8}
\begin{sphinxuseclass}{sd-col-md-8}
\begin{sphinxuseclass}{sd-col-lg-8}
\begin{sphinxuseclass}{sd-card}
\begin{sphinxuseclass}{sd-sphinx-override}
\begin{sphinxuseclass}{sd-w-100}
\begin{sphinxuseclass}{sd-shadow-sm}
\begin{sphinxuseclass}{sd-card-body}
\begin{sphinxuseclass}{sd-card-title}
\begin{sphinxuseclass}{sd-font-weight-bold}Problema … Calore latente \sphinxhyphen{} J.Black \sphinxstyleemphasis{Lectures on the Elements of Chemistry: Delivered in the University of Edinburgh}
\end{sphinxuseclass}
\end{sphinxuseclass}\begin{itemize}
\item {} 
\sphinxAtStartPar
…

\item {} 
\sphinxAtStartPar
Durante i suoi esperimenti, J.Black scopre che una massa \(m\) di acqua a temperatura \(T_1 = 176 \, °F\) è necessaria e sufficiente per sciogliere una massa uguale di ghiaccio a temperatura costante \(T_0 = 32 \, °F\).

\item {} 
\sphinxAtStartPar
…

\end{itemize}

\end{sphinxuseclass}
\end{sphinxuseclass}
\end{sphinxuseclass}
\end{sphinxuseclass}
\end{sphinxuseclass}
\end{sphinxuseclass}
\end{sphinxuseclass}
\end{sphinxuseclass}
\end{sphinxuseclass}
\end{sphinxuseclass}
\end{sphinxuseclass}
\end{sphinxuseclass}
\begin{sphinxuseclass}{sd-col}
\begin{sphinxuseclass}{sd-d-flex-row}
\begin{sphinxuseclass}{sd-col-4}
\begin{sphinxuseclass}{sd-col-xs-4}
\begin{sphinxuseclass}{sd-col-sm-4}
\begin{sphinxuseclass}{sd-col-md-4}
\begin{sphinxuseclass}{sd-col-lg-4}
\begin{sphinxuseclass}{sd-card}
\begin{sphinxuseclass}{sd-sphinx-override}
\begin{sphinxuseclass}{sd-w-100}
\begin{sphinxuseclass}{sd-shadow-sm}
\begin{sphinxuseclass}{sd-card-body}


\end{sphinxuseclass}
\end{sphinxuseclass}
\end{sphinxuseclass}
\end{sphinxuseclass}
\end{sphinxuseclass}
\end{sphinxuseclass}
\end{sphinxuseclass}
\end{sphinxuseclass}
\end{sphinxuseclass}
\end{sphinxuseclass}
\end{sphinxuseclass}
\end{sphinxuseclass}
\end{sphinxuseclass}
\end{sphinxuseclass}
\end{sphinxuseclass}
\end{sphinxuseclass}
\end{sphinxuseclass}
\end{sphinxuseclass}
\end{sphinxuseclass}
\end{sphinxuseclass}
\end{sphinxuseclass}\subsubsection*{Soluzione.}



\begin{sphinxuseclass}{sd-container-fluid}
\begin{sphinxuseclass}{sd-sphinx-override}
\begin{sphinxuseclass}{sd-mb-4}
\begin{sphinxuseclass}{sd-row}
\begin{sphinxuseclass}{sd-g-2}
\begin{sphinxuseclass}{sd-g-xs-2}
\begin{sphinxuseclass}{sd-g-sm-2}
\begin{sphinxuseclass}{sd-g-md-2}
\begin{sphinxuseclass}{sd-g-lg-2}
\begin{sphinxuseclass}{sd-col}
\begin{sphinxuseclass}{sd-d-flex-row}
\begin{sphinxuseclass}{sd-col-8}
\begin{sphinxuseclass}{sd-col-xs-8}
\begin{sphinxuseclass}{sd-col-sm-8}
\begin{sphinxuseclass}{sd-col-md-8}
\begin{sphinxuseclass}{sd-col-lg-8}
\begin{sphinxuseclass}{sd-card}
\begin{sphinxuseclass}{sd-sphinx-override}
\begin{sphinxuseclass}{sd-w-100}
\begin{sphinxuseclass}{sd-shadow-sm}
\begin{sphinxuseclass}{sd-card-body}
\begin{sphinxuseclass}{sd-card-title}
\begin{sphinxuseclass}{sd-font-weight-bold}Problema … Calore latente \sphinxhyphen{} Refrigeratore evaporativo
\end{sphinxuseclass}
\end{sphinxuseclass}
\sphinxAtStartPar
Qualche applicazione…

\end{sphinxuseclass}
\end{sphinxuseclass}
\end{sphinxuseclass}
\end{sphinxuseclass}
\end{sphinxuseclass}
\end{sphinxuseclass}
\end{sphinxuseclass}
\end{sphinxuseclass}
\end{sphinxuseclass}
\end{sphinxuseclass}
\end{sphinxuseclass}
\end{sphinxuseclass}
\begin{sphinxuseclass}{sd-col}
\begin{sphinxuseclass}{sd-d-flex-row}
\begin{sphinxuseclass}{sd-col-4}
\begin{sphinxuseclass}{sd-col-xs-4}
\begin{sphinxuseclass}{sd-col-sm-4}
\begin{sphinxuseclass}{sd-col-md-4}
\begin{sphinxuseclass}{sd-col-lg-4}
\begin{sphinxuseclass}{sd-card}
\begin{sphinxuseclass}{sd-sphinx-override}
\begin{sphinxuseclass}{sd-w-100}
\begin{sphinxuseclass}{sd-shadow-sm}
\begin{sphinxuseclass}{sd-card-body}




\end{sphinxuseclass}
\end{sphinxuseclass}
\end{sphinxuseclass}
\end{sphinxuseclass}
\end{sphinxuseclass}
\end{sphinxuseclass}
\end{sphinxuseclass}
\end{sphinxuseclass}
\end{sphinxuseclass}
\end{sphinxuseclass}
\end{sphinxuseclass}
\end{sphinxuseclass}
\end{sphinxuseclass}
\end{sphinxuseclass}
\end{sphinxuseclass}
\end{sphinxuseclass}
\end{sphinxuseclass}
\end{sphinxuseclass}
\end{sphinxuseclass}
\end{sphinxuseclass}
\end{sphinxuseclass}\subsubsection*{Soluzione.}



\begin{sphinxuseclass}{sd-container-fluid}
\begin{sphinxuseclass}{sd-sphinx-override}
\begin{sphinxuseclass}{sd-mb-4}
\begin{sphinxuseclass}{sd-row}
\begin{sphinxuseclass}{sd-g-2}
\begin{sphinxuseclass}{sd-g-xs-2}
\begin{sphinxuseclass}{sd-g-sm-2}
\begin{sphinxuseclass}{sd-g-md-2}
\begin{sphinxuseclass}{sd-g-lg-2}
\begin{sphinxuseclass}{sd-col}
\begin{sphinxuseclass}{sd-d-flex-row}
\begin{sphinxuseclass}{sd-col-8}
\begin{sphinxuseclass}{sd-col-xs-8}
\begin{sphinxuseclass}{sd-col-sm-8}
\begin{sphinxuseclass}{sd-col-md-8}
\begin{sphinxuseclass}{sd-col-lg-8}
\begin{sphinxuseclass}{sd-card}
\begin{sphinxuseclass}{sd-sphinx-override}
\begin{sphinxuseclass}{sd-w-100}
\begin{sphinxuseclass}{sd-shadow-sm}
\begin{sphinxuseclass}{sd-card-body}
\begin{sphinxuseclass}{sd-card-title}
\begin{sphinxuseclass}{sd-font-weight-bold}Problema … Calore latente \sphinxhyphen{} Sudore e termoregolazione
\end{sphinxuseclass}
\end{sphinxuseclass}
\sphinxAtStartPar
Refrigerazione evaporativa tramite evaporazione del sudore sulla pelle

\end{sphinxuseclass}
\end{sphinxuseclass}
\end{sphinxuseclass}
\end{sphinxuseclass}
\end{sphinxuseclass}
\end{sphinxuseclass}
\end{sphinxuseclass}
\end{sphinxuseclass}
\end{sphinxuseclass}
\end{sphinxuseclass}
\end{sphinxuseclass}
\end{sphinxuseclass}
\begin{sphinxuseclass}{sd-col}
\begin{sphinxuseclass}{sd-d-flex-row}
\begin{sphinxuseclass}{sd-col-4}
\begin{sphinxuseclass}{sd-col-xs-4}
\begin{sphinxuseclass}{sd-col-sm-4}
\begin{sphinxuseclass}{sd-col-md-4}
\begin{sphinxuseclass}{sd-col-lg-4}
\begin{sphinxuseclass}{sd-card}
\begin{sphinxuseclass}{sd-sphinx-override}
\begin{sphinxuseclass}{sd-w-100}
\begin{sphinxuseclass}{sd-shadow-sm}
\begin{sphinxuseclass}{sd-card-body}




\end{sphinxuseclass}
\end{sphinxuseclass}
\end{sphinxuseclass}
\end{sphinxuseclass}
\end{sphinxuseclass}
\end{sphinxuseclass}
\end{sphinxuseclass}
\end{sphinxuseclass}
\end{sphinxuseclass}
\end{sphinxuseclass}
\end{sphinxuseclass}
\end{sphinxuseclass}
\end{sphinxuseclass}
\end{sphinxuseclass}
\end{sphinxuseclass}
\end{sphinxuseclass}
\end{sphinxuseclass}
\end{sphinxuseclass}
\end{sphinxuseclass}
\end{sphinxuseclass}
\end{sphinxuseclass}\subsubsection*{Soluzione.}

\sphinxstepscope


\chapter{Princìpi della termodinamica}
\label{\detokenize{ch/thermodynamics/principles:principi-della-termodinamica}}\label{\detokenize{ch/thermodynamics/principles:physics-hs-thermodynamics-foundation-principles}}\label{\detokenize{ch/thermodynamics/principles::doc}}
\sphinxAtStartPar
Sistemare come presentazione! I contenuti vengono divisi nelle sezioni successive.

\sphinxAtStartPar
In questa sezione vengono presentati i princìpi fondamentali della termodinamica classica. \sphinxstylestrong{todo}

\sphinxAtStartPar
\sphinxstylestrong{Principio di conservazione della massa.}
Nell’ambito della fisica classica, la massa di un sistema chiuso è costante.

\sphinxAtStartPar
\sphinxstylestrong{Primo principio della termodinamica \sphinxhyphen{} bilancio dell’energia totale.}
Il primo principio della termodinamica rappresenta il bilancio di energia totale per un sistema chiuso (\sphinxstylestrong{todo} \sphinxstyleemphasis{riferimenti a sistemi aperti e chiusi}),
\begin{equation*}
\begin{split}d E^{tot} = \delta L^{ext} + \delta Q^{ext} \ . \end{split}
\end{equation*}
\sphinxAtStartPar
Usando il teorema dell’energia cinetica (\sphinxstylestrong{todo} riferimento alla meccanica), \(dK = \delta L^{ext} + \delta L^{int}\), e la definizione di energia interna come differenza tra energia totale ed energia cinetica macroscopica, \(E := E^{tot} - K\),
\begin{equation*}
\begin{split}d E = - \delta L^{int} + \delta Q^{ext} \ .\end{split}
\end{equation*}
\sphinxAtStartPar
\sphinxstylestrong{Regola delle fasi di Gibbs.}
L’energia interna può essere scritta come funzione di stato, \(E(S, X_k)\), \sphinxstylestrong{todo} con variabili indipendenti …
\begin{equation*}
\begin{split}\begin{aligned}
dE & = \left(\dfrac{\partial E}{\partial S}\right)_{\mathbf{X}} d S 
     + \left(\dfrac{\partial E}{\partial X_k}\right)_{S} d X_k  = \\
   & = T \, d S + \sum_k F_k \, d X_k
\end{aligned}\end{split}
\end{equation*}
\sphinxAtStartPar
La variazione di energia interna rispetto alla variabile \(S\) corrisponde alla temperatura,
\begin{equation*}
\begin{split}T = \left(\dfrac{\partial E}{\partial S}\right)_{\mathbf{X}} \ge 0 \ .\end{split}
\end{equation*}
\sphinxAtStartPar
\sphinxstylestrong{Secondo principio della termodinamica \sphinxhyphen{} irreversibilità.}
\begin{itemize}
\item {} 
\sphinxAtStartPar
Secondo principio per sistemi semplici \sphinxstylestrong{todo} \sphinxstyleemphasis{temperatura uniforme}
\begin{equation*}
\begin{split}\begin{aligned}
    dE & = \delta Q^{ext} - \delta L^{int} = \\
       & = \underbrace{\delta Q^{ext} + \delta^+ D}_{\delta U} - \delta L^{int, rev} = \\
  \end{aligned}\end{split}
\end{equation*}\begin{equation*}
\begin{split}\begin{cases}
  -\delta L^{int,rev} & = \displaystyle\sum_k F_k \, d X_k \\
  \delta U            & = T \, dS
  \end{cases}\end{split}
\end{equation*}
\end{itemize}

\sphinxstepscope


\section{Principio di Lavoisier}
\label{\detokenize{ch/thermodynamics/principles-lavoisier:principio-di-lavoisier}}\label{\detokenize{ch/thermodynamics/principles-lavoisier:physics-hs-thermodynamics-foundation-principles-lavoisier}}\label{\detokenize{ch/thermodynamics/principles-lavoisier::doc}}
\sphinxAtStartPar
Il principio di Lavoisier descrive la conservazione della massa per sistemi chiusi,
\begin{equation*}
\begin{split}d M = 0 \ .\end{split}
\end{equation*}
\sphinxstepscope


\section{Primo principio della termodinamica}
\label{\detokenize{ch/thermodynamics/principles-first:primo-principio-della-termodinamica}}\label{\detokenize{ch/thermodynamics/principles-first:physics-hs-thermodynamics-foundation-principles-first}}\label{\detokenize{ch/thermodynamics/principles-first::doc}}
\sphinxAtStartPar
Il primo principio della termodinamica è il bilancio di energia totale per sistemi chiusi,
\begin{equation*}
\begin{split}d E^{tot} = \delta L^{ext} + \delta Q^{ext}\end{split}
\end{equation*}
\sphinxstepscope


\section{Energia interna e regola delle fasi di Gibbs}
\label{\detokenize{ch/thermodynamics/principles-gibbs-phase-rule:energia-interna-e-regola-delle-fasi-di-gibbs}}\label{\detokenize{ch/thermodynamics/principles-gibbs-phase-rule:physics-hs-thermodynamics-foundation-principles-gibbs-phase-rule}}\label{\detokenize{ch/thermodynamics/principles-gibbs-phase-rule::doc}}

\subsection{Energia interna}
\label{\detokenize{ch/thermodynamics/principles-gibbs-phase-rule:energia-interna}}
\sphinxAtStartPar
Gibbs definisce l’energia interna del sistema come differenza tra la sua energia totale e l’energia cinetica “macroscopica”,
\begin{equation*}
\begin{split}E = E^{tot} - K \ .\end{split}
\end{equation*}
\sphinxAtStartPar
Il bilancio dell’energia interna viene ricavato come differenza dei bilanci dell’energia totale, descritto dal primo principio della termodinamica
\begin{equation*}
\begin{split}d E^{tot} = \delta L^{ext} + \delta Q^{ext} \ ,\end{split}
\end{equation*}
\sphinxAtStartPar
e il bilancio dell’energia cinetica, fornito dal teorema dell’energia cinetica ricavato in meccanica,
\begin{equation*}
\begin{split}d K = \delta L^{ext} + \delta L^{int} \ .\end{split}
\end{equation*}
\sphinxAtStartPar
Il bilancio dell’energia interna diventa quindi
\begin{equation*}
\begin{split}d E = \delta Q^{ext} - \delta L^{int} \ .\end{split}
\end{equation*}

\subsection{Variabili di stato e regola delle fasi}
\label{\detokenize{ch/thermodynamics/principles-gibbs-phase-rule:variabili-di-stato-e-regola-delle-fasi}}
\sphinxAtStartPar
\sphinxstylestrong{todo} \sphinxstyleemphasis{def di variabile di stato}
Lo stato termodinamico di un sistema può essere descritto da un numero definito di variabili termodinamiche indipendenti, descritto dalla \sphinxstylestrong{regola delle fasi di Gibbs},
\begin{equation*}
\begin{split}F = C - P + 1 + W \ ,\end{split}
\end{equation*}
\sphinxAtStartPar
cioè il numero di variabili indipendenti (o gradi di libertà), \(F\), di un sistema è una funzione del numero di componenti \(C\) di un sistema, il numero di fasi \(P\) e il numero \(W\) di modi del sistema di manifestare lavoro interno, come ad esempio:
\begin{itemize}
\item {} 
\sphinxAtStartPar
sforzi meccanici interni

\item {} 
\sphinxAtStartPar
contributo della tensione superficiale

\item {} 
\sphinxAtStartPar
energia dei legami delle molecole dei componenti

\item {} 
\sphinxAtStartPar
contributo del campo elettromagnetico

\end{itemize}

\sphinxAtStartPar
\sphinxstylestrong{Esempi.}
\begin{itemize}
\item {} 
\sphinxAtStartPar
In un sistema composto da un gas comprimibile, monocomponente e monofase (gassosa), elettricamente neutro, \sphinxstylestrong{todo} \sphinxstyleemphasis{altro?}, l’unica forma di lavoro interno è quello legato alla compressione, \(\delta L^{int,rev} = P dV\), e quindi \(W = 1\). Per questo sistema servono quindi,
\begin{equation*}
\begin{split}F = C - P + 1 + W = 1 - 1 + 1 + 1 = 2 \ ,\end{split}
\end{equation*}
\sphinxAtStartPar
variabili di stato per definire lo stato del sistema.

\item {} 
\sphinxAtStartPar
Solido elastico, … \sphinxstylestrong{todo}

\item {} 
\sphinxAtStartPar
Miscela reattiva di gas

\item {} 
\sphinxAtStartPar
Sistema monocomponente durante una transizione di fase
\begin{itemize}
\item {} 
\sphinxAtStartPar
transizione di fase del primo ordine

\item {} 
\sphinxAtStartPar
punto critico

\end{itemize}

\item {} 
\sphinxAtStartPar
Fasi nelle miscele solide

\item {} 
\sphinxAtStartPar
Campo elettrico e magnetizzazione

\end{itemize}


\subsection{Primo principio in termini delle variabili di stato}
\label{\detokenize{ch/thermodynamics/principles-gibbs-phase-rule:primo-principio-in-termini-delle-variabili-di-stato}}
\sphinxAtStartPar
L’energia interna è una variabile di stato che, secondo la regola delle fasi, può essere scritta come
\begin{equation*}
\begin{split}E(S, X_k) \ ,\end{split}
\end{equation*}
\sphinxAtStartPar
avendo indicato con \(X_k\) tutte le variabili (\sphinxstylestrong{todo} \sphinxstyleemphasis{di stato?}) la cui variazione è associata a un lavoro interno reversibile, ed \(S\) la variabile di stato la cui variazione è associata al calore scambiato con l’ambiente esterno e alle azioni interne dissipative. Il differenziale \sphinxhyphen{} \sphinxstylestrong{esatto} \sphinxhyphen{} dell’energia interna
\begin{equation*}
\begin{split}\begin{aligned}
dE & = \left(\dfrac{\partial E}{\partial S}\right)_{\mathbf{X}} d S 
     + \left(\dfrac{\partial E}{\partial X_k}\right)_{S} d X_k  = \\
   & = T \, d S + \sum_k F_k \, d X_k
\end{aligned}\end{split}
\end{equation*}
\sphinxAtStartPar
può essere confrontato con il bilancio dell’energia interna,
\begin{equation*}
\begin{split}\begin{aligned}
  d E & = \delta Q^{ext} + \delta L^{int} = \\
      & = \delta Q^{ext} + \delta^+ D - \delta L^{int,rev} \ .
\end{aligned}\end{split}
\end{equation*}
\sphinxAtStartPar
Poiché \(d E\) è un differenziale esatto e \(\delta L^{int,rev}\) è un contributo reversibile, segue che la somma dei due contributi in generale non reversibili, \(\delta U := \delta Q^{ext} + \delta^+ D\), è un contributo reversibile. Confrontando le due espressioni del differenziale dell’energia interna, si può associare questo contributo al termine \(T \, dS\), il lavoro interno reversibile alla somma dei contributi formati come prodotto delle forze generalizzate \(F_k\) e le variazioni delle variabili di stato \(X_k\),
\begin{equation*}
\begin{split}\begin{cases}
  -\delta L^{int,rev} & = \displaystyle\sum_k F_k \, d X_k \\
  \delta U            & = T \, dS
\end{cases}\end{split}
\end{equation*}
\sphinxstepscope


\section{Diagrammi termodinamici}
\label{\detokenize{ch/thermodynamics/principles-phase-diagrams:diagrammi-termodinamici}}\label{\detokenize{ch/thermodynamics/principles-phase-diagrams:physics-hs-thermodynamics-foundation-principles-phase-diagrams}}\label{\detokenize{ch/thermodynamics/principles-phase-diagrams::doc}}

\subsection{Diagramma di stato di un sistema mono\sphinxhyphen{}componente}
\label{\detokenize{ch/thermodynamics/principles-phase-diagrams:diagramma-di-stato-di-un-sistema-mono-componente}}
\sphinxAtStartPar
Si consideri un sistema ad un componente, in grado di scambiare calore e con un unico modo di manifestare il lavoro reversibile interno al sistema, quello meccanico dovuto a un’espansione isotropa del volume, \(\delta L^{int,rev} = P \, d V\).

\sphinxAtStartPar
Come applicarla? \(P\), \(T\) e concentrazioni? In una rappresentazione 3D del diagramma di stato, la superficie di stato è una superficie 2D, anche durante le transizioni di fase: queste sono descritte dalla composizione delle fasi, come frazioni molari. Queste non sono contate come gradi di libertà? I \sphinxstylestrong{gradi di libertà} sono cons le proprietà termodinamiche \sphinxstylestrong{intensive} indipendenti, come P,T
Applicando la regola delle fasi di Gibbs,…

\sphinxAtStartPar
Diagramma 3D di sistema monocomponente
\begin{itemize}
\item {} 
\sphinxAtStartPar
regione di una superficie 2D: ha 2 gradi di libertà

\item {} 
\sphinxAtStartPar
curva sulla superficie di stato, es: trasformazione termodinamica: ha un grado di libertà

\item {} 
\sphinxAtStartPar
punto: stato TD determinato, es. punto triplo, eutettico: zero gradi di libertà

\end{itemize}


\subsection{Piani termodinamici}
\label{\detokenize{ch/thermodynamics/principles-phase-diagrams:piani-termodinamici}}
\sphinxAtStartPar
Piani termodinamici: proiezioni 2D di un diagramma multi\sphinxhyphen{}dimensionale
Esempi:
\begin{itemize}
\item {} 
\sphinxAtStartPar
piano P\sphinxhyphen{}V, Clapeyron

\item {} 
\sphinxAtStartPar
piano T\sphinxhyphen{}S, entropico

\item {} 
\sphinxAtStartPar
piano H\sphinxhyphen{}S, Mollier

\item {} 
\sphinxAtStartPar
…

\end{itemize}


\subsubsection{Piano di Clapeyron, P\sphinxhyphen{}V}
\label{\detokenize{ch/thermodynamics/principles-phase-diagrams:piano-di-clapeyron-p-v}}
\sphinxAtStartPar
\sphinxstylestrong{Lavoro.}
Nel caso di \sphinxstylestrong{sistemi chiusi} e \sphinxstylestrong{processi ideali}, il primo principio della termodinamica viene scritto
\begin{equation*}
\begin{split}\begin{aligned}
  d E & = - \delta L^{int,rev} + \delta Q^{ext} = \\
      & = - P \, dV + T \, dS \ .
\end{aligned}\end{split}
\end{equation*}
\sphinxAtStartPar
Nel caso in cui il contributo dell’\sphinxstylestrong{energia cinetica sia trascurabile}, il lavoro compiuto dal sistema sull’ambiente esterno coincide con
\begin{equation*}
\begin{split}\delta L^{done} = - \delta L^{ext} = - d K + \delta L^{int} \approx \delta L^{int} \approx \delta L^{int,rev} =  P \, dV\end{split}
\end{equation*}
\sphinxAtStartPar
Un sistema che compie una trasformazione termodinamica descritta dalla curva \(\gamma\) nel piano \(P-V\) di Clapeyron, compie un lavoro verso l’ambiente esterno che è uguale alla somma dei contributi elementari \sphinxhyphen{} e quindi l’integrale
\begin{equation*}
\begin{split}L^{done} = \int_{\gamma} \delta L^{done} \approx \int_{\gamma} P \, d V \ ,\end{split}
\end{equation*}
\sphinxAtStartPar
che ha l’immediata rappresentazione grafica corrispondente all’area (con segno) tra il grafico della trasformazione e l’asse delle ascisse, \(P=0\).

\sphinxAtStartPar
\sphinxstylestrong{Esempi di trasformazioni.}
…


\subsubsection{Piano entropico, T\sphinxhyphen{}S}
\label{\detokenize{ch/thermodynamics/principles-phase-diagrams:piano-entropico-t-s}}
\sphinxAtStartPar
Nel caso di trasformazioni ideali, il calore entrante nel sistema può essere identificato con il termines
\begin{equation*}
\begin{split}\delta Q^{ext} = T \, d S - \underbrace{\delta^+ D}_{=0 \text{ ideal, rev.}} = T \, dS \ ..\end{split}
\end{equation*}
\sphinxAtStartPar
Un sistema chiuso che compie una trasformazione termodinamica descritta dalla curva \(\gamma\) nel piano \(P-V\) di Clapeyron, assorbe calore dall’ambiente esterno che è uguale alla somma dei contributi elementari \sphinxhyphen{} e quindi l’integrale
\begin{equation*}
\begin{split}Q^{ext} = \int_{\gamma} \delta Q^{ext} \approx \int_{\gamma} T \, d S ,\end{split}
\end{equation*}
\sphinxAtStartPar
che ha l’immediata rappresentazione grafica corrispondente all’area (con segno) tra il grafico della trasformazione e l’asse delle ascisse, \(T=0\).

\sphinxstepscope


\section{Secondo principio della termodinamica}
\label{\detokenize{ch/thermodynamics/principles-second:secondo-principio-della-termodinamica}}\label{\detokenize{ch/thermodynamics/principles-second:physics-hs-thermodynamics-foundation-principles-second}}\label{\detokenize{ch/thermodynamics/principles-second::doc}}

\subsection{Enunciato di Clausius}
\label{\detokenize{ch/thermodynamics/principles-second:enunciato-di-clausius}}

\subsubsection{Sistemi semplici}
\label{\detokenize{ch/thermodynamics/principles-second:sistemi-semplici}}
\sphinxAtStartPar
La variazione elementare di entropia \(d S\) di un sistema a temperatura uniforme \(T\) è maggiore o uguale al rapporto tra il flusso di calore elementare introdotto nel sistema e la temperatura del sistema stesso,
\begin{equation*}
\begin{split}dS = \underbrace{\dfrac{\delta^+ D}{T}}_{\ge 0} + \dfrac{\delta Q^{ext}}{T} \ge \dfrac{\delta Q^{ext}}{T} \ .\end{split}
\end{equation*}
\sphinxAtStartPar
Questo è l’enunciato di Clausius del secondo principio della termodinamica per sistemi semplici con temperatura omogenea.


\subsubsection{Sistemi composti}
\label{\detokenize{ch/thermodynamics/principles-second:sistemi-composti}}
\sphinxAtStartPar
\sphinxstylestrong{todo} definizione di sistema composto. Avviene conduzione tra i sotto\sphinxhyphen{}sistemi.

\sphinxAtStartPar
Definendo l’entropia come una proprietà additiva, l’entropia di un sistema composto da \(N\) sotto\sphinxhyphen{}sistemi semplici è la somma dell’entropia dei sotto\sphinxhyphen{}sistemi,
\begin{equation*}
\begin{split}S = \sum_{n=1:N} S_n \ .\end{split}
\end{equation*}
\sphinxAtStartPar
Il bilancio dell’entropia del singolo sotto\sphinxhyphen{}sistema che scambia calore con gli altri sotto\sphinxhyphen{}sistemi e l’ambiente esterno viene scritto come
\begin{equation*}
\begin{split}\begin{aligned}
    dS_i & = \dfrac{\delta Q^{ext,i}_i}{T_i} - \dfrac{\delta D_i}{T_i} = \\
         & = \dfrac{\delta Q^{ext}_i}{T_i} + \dfrac{\sum_{k \ne i} \delta Q_{ik}}{T_i} - \dfrac{\delta D_i}{T_i} \ge \\
         & \ge \dfrac{\delta Q^{ext}_i}{T_i} + \dfrac{\sum_{k \ne i} \delta Q_{ik}}{T_i} \ . 
  \end{aligned}\end{split}
\end{equation*}
\sphinxAtStartPar
Il bilancio dell’entropia dell’intero sistema viene ricavato sommando i bilanci dell’entropia dei singoli sotto\sphinxhyphen{}sistemi,
\begin{equation*}
\begin{split}\begin{aligned}
    dS & = \sum_i d S_i \ge \\
       & \ge \sum_i \left\{ \dfrac{\delta Q^{ext}_i}{T_i} + \dfrac{\sum_{k \ne i} \delta Q_{ik}}{T_i} \right\} = \\
       & = \sum_i \dfrac{\delta Q^{ext}_i}{T_i} + \underbrace{\sum_{\left\{i,k\right\}} \delta Q_{ik} \left( \dfrac{1}{T_i} - \dfrac{1}{T_k} \right)}_{\ge 0} \ge \\
       & \ge \sum_i \dfrac{\delta Q^{ext}_i}{T_i} \ . 
  \end{aligned}\end{split}
\end{equation*}
\sphinxAtStartPar
avendo usato la relazione che rappresenta la tendenza naturale della trasmissione del calore “da un sistema a temperatura maggiore a un sistema a temperatura minore”,
\begin{equation*}
\begin{split}\delta Q_{ik} \left( \dfrac{1}{T_i} - \dfrac{1}{T_k} \right) \ge 0 \ .\end{split}
\end{equation*}
\sphinxAtStartPar
\sphinxstylestrong{todo} \sphinxstyleemphasis{aggiungere riferimento alla tendenza naturale nella trasmissione del calore}


\subsection{Aumento dell’entropia nell’universo}
\label{\detokenize{ch/thermodynamics/principles-second:aumento-dell-entropia-nell-universo}}
\sphinxAtStartPar
Se consideriamo l’universo come il sistema chiuso e isolato (ma sarà vero? E chi lo sa? Forse è sensato che lo sia, ma tante cose che sembrano sensate oggi saranno fregnacce tra qualche anno) formato da un sistema di interesse \(sys\) e dall’ambiente esterno \(env\).

\sphinxAtStartPar
La variazione dell’entropia dell’universo è la somma della variazione nel sistema e nell’ambiente esterno. Si indica con \(\delta Q_{sys,env}\) il flusso di calore che, se positivo, fa aumentare l’energia del sistema e diminuire quella dell’ambiente esterno. Assumendo che i due sotto\sphinxhyphen{}sistemi siano internamente omogenei,
\begin{equation*}
\begin{split}\begin{aligned}
d S^{univ} & = d S^{sys} + d S^{env} = \\
           & = \dfrac{\delta Q_{sys,env}}{T^{sys}} + \dfrac{\delta Q_{env,sys}}{T^{env}} = \\
           & = \dfrac{\delta Q_{sys,env}}{T^{sys}} - \dfrac{\delta Q_{sys,env}}{T^{env}} = \\
           & = \delta Q_{sys,env} \left( \dfrac{1}{T^{sys}} - \dfrac{1}{T^{env}} \right) \ge 0 \ ,
\end{aligned}\end{split}
\end{equation*}
\sphinxAtStartPar
si ottiene la relazione
\begin{equation*}
\begin{split}dS^{univ} \ge 0 \ ,\end{split}
\end{equation*}
\sphinxAtStartPar
che prevede la “non\sphinxhyphen{}diminuzione” dell’entropia dell’universo.

\sphinxstepscope


\section{Sistemi aperti}
\label{\detokenize{ch/thermodynamics/principles-open:sistemi-aperti}}\label{\detokenize{ch/thermodynamics/principles-open:physics-hs-thermodynamics-foundation-principles-open}}\label{\detokenize{ch/thermodynamics/principles-open::doc}}
\sphinxstepscope


\chapter{Stati della materia}
\label{\detokenize{ch/thermodynamics/matter:stati-della-materia}}\label{\detokenize{ch/thermodynamics/matter:physics-hs-thermodynamics-matter}}\label{\detokenize{ch/thermodynamics/matter::doc}}
\sphinxstepscope


\section{Gas ideali}
\label{\detokenize{ch/thermodynamics/ideal-gas:gas-ideali}}\label{\detokenize{ch/thermodynamics/ideal-gas:physics-hs-thermodynamics-matter-gases-ideal}}\label{\detokenize{ch/thermodynamics/ideal-gas::doc}}
\sphinxAtStartPar
Il modello di gas ideale rappresenta un gas in cui
\begin{itemize}
\item {} 
\sphinxAtStartPar
le molecole hanno volume trascurabile rispetto al volume disponibile

\item {} 
\sphinxAtStartPar
le molecole non interagiscono tra di loro e interagiscono le pareti solide di un contenitore con urti perfettamente elastici; è possibile rilassare l’ipotesi di assenza di interazioni tra le particelle con l’ipotesi di interazioni perfettamente elastiche

\item {} 
\sphinxAtStartPar
le molecole sono identiche

\item {} 
\sphinxAtStartPar
il moto delle molecole è casuale e isotropo, cioè non esistono direzioni preferenziali del moto

\end{itemize}

\sphinxAtStartPar
Il modello di gas ideale può essere un buon modello per gas:
\begin{itemize}
\item {} 
\sphinxAtStartPar
\sphinxstylestrong{alta temperatura e molecole semplici}: l’energia cinetica delle molecole rende trascurabile l’energia delle forze intermolecolari tra molecole distanti; l’interazione tra molecole semplici o non\sphinxhyphen{}polari è debole se confrontata rispetto a molecole complesse o polari

\item {} 
\sphinxAtStartPar
\sphinxstylestrong{bassa pressione e bassa densità}: la bassa concentrazione di molecole rende le loro interazioni rare.

\end{itemize}

\sphinxstepscope


\subsection{Esperimenti}
\label{\detokenize{ch/thermodynamics/ideal-gas-experiments:esperimenti}}\label{\detokenize{ch/thermodynamics/ideal-gas-experiments:physics-hs-thermodynamics-matter-gases-ideal-experiments}}\label{\detokenize{ch/thermodynamics/ideal-gas-experiments::doc}}
\sphinxAtStartPar
\sphinxstylestrong{todo} Aggiungere immagini e grafici dei dati sperimentali per le leggi di Charles e Gay\sphinxhyphen{}Lussac con estrapolazione verso lo zero assoluto.


\subsubsection{Esperimenti e leggi}
\label{\detokenize{ch/thermodynamics/ideal-gas-experiments:esperimenti-e-leggi}}

\paragraph{Legge di Boyle}
\label{\detokenize{ch/thermodynamics/ideal-gas-experiments:legge-di-boyle}}\label{\detokenize{ch/thermodynamics/ideal-gas-experiments:physics-hs-thermodynamics-matter-gases-ideal-experiments-boyle}}
\sphinxAtStartPar
Per gas semplici, a temperatura sufficientemente elevata, e pressione sufficientemente ridotta
\begin{equation*}
\begin{split}T, n \text{ const} \quad \rightarrow \quad P V = \text{const}\end{split}
\end{equation*}

\paragraph{Legge di Charles}
\label{\detokenize{ch/thermodynamics/ideal-gas-experiments:legge-di-charles}}\label{\detokenize{ch/thermodynamics/ideal-gas-experiments:physics-hs-thermodynamics-matter-gases-ideal-experiments-charles}}
\sphinxAtStartPar
Per gas semplici, a temperatura sufficientemente elevata, e pressione sufficientemente ridotta
\begin{equation*}
\begin{split}P, n \text{ const} \quad \rightarrow \quad \dfrac{\Delta V}{\Delta T} = V_0 \, \alpha_P = \text{const}\end{split}
\end{equation*}\begin{equation*}
\begin{split}V = V_0 \, ( 1 + \alpha_P \, T ) \ ,\end{split}
\end{equation*}
\sphinxAtStartPar
avendo indicato con \(\alpha_0\) il coefficiente di dilatazione termica a pressione costante.

\sphinxAtStartPar
I dati sperimentali misurati mostrano un andamento lineare, e la loro estrapolazione verso il valore limite del volume \(V = 0\) porta a un valore di temperatura \(T = -273.15 \text{°C}\).


\paragraph{Legge di Gay\sphinxhyphen{}Lussac}
\label{\detokenize{ch/thermodynamics/ideal-gas-experiments:legge-di-gay-lussac}}\label{\detokenize{ch/thermodynamics/ideal-gas-experiments:physics-hs-thermodynamics-matter-gases-ideal-experiments-gay-lussac}}
\sphinxAtStartPar
Per gas semplici, a temperatura sufficientemente elevata, e pressione sufficientemente ridotta
\begin{equation*}
\begin{split}V, n \text{ const} \quad \rightarrow \quad \frac{\Delta P}{\Delta T} = P_0 \, k_V = \text{const}\end{split}
\end{equation*}\begin{equation*}
\begin{split}P = P_0 \, ( 1 + k_V \, T ) \ ,\end{split}
\end{equation*}
\sphinxAtStartPar
I dati sperimentali misurati mostrano un andamento lineare, e la loro estrapolazione verso il valore limite del volume \(P = 0\) porta allo stesso valore di temperatura \(T = -273.15 \text{°C}\) trovato nell’esperimento di Charles.


\subparagraph{Scala di temperatura assoluta}
\label{\detokenize{ch/thermodynamics/ideal-gas-experiments:scala-di-temperatura-assoluta}}\label{\detokenize{ch/thermodynamics/ideal-gas-experiments:physics-hs-thermodynamics-matter-gases-ideal-experiments-t-kelvin}}
\sphinxAtStartPar
Questa osservazione porta alla scelta di una nuova scala di temperatura, quella che diverrà la scala di temperatura termodinamica, o assoluta, di Kelvin:
\begin{itemize}
\item {} 
\sphinxAtStartPar
viene definito il punto a temperatura, \(0 \text{ K} = -273.15 \text{°C}\)

\item {} 
\sphinxAtStartPar
viene mantenuta l’ampiezza del grado,

\end{itemize}

\sphinxAtStartPar
così che la legge di conversione tra il valore numerico della misura di temperatura con la scala Celsius e la scala Kelvin è
\begin{equation*}
\begin{split}T[\text{K}] = T[\text{°C}] + 273.15 \ .\end{split}
\end{equation*}
\sphinxAtStartPar
Usando la scala di temperatura assoluta, le leggi di Charles e di Gay\sphinxhyphen{}Lussac possono essere riscritte come \sphinxstylestrong{todo} (\sphinxstyleemphasis{evitare singolarità})
\begin{equation*}
\begin{split}\begin{aligned}
  V & \propto T \qquad \text{se $P, n$ const.} \\
  P & \propto T \qquad \text{se $T, n$ const.} \\
\end{aligned}\end{split}
\end{equation*}

\paragraph{Legge di Avogadro}
\label{\detokenize{ch/thermodynamics/ideal-gas-experiments:legge-di-avogadro}}
\sphinxAtStartPar
Per gas semplici, a temperatura sufficientemente elevata, e pressione sufficientemente ridotta
\begin{equation*}
\begin{split}P, T \text{ const} \quad \rightarrow \quad \frac{n}{V} = \text{const}\end{split}
\end{equation*}

\subsubsection{Legge dei gas ideali}
\label{\detokenize{ch/thermodynamics/ideal-gas-experiments:legge-dei-gas-ideali}}
\sphinxAtStartPar
La legge dei gas ideali permette di riassumere le quattro leggi di Boyle, Charles, Gay\sphinxhyphen{}Lussac, Avogadro in un’unica equazione di stato,
\begin{equation*}
\begin{split}\dfrac{P \, V}{n \, T} = R \ ,\end{split}
\end{equation*}
\sphinxAtStartPar
avendo introdotto \(R \approx 8.314 \frac{\text{J}}{\text{mol} \,\text{K}}\) la \sphinxstylestrong{costante universale dei gas}.

\sphinxstepscope


\subsection{Espressioni diverse dell’equazione di stato dei gas perfetti}
\label{\detokenize{ch/thermodynamics/ideal-gas-expressions:espressioni-diverse-dell-equazione-di-stato-dei-gas-perfetti}}\label{\detokenize{ch/thermodynamics/ideal-gas-expressions:physics-hs-thermodynamics-matter-gases-ideal-expressions}}\label{\detokenize{ch/thermodynamics/ideal-gas-expressions::doc}}
\sphinxAtStartPar
Formule alternative dell’equazione di stato dei gas perfetti
\begin{itemize}
\item {} 
\sphinxAtStartPar
\(n\) numero di moli, \(R\) costante universale dei gas
\begin{equation*}
\begin{split}P \, V = n \, R \, T\end{split}
\end{equation*}
\item {} 
\sphinxAtStartPar
il numero di moli \(n\) può essere scritto come rapporto della massa \(m\) del sistema e la massa molare \(M_m\) del gas considerato,
\begin{equation*}
\begin{split}m = M_m \, n\end{split}
\end{equation*}
\sphinxAtStartPar
Usando questa espressione per sostituire \(n\) nella legge dei gas perfetti, e dividendo per \(V\) si può trovare una nuova espressione dell’equazione di stato di un gas perfetto,
\begin{equation*}
\begin{split}\begin{aligned}
    P   = \dfrac{m}{V} \, \dfrac{R}{M_m} \, T     
        = \rho \, R_g \, T \ ,
  \end{aligned}\end{split}
\end{equation*}
\sphinxAtStartPar
avendo riconosciuto la densità come rapporto tra massa e volume del sistema \(\rho = \frac{m}{V}\) e definito la costante    del gas specifica per il gas considerato come rapporto della costante universale e la massa molare, \(R_g = \frac{R}{M_m}\)

\item {} 
\sphinxAtStartPar
la relazione di Avogadro lega il numero di moli \(n\) e il numero di molecole \(N\) (\sphinxstylestrong{todo} *può essere solo una comoda unità di conto? Da dove arriva?…),
\begin{equation*}
\begin{split}N = N_A \, n \ ,\end{split}
\end{equation*}
\sphinxAtStartPar
essendo \(N_A \approx 6.022 \cdot 10^{23} \text{mol}^-1\) il \sphinxstylestrong{numero di Avogadro}. La legge di stato dei gas perfetti può quindi essere riscritta come
\begin{equation*}
\begin{split}P \, V = N \, \frac{R}{N_A} \, T = N \, k_B \, T \ , \end{split}
\end{equation*}
\sphinxAtStartPar
dove è stata introdotta la costante di Boltzmann, \(k_B = \frac{R}{N_A} \approx 1.38 \cdot 10^{-23} \frac{\text{J}}{\text{K}}\).

\sphinxAtStartPar
La costante di Boltzmann (\sphinxstylestrong{todo} \sphinxstyleemphasis{introdotta da Planck e da lui dedicata a Boltzmann}) è il fattore di conversione tra l’energia dovuta all’agitazione termica del sistema e la sua temperatura, come mostrato nella sezione dedicata alla {\hyperref[\detokenize{ch/thermodynamics/ideal-gas-kinetic-theory:physics-hs-thermodynamics-matter-gases-ideal-kinetic-theory}]{\sphinxcrossref{\DUrole{std,std-ref}{teoria cinetica dei gas}}}}.

\end{itemize}

\sphinxstepscope


\subsection{Teoria cinetica dei gas}
\label{\detokenize{ch/thermodynamics/ideal-gas-kinetic-theory:teoria-cinetica-dei-gas}}\label{\detokenize{ch/thermodynamics/ideal-gas-kinetic-theory:physics-hs-thermodynamics-matter-gases-ideal-kinetic-theory}}\label{\detokenize{ch/thermodynamics/ideal-gas-kinetic-theory::doc}}
\sphinxAtStartPar
Nel 1738, Daniel Bernoulli pubblica il \sphinxstyleemphasis{Hydrodynamica} nel quale fornisce un primo modello microscopico di un gas, pensato come un insieme di un numero enorme \sphinxstylestrong{todo} di particelle elementari (molecole), e il legame tra le grandezze macroscopiche e la media delle grandezze microscopiche.

\sphinxAtStartPar
Considerando: \sphinxstylestrong{todo}
\begin{itemize}
\item {} 
\sphinxAtStartPar
un volume retto di lati \(\Delta L_x\), \(\Delta L_y\), \(\Delta L_z\), \(\Delta V = \Delta L_x \, \Delta L_y \, \Delta L_z\)

\item {} 
\sphinxAtStartPar
che contiene un numero \(\Delta N\) di particelle identiche che non interagiscono tra di loro ma solo con urti elastici con le pareti rigide del volume

\end{itemize}

\sphinxAtStartPar
La forza sulla parete del volume con normale in direzione \(x\), può essere calcolata come rapporto tra l’impulso esercitato dalla parete e l’intervallo di tempo tra 2 urti della stessa molecola con la stessa parete,
\begin{equation*}
\begin{split}\Delta F_{x,i} = -\frac{\Delta I_{x,i}}{\Delta t_i} = \frac{2 m_m v_{x,i}}{\frac{2 \Delta L_x}{v_{x,i}}} = m_m \frac{v_{x,i}^2}{\Delta L_x}\end{split}
\end{equation*}
\sphinxAtStartPar
La forza media per unità di superficie sulla parete è
\begin{equation*}
\begin{split}\frac{\Delta F_{x,i}}{\Delta S_x} = \frac{\Delta F_{x,i}}{\Delta L_y \Delta L_z} = \dfrac{m_m v_{x_i}^2}{\Delta L_x \, \Delta L_y \, \Delta L_z} = \dfrac{m_m v_{x_i}^2}{\Delta V}\end{split}
\end{equation*}
\sphinxAtStartPar
L’energia cinetica della \(i\)\sphinxhyphen{}esima particella è
\begin{equation*}
\begin{split}
K_i = \frac{1}{2} m_m |\vec{v}_i|^2 = 
\frac{1}{2} m_m \left( v_{x,i}^2 + v_{y,i}^2 + v_{z,i}^2  \right) = 
\end{split}
\end{equation*}
\sphinxAtStartPar
L’energia dell’insieme delle particelle contenute nel volume è uguale alla somma delle loro energie cinetiche
\begin{equation*}
\begin{split}K = \sum_i K_i = \sum_i \frac{1}{2} m_m \left( v_{x,i}^2 + v_{y,i}^2 + v_{z,i}^2 \right) \ .\end{split}
\end{equation*}
\sphinxAtStartPar
Assumendo che la velocità delle particelle abbia una distribuzione isotropa nello spazio, ossia che non ci siano direzioni preferenziali, la media dei quadrati delle singole componenti cartesiane è uguale
\begin{equation*}
\begin{split}\langle \Delta K \rangle = \langle K_1 \rangle \, \Delta N = \Delta N \frac{3}{2} m_m v_{rms}^2 \ .\end{split}
\end{equation*}
\sphinxAtStartPar
\sphinxstylestrong{todo} L’energia cinetica può essere scritta in funzione della temperatura, \(T\),
\begin{equation*}
\begin{split}\frac{\langle \Delta K \rangle}{\Delta N} = \dfrac{3}{2} m_m v^2_{rms} = \frac{3}{2} k_B \, T \ ,\end{split}
\end{equation*}
\sphinxAtStartPar
questa espressione prevede che l’energia cinetica di una molecola sia direttamente proporizonale alla temperatura e al numero di gradi di libertà della particella, qui \(f = 3\), tramite la costante di proporzionalità \(k_B = \dots\), la \sphinxstylestrong{costante di Boltzmann}.

\sphinxAtStartPar
La forza media esercitata dalle \(\Delta N\) molecole sulla superficie con normale \(x\) può essere quindi scritta come

\sphinxAtStartPar
La \sphinxstylestrong{costante di Avogadro} (\sphinxstylestrong{todo} da dove arriva? Esperimenti sui gas a pari volume e condizioni TD, fatti da?? Gay\sphinxhyphen{}Lussac?? Charles?? Controllare video di Bressanini e altre fonti) permette di convertire il numero di molecole \(N\) nel numero di moli \(n\), \(\Delta N = N_A \, \Delta n\), e calcolare la massa di una mole, la massa molare, una volta nota la massa di una molecola \(M_m = N_A m_m\)
\begin{equation*}
\begin{split}P 
  & = \dfrac{\Delta N}{\Delta V} m_m \, v_{rms}^2 = \\  
  & = \dfrac{\Delta N}{\Delta V} k_B \, T
    = \dfrac{\Delta n}{\Delta V} \underbrace{N_A k_B}_{= R_u} \, T \\
  & = \dfrac{m_m \Delta N}{\Delta V} \dfrac{k_B}{m_m} \, T
    = \dfrac{\Delta m }{\Delta V} \dfrac{k_B}{m_m} \, T  
    = \dfrac{\Delta m }{\Delta V} \underbrace{\dfrac{N_A \, k_B}{M_m}}_{=\frac{R_u}{M_m} = R} \, T   \ , \end{split}
\end{equation*}
\sphinxAtStartPar
avendo introdotto la definizione della \sphinxstylestrong{costante universale} \(R_u = N_A \, k_B\) come prodotto del numero di Avogadro e la costante di Boltzmann, e una costante del gas considerato come rapporto tra la costante universale e la sua massa molare, \(R = \dfrac{R_u}{M_m}\).

\sphinxAtStartPar
Valori numerici; cenni storici

\sphinxstepscope


\subsection{Caratteristiche dei gas perfetti}
\label{\detokenize{ch/thermodynamics/ideal-gas-formulas:caratteristiche-dei-gas-perfetti}}\label{\detokenize{ch/thermodynamics/ideal-gas-formulas:physics-hs-thermodynamics-matter-gases-ideal-formulas}}\label{\detokenize{ch/thermodynamics/ideal-gas-formulas::doc}}

\subsubsection{Legge di stato}
\label{\detokenize{ch/thermodynamics/ideal-gas-formulas:legge-di-stato}}\begin{equation*}
\begin{split}\begin{aligned}
  P V & =    N \, k_B \, T & \\
                           & & \qquad \text{$\left( N = N_A \, n \ , \  N_A \, k_B = R \right)$} \\ \\
  P V & =    n \, R   \, T & \\
                           & & \qquad \text{$\left( m = M_m \, n \ , \  R_g = \frac{R}{M_m} \right)$} \\ \\
  P V & =    m \, R_g \, T & \\
                           & & \qquad \text{$\left( m = \rho \, V \right)$} \\ \\
  P   & = \rho \, R_g \, T & \qquad \\
\end{aligned}\end{split}
\end{equation*}

\subsubsection{Primo principio della termodinamica}
\label{\detokenize{ch/thermodynamics/ideal-gas-formulas:primo-principio-della-termodinamica}}
\sphinxAtStartPar
Per un gas comprimibile monocomponente, il lavoro interno meccanico reversibile è
\begin{equation*}
\begin{split}\delta L^{int,rev, mech} = P \, d V\end{split}
\end{equation*}
\sphinxAtStartPar
In assenza di altre interazioni di lavoro, il bilancio di energia interna per un gas comprimible diventa
\begin{equation*}
\begin{split}\begin{aligned}
  d E & = \delta Q^{ext} + \delta^+ D - \delta L^{int,rev} = \\
      & = T \, dS - P \, dV \ .
\end{aligned}\end{split}
\end{equation*}

\subsubsection{Energia interna, entalpia e calori specifici}
\label{\detokenize{ch/thermodynamics/ideal-gas-formulas:energia-interna-entalpia-e-calori-specifici}}
\sphinxAtStartPar
\sphinxstylestrong{Energia interna.} Seguendo le conclusioni del modello di gas ideale fornito dalla {\hyperref[\detokenize{ch/thermodynamics/ideal-gas-kinetic-theory:physics-hs-thermodynamics-matter-gases-ideal-kinetic-theory}]{\sphinxcrossref{\DUrole{std,std-ref}{teoria cinetica dei gas}}}}, l’espressione dell’energia interna di un gas perfetto può essere scritta come,
\begin{equation*}
\begin{split}E = \frac{f}{2} \, N \, k_B \, T = \frac{f}{2} \, n \, R \, T = m \frac{f}{2} \, R_g \, T \ .\end{split}
\end{equation*}
\sphinxAtStartPar
\sphinxstylestrong{Entalpia.} Usando la definizione di entalpia \(H = E + F_i \, X_i = E + P \, V\), l’equazione di stato e l’espressione dell’energia interna dei gas perfetti, l’entalpia di un gas perfetto può essere scritta come
\begin{equation*}
\begin{split}H = \left(\frac{f}{2} + 1 \right) \, N \, k_B \, T = \left( \frac{f}{2} + 1 \right) \, n \, R \, T = m \left( \frac{f}{2} + 1 \right) \, R_g \, T \ .\end{split}
\end{equation*}
\sphinxAtStartPar
\sphinxstylestrong{Calore specifico a volume costante.} Se il volume del sistema è costante, il lavoro interno è nullo (\sphinxstylestrong{todo} complessivo, reversibile, aggiungere ipotesi di stato di equilibrio una volta per tutte?), \(\delta L = 0\), \(dE = \delta Q^{ext} = T \, dS\)
\begin{equation*}
\begin{split}m \, c_v \, d T := \delta Q^{ext}\big|_v = dE\big|_v = m \frac{f}{2} \, R_g \, d T 
\qquad \rightarrow \qquad
c_v = \frac{f}{2} \, R_g \ .\end{split}
\end{equation*}
\sphinxAtStartPar
\sphinxstylestrong{Calore specifico a pressione costante.} Il differenziale dell’entropia a pressione costante,
\begin{equation*}
\begin{split}dH\big|_P = d ( E + P \, V )\big|_P  = d E\big|_P  + \underbrace{d P}_{=0} \, V + P \, dV\big|_P \ ,\end{split}
\end{equation*}
\sphinxAtStartPar
può essere utilizzato per riscrivere il bilancio di energia intenra a pressione costante,
\begin{equation*}
\begin{split}dH\big|_P = dE\big|_P + P \delta V\big|_P = \delta Q^{ext}\big|_P + \delta^+ D\big|_P \ .\end{split}
\end{equation*}
\sphinxAtStartPar
Nell’ipotesi che la dissipazione sia nulla, (\sphinxstylestrong{todo} aggiungere ipotesi di stato di equilibrio una volta per tutte?)), si può quindi legare la variazione di entalpia del sistema all’apporto di calore al sistema, e al calore specifico a pressione costante,
\begin{equation*}
\begin{split}m \, c_P \, dT := \delta Q^{ext}\big|_P = d H \big|_P = m \left( \frac{f}{2} + 1 \right) \, R_g \, d T 
\qquad \rightarrow \qquad
c_P = \left( \frac{f}{2} + 1 \right) \, R_g \ .\end{split}
\end{equation*}
\sphinxAtStartPar
\sphinxstylestrong{Esempi: calcolo del calore specifico di gas}
\subsubsection*{Idrogeno molecolare, \protect\(\text{ H}_2\protect\)}

\sphinxAtStartPar
Assumendo che l’idrogeno, \(\text{H}_2\), con massa molare \(M_m = 2.0 \frac{\text{kg}}{\text{kmol}}\), si comporti come un gas perfetto nella condizione di interesse, la costante specifica dell’idrogeno molecolare vale
\begin{equation*}
\begin{split}R_g = \frac{R}{M_m} = \frac{8314 \frac{\text{J}}{\text{kmol} \text{ K}}}{2 \frac{\text{kg}}{\text{kmol}}} = 4157 \frac{\text{J}}{\text{kg} \text{ K}} .\end{split}
\end{equation*}
\sphinxAtStartPar
e i calori specifici
\begin{equation*}
\begin{split}c_v = \frac{5}{2} \, R_g = \frac{5}{2} \, 4157 \, \frac{\text{J}}{\text{kg} \text{ K}} = 10392.5 \, \frac{\text{J}}{\text{kg} \text{ K}} \end{split}
\end{equation*}\begin{equation*}
\begin{split}c_P = \frac{7}{2} \, R_g = \frac{7}{2} \, 4157 \, \frac{\text{J}}{\text{kg} \text{ K}} = 14549.5 \, \frac{\text{J}}{\text{kg} \text{ K}} \end{split}
\end{equation*}\subsubsection*{Elio, \protect\(\text{ He}\protect\)}

\sphinxAtStartPar
Assumendo che l’elio, \(\text{He}\), con massa molare \(M_m = 4 \frac{\text{kg}}{\text{kmol}}\), si comporti come un gas perfetto nella condizione di interesse, la costante specifica dell’idrogeno molecolare vale
\begin{equation*}
\begin{split}R_g = \frac{R}{M_m} = \frac{8314 \frac{\text{J}}{\text{kmol} \text{ K}}}{4 \frac{\text{kg}}{\text{kmol}}} = 2078.5 \frac{\text{J}}{\text{kg} \text{ K}} .\end{split}
\end{equation*}
\sphinxAtStartPar
e i calori specifici
\begin{equation*}
\begin{split}c_v = \frac{3}{2} \, R_g = \frac{3}{2} \, 2078.5 \, \frac{\text{J}}{\text{kg} \text{ K}} = 3117.8 \, \frac{\text{J}}{\text{kg} \text{ K}} \end{split}
\end{equation*}\begin{equation*}
\begin{split}c_P = \frac{5}{2} \, R_g = \frac{5}{2} \, 2078.5 \, \frac{\text{J}}{\text{kg} \text{ K}} = 5196.2 \, \frac{\text{J}}{\text{kg} \text{ K}} \end{split}
\end{equation*}

\subsubsection*{Aria, miscela di gas}

\sphinxAtStartPar
L’aria è una miscela di gas (\sphinxstylestrong{todo} \sphinxstyleemphasis{riferimento a miscele?}) composta da \(\text{N}_2\), \(\text{O}_2\),… la cui massa molare è la media pesata delle masse molari dei suoi componenti, \(M_m = 28.97 \frac{\text{kg}}{\text{kmol}}\).
La costante specifica dell’aria è quindi
\begin{equation*}
\begin{split}R_g = \frac{R}{M_m} = \frac{8314 \frac{\text{J}}{\text{kmol} \text{ K}}}{28.97 \frac{\text{kg}}{\text{kmol}}} = 287 \frac{\text{J}}{\text{kg} \text{ K}} .\end{split}
\end{equation*}
\sphinxAtStartPar
Essendo composta da molecole di gas biatomiche, i gradi di libertà della singola molecola sono \(f = 5\) (3 legati alla traslazione, 2 alla rotazione; manca la rotazione attorno all’asse della molecola, assumendo trascurabile l’inerzia attorno a quell’asse). I calori specifici valgono quindi
\begin{equation*}
\begin{split}c_v = \frac{5}{2} \, R_g = \frac{5}{2} \, 287 \, \frac{\text{J}}{\text{kg} \text{ K}} =  717.5 \, \frac{\text{J}}{\text{kg} \text{ K}} \end{split}
\end{equation*}\begin{equation*}
\begin{split}c_P = \frac{7}{2} \, R_g = \frac{7}{2} \, 287 \, \frac{\text{J}}{\text{kg} \text{ K}} = 1004.5 \, \frac{\text{J}}{\text{kg} \text{ K}} \end{split}
\end{equation*}

\subsubsection{Variazioni di entropia}
\label{\detokenize{ch/thermodynamics/ideal-gas-formulas:variazioni-di-entropia}}
\sphinxAtStartPar
La variazione dell’entropia di un gas perfetto può essere scritta in diverse forme partendo dal primo principio della termodinamica e usando l’espressione dell’energia interna e la legge di stato dei gas perfetti per cambiare le variabili indipendenti,
\begin{equation*}
\begin{split}\begin{aligned}
  ds & = \frac{1}{T} \left( d e + \frac{P}{\rho^2} d \rho \right) = \\
     & = c_v \frac{dT}{T} + R_g \frac{d \rho}{\rho} = \\
     & = c_P \frac{dT}{T} + R_g \frac{d P}{P} = \\
     & = c_P \frac{d \rho}{\rho} + c_v \frac{d P}{P}  \ .
\end{aligned}\end{split}
\end{equation*}
\sphinxstepscope


\section{Solidi elastici}
\label{\detokenize{ch/thermodynamics/elastic-solid-1d:solidi-elastici}}\label{\detokenize{ch/thermodynamics/elastic-solid-1d:physics-hs-thermodynamics-matter-elastic-1d}}\label{\detokenize{ch/thermodynamics/elastic-solid-1d::doc}}

\subsection{Solido elastico lineare 1\sphinxhyphen{}dimensionale}
\label{\detokenize{ch/thermodynamics/elastic-solid-1d:solido-elastico-lineare-1-dimensionale}}
\sphinxAtStartPar
\sphinxstylestrong{Legge costitutiva lineare con espansione termica.}
Sia data la legge costitutiva elastica che esprime la lunghezza della trave \(L\) in funzione dell’azione assiale \(f\) e della differenza di temperatura \(T-T_0\) rispetto alla temperatura di riferimento \(T_0\),
\begin{equation*}
\begin{split}L(f,T) - L_0 = \frac{1}{K} f + \alpha L_0 (T-T_0) \ ,\end{split}
\end{equation*}
\sphinxAtStartPar
assumendo che la costante elastica isoterma \(K\), e il coefficiente di dilatazione termica a carico costante \(\alpha\) siano costanti, parametri caratteristici del materiale e della configurazione di riferimento. Sotto queste ipotesi, è possibile invertire la relazione per scrivere l’azione assiale in funzione dell’allungamento e della temperatura,
\begin{equation*}
\begin{split}f(\Delta L, \, \Delta T) = K \Delta L - \alpha \, L_0 \, K \, \Delta T \ .\end{split}
\end{equation*}
\sphinxAtStartPar
\sphinxstylestrong{Potenziali termodinamici.}
\begin{equation*}
\begin{split}dE = T dS + f dL  \qquad , \qquad \text{energia interna} \end{split}
\end{equation*}\begin{equation*}
\begin{split}dH = T dS - L df  \qquad , \qquad \text{entalpia, $H = E - f \, L$} \end{split}
\end{equation*}\begin{equation*}
\begin{split}dF =-S dT + f dL  \qquad , \qquad \text{Helmholtz, $F = E + T \, S$} \end{split}
\end{equation*}\begin{equation*}
\begin{split}dG =-S dT - L df  \qquad , \qquad \text{Gibbs, $G = H + T \, S$} \end{split}
\end{equation*}
\sphinxAtStartPar
\sphinxstylestrong{Energia libera di Helmholtz.}
\begin{equation*}
\begin{split}d E = \delta Q^{ext} - \delta L^{int} = T \, dS + f \, dL\end{split}
\end{equation*}
\sphinxAtStartPar
La variazione dell’energia libera di Helmholtz, \(F := E - TS\),
\begin{equation*}
\begin{split}dF = d E - T \, dS - S \, dT = f \, dL - S \, dT \ ,\end{split}
\end{equation*}
\sphinxAtStartPar
permette di riconoscere l’azione assiale e l’entropia come le derivate parziali di \(F\),
\begin{equation*}
\begin{split}
f = \left(\frac{\partial F}{\partial L} \right)_T
\qquad , \qquad
S = - \left(\frac{\partial F}{\partial T} \right)_L
\end{split}
\end{equation*}
\sphinxAtStartPar
Integrando la relazione dell’azione assiale, si ottiene
\begin{equation*}
\begin{split}F(\Delta L, \Delta T) = \dfrac{1}{2} \, K \, \Delta L^2 - \alpha \, L_0 \, K \, \Delta T \, \Delta L + F_0(T) \ ,\end{split}
\end{equation*}
\sphinxAtStartPar
avendo introdotto la funzione \(F_0(T)\), dipendente al massimo dalla temperatura \(T\), come risultato dell’integrazione in \(L\).
Dall’espressione dell’energia libera di Helmholtz si può poi ricavare l’espressione dell’entropia
\begin{equation*}
\begin{split}
S(\Delta L, \Delta T) = - \left(\frac{\partial F}{\partial T} \right)_L = \alpha \, L_0 \, K \, \Delta L - F_0'(T) \ .
\end{split}
\end{equation*}
\sphinxAtStartPar
\sphinxstylestrong{Calori specifici.}
Il calore specifico a lunghezza costante viene calcolato direttamente usando l’espressione dell’entropia,
\begin{equation*}
\begin{split}C_L = T \left(\frac{\partial S}{\partial T} \right)_L = - T \, F_0''(T) \ .\end{split}
\end{equation*}
\sphinxAtStartPar
Assumendo che il calore specifico \(C_L\) sia costante, l’integrazione ci fornisce un’espressione della funzione \(F_0'(T)\),
\begin{equation*}
\begin{split}F_0'(T) - F_0'(T_0) = - C_L \ln \left( \dfrac{T}{T_0} \right) \ ,\end{split}
\end{equation*}
\sphinxAtStartPar
che consente di esprimere l’entropia in fuznione del calore specifico,
\begin{equation*}
\begin{split}S(\Delta L, \, \Delta T) = \alpha \, L_0 \, K \, \Delta L + C_L \ln \left( 1 + \dfrac {\Delta T}{T_0} \right) + S_0\end{split}
\end{equation*}
\sphinxAtStartPar
Usando la legge costitutiva per esprimere l’allungamento in funzione dell’azione assiale e dell’incremento di temperatura,
\begin{equation*}
\begin{split}S(f, \, \Delta T) = \alpha \, L_0 \, K \, \left( \dfrac{1}{K} \, f + \alpha \, L_0 \, \Delta T \right) + C_L \ln \left( 1 + \dfrac {\Delta T}{T_0} \right) + S_0 \ ,\end{split}
\end{equation*}
\sphinxAtStartPar
è possibile calcolare il calore specifico a carico costante,
\begin{equation*}
\begin{split}\begin{aligned}
C_f & = T \left(\frac{\partial S}{\partial T} \right)_f = \\
    & = T \, \left[ K \, ( \alpha \, L_0 )^2 + \frac{C_L}{T}\right] \\
    & = T K \, ( \alpha \, L_0 )^2 + C_L \ .
\end{aligned}\end{split}
\end{equation*}
\sphinxAtStartPar
\sphinxstylestrong{Coefficienti termodinamici: costanti elastiche, coefficiente di dilatazione.}
Dall’espressione della legge costitutiva, si definiscono la costante elastica isoterma
\begin{equation*}
\begin{split}\frac{1}{K} := \left(\frac{\partial L}{\partial f}\right)_T \ ,\end{split}
\end{equation*}
\sphinxAtStartPar
e il coefficiente di dilatazione termica a carico costante
\begin{equation*}
\begin{split}\alpha_f := \frac{1}{L_0} \left(\frac{\partial L}{\partial T}\right)_f \ .\end{split}
\end{equation*}
\sphinxAtStartPar
La costante elastica adiabatica,
\begin{equation*}
\begin{split}\frac{1}{K_{ad}} := \left(\frac{\partial L}{\partial f}\right)_S \ ,\end{split}
\end{equation*}
\sphinxAtStartPar
può essere calcolata derivando la funzione che esprime la lunghezza \(L\) in funzione delle variabili indipendente \(f\), \(S\) che si può ricavare sostituendo il legame \(\Delta T(\Delta L, \, F)\) della relazione costitutiva nell’espressione dell’entropia, per ottenere
\begin{equation*}
\begin{split}S = \alpha \, L_0 \, K \, \Delta L + C_L \ln \left( 1 + \frac{1}{T_0} \frac{1}{\alpha \, L_0 \, K} \left( K \, \Delta L - f \right) \right) + S_0\end{split}
\end{equation*}
\sphinxAtStartPar
la cui derivata \(\frac{\partial}{\partial f}\big|_S\) vale
\begin{equation*}
\begin{split}0 = \alpha \, L_0 \, K \left(\frac{\partial L}{\partial f}\right)_S + C_L \dfrac{1}{1 + \frac{K \, \Delta L - f}{\alpha \, T_0 \, L_0 \, K}}\frac{1}{\alpha \, L_0 \, K \, T_0} \left( K \left(\frac{\partial L}{\partial f}\right)_S - 1\right) \ .\end{split}
\end{equation*}
\sphinxAtStartPar
Introducendo la definizione della costante elastica in condizioni adiabatiche, \(K_{ad}(T; K, \alpha)\),
\begin{equation*}
\begin{split}\dfrac{K}{K_{ad}} \left[ \frac{(\alpha \, L_0)^2 \, T \, K }{C_L} + 1 \right] = 1\end{split}
\end{equation*}
\sphinxAtStartPar
si trova la relazione tra le costanti elastiche isoterma e adiabatica,
\begin{equation*}
\begin{split}\begin{aligned}
  K_{ad} & = K \left( 1 + \frac{(\alpha \, L_0)^2 \, T \, K }{C_L} \right)     
           = K \frac{1}{1 - \dfrac{(\alpha \, L_0)^2 \, K \, T }{C_f}} \ . 
\end{aligned}\end{split}
\end{equation*}
\sphinxAtStartPar
\sphinxstylestrong{Energia interna.}
L’energia interna del sistema può essere ricavata da \(E = F + T \, S\),
\begin{equation*}
\begin{split}\begin{aligned}
  E & = \dfrac{1}{2} \, K \, \Delta L^2 - \alpha \, L_0 \, K \, \Delta T \, \Delta L + F_0(T) + T \left( \alpha \, L_0 \, K \, \Delta L + C_L \ln \left( 1 + \dfrac{\Delta T}{T_0} \right) + S_0 \right) = \\
    & = \frac{1}{2} \, K \, \Delta L^2 + F_0(T) + T \, C_L \, \ln \left( 1 + \dfrac{\Delta T}{T_0} \right) + T \, S_0 \ .
\end{aligned}\end{split}
\end{equation*}
\sphinxAtStartPar
E’ quindi possibile riconoscere \(E_0 := E(\Delta L = 0, \Delta T = 0) = F_0(T_0) + T_0 \, S_0\).
La variazione di quest’ultima relazione nei confronti delle variabili \(\Delta L\), \(\Delta T\),
\begin{equation*}
\begin{split}\begin{aligned}
  d E & = K \Delta L \, d L + \underbrace{F_0'(T)}_{- C_L \ln \left(\frac{T}{T_0}\right) - S_0} dT + C_L \, d T \left[ \ln \left(\frac{T}{T_0}\right) + 1 \right] + S_0 \, d T = \\
      & = K \Delta L \, d L + C_L \, d T
\end{aligned}\end{split}
\end{equation*}
\sphinxAtStartPar
può essere espressa in funzione degli incrementi \(d L\), \(d S\), grazie all’incremento della relazione che lega le tre variabili \(S, L, T\),
\begin{equation*}
\begin{split}d S = \alpha \, L_0 \, K \, d L + \frac{C_L}{T} \, dT \ ,\end{split}
\end{equation*}
\sphinxAtStartPar
in
\begin{equation*}
\begin{split}\begin{aligned}
  d E & = K \Delta L \, d L + C_L \, dT = \\
      & = K \Delta L \, d L + \, T \, ( dS - \alpha \, L_0 \, K \, dL ) = \\
      & = ( K \Delta L - K \, \alpha \, L_0 \, T ) \, d L + \, T \, dS  \ .
\end{aligned}\end{split}
\end{equation*}
\sphinxAtStartPar
\sphinxstylestrong{todo}
Controllare! Non torna l’espressione della forza: c’è solo la temperatura, ma ci dovrebbe essere la differenza di temperatura rispetto a quella di riferimento?

\sphinxstepscope


\chapter{Macchine termiche}
\label{\detokenize{ch/thermodynamics/heat-engine:macchine-termiche}}\label{\detokenize{ch/thermodynamics/heat-engine:physics-hs-thermodynamics-heat-engine}}\label{\detokenize{ch/thermodynamics/heat-engine::doc}}
\sphinxAtStartPar
\sphinxstylestrong{Storia.}
\begin{itemize}
\item {} 
\sphinxAtStartPar
rivoluzione industriale

\item {} 
\sphinxAtStartPar
…

\item {} 
\sphinxAtStartPar
macchine termiche oggi: motori a combustione interna, motori aeronautici, centrali di generazione di energia elettrica (conversione di forme di energia),…

\item {} 
\sphinxAtStartPar
…

\end{itemize}

\sphinxAtStartPar
\sphinxstylestrong{Classificazione.}
\begin{itemize}
\item {} 
\sphinxAtStartPar
Combustione: interna/esterna

\item {} 
\sphinxAtStartPar
Funzionamento: in fasi (o volumetrico; a sua volta alternativo/rotativo)/continuo

\end{itemize}

\sphinxAtStartPar
\sphinxstylestrong{Componenti.}
\begin{itemize}
\item {} 
\sphinxAtStartPar
trasformazioni TD e componenti (turbine, compressori, scambiatori di calore,…)

\item {} 
\sphinxAtStartPar
cicli termodinamici e macchine termiche
\begin{itemize}
\item {} 
\sphinxAtStartPar
macchina ideale di Carnot: efficienza massima, formulazioni equivalenti del \sphinxstyleemphasis{secondo principio della TD} per le macchine termiche (Planck, Kelvin)

\item {} 
\sphinxAtStartPar
macchine reali

\end{itemize}

\end{itemize}

\sphinxstepscope


\section{Cicli termodinamici}
\label{\detokenize{ch/thermodynamics/heat-engine-td-cycles:cicli-termodinamici}}\label{\detokenize{ch/thermodynamics/heat-engine-td-cycles:physics-hs-thermodynamics-heat-engine-td-cycles}}\label{\detokenize{ch/thermodynamics/heat-engine-td-cycles::doc}}
\sphinxAtStartPar
Un ciclo termodinamico è una sequenza di trasformazioni termodinamiche che riportano il sistema al suo stato di partenza.
In un piano termodinamico, un ciclo termodinamico è rappresentato da una curva chiusa.
\begin{itemize}
\item {} 
\sphinxAtStartPar
\sphinxstylestrong{todo.} Sistemi aperti/sistemi chiusi

\end{itemize}

\sphinxAtStartPar
Per un \sphinxstylestrong{sistema chiuso}, il primo principio della termodinamica
\begin{equation*}
\begin{split}d E^{tot} = \delta Q^e + \delta L^e\end{split}
\end{equation*}
\sphinxAtStartPar
Nell’ipotesi di regime periodico dello stato del sistema descritto da un ciclo termodinamico, alla fine del ciclo
\begin{equation*}
\begin{split}0 =  \oint_\gamma d E^{tot} = \oint_\gamma \delta Q^e + \oint_\gamma \delta L^e \ ,\end{split}
\end{equation*}
\sphinxAtStartPar
e da questo segue che il lavoro netto fatto dal sistema \(-L^e\) è uguale al calore netto entrato nel sistema \(Q^e\),
\begin{equation*}
\begin{split}-L^e = Q^e \ .\end{split}
\end{equation*}
\sphinxAtStartPar
\sphinxstylestrong{todo} In un sistema in cui sia trascurabile l’energia cinetica del sistema sia trascurabile rispetto alla variazione di energia interna, si può approssimare \(E^{tot} = K + E \approx E\)

\sphinxstepscope


\section{Macchina termica di Carnot}
\label{\detokenize{ch/thermodynamics/heat-engine-carnot:macchina-termica-di-carnot}}\label{\detokenize{ch/thermodynamics/heat-engine-carnot:physics-hs-thermodynamics-heat-engine-carnot}}\label{\detokenize{ch/thermodynamics/heat-engine-carnot::doc}}\begin{equation*}
\begin{split}d E^{tot} = \delta Q^{ext} + \delta L^{ext} \ ,\end{split}
\end{equation*}\begin{equation*}
\begin{split}dS = \dfrac{\delta Q^{ext}}{T} + \dfrac{\delta^+ D}{T} \ge  \dfrac{\delta Q^{ext}}{T} \ .\end{split}
\end{equation*}\begin{equation*}
\begin{split}0 = \oint_{\gamma} d E^{tot} = \oint_{\gamma} \delta Q^{ext} + \oint_{\gamma} \delta L^{ext}\end{split}
\end{equation*}
\sphinxAtStartPar
Il lavoro fatto in un ciclo è
\begin{equation*}
\begin{split}\Delta L^{1-cycle} = - \Delta L^{ext} = \Delta Q^{ext} \ .\end{split}
\end{equation*}
\sphinxAtStartPar
\sphinxstylestrong{Ciclo di Carnot.} Due adiabatiche ideali e due isoterme ideali.

\sphinxAtStartPar
\sphinxstylestrong{Principio di Carnot.} Il rendimento massimo di una macchina termica che scambia calore con due sorgenti di calore a temperatura costante \(T_1\), \(T_2 < T_2\) è
\begin{equation*}
\begin{split}\eta_C = 1 - \dfrac{T_2}{T_1} \ .\end{split}
\end{equation*}
\sphinxstepscope


\section{Secondo principio della termodinamica per cicli termodinamici}
\label{\detokenize{ch/thermodynamics/heat-engine-second-principle:secondo-principio-della-termodinamica-per-cicli-termodinamici}}\label{\detokenize{ch/thermodynamics/heat-engine-second-principle:physics-hs-thermodynamics-heat-engine-second-principle}}\label{\detokenize{ch/thermodynamics/heat-engine-second-principle::doc}}
\sphinxAtStartPar
Esistono due enunciati equivalenti del secondo principio della termodinamica per una macchina termica che realizza un ciclo termodinamico.

\sphinxAtStartPar
\sphinxstylestrong{Enunciato di Kelvin.} Una macchina termodinamica non può trasformare in lavoro tutto il calore assorbito da una sorgente a temperatura costante.
\begin{equation*}
\begin{split}d E^{tot} = \delta Q^{ext} + \delta L^{ext} = \delta Q^{ext} - \delta L\end{split}
\end{equation*}\begin{equation*}
\begin{split}\begin{aligned}
  0 & = \oint_{\gamma} \dfrac{d E^{tot}}{T_1} = \\
    & = \oint_{\gamma} \dfrac{\delta Q^{ext}}{T_1} - \oint_{\gamma} \dfrac{\delta L}{T_1} \le & \delta Q^{ext} \left(\dfrac{1}{T_1} - \dfrac{1}{T} \right) \le 0 \\
    & \le \oint_{\gamma} \dfrac{\delta Q^{ext}}{T} - \dfrac{L}{T_1} \le \\
    & \le \underbrace{ \oint_{\gamma} d S}_{=0} - \dfrac{L}{T_1} \ ,
\end{aligned}\end{split}
\end{equation*}
\sphinxAtStartPar
che implica
\begin{equation*}
\begin{split}L \le 0 \ ,\end{split}
\end{equation*}
\sphinxAtStartPar
cioè che una macchina termica che scambia calore unicamente con una sorgente a temperatura costante produce un lavoro negativo, ossia assorbe lavoro esterno, e cede calore.

\sphinxAtStartPar
\sphinxstylestrong{Enunciato di Planck.} Non è possibile trasferire calore da una sorgente a temperatura \(T_2\) a una sorgente a temperatura maggiore \(T_1 > T_2\) con una macchina termica che non assorba lavoro.

\sphinxAtStartPar
\sphinxstylestrong{todo} dim

\sphinxstepscope


\section{Ciclo Otto}
\label{\detokenize{ch/thermodynamics/heat-engine-otto:ciclo-otto}}\label{\detokenize{ch/thermodynamics/heat-engine-otto:physics-hs-thermodynamics-heat-engine-otto}}\label{\detokenize{ch/thermodynamics/heat-engine-otto::doc}}
\sphinxAtStartPar
\sphinxstylestrong{Storia e applicazioni.}


\subsection{Ciclo Otto reale}
\label{\detokenize{ch/thermodynamics/heat-engine-otto:ciclo-otto-reale}}
\sphinxAtStartPar
…


\subsection{Ciclo Otto ideale}
\label{\detokenize{ch/thermodynamics/heat-engine-otto:ciclo-otto-ideale}}
\sphinxAtStartPar
Un modello ideale del ciclo Otto è formato da:
\begin{itemize}
\item {} 
\sphinxAtStartPar
\(0 \rightarrow 1\) aspirazione a pressione costante, \(P_1\). Durante l’aspirazione, il sistema è aperto: le valvole di aspirazione sono aperte per far entrare l’aria in camera di combustione. Alla fine dell’aspirazione, le valvole vengono chiuse e il sistema di interesse è un sistema chiuso

\item {} 
\sphinxAtStartPar
\(1 \rightarrow 2\) compressione adiabatica in sistema chiuso

\item {} 
\sphinxAtStartPar
\(2 \rightarrow 3\) combustione a volume costante: la combustione avviene in maniera sufficientemente veloce da poter essere modellata come una trasformazione termodinamica a volume costante, in corrispondenza del punto morto superiore; in prima approssimazione, si può trascurare il flusso di massa del combustibile e la variazione delle proprietà chimico\sphinxhyphen{}fisiche del fluido di lavoro; la reazione di combustione produce il calore in ingresso al sistema

\item {} 
\sphinxAtStartPar
\(3 \rightarrow 4\) espansione adiabatica

\item {} 
\sphinxAtStartPar
\(4 \rightarrow 1\), \(1 \rightarrow 0\) scarico libero e scarico forzato. \sphinxstylestrong{todo} in prima approssimazione, la parte di scarico al punto morto inferiore non produce lavoro poiché \(\Delta V_{14} = 0\) e la fase di scarico forzata è equilibrata dalla fase di aspirazione.

\end{itemize}


\subsection{Rendimento del ciclo Otto}
\label{\detokenize{ch/thermodynamics/heat-engine-otto:rendimento-del-ciclo-otto}}\begin{equation*}
\begin{split}\eta = 1 + \dfrac{\Delta Q_{41}}{\Delta Q_{23}}
       = 1 + \dfrac{m \, c_V \, (T_1 - T_4)}{m \, c_V \, (T_3 - T_2)}
       = 1 + \dfrac{T_1 - T_4}{T_3 - T_2}
\end{split}
\end{equation*}
\sphinxAtStartPar
Usando le condizioni, \sphinxstylestrong{todo} \sphinxstyleemphasis{usare direttamente le espressioni delle adiabatiche ideali ricavate nella sezione delle trasformazioni termodinamiche con gas ideali}
\begin{equation*}
\begin{split}V_2 = V_3 \qquad , \qquad V_1 = V_4\end{split}
\end{equation*}\begin{equation*}
\begin{split}P_1 \, V_1^{\gamma} = P_2 \, V_2^{\gamma}\end{split}
\end{equation*}\begin{equation*}
\begin{split}P_3 \, V_3^{\gamma} = P_4 \, V_4^{\gamma}\end{split}
\end{equation*}
\sphinxAtStartPar
e la legge dei gas ideali, \(P V = m R T\), assumendo che sia un’equazione di stato adatta a descrivere il fluido di lavoro, per riscrivere l’equazione delle trasformazioni adiabatiche
\begin{equation*}
\begin{split}\begin{aligned}
  T_1 \, V_1^{\gamma-1} & = T_2 \, V_2^{\gamma-1} \\
  T_3 \, V_3^{\gamma-1} & = T_4 \, V_4^{\gamma-1}
\end{aligned}
\begin{aligned}
  & \qquad \rightarrow \qquad  (T_4 - T_1) \, V_1^{\gamma-1} = (T_3 - T_2) \, V_2^{\gamma - 1} \\
  & \qquad \rightarrow \qquad  \dfrac{T_4 - T_1}{T_3 - T_2} = \left( \dfrac{V_2}{V_1} \right)^{\gamma-1} = \dfrac{1}{\beta^{\gamma - 1}} \\
\end{aligned}
\end{split}
\end{equation*}
\sphinxAtStartPar
è possibile riscrivere l’espressione del rendimento del ciclo Otto in funzione unicamente del rapporto di compressione volumetrico \(\beta := \dfrac{V_1}{V_2}\),
\begin{equation*}
\begin{split}\eta = 1 - \dfrac{1}{\beta^{\gamma-1}} \ .\end{split}
\end{equation*}

\subsection{Funzionamento di un motore a combustione interna}
\label{\detokenize{ch/thermodynamics/heat-engine-otto:funzionamento-di-un-motore-a-combustione-interna}}
\sphinxAtStartPar
\sphinxstylestrong{todo}


\subsection{Esempio}
\label{\detokenize{ch/thermodynamics/heat-engine-otto:esempio}}
\sphinxAtStartPar
\sphinxstylestrong{todo}

\sphinxstepscope


\section{Ciclo Diesel}
\label{\detokenize{ch/thermodynamics/heat-engine-diesel:ciclo-diesel}}\label{\detokenize{ch/thermodynamics/heat-engine-diesel:physics-hs-thermodynamics-heat-engine-diesel}}\label{\detokenize{ch/thermodynamics/heat-engine-diesel::doc}}

\subsection{Ciclo Diesel reale}
\label{\detokenize{ch/thermodynamics/heat-engine-diesel:ciclo-diesel-reale}}\begin{itemize}
\item {} 
\sphinxAtStartPar
aspirazione

\item {} 
\sphinxAtStartPar
compressione adiabatica

\item {} 
\sphinxAtStartPar
combustione

\item {} 
\sphinxAtStartPar
espansione adaibatica

\item {} 
\sphinxAtStartPar
scarico

\end{itemize}

\sphinxstepscope


\section{Ciclo Joule\sphinxhyphen{}Brayton}
\label{\detokenize{ch/thermodynamics/heat-engine-joule-brayton:ciclo-joule-brayton}}\label{\detokenize{ch/thermodynamics/heat-engine-joule-brayton:physics-hs-thermodynamics-heat-engine-joule-brayton}}\label{\detokenize{ch/thermodynamics/heat-engine-joule-brayton::doc}}
\sphinxAtStartPar
\sphinxstylestrong{Storia e applicazioni.}
Il ciclo Joule\sphinxhyphen{}Brayton rappresenta il ciclo termodinamico ideale per il funzionamento a ciclo continuo delle macchine a gas.

\sphinxAtStartPar
Nelle moderne applicazioni, le turbine a gas possono operare
\begin{itemize}
\item {} 
\sphinxAtStartPar
a ciclo aperto: motori a getto, ad esempio per propulsione aeronautica

\item {} 
\sphinxAtStartPar
ciclo chiuso: turbine con rigenerazione

\item {} 
\sphinxAtStartPar
cicli combinati

\end{itemize}

\sphinxAtStartPar
Entrambe le configurazioni sono realizzate con macchine termiche continue, che sono \sphinxstylestrong{sistemi aperti} \sphinxstylestrong{todo} \sphinxstyleemphasis{scrivere la sezione per i sistemi aperti e aggiungere riferimento}


\subsection{Ciclo Joule\sphinxhyphen{}Brayton aperto}
\label{\detokenize{ch/thermodynamics/heat-engine-joule-brayton:ciclo-joule-brayton-aperto}}

\subsection{Ciclo Joule\sphinxhyphen{}Brayton chiuso}
\label{\detokenize{ch/thermodynamics/heat-engine-joule-brayton:ciclo-joule-brayton-chiuso}}
\sphinxAtStartPar
Un modello ideale del ciclo Joule\sphinxhyphen{}Brayton è formato da:
\begin{itemize}
\item {} 
\sphinxAtStartPar
\(1 \rightarrow 2\) compressione adiabatica in compressore, tipicamente dinamico assiale \sphinxhyphen{} sistema aperto

\item {} 
\sphinxAtStartPar
\(2 \rightarrow 3\) combustione a pressione costante: la combustione avviene in camera di combustione aperta e viene modellata come una trasformazione termodinamica a pressione costante; in prima approssimazione, si può trascurare il flusso di massa del combustibile e la variazione delle proprietà chimico\sphinxhyphen{}fisiche del fluido di lavoro; la reazione di combustione produce il calore in ingresso al sistema

\item {} 
\sphinxAtStartPar
\(3 \rightarrow 4\) espansione adiabatica in turbina \sphinxhyphen{} sistema aperto

\item {} 
\sphinxAtStartPar
\(4 \rightarrow 1\), raffreddamento a pressione costante

\end{itemize}


\subsection{Rendimento del ciclo Joule\sphinxhyphen{}Brayton}
\label{\detokenize{ch/thermodynamics/heat-engine-joule-brayton:rendimento-del-ciclo-joule-brayton}}\begin{equation*}
\begin{split}\eta = 1 + \dfrac{\dot{Q}_{41}}{\dot{Q}_{23}}
       = 1 + \dfrac{\dot{m} \, c_P \, (T_1 - T_4)}{\dot{m} \, c_P \, (T_3 - T_2)}
       = 1 + \dfrac{T_1 - T_4}{T_3 - T_2}
\end{split}
\end{equation*}
\sphinxAtStartPar
Usando le condizioni, \sphinxstylestrong{todo} \sphinxstyleemphasis{usare direttamente le espressioni delle adiabatiche ideali ricavate nella sezione delle trasformazioni termodinamiche con gas ideali}
\begin{equation*}
\begin{split}P_2 = P_3 \qquad , \qquad P_1 = P_4\end{split}
\end{equation*}\begin{equation*}
\begin{split}P_1 \, V_1^{\gamma} = P_2 \, V_2^{\gamma}\end{split}
\end{equation*}\begin{equation*}
\begin{split}P_3 \, V_3^{\gamma} = P_4 \, V_4^{\gamma}\end{split}
\end{equation*}
\sphinxAtStartPar
e la legge dei gas ideali, \(P V = m R T\), assumendo che sia un’equazione di stato adatta a descrivere il fluido di lavoro, per riscrivere l’equazione delle trasformazioni adiabatiche
\begin{equation*}
\begin{split}\begin{aligned}
  P_1^{1-\gamma} \, T_1^{\gamma} & = P_2^{1-\gamma} \, T_2^{\gamma} \\
  P_3^{1-\gamma} \, T_3^{\gamma} & = P_4^{1-\gamma} \, T_4^{\gamma}
\end{aligned}
\begin{aligned}
  & \qquad \rightarrow \qquad  (T_4 - T_1) \, P_1^{\frac{1-\gamma}{\gamma}} = (T_3 - T_2) \, P_2^{\frac{1-\gamma}{\gamma}} \\
  & \qquad \rightarrow \qquad  \dfrac{T_4 - T_1}{T_3 - T_2} = \left( \dfrac{P_1}{P_2} \right)^{\frac{\gamma-1}{\gamma}} = \dfrac{1}{\beta^{\frac{\gamma - 1}{\gamma}}} \\
\end{aligned}
\end{split}
\end{equation*}
\sphinxAtStartPar
è possibile riscrivere l’espressione del rendimento del ciclo Otto in funzione unicamente del rapporto di compressione \(\beta := \dfrac{P_2}{P_1}\),
\begin{equation*}
\begin{split}\eta = 1 - \dfrac{1}{\beta^{\frac{\gamma-1}{\gamma}}} \ .\end{split}
\end{equation*}

\subsection{Esempio}
\label{\detokenize{ch/thermodynamics/heat-engine-joule-brayton:esempio}}
\sphinxAtStartPar
\sphinxstylestrong{todo}

\sphinxstepscope


\section{Ciclo Rankine}
\label{\detokenize{ch/thermodynamics/heat-engine-rankine:ciclo-rankine}}\label{\detokenize{ch/thermodynamics/heat-engine-rankine:physics-hs-thermodynamics-heat-engine-rankine}}\label{\detokenize{ch/thermodynamics/heat-engine-rankine::doc}}
\sphinxAtStartPar
Il ciclo Rankine rappresenta il ciclo termodinamico ideale per il funzionamento a ciclo continuo delle macchine a vapore.

\sphinxAtStartPar
Il sistema sfrutta il cambio di fase tra liquido e vapore di un fluido di lavoro, di solito acqua, oggi anche ORC
\begin{itemize}
\item {} 
\sphinxAtStartPar
ciclo aperto nelle applicazioni storiche, come nelle prime locomotive

\item {} 
\sphinxAtStartPar
ciclo chiuso nelle moderne applicazioni nelle centrali elettriche

\end{itemize}

\sphinxAtStartPar
Nelle moderne applicazioni, alcune modifiche/migioramenti:
\begin{itemize}
\item {} 
\sphinxAtStartPar
con surriscaldamento

\item {} 
\sphinxAtStartPar
con rigenerazione

\item {} 
\sphinxAtStartPar
cicli combinati

\end{itemize}

\sphinxstepscope


\section{Problemi}
\label{\detokenize{ch/thermodynamics/heat-engine-problems:problemi}}\label{\detokenize{ch/thermodynamics/heat-engine-problems:physics-hs-thermodynamics-heat-engine-problems}}\label{\detokenize{ch/thermodynamics/heat-engine-problems::doc}}


\begin{sphinxuseclass}{sd-container-fluid}
\begin{sphinxuseclass}{sd-sphinx-override}
\begin{sphinxuseclass}{sd-mb-4}
\begin{sphinxuseclass}{sd-row}
\begin{sphinxuseclass}{sd-g-2}
\begin{sphinxuseclass}{sd-g-xs-2}
\begin{sphinxuseclass}{sd-g-sm-2}
\begin{sphinxuseclass}{sd-g-md-2}
\begin{sphinxuseclass}{sd-g-lg-2}
\begin{sphinxuseclass}{sd-col}
\begin{sphinxuseclass}{sd-d-flex-row}
\begin{sphinxuseclass}{sd-col-8}
\begin{sphinxuseclass}{sd-col-xs-8}
\begin{sphinxuseclass}{sd-col-sm-8}
\begin{sphinxuseclass}{sd-col-md-8}
\begin{sphinxuseclass}{sd-col-lg-8}
\begin{sphinxuseclass}{sd-card}
\begin{sphinxuseclass}{sd-sphinx-override}
\begin{sphinxuseclass}{sd-w-100}
\begin{sphinxuseclass}{sd-shadow-sm}
\begin{sphinxuseclass}{sd-card-body}
\begin{sphinxuseclass}{sd-card-title}
\begin{sphinxuseclass}{sd-font-weight-bold}Problema 1.
\end{sphinxuseclass}
\end{sphinxuseclass}
\sphinxAtStartPar
Testo…

\end{sphinxuseclass}
\end{sphinxuseclass}
\end{sphinxuseclass}
\end{sphinxuseclass}
\end{sphinxuseclass}
\end{sphinxuseclass}
\end{sphinxuseclass}
\end{sphinxuseclass}
\end{sphinxuseclass}
\end{sphinxuseclass}
\end{sphinxuseclass}
\end{sphinxuseclass}
\begin{sphinxuseclass}{sd-col}
\begin{sphinxuseclass}{sd-d-flex-row}
\begin{sphinxuseclass}{sd-col-4}
\begin{sphinxuseclass}{sd-col-xs-4}
\begin{sphinxuseclass}{sd-col-sm-4}
\begin{sphinxuseclass}{sd-col-md-4}
\begin{sphinxuseclass}{sd-col-lg-4}
\begin{sphinxuseclass}{sd-card}
\begin{sphinxuseclass}{sd-sphinx-override}
\begin{sphinxuseclass}{sd-w-100}
\begin{sphinxuseclass}{sd-shadow-sm}
\begin{sphinxuseclass}{sd-card-body}




\end{sphinxuseclass}
\end{sphinxuseclass}
\end{sphinxuseclass}
\end{sphinxuseclass}
\end{sphinxuseclass}
\end{sphinxuseclass}
\end{sphinxuseclass}
\end{sphinxuseclass}
\end{sphinxuseclass}
\end{sphinxuseclass}
\end{sphinxuseclass}
\end{sphinxuseclass}
\end{sphinxuseclass}
\end{sphinxuseclass}
\end{sphinxuseclass}
\end{sphinxuseclass}
\end{sphinxuseclass}
\end{sphinxuseclass}
\end{sphinxuseclass}
\end{sphinxuseclass}
\end{sphinxuseclass}\subsubsection*{Soluzione.}



\begin{sphinxuseclass}{sd-container-fluid}
\begin{sphinxuseclass}{sd-sphinx-override}
\begin{sphinxuseclass}{sd-mb-4}
\begin{sphinxuseclass}{sd-row}
\begin{sphinxuseclass}{sd-g-2}
\begin{sphinxuseclass}{sd-g-xs-2}
\begin{sphinxuseclass}{sd-g-sm-2}
\begin{sphinxuseclass}{sd-g-md-2}
\begin{sphinxuseclass}{sd-g-lg-2}
\begin{sphinxuseclass}{sd-col}
\begin{sphinxuseclass}{sd-d-flex-row}
\begin{sphinxuseclass}{sd-col-8}
\begin{sphinxuseclass}{sd-col-xs-8}
\begin{sphinxuseclass}{sd-col-sm-8}
\begin{sphinxuseclass}{sd-col-md-8}
\begin{sphinxuseclass}{sd-col-lg-8}
\begin{sphinxuseclass}{sd-card}
\begin{sphinxuseclass}{sd-sphinx-override}
\begin{sphinxuseclass}{sd-w-100}
\begin{sphinxuseclass}{sd-shadow-sm}
\begin{sphinxuseclass}{sd-card-body}
\begin{sphinxuseclass}{sd-card-title}
\begin{sphinxuseclass}{sd-font-weight-bold}Problema 2.
\end{sphinxuseclass}
\end{sphinxuseclass}
\sphinxAtStartPar
Testo…

\end{sphinxuseclass}
\end{sphinxuseclass}
\end{sphinxuseclass}
\end{sphinxuseclass}
\end{sphinxuseclass}
\end{sphinxuseclass}
\end{sphinxuseclass}
\end{sphinxuseclass}
\end{sphinxuseclass}
\end{sphinxuseclass}
\end{sphinxuseclass}
\end{sphinxuseclass}
\begin{sphinxuseclass}{sd-col}
\begin{sphinxuseclass}{sd-d-flex-row}
\begin{sphinxuseclass}{sd-col-4}
\begin{sphinxuseclass}{sd-col-xs-4}
\begin{sphinxuseclass}{sd-col-sm-4}
\begin{sphinxuseclass}{sd-col-md-4}
\begin{sphinxuseclass}{sd-col-lg-4}
\begin{sphinxuseclass}{sd-card}
\begin{sphinxuseclass}{sd-sphinx-override}
\begin{sphinxuseclass}{sd-w-100}
\begin{sphinxuseclass}{sd-shadow-sm}
\begin{sphinxuseclass}{sd-card-body}


\end{sphinxuseclass}
\end{sphinxuseclass}
\end{sphinxuseclass}
\end{sphinxuseclass}
\end{sphinxuseclass}
\end{sphinxuseclass}
\end{sphinxuseclass}
\end{sphinxuseclass}
\end{sphinxuseclass}
\end{sphinxuseclass}
\end{sphinxuseclass}
\end{sphinxuseclass}
\end{sphinxuseclass}
\end{sphinxuseclass}
\end{sphinxuseclass}
\end{sphinxuseclass}
\end{sphinxuseclass}
\end{sphinxuseclass}
\end{sphinxuseclass}
\end{sphinxuseclass}
\end{sphinxuseclass}\subsubsection*{Soluzione.}

\sphinxstepscope


\chapter{Meccanismi di trasmissione del calore}
\label{\detokenize{ch/thermodynamics/heat-transmission:meccanismi-di-trasmissione-del-calore}}\label{\detokenize{ch/thermodynamics/heat-transmission:physics-hs-thermodynamics-heat-transmission}}\label{\detokenize{ch/thermodynamics/heat-transmission::doc}}
\sphinxstepscope


\part{Elettromagnetismo}

\sphinxstepscope


\chapter{Elettromagnetismo}
\label{\detokenize{ch/electromagnetism:elettromagnetismo}}\label{\detokenize{ch/electromagnetism:physics-hs-electromagnetism}}\label{\detokenize{ch/electromagnetism::doc}}
\sphinxAtStartPar
\sphinxstylestrong{Presentazione libro}
I contenuti sono suddivisi in quattro diverse sezioni:
\begin{enumerate}
\sphinxsetlistlabels{\arabic}{enumi}{enumii}{}{.}%
\item {} 
\sphinxAtStartPar
\sphinxstylestrong{Introduzione all’elettromagnetismo.} Viene presentata la cronologia e una la descrizione dei \sphinxstylestrong{concetti} e le prime \sphinxstylestrong{esperienze} che hanno permesso di riconoscere i principi fisici che governano i fenomeni elettromagnetici e di costruire una teoria consistente, la teoria dell’elettromagnetismo classico;

\item {} 
\sphinxAtStartPar
\sphinxstylestrong{Principi dell’elettromagnetismo.} Partendo dai concetti e dalle esperienze, vengono formalizzati i principi dell’elettromagnetismo \sphinxstylestrong{todo}

\item {} 
\sphinxAtStartPar
\sphinxstylestrong{Principi di elettrotecnica.} Partendo dai principi generali dell’elettromagnetismo, viene introdotta l’\sphinxstylestrong{approssimazione circuitale} per ridurre il problema elettromagnetico alla soluzione di circuiti.  \sphinxstylestrong{todo}

\item {} 
\sphinxAtStartPar
\sphinxstylestrong{Onde elettromagnetiche.} \sphinxstylestrong{todo}

\end{enumerate}



\sphinxstepscope


\chapter{Introduzione all’elettromagnetismo}
\label{\detokenize{ch/electromagnetism/intro:introduzione-all-elettromagnetismo}}\label{\detokenize{ch/electromagnetism/intro:physics-hs-electromagnetism-intro}}\label{\detokenize{ch/electromagnetism/intro::doc}}
\sphinxstepscope


\section{Breve storia dell’elettromagnetismo}
\label{\detokenize{ch/electromagnetism/intro-history:breve-storia-dell-elettromagnetismo}}\label{\detokenize{ch/electromagnetism/intro-history:physics-hs-electromagnetism-intro-history}}\label{\detokenize{ch/electromagnetism/intro-history::doc}}
\sphinxAtStartPar
\sphinxstylestrong{todo}
…

\sphinxAtStartPar
\sphinxstylestrong{Applicazioni.}
\begin{itemize}
\item {} 
\sphinxAtStartPar
trasferimento energia:

\item {} 
\sphinxAtStartPar
trasferimento informazione:
\begin{itemize}
\item {} 
\sphinxAtStartPar
studi primordiali di Lesage, 1774

\item {} 
\sphinxAtStartPar
telegrafo

\item {} 
\sphinxAtStartPar
onde EM

\end{itemize}

\item {} 
\sphinxAtStartPar
Antichità: …

\item {} 
\sphinxAtStartPar
Primi esperienze e strumenti:
\begin{itemize}
\item {} 
\sphinxAtStartPar
elettrizzazione

\item {} 
\sphinxAtStartPar
macchine elettrostatiche, bottiglia di Leida

\item {} 
\sphinxAtStartPar
…

\end{itemize}

\item {} 
\sphinxAtStartPar
1747, B.Franklin intuisce la legge di conservazione della carica elettrica, \sphinxstyleemphasis{“not created by the the friction, but collected”}

\item {} 
\sphinxAtStartPar
1784, C.A.Coulomb formula la legge di Coulomb usando una bilancia a torsione

\item {} 
\sphinxAtStartPar
1800, A.Volta: pila. Conversione di energia chimia in energia elettrica.
\begin{itemize}
\item {} 
\sphinxAtStartPar
\sphinxstylestrong{todo} principi di funzionamento ed esercizio

\end{itemize}

\item {} 
\sphinxAtStartPar
1806: H.Davy dà origine all’elettrochimica, usando una pila per scomporre sostanze. \sphinxstylestrong{todo} \sphinxstyleemphasis{negli anni successivi, conclusioni su natura elettricità prodotta in maniera differente, ed energia}

\item {} 
\sphinxAtStartPar
Elettromagnetismo:
\begin{itemize}
\item {} 
\sphinxAtStartPar
1820, Oersted

\item {} 
\sphinxAtStartPar
1820\sphinxhyphen{}27, Ampére

\item {} 
\sphinxAtStartPar
1831\sphinxhyphen{}55, Faraday:
\begin{itemize}
\item {} 
\sphinxAtStartPar
induzione EM

\item {} 
\sphinxAtStartPar
…

\end{itemize}

\end{itemize}

\item {} 
\sphinxAtStartPar
Applicazioni e sviluppi della matematica in fisica, “nascita della fisica matematica”:
\begin{itemize}
\item {} 
\sphinxAtStartPar
Poisson

\item {} 
\sphinxAtStartPar
1828, Green \sphinxstyleemphasis{An Essay on the Application on Mathematical Analysis to the Theories of Electricity and Magnetism}

\item {} 
\sphinxAtStartPar
1884, O.Heaviside riformula le equazioni di Maxwell nella forma attualmente conosciuta, usando gli strumenti del calcolo differenziale

\end{itemize}

\item {} 
\sphinxAtStartPar
Strumenti:
\begin{itemize}
\item {} 
\sphinxAtStartPar
1822\sphinxhyphen{}37, galvanometri: Schweigger, Weber + Gauss; galvanometro a riflessione?

\end{itemize}

\item {} 
\sphinxAtStartPar
Elettricità e termodinamica:
\begin{itemize}
\item {} 
\sphinxAtStartPar
1821, Seeback

\item {} 
\sphinxAtStartPar
1827, Ohm

\item {} 
\sphinxAtStartPar
1834, Peltier

\end{itemize}

\item {} 
\sphinxAtStartPar
1850, Kirchhoff e leggi sui circuiti

\item {} 
\sphinxAtStartPar
Primi generatori/motori elettrici; circuiti in AC

\item {} 
\sphinxAtStartPar
\sphinxstylestrong{Maxwell}
\begin{itemize}
\item {} 
\sphinxAtStartPar
correzione e formalizzazione delle equazioni dell’elettromagnetismo

\item {} 
\sphinxAtStartPar
onde EM: velocità di propagazione del campo EM \textasciitilde{} velocità della luce

\end{itemize}

\item {} 
\sphinxAtStartPar
Hertz e onde EM

\item {} 
\sphinxAtStartPar
Elettromagnetismo negli ultimi anni del XIX secolo

\item {} 
\sphinxAtStartPar
Elettromagnetismo all’inizio del XX secolo:
\begin{itemize}
\item {} 
\sphinxAtStartPar
crisi e nuove teorie

\end{itemize}

\end{itemize}

\sphinxstepscope


\section{Esperienze ed esperimenti}
\label{\detokenize{ch/electromagnetism/intro-experiments:esperienze-ed-esperimenti}}\label{\detokenize{ch/electromagnetism/intro-experiments:physics-hs-electromagnetism-intro-experiments}}\label{\detokenize{ch/electromagnetism/intro-experiments::doc}}

\subsection{Elettrizzazione}
\label{\detokenize{ch/electromagnetism/intro-experiments:elettrizzazione}}
\sphinxAtStartPar
…


\subsection{Conservazione della carica}
\label{\detokenize{ch/electromagnetism/intro-experiments:conservazione-della-carica}}
\sphinxAtStartPar
Conservazione della carica, e corrente elettrica


\subsection{Coulomb}
\label{\detokenize{ch/electromagnetism/intro-experiments:coulomb}}
\sphinxAtStartPar
Legge di Coulomb
\begin{equation*}
\begin{split}\vec{F}_{12} = k \frac{q_1 \, q_2}{|\vec{r}_{12}|^2} \hat{r}_{21} \ ,\end{split}
\end{equation*}
\sphinxAtStartPar
avendo definito il vettore \(\vec{r}_{21} = \vec{r}_1 - \vec{r}_2\).
\begin{itemize}
\item {} 
\sphinxAtStartPar
bilancia a torsione

\item {} 
\sphinxAtStartPar
esercizi
\begin{itemize}
\item {} 
\sphinxAtStartPar
bilancia a torsione

\item {} 
\sphinxAtStartPar
bilancia lineare

\item {} 
\sphinxAtStartPar
pendolo
…

\end{itemize}

\end{itemize}


\subsection{Campo elettrico ed energia del campo elettrico}
\label{\detokenize{ch/electromagnetism/intro-experiments:campo-elettrico-ed-energia-del-campo-elettrico}}
\sphinxAtStartPar
…


\subsection{Pila di Volta}
\label{\detokenize{ch/electromagnetism/intro-experiments:pila-di-volta}}
\sphinxAtStartPar
… applicazione delle leggi della termodinamica …


\subsection{…}
\label{\detokenize{ch/electromagnetism/intro-experiments:id1}}
\sphinxstepscope


\section{Problemi}
\label{\detokenize{ch/electromagnetism/intro-problems:problemi}}\label{\detokenize{ch/electromagnetism/intro-problems:physics-hs-electromagnetism-intro-problems}}\label{\detokenize{ch/electromagnetism/intro-problems::doc}}


\begin{sphinxuseclass}{sd-container-fluid}
\begin{sphinxuseclass}{sd-sphinx-override}
\begin{sphinxuseclass}{sd-mb-4}
\begin{sphinxuseclass}{sd-row}
\begin{sphinxuseclass}{sd-g-2}
\begin{sphinxuseclass}{sd-g-xs-2}
\begin{sphinxuseclass}{sd-g-sm-2}
\begin{sphinxuseclass}{sd-g-md-2}
\begin{sphinxuseclass}{sd-g-lg-2}
\begin{sphinxuseclass}{sd-col}
\begin{sphinxuseclass}{sd-d-flex-row}
\begin{sphinxuseclass}{sd-col-8}
\begin{sphinxuseclass}{sd-col-xs-8}
\begin{sphinxuseclass}{sd-col-sm-8}
\begin{sphinxuseclass}{sd-col-md-8}
\begin{sphinxuseclass}{sd-col-lg-8}
\begin{sphinxuseclass}{sd-card}
\begin{sphinxuseclass}{sd-sphinx-override}
\begin{sphinxuseclass}{sd-w-100}
\begin{sphinxuseclass}{sd-shadow-sm}
\begin{sphinxuseclass}{sd-card-body}
\begin{sphinxuseclass}{sd-card-title}
\begin{sphinxuseclass}{sd-font-weight-bold}Problema … Bilancia a torsione
\end{sphinxuseclass}
\end{sphinxuseclass}
\end{sphinxuseclass}
\end{sphinxuseclass}
\end{sphinxuseclass}
\end{sphinxuseclass}
\end{sphinxuseclass}
\end{sphinxuseclass}
\end{sphinxuseclass}
\end{sphinxuseclass}
\end{sphinxuseclass}
\end{sphinxuseclass}
\end{sphinxuseclass}
\end{sphinxuseclass}
\begin{sphinxuseclass}{sd-col}
\begin{sphinxuseclass}{sd-d-flex-row}
\begin{sphinxuseclass}{sd-col-4}
\begin{sphinxuseclass}{sd-col-xs-4}
\begin{sphinxuseclass}{sd-col-sm-4}
\begin{sphinxuseclass}{sd-col-md-4}
\begin{sphinxuseclass}{sd-col-lg-4}
\begin{sphinxuseclass}{sd-card}
\begin{sphinxuseclass}{sd-sphinx-override}
\begin{sphinxuseclass}{sd-w-100}
\begin{sphinxuseclass}{sd-shadow-sm}
\begin{sphinxuseclass}{sd-card-body}
\sphinxAtStartPar
\sphinxincludegraphics{{electrostatics-torsion-balance}.png}



\end{sphinxuseclass}
\end{sphinxuseclass}
\end{sphinxuseclass}
\end{sphinxuseclass}
\end{sphinxuseclass}
\end{sphinxuseclass}
\end{sphinxuseclass}
\end{sphinxuseclass}
\end{sphinxuseclass}
\end{sphinxuseclass}
\end{sphinxuseclass}
\end{sphinxuseclass}
\end{sphinxuseclass}
\end{sphinxuseclass}
\end{sphinxuseclass}
\end{sphinxuseclass}
\end{sphinxuseclass}
\end{sphinxuseclass}
\end{sphinxuseclass}
\end{sphinxuseclass}
\end{sphinxuseclass}\subsubsection*{Soluzione.}



\begin{sphinxuseclass}{sd-container-fluid}
\begin{sphinxuseclass}{sd-sphinx-override}
\begin{sphinxuseclass}{sd-mb-4}
\begin{sphinxuseclass}{sd-row}
\begin{sphinxuseclass}{sd-g-2}
\begin{sphinxuseclass}{sd-g-xs-2}
\begin{sphinxuseclass}{sd-g-sm-2}
\begin{sphinxuseclass}{sd-g-md-2}
\begin{sphinxuseclass}{sd-g-lg-2}
\begin{sphinxuseclass}{sd-col}
\begin{sphinxuseclass}{sd-d-flex-row}
\begin{sphinxuseclass}{sd-col-8}
\begin{sphinxuseclass}{sd-col-xs-8}
\begin{sphinxuseclass}{sd-col-sm-8}
\begin{sphinxuseclass}{sd-col-md-8}
\begin{sphinxuseclass}{sd-col-lg-8}
\begin{sphinxuseclass}{sd-card}
\begin{sphinxuseclass}{sd-sphinx-override}
\begin{sphinxuseclass}{sd-w-100}
\begin{sphinxuseclass}{sd-shadow-sm}
\begin{sphinxuseclass}{sd-card-body}
\begin{sphinxuseclass}{sd-card-title}
\begin{sphinxuseclass}{sd-font-weight-bold}Problema …
\end{sphinxuseclass}
\end{sphinxuseclass}
\end{sphinxuseclass}
\end{sphinxuseclass}
\end{sphinxuseclass}
\end{sphinxuseclass}
\end{sphinxuseclass}
\end{sphinxuseclass}
\end{sphinxuseclass}
\end{sphinxuseclass}
\end{sphinxuseclass}
\end{sphinxuseclass}
\end{sphinxuseclass}
\end{sphinxuseclass}
\begin{sphinxuseclass}{sd-col}
\begin{sphinxuseclass}{sd-d-flex-row}
\begin{sphinxuseclass}{sd-col-4}
\begin{sphinxuseclass}{sd-col-xs-4}
\begin{sphinxuseclass}{sd-col-sm-4}
\begin{sphinxuseclass}{sd-col-md-4}
\begin{sphinxuseclass}{sd-col-lg-4}
\begin{sphinxuseclass}{sd-card}
\begin{sphinxuseclass}{sd-sphinx-override}
\begin{sphinxuseclass}{sd-w-100}
\begin{sphinxuseclass}{sd-shadow-sm}
\begin{sphinxuseclass}{sd-card-body}
\sphinxAtStartPar
\sphinxincludegraphics{{electrostatics-pendulum-charges}.png}



\end{sphinxuseclass}
\end{sphinxuseclass}
\end{sphinxuseclass}
\end{sphinxuseclass}
\end{sphinxuseclass}
\end{sphinxuseclass}
\end{sphinxuseclass}
\end{sphinxuseclass}
\end{sphinxuseclass}
\end{sphinxuseclass}
\end{sphinxuseclass}
\end{sphinxuseclass}
\end{sphinxuseclass}
\end{sphinxuseclass}
\end{sphinxuseclass}
\end{sphinxuseclass}
\end{sphinxuseclass}
\end{sphinxuseclass}
\end{sphinxuseclass}
\end{sphinxuseclass}
\end{sphinxuseclass}\subsubsection*{Soluzione.}



\begin{sphinxuseclass}{sd-container-fluid}
\begin{sphinxuseclass}{sd-sphinx-override}
\begin{sphinxuseclass}{sd-mb-4}
\begin{sphinxuseclass}{sd-row}
\begin{sphinxuseclass}{sd-g-2}
\begin{sphinxuseclass}{sd-g-xs-2}
\begin{sphinxuseclass}{sd-g-sm-2}
\begin{sphinxuseclass}{sd-g-md-2}
\begin{sphinxuseclass}{sd-g-lg-2}
\begin{sphinxuseclass}{sd-col}
\begin{sphinxuseclass}{sd-d-flex-row}
\begin{sphinxuseclass}{sd-col-8}
\begin{sphinxuseclass}{sd-col-xs-8}
\begin{sphinxuseclass}{sd-col-sm-8}
\begin{sphinxuseclass}{sd-col-md-8}
\begin{sphinxuseclass}{sd-col-lg-8}
\begin{sphinxuseclass}{sd-card}
\begin{sphinxuseclass}{sd-sphinx-override}
\begin{sphinxuseclass}{sd-w-100}
\begin{sphinxuseclass}{sd-shadow-sm}
\begin{sphinxuseclass}{sd-card-body}
\begin{sphinxuseclass}{sd-card-title}
\begin{sphinxuseclass}{sd-font-weight-bold}Problema …
\end{sphinxuseclass}
\end{sphinxuseclass}
\end{sphinxuseclass}
\end{sphinxuseclass}
\end{sphinxuseclass}
\end{sphinxuseclass}
\end{sphinxuseclass}
\end{sphinxuseclass}
\end{sphinxuseclass}
\end{sphinxuseclass}
\end{sphinxuseclass}
\end{sphinxuseclass}
\end{sphinxuseclass}
\end{sphinxuseclass}
\begin{sphinxuseclass}{sd-col}
\begin{sphinxuseclass}{sd-d-flex-row}
\begin{sphinxuseclass}{sd-col-4}
\begin{sphinxuseclass}{sd-col-xs-4}
\begin{sphinxuseclass}{sd-col-sm-4}
\begin{sphinxuseclass}{sd-col-md-4}
\begin{sphinxuseclass}{sd-col-lg-4}
\begin{sphinxuseclass}{sd-card}
\begin{sphinxuseclass}{sd-sphinx-override}
\begin{sphinxuseclass}{sd-w-100}
\begin{sphinxuseclass}{sd-shadow-sm}
\begin{sphinxuseclass}{sd-card-body}
\sphinxAtStartPar
\sphinxincludegraphics{{electrostatics-charge-equilibrium}.png}



\end{sphinxuseclass}
\end{sphinxuseclass}
\end{sphinxuseclass}
\end{sphinxuseclass}
\end{sphinxuseclass}
\end{sphinxuseclass}
\end{sphinxuseclass}
\end{sphinxuseclass}
\end{sphinxuseclass}
\end{sphinxuseclass}
\end{sphinxuseclass}
\end{sphinxuseclass}
\end{sphinxuseclass}
\end{sphinxuseclass}
\end{sphinxuseclass}
\end{sphinxuseclass}
\end{sphinxuseclass}
\end{sphinxuseclass}
\end{sphinxuseclass}
\end{sphinxuseclass}
\end{sphinxuseclass}\subsubsection*{Soluzione.}



\begin{sphinxuseclass}{sd-container-fluid}
\begin{sphinxuseclass}{sd-sphinx-override}
\begin{sphinxuseclass}{sd-mb-4}
\begin{sphinxuseclass}{sd-row}
\begin{sphinxuseclass}{sd-g-2}
\begin{sphinxuseclass}{sd-g-xs-2}
\begin{sphinxuseclass}{sd-g-sm-2}
\begin{sphinxuseclass}{sd-g-md-2}
\begin{sphinxuseclass}{sd-g-lg-2}
\begin{sphinxuseclass}{sd-col}
\begin{sphinxuseclass}{sd-d-flex-row}
\begin{sphinxuseclass}{sd-col-8}
\begin{sphinxuseclass}{sd-col-xs-8}
\begin{sphinxuseclass}{sd-col-sm-8}
\begin{sphinxuseclass}{sd-col-md-8}
\begin{sphinxuseclass}{sd-col-lg-8}
\begin{sphinxuseclass}{sd-card}
\begin{sphinxuseclass}{sd-sphinx-override}
\begin{sphinxuseclass}{sd-w-100}
\begin{sphinxuseclass}{sd-shadow-sm}
\begin{sphinxuseclass}{sd-card-body}
\begin{sphinxuseclass}{sd-card-title}
\begin{sphinxuseclass}{sd-font-weight-bold}Problema …
\end{sphinxuseclass}
\end{sphinxuseclass}
\end{sphinxuseclass}
\end{sphinxuseclass}
\end{sphinxuseclass}
\end{sphinxuseclass}
\end{sphinxuseclass}
\end{sphinxuseclass}
\end{sphinxuseclass}
\end{sphinxuseclass}
\end{sphinxuseclass}
\end{sphinxuseclass}
\end{sphinxuseclass}
\end{sphinxuseclass}
\begin{sphinxuseclass}{sd-col}
\begin{sphinxuseclass}{sd-d-flex-row}
\begin{sphinxuseclass}{sd-col-4}
\begin{sphinxuseclass}{sd-col-xs-4}
\begin{sphinxuseclass}{sd-col-sm-4}
\begin{sphinxuseclass}{sd-col-md-4}
\begin{sphinxuseclass}{sd-col-lg-4}
\begin{sphinxuseclass}{sd-card}
\begin{sphinxuseclass}{sd-sphinx-override}
\begin{sphinxuseclass}{sd-w-100}
\begin{sphinxuseclass}{sd-shadow-sm}
\begin{sphinxuseclass}{sd-card-body}
\sphinxAtStartPar
\sphinxincludegraphics{{electrostatics-pendulum-e-field}.png}



\end{sphinxuseclass}
\end{sphinxuseclass}
\end{sphinxuseclass}
\end{sphinxuseclass}
\end{sphinxuseclass}
\end{sphinxuseclass}
\end{sphinxuseclass}
\end{sphinxuseclass}
\end{sphinxuseclass}
\end{sphinxuseclass}
\end{sphinxuseclass}
\end{sphinxuseclass}
\end{sphinxuseclass}
\end{sphinxuseclass}
\end{sphinxuseclass}
\end{sphinxuseclass}
\end{sphinxuseclass}
\end{sphinxuseclass}
\end{sphinxuseclass}
\end{sphinxuseclass}
\end{sphinxuseclass}\subsubsection*{Soluzione.}

\sphinxstepscope


\chapter{Fondamenti di elettromagnetismo}
\label{\detokenize{ch/electromagnetism/em:fondamenti-di-elettromagnetismo}}\label{\detokenize{ch/electromagnetism/em:physics-hs-electromagnetism-em}}\label{\detokenize{ch/electromagnetism/em::doc}}
\sphinxAtStartPar
In questa sezione vengono ripresi i concetti e le esperienze fondamentali per formulare i \sphinxstylestrong{principi} dell’elettromagnetismo:
\begin{itemize}
\item {} 
\sphinxAtStartPar
principio di conservazione della carica elettrica

\item {} 
\sphinxAtStartPar
equazioni di Maxwell per il campo elettromagnetico

\item {} 
\sphinxAtStartPar
forza di Lorentz, agente su cariche elettriche in un campo magnetico

\end{itemize}

\sphinxAtStartPar
Lo sviluppo della materia include dei cenni sul comportamento dei materiali sottoposti a fenomeni elettromagnetici, riassumibile con le equazioni costitutive del materiale, e alcune applicazioni.

\sphinxAtStartPar
La presentazione degli argomenti segue qualitativamente un’ordine cronologico e di complessità della descrizione dei fenomeni coinvolti.

\sphinxAtStartPar
\sphinxstylestrong{Elettrostatica.} Partendo dalla \sphinxstyleemphasis{forza di Coulomb} scambiata tra due cariche puntiformi in quiete nello spazio, viene introdotto il concetto di \sphinxstyleemphasis{campo elettrico} tramite una sua definzione operativa. Il campo elettrico è \sphinxstyleemphasis{conservativo in regime stazionario} ed è quindi possibile introdurre un’\sphinxstyleemphasis{energia potenziale} e un \sphinxstyleemphasis{potenziale elettrico}. Viene descitta la risposta in un campo elettrico di materiali suscettibili alla \sphinxstyleemphasis{polarizzazione}. Vengono riassunte le proprietà del campo elettrico in regime stazionario in termini di \sphinxstyleemphasis{flusso} e \sphinxstyleemphasis{circuitazione}, con quelle che saranno le prime due equazioni di Maxwell: la \sphinxstyleemphasis{legge di Gauss per il campo elettrico} e la \sphinxstyleemphasis{legge di Faraday} in regime stazionario. Infine vengono analizzati modelli ideali di \sphinxstyleemphasis{condensatore}, componente elementare di molti circuiti elettrici.

\sphinxAtStartPar
\sphinxstylestrong{Corrente elettrica.} Viene introdotto il concetto di \sphinxstyleemphasis{corrente elettrica}, partendo da una descrizione microscopica del moto di cariche elementari discrete. Viene formulato il \sphinxstyleemphasis{principio di conservazione della carica eletttrica}. Infine viene discusso il fenomeno della conduzione elettrica in diversi materiali: viene descritto il modello ideale di \sphinxstyleemphasis{resistenza elettrica (di Ohm)}, componente elementare di molti circuiti elettrici; la conduzione elettrica nei gas permette di discutere dei primi esperimenti sulla natura della materia; l’analisi dei semiconduttori permette di discutere materiali e componenti elettrici fondamentali per l’elettronica contemporanea.

\sphinxAtStartPar
\sphinxstylestrong{Magnetismo ed elettromagnetismo stazionario.} Vengono introdotti i fenomeni magnetici. Con le esperienze di Faraday, Oersted e Ampére, viene descritto il legame “monodirezionale” in regime stazionario tra fenomeni elettrici e fenomeni magnetici: la corrente elettrica produce un magnetico, descritto dalla \sphinxstyleemphasis{legge di Biot\sphinxhyphen{}Savart}. I risultati dell’esperienza di Faraday permettono la descrizione di versioni rudimentali degli strumenti di misura della corrente e della differenza di potenziale. Le proprietà del campo magnetico vengono riassunte in termini di \sphinxstyleemphasis{flusso} e \sphinxstyleemphasis{circuitazione} con quelle che saranno altre due equazioni di Mawell: la \sphinxstyleemphasis{legge di Gauss per il campo magnetico} e la \sphinxstyleemphasis{legge di Ampére}. Queste leggi fisiche vengono utilizzate per l’analisi di modelli ideali di \sphinxstyleemphasis{induttore}, componente elementare di molti circuiti elettrici, elettromagnetici ed elettromeccanici. Viene presentata infine la \sphinxstyleemphasis{correzione di Maxwell} della legge di Ampére con l’aggiunta del termine non\sphinxhyphen{}stazionario che la rende compatibile con l’equazione di conservazione della carica elettrica: la versione corretta viene infine applicata al processo di carica di un condensatore.

\sphinxAtStartPar
\sphinxstylestrong{Elettromagnetismo.} Con la \sphinxstyleemphasis{legge di induzione di Faraday}, viene introdotto l’accoppiamento inverso a quello descritto nella sezione precedente: un flusso di campo magnetico variabile nel tempo, induce un campo elettrico.

\sphinxstepscope


\section{Elettrostatica}
\label{\detokenize{ch/electromagnetism/electrostatics:elettrostatica}}\label{\detokenize{ch/electromagnetism/electrostatics:physics-hs-electromagnetism-electrostatics}}\label{\detokenize{ch/electromagnetism/electrostatics::doc}}

\subsection{Legge di Coulomb}
\label{\detokenize{ch/electromagnetism/electrostatics:legge-di-coulomb}}
\sphinxAtStartPar
Date due cariche elettriche puntiformi \(q_1\), \(q_2\), nella posizione \(P_1\), \(P_2\) nello spazio, la forza
\begin{equation*}
\begin{split}\vec{F}_{12} = k \frac{q_1 \, q_2}{|\vec{r}_{12}|^2} \, \hat{r}_{21} = \frac{1}{4 \pi \varepsilon}\frac{q_1 \, q_2}{|\vec{r}_{12}|^2} \, \hat{r}_{21}\end{split}
\end{equation*}
\sphinxAtStartPar
essendo \(\vec{r}_{21}\) il vettore che congiunge il punto \(P_2\) con il punto \(P_{1}\), \(\vec{r}_{21} = \vec{r}_1 - \vec{r}_2\).


\begin{savenotes}\sphinxattablestart
\centering
\begin{tabulary}{\linewidth}[t]{|T|T|T|}
\hline

\sphinxAtStartPar
\sphinxincludegraphics{{electrostatics-coulomb-pp}.png}
&
\sphinxAtStartPar
\sphinxincludegraphics{{electrostatics-coulomb-pn}.png}
&
\sphinxAtStartPar
\sphinxincludegraphics{{electrostatics-coulomb-nn}.png}
\\
\hline
\end{tabulary}
\par
\sphinxattableend\end{savenotes}







\sphinxAtStartPar
La scelta della definizione della costante di proporzionalità, \(k = \frac{1}{4 \pi \varepsilon}\), viene fatta per ottenere un’espressione della {\hyperref[\detokenize{ch/electromagnetism/electrostatics:physics-hs-electromagnetism-electrostatics-maxwell-gauss}]{\sphinxcrossref{\DUrole{std,std-ref}{legge di Gauss per il campo elettrico}}}} senza fattori numerici.

\sphinxAtStartPar
La costante \(\varepsilon\) viene definita costante dielettrica del mezzo. Per cariche elettriche posizionate nello spazio “vuoto” (di materia ma non di proprietà fisiche), nell’espressione della legge di Coulomb compare la \sphinxstylestrong{costante dielettrica nel vuoto},
\begin{equation*}
\begin{split}\varepsilon_0 = 8.854 \cdot 10^{-12} \, C^2 \, N^{-1} \, m^{-2} \ .\end{split}
\end{equation*}
\sphinxAtStartPar
Materiali isotropi lineari non dispersivi possono essere caratterizzati da una sola costante, la costante dielettrica del materiale. Questa caratteristica del materiale viene di solito definita come multiplo della costante dielettrica del vuoto, tramite la costante dielettrica relativa \(\varepsilon_r\),
\begin{equation*}
\begin{split}\varepsilon = \varepsilon_r \,\varepsilon_0 \ . \end{split}
\end{equation*}
\sphinxAtStartPar
Vale il \sphinxstylestrong{principio di sovrapposizione delle cause e degli effetti}. In presenza di 3 cariche puntiformi, \(q_1\), \(q_2\), \(q_3\), la forza totale agente sulla carica \(q_1\) è uguale alla somma delle forze dovute a \(q_2\) e \(q_3\),
\begin{equation*}
\begin{split}\vec{F}_1 = \vec{F}_{12} + \vec{F}_{13} = \frac{q_1 \, q_2}{4 \pi \varepsilon}\frac{\vec{r}_1 - \vec{r}_2}{|\vec{r}_1 - \vec{r}_2|^2} +  \frac{q_1 \, q_3}{4 \pi \varepsilon}\frac{\vec{r}_1 - \vec{r}_3}{|\vec{r}_1 - \vec{r}_3|^2} \ .\end{split}
\end{equation*}
\begin{figure}[htbp]
\centering

\noindent\sphinxincludegraphics{{electrostatics-coulomb-psce}.png}
\end{figure}




\subsubsection{Misura della carica elettrica}
\label{\detokenize{ch/electromagnetism/electrostatics:misura-della-carica-elettrica}}
\sphinxAtStartPar
Un elettrometro è uno strumento di misura della carica elettrica. Una versione rudimentale di un elettrometro è la bilancia di torsione usata da Coulomb nei suoi esperimenti.

\sphinxAtStartPar
Il momento generato dalla forza di Coulomb sulla carica elettrica incognita \(q_1\) dalla carica elettrica \(q_2\) equilibria il momento elastico della bilancia di torsione. Se la struttura ha una equazione costitutiva il momento strutturale è proproporzionale alla rotazione, \(M_z = K \, \theta\).

\sphinxAtStartPar
\sphinxstylestrong{todo} \sphinxstyleemphasis{svolgere conti qui o rimandare a esercizi?}

\begin{figure}[htbp]
\centering

\noindent\sphinxincludegraphics{{electrostatics-torsion-balance-coulomb}.png}
\end{figure}


\subsection{Il campo elettrico}
\label{\detokenize{ch/electromagnetism/electrostatics:il-campo-elettrico}}
\sphinxAtStartPar
Data una distribuzione di cariche nello spazio, è possibile descriverla tramite l’effetto che avrebbe su una carica qualsiasi posta in un punto arbitrario dello spazio, introducendo la definizione di campo elettrico.

\sphinxAtStartPar
Viene data qui una \sphinxstylestrong{definizione operativa} del campo elettrico. Data una distribuzione di cariche, \(q_i\), nei punti dello spazio \(P_i\), si prende una carica test \sphinxhyphen{} di prova \sphinxhyphen{} di intensità nota \(q^{test}\), che può essere posizionata in ogni punto \(P\) dello spazio. E’ inoltre possibile misurare la forza \(\vec{F}(P; q^{test})\) agente sulla carica di prova dovuta all’interazione con la distribuzione di cariche in esame,
\begin{equation*}
\begin{split}\begin{aligned}
  \vec{F}_{test}(P, q^{test})
  & = \sum_i \vec{F}_{test,i}(P) = \\
  & = \sum_i \frac{1}{4 \pi \varepsilon}\frac{q_i \, q_{test}}{|\vec{r}_{i,test,i}|^2} \, \hat{r}_{i,test} = \\
  & = q_{test} \sum_i \frac{1}{4 \pi \varepsilon}\frac{q_i}{|\vec{r}_{i,test,i}|^2} \, \hat{r}_{i,test} = \\
  & = q_{test} \, \vec{e}(P; \, q_i, \, P_i) \ .
\end{aligned}\end{split}
\end{equation*}
\sphinxAtStartPar
Poichè la forza sulla carica di prova è proporzionale alla sua carica elettrica, è possibile descrivere l’effetto della distribuzione nota di cariche nello spazio con la funzione \(\vec{e}(P; \, q_i, \, P_i)\). Questa funzione viene definita \sphinxstylestrong{campo elettrico} della distribuzione delle cariche.

\sphinxAtStartPar
Viceversa, noto il campo elettrico di una distribuzione di cariche, la forza agente su una carica elettrica \(q\) posta nel punto \(P\) dello spazio è
\begin{equation*}
\begin{split}\vec{F} = q \, \vec{e}(P) \ .\end{split}
\end{equation*}\begin{itemize}
\item {} 
\sphinxAtStartPar
\sphinxstylestrong{todo} Poichè il PSCE vale per la forza, il \sphinxstylestrong{PSCE} vale per il campo elettrico

\end{itemize}


\subsubsection{Campo conservativo}
\label{\detokenize{ch/electromagnetism/electrostatics:campo-conservativo}}
\sphinxAtStartPar
Come mostrato (\sphinxstylestrong{todo}  ah sì? fare riferimenti qui?) per il campo gravitazionale, anche il campo elettrostatico è un campo conservativo.

\sphinxAtStartPar
Il lavoro fatto dal campo su una carica che descrive una traiettoria \(\gamma\), con estremi \(A\), \(B\) è uguale a
\begin{equation*}
\begin{split}\begin{aligned}
  L & = \int_{\gamma} \vec{F}(P) \cdot d \vec{r} = - \int_{\gamma} \nabla U(P) \cdot d \vec{r} = - \Delta U = U(A) - U(B) \\
    & = q \, \int_{\gamma} \vec{e}(P) \cdot d \vec{r} = - q \int_{\gamma} \nabla V(P) \cdot d \vec{r} = - q \, \Delta V = q \, \left( V(A) - V(B) \right) \\
\end{aligned}\end{split}
\end{equation*}
\sphinxAtStartPar
avendo definito l’\sphinxstylestrong{energia potenziale} \(U(P)\) del sistema di cariche che produce il campo elettrico \(\vec{e}(P)\) e il \sphinxstylestrong{potenziale elettrico} \(V(P)\) come l’energia potenziale per unità di carica \(q\). Sia l’energia potenziale sia il potenziale sono definiti a meno di una costante additiva.

\sphinxAtStartPar
Il potenziale generato da una carica \(q_i\) posizionata punto “potenziante” \(P_i\) nel punto “potenziato” \(P\)
\begin{equation*}
\begin{split}V_i(P) = \frac{1}{4 \pi \varepsilon} \frac{q_i}{|\vec{r}_i|} \ ,\end{split}
\end{equation*}
\sphinxAtStartPar
con \(\vec{r}_i = P - P_i\). Poichè il PSCE vale per la forza e il campo elettrico, il \sphinxstylestrong{PSCE} vale per il potenziale, e quindi il potenziale elettrico generato da un sistema di cariche è la somma del potenziale elettrico generato dalle singole cariche,
\begin{equation*}
\begin{split}V_i(P) = \frac{1}{4 \pi \varepsilon} \sum_i \frac{q_{i}}{\left|\vec{r}_{i}\right|} \end{split}
\end{equation*}

\subsubsection{Energia potenziale di una distribuzione di cariche}
\label{\detokenize{ch/electromagnetism/electrostatics:energia-potenziale-di-una-distribuzione-di-cariche}}
\sphinxAtStartPar
L’energia potenziale di un sistema di cariche è uguale al lavoro (delle forze esterne = \sphinxhyphen{} lavoro forza elettrica) fatto per costruire tale distribuzione. Poiché in meccanica classica l’energia è definita a meno di una costante additiva arbitraria, si può considerare la condizione di riferimento con le cariche poste all‘“infinito” o, meglio, infinitamente distanti una dalle altre.

\sphinxAtStartPar
Per un sistema di cariche puntiformi, l’energia potenziale del sistema è uguale alla somma dell’energia potenziale tra le singole coppie di cariche
\begin{equation*}
\begin{split}E^{pot} = \sum_{\{i,j\}, i \ne j} V_{ij} = \sum_{\{i,j\}, i \ne j} \frac{1}{4 \pi \varepsilon} \frac{{q}_{i} \, q_{j}}{r_{ij}} \ ,\end{split}
\end{equation*}
\sphinxAtStartPar
senza ripetere la sommatoria sulle coppie con gli elementi invertiti.

\sphinxAtStartPar
Seguono due dimostrazioni di questa formula, ottenute costruendo il sistema di cariche dall’infinito in due maneire diverse.



\sphinxAtStartPar
 \sphinxstylestrong{todo} 
\subsubsection*{Posizionando una carica alla volta}
\begin{equation*}
\begin{split}\begin{aligned}
  L^{ext}_1 & = 0 \\
  L^{ext}_2 & = \frac{1}{4 \pi \varepsilon} \frac{q_1 \, q_2}{r_{12}}  \\
  L^{ext}_3 & = \frac{1}{4 \pi \varepsilon} \frac{q_1 \, q_3}{r_{13}} + \frac{1}{4 \pi \varepsilon} \frac{q_2 \, q_3}{r_{23}}  \\
            & ... \\
  L^{ext}_n & = \sum_{j=1}^{n-1} \frac{1}{4 \pi \varepsilon} \frac{q_1 \, q_n}{r_{1n}} \\
\end{aligned}\end{split}
\end{equation*}\begin{equation*}
\begin{split}E^{pot} = L^{ext} = \sum_i L^{ext}_i = \sum_{\{i, j\}, i \ne j} \frac{1}{4 \pi \varepsilon} \frac{q_1 \, q_j}{r_{ij}} \ .\end{split}
\end{equation*}\subsubsection*{Posizionando le cariche contemporanamente}

\sphinxAtStartPar
Posizionando tutte le cariche contamporaneamente con una “scalatura” delle distanze, \(\vec{r}_i(\alpha) = \frac{\vec{r}_i}{\alpha}\), \(\alpha \in (0, 1]\), il lavoro delle forze elettriche è
\begin{equation*}
\begin{split}\begin{aligned}
  dL_i(\alpha) & = \sum_{j \ne i} \vec{F}_{ij}(\alpha) \cdot d \vec{r}_i(\alpha) = \\
  & = \sum_{j \ne i} \frac{q_i \, q_j}{4 \pi \varepsilon}  \frac{1}{\left| \frac{\vec{r}_i}{\alpha} - \frac{\vec{r}_j}{\alpha}\right|^2} \hat{r}_{ji} \cdot \left(-\frac{\vec{r}_i}{\alpha^2}\right) \, d \alpha = \\
  & = - \sum_{j \ne i} \frac{q_i \, q_j}{4 \pi \varepsilon}  \frac{\hat{r}_{ji} \cdot\vec{r}_i}{\left| \vec{r}_i - \vec{r}_j\right|^2}  \, d \alpha
\end{aligned}\end{split}
\end{equation*}\begin{equation*}
\begin{split}\begin{aligned}
 dL(\alpha) & = \sum_i d L_i = \\
  & = - \sum_{i} \sum_{j \ne i} \frac{q_i \, q_j}{4 \pi \varepsilon}  \frac{\hat{r}_{ji} \cdot\vec{r}_i}{\left| \vec{r}_i - \vec{r}_j\right|^2}  \, d \alpha = \\
  & = - \sum_{\{i,j\}, i \ne j} \frac{q_i \, q_j}{4 \pi \varepsilon}  \frac{\hat{r}_{ji} \cdot \left( \vec{r}_i - \vec{r}_j \right)}{\left| \vec{r}_i - \vec{r}_j\right|^2}  \, d \alpha = \\
  & = - \sum_{\{i,j\}, i \ne j} \frac{q_i \, q_j}{4 \pi \varepsilon}  \frac{1}{r_{ij}}  \, d \alpha  \ ,
\end{aligned}\end{split}
\end{equation*}
\sphinxAtStartPar
e il lavoro diventa
\begin{equation*}
\begin{split}
 L = \int_{\alpha = 0}^{1} dL (\alpha) =  - \int_{\alpha=0}^{1} \sum_{\{i,j\}, i \ne j} \frac{q_i \, q_j}{4 \pi \varepsilon}  \frac{1}{r_{ij}}  \, d \alpha = - \sum_{\{i,j\}, i \ne j} \frac{q_i \, q_j}{4 \pi \varepsilon}  \frac{1}{r_{ij}} \ .
\end{split}
\end{equation*}

\subsection{Campo elettrico nei materiali}
\label{\detokenize{ch/electromagnetism/electrostatics:campo-elettrico-nei-materiali}}\begin{itemize}
\item {} 
\sphinxAtStartPar
polarizzazione…

\end{itemize}

\sphinxAtStartPar
Per materiali lineari isotropi,
\begin{equation*}
\begin{split}\vec{d} := \varepsilon \vec{e} = \varepsilon_0 \vec{e} + \vec{p}\end{split}
\end{equation*}
\sphinxAtStartPar
\sphinxstylestrong{todo} polarizzazione, cariche libere e cariche “vincolate”


\subsection{Verso le equazioni di Maxwell}
\label{\detokenize{ch/electromagnetism/electrostatics:verso-le-equazioni-di-maxwell}}\label{\detokenize{ch/electromagnetism/electrostatics:physics-hs-electromagnetism-electrostatics-maxwell}}

\subsubsection{Legge di Gauss per il flusso del campo elettrico}
\label{\detokenize{ch/electromagnetism/electrostatics:legge-di-gauss-per-il-flusso-del-campo-elettrico}}\label{\detokenize{ch/electromagnetism/electrostatics:physics-hs-electromagnetism-electrostatics-maxwell-gauss}}\begin{equation*}
\begin{split}\Phi_{\partial V}(\vec{d}) = Q_V\end{split}
\end{equation*}\subsubsection*{Dimostrazione della legge di Gauss}

\sphinxAtStartPar
\sphinxstylestrong{Dimostrazione per una carica puntiforme e una superficie sferica.}
Il calcolo diretto del flusso del campo elettrico generato da una carica puntiforme attraverso una superficie sferica di raggio \(r\) centrata nella carica
\begin{equation*}
\begin{split}\Phi_{S^{sphere}}(\vec{d}) = \oint_{S^{sphere}} \vec{d} \cdot \hat{n} = \oint_{S^{sphere}} \frac{1}{4 \pi }\frac{q}{r^2} \underbrace{\hat{r} \cdot \hat{r}}_{=1} \ .  \end{split}
\end{equation*}
\sphinxAtStartPar
L’integranda è costante, essendo \(r\) costante sulla superficie sferica, e quindi si riduce al prodotto della funzione integranda per l’estensione del dominio di integrazione, qui la superficie estenra della sfera. Ricordando che la superficie di una superficie sferica di raggio \(r\) è \(S = 4 \pi r^2\), si ottiene l’espressione della legge di Gauss per il campo elettrico di una carica puntiforme attraverso una superficie sferica,
\begin{equation*}
\begin{split}\Phi_{S^{sphere}}(\vec{d}) = 4 \, \pi \, r^2 \frac{1}{4 \, \pi \, r^2} q = q \ .\end{split}
\end{equation*}
\sphinxAtStartPar
\sphinxstylestrong{todo} obs: andamento del campo come \(r^{-2}\) implica andamento del flusso costante attraverso superfici che sottengono lo stesso \sphinxstylestrong{angolo solido}

\sphinxAtStartPar
\sphinxstylestrong{todo} … altra osservazione che ora non ricordo…

\sphinxAtStartPar
\sphinxstylestrong{Dimostrazione per una carica puntiforme e per una superficie arbitraria.}
Usando l’osservazione sull’andamento del campo, e la definizione di angolo solido
\begin{equation*}
\begin{split}\oint_S \frac{q}{4 \pi} \frac{1}{r^2} \hat{r} \cdot \hat{n} \, dS =
\oint_{\Omega} \frac{q}{4 \pi}  \, d \Omega = q \end{split}
\end{equation*}
\sphinxAtStartPar
\sphinxstylestrong{Dimostrazione per una distribuzione di carica qualsiasi e superficie arbitraria.}
Avendo dimostrato la legge di Gauss per una carica puntiforme attraverso una superficie arbitraria, la legge di Gauss per il campo \(\vec{d}\) generato da una distribuzione di carica qualsiasi segue immediatamente, ricordando che vale il PSCE
\begin{equation*}
\begin{split}\Phi_{\partial V}(\vec{d}_i) = q_i\end{split}
\end{equation*}\begin{equation*}
\begin{split}\sum_i \Phi_{\partial V}(\vec{d}_i) = \Phi_{\partial V} \left(\sum_i \vec{d}_i \right) = \sum_i q_i\end{split}
\end{equation*}\begin{equation*}
\begin{split}\Phi_{\partial V}(\vec{d}) = Q_V\end{split}
\end{equation*}

\subsubsection{Legge di Faraday, in elettrostatica}
\label{\detokenize{ch/electromagnetism/electrostatics:legge-di-faraday-in-elettrostatica}}\begin{itemize}
\item {} 
\sphinxAtStartPar
La legge di Faraday in elettrostatica è una diretta conseguenza della conservatività del campo elettrico
\begin{equation*}
\begin{split}\Gamma_{\ell}(\vec{e}) = \oint_{\ell} \vec{e} \cdot \hat{t} = 0 \ .\end{split}
\end{equation*}
\item {} 
\sphinxAtStartPar
Questa equazione è valida \sphinxstylestrong{solo} in un regime elettrostatico: la forma generale dell’equazione di Faraday prevede un termine dipendente dal tempo, che è identicamente nullo nel regime elettrostatico.

\end{itemize}
\subsubsection*{Dimostrazione della legge di Faraday}

\sphinxAtStartPar
\sphinxstylestrong{Dimostrazione per una carica puntiforme e un percorso circolare.}
Il calcolo diretto della circuitazione del campo elettrico generato da una carica puntiforme lungo un percorso circolare di raggio \(r\) centrato nella carica
\begin{equation*}
\begin{split}\Gamma_{\ell^{circle}}(\vec{e}) = \oint_{\ell^{circle}} \vec{e} \cdot \hat{t} = \oint_{S^{sphere}} \frac{1}{4 \pi \varepsilon}\frac{q}{r^2} \underbrace{\hat{r} \cdot \hat{t}}_{=0} = 0 \ ,  \end{split}
\end{equation*}
\sphinxAtStartPar
poiché il versore tangente al percorso circolare è ortogonale al campo elettrico, diretto in direzione radiale.

\sphinxAtStartPar
\sphinxstylestrong{Dimostrazione per una carica puntiforme e un percorso arbitrario.}

\sphinxAtStartPar
\sphinxstylestrong{Dimostrazione per una distribuzione di carica qualsiasi e percorso arbitrario.}
Avendo dimostrato la legge di Faraday nel caso stazionario per una carica puntiforme lungo un percorso arbitrario, la legge di Faraday in regime stazionario per il \(\vec{e}\) generato da una distribuzione di carica qualsiasi segue immediatamente, ricordando che vale il PSCE
\begin{equation*}
\begin{split}\Gamma_{\partial S}(\vec{e}_i) = 0\end{split}
\end{equation*}\begin{equation*}
\begin{split}\sum_i \Gamma_{\partial S}(\vec{e}_i) = \Gamma_{\partial S} \left(\sum_i \vec{e}_i \right) = 0\end{split}
\end{equation*}\begin{equation*}
\begin{split}\Gamma_{\partial S}(\vec{e}) = 0\end{split}
\end{equation*}

\subsection{Moto di una carica in un campo elettrico}
\label{\detokenize{ch/electromagnetism/electrostatics:moto-di-una-carica-in-un-campo-elettrico}}
\sphinxAtStartPar
Il moto di una corpo puntiforme di massa \(m\) e carica elettrica \(q\) in una regione dello spazio nel quale c’è un campo elettrico \(\vec{e}(\vec{r})\) è soggetto a una forza esterna \(\vec{F}^{el} = q \, \vec{e}(P)\). L’equazione del moto diventa quindi
\begin{equation*}
\begin{split}m \ddot{ \vec{r} } = \vec{R}^{ext} = q \, \vec{e}(P) + \vec{F}^{\text{non }\vec{e}}\end{split}
\end{equation*}\begin{itemize}
\item {} 
\sphinxAtStartPar
\sphinxstylestrong{todo} esempi

\end{itemize}


\subsection{Condensatore}
\label{\detokenize{ch/electromagnetism/electrostatics:condensatore}}

\subsubsection{Condensatore infinito piano}
\label{\detokenize{ch/electromagnetism/electrostatics:condensatore-infinito-piano}}\begin{equation*}
\begin{split}e = \frac{\sigma}{\varepsilon}\end{split}
\end{equation*}\begin{equation*}
\begin{split}Q = \sigma \, A\end{split}
\end{equation*}\begin{equation*}
\begin{split}\Delta V = \int_{\ell} \vec{e} \cdot d \vec{r} = \ell \, e\end{split}
\end{equation*}\begin{equation*}
\begin{split} Q = \sigma \, A = \varepsilon \, e \, A = \frac{\varepsilon \, \ell}{A} \, \Delta V = C \, \Delta V \ ,\end{split}
\end{equation*}
\sphinxAtStartPar
\(C\) capacità, \(C = \frac{\varepsilon \, A}{\ell}\) capacità per un condensatore piano.
\subsubsection*{Condensatore cilindrico}

\sphinxAtStartPar
\sphinxstylestrong{todo}
\subsubsection*{Condensatore sferico}

\sphinxAtStartPar
Tra le sfere del condensatore, il campo elettrico è ha direzione radiale e valore assoluto \(\propto r^{-2}\),
\begin{equation*}
\begin{split}\vec{e}(r) = \frac{1}{4 \pi \varepsilon} \frac{Q}{r^2} \hat{r} \ .\end{split}
\end{equation*}
\sphinxAtStartPar
dove la carica totale della superficie sferica con distribuzione di carica uniforme è data dal prodotto della densità superficiale di carica e la superficie, \(Q = \sigma \, S_1 = \sigma \, 4 \pi \, R_1^2\).
La differenza di potenziale tra le due armature è quindi
\begin{equation*}
\begin{split}\Delta V = - \int_{\ell} \vec{e}(r) \cdot \hat{r} = \int_{r=R_1}^{R_2} \frac{Q}{4 \pi r^2} dr = \frac{1}{4 \pi \varepsilon} \frac{1}{r} \bigg|_{R_1}^{R_2} = - \frac{1}{4 \pi \varepsilon} \left(\frac{1}{R_1} - \frac{1}{R_2} \right) \ , Q \ .\end{split}
\end{equation*}
\sphinxAtStartPar
La formula precedente e la definizione di capacità, \$\(Q = C , \Delta V\)\$, consente di determinare la capacità di un condensatore sferico ideale,
\begin{equation*}
\begin{split}C = 4 \pi \, \varepsilon \,  \frac{R_1 \, R_2}{R_2 - R_1} \ .\end{split}
\end{equation*}


\sphinxstepscope


\section{Corrente elettrica}
\label{\detokenize{ch/electromagnetism/electric-current:corrente-elettrica}}\label{\detokenize{ch/electromagnetism/electric-current:physics-hs-electromagnetism-electric-current}}\label{\detokenize{ch/electromagnetism/electric-current::doc}}\begin{itemize}
\item {} 
\sphinxAtStartPar
corrente elettrica:
\begin{itemize}
\item {} 
\sphinxAtStartPar
descrizione microscopica: materiale elettricamente neutro, con \(e^-\) liberi di conduzione

\item {} 
\sphinxAtStartPar
def come flusso di carica: dalla descrizione micro alla descrizione macroscopica, media, (“fenomenologica”?)

\end{itemize}

\end{itemize}

\sphinxAtStartPar
Localmente, è possibile definire una \sphinxstylestrong{densità “macroscopica” di corrente elettrica} come la media pesata delle velocità delle cariche, \(\vec{v}\).
\begin{equation*}
\begin{split}\lim_{\Delta V(P) \rightarrow 0} \dfrac{\sum_{k \in \Delta V(P)} q_k}{\Delta V} = \rho(P)\end{split}
\end{equation*}\begin{equation*}
\begin{split}\lim_{\Delta V(P) \rightarrow 0} \dfrac{\sum_{k \in \Delta V(P)} q_k \vec{v}_k}{\Delta V_P} = \vec{j}(P)\end{split}
\end{equation*}
\sphinxAtStartPar
La \sphinxstylestrong{corrente elettrica} attraverso una superficie \(S\) viene definita come il flusso di carica elettrica attraverso la superficie \(S\),
\begin{equation*}
\begin{split}i = I_{S} = \Phi_{S}(\vec{j}) \ .\end{split}
\end{equation*}

\begin{savenotes}\sphinxattablestart
\centering
\begin{tabulary}{\linewidth}[t]{|T|T|}
\hline

\sphinxAtStartPar
\sphinxincludegraphics{{electric-current}.png}
&
\sphinxAtStartPar
\sphinxincludegraphics{{electric-current-solids}.png}
\\
\hline
\end{tabulary}
\par
\sphinxattableend\end{savenotes}

\sphinxAtStartPar
\sphinxstylestrong{Oss. Corrente elettrica in materiali neutri}: è possibile avere corrente elettrica anche in materiali elettricamente neutri, anche localmente. Ad esempio, nell’ipotesi di avere due sostanze diverse con carica \(\rho^+\), \(\rho^-\) e velocità media delle due sostanze \(\vec{v}^+\), \(\vec{v}^-\), la corrente densità di corrente elettrica è
\begin{equation*}
\begin{split}\vec{j} = \rho \vec{v} = \rho^+ \vec{v}^+ + \rho^- \vec{v}^- \ . \end{split}
\end{equation*}
\sphinxAtStartPar
Nel caso in cui il materiale sia neutro, la densità di carica elettrica è nulla, \(0 = \rho = \rho^+ + \rho^- =\) e quindi \(\rho^+ = - \rho^-\) e la densità di corrente elettrica può essere scritta come \(\vec{j} = \rho^- (\vec{v}^- - \vec{v}^+)\).

\sphinxAtStartPar
\sphinxstylestrong{Oss. Corrente elettrica in solidi conduttori neutri.} I solidi hanno una struttura microscopica con gli atomi disposti in un reticolo, senza libertà di movimento. Nei solidi conduttori, gli elettroni “più esterni” della struttura atomica non sono localizzati attorno al singolo atono, ma sono “condivisi” e libersi di muoversi tra tutti gli atomi del solido: queste cariche elettriche libere di muoversi permettono una buona conduzione di corrente elettrica, e vengono chiamati \sphinxstylestrong{elettroni di conduzione}

\sphinxAtStartPar
Senza “forzanti esterne”, come ad esempio campi elettrici, il moto degli elettroni di conduzione non ha direzioni privilegiate: poiché il moto delle cariche libere è casuale senza direzioni privilegiate, la velocità media è nulla (la velocità è una grandezza vettoriale!) e la corrente elettrica è nulla. Se le velocità delle cariche libere ha una direzione preferenziale, la loro velocità media, \(\vec{v}^-\), e quindi la corrente elettrica, non è nulla. Assumendo che le cariche elettriche positive abbiano velocità media nulla rispetto all’osservatore \(\vec{v}^+ = \vec{0}\), la densità di corrente elettrica diventa \(\vec{j} = \rho^- \vec{v}^-\).






\subsection{Strumenti: misura e generazione}
\label{\detokenize{ch/electromagnetism/electric-current:strumenti-misura-e-generazione}}
\sphinxAtStartPar
\sphinxstylestrong{todo}
\begin{itemize}
\item {} 
\sphinxAtStartPar
strumenti per misurare corrente e tensione: amperometro e voltmetro

\item {} 
\sphinxAtStartPar
generatori di “spinta”: generatori di tensione

\item {} 
\sphinxAtStartPar
resistenza al moto: la resistenza elettica

\end{itemize}


\subsection{Legge di conservazione della carica elettrica}
\label{\detokenize{ch/electromagnetism/electric-current:legge-di-conservazione-della-carica-elettrica}}
\sphinxAtStartPar
Il principio di conservazione della carica elettrica
\begin{equation*}
\begin{split}\dot{Q}_V = - \Phi_{\partial V}(\vec{j}) = - I_{\partial V}\end{split}
\end{equation*}\begin{itemize}
\item {} 
\sphinxAtStartPar
\sphinxstylestrong{todo} esempi/esercizi con misura della corrente e della carica elettrica, con strumenti di misura (misura o modello di strumento, come bilance)

\end{itemize}


\subsection{Conduzione}
\label{\detokenize{ch/electromagnetism/electric-current:conduzione}}

\subsubsection{Conduzione nei solidi “di Ohm”}
\label{\detokenize{ch/electromagnetism/electric-current:conduzione-nei-solidi-di-ohm}}
\sphinxAtStartPar
In un materiale di Ohm, il campo elettrico \(\vec{e}\) è proporzionale alla densità di corrente elettrica \(\vec{j}\). Per un solido isotropo, senza direzioni preferenziali, la \sphinxstylestrong{forma locale \sphinxhyphen{} differenziale \sphinxhyphen{} della legge di Ohm} è
\sphinxstylestrong{Legge di Ohm} in forma locale:
\begin{equation*}
\begin{split}
\vec{j}(P) = \sigma(P) \, \vec{e}(P)
\qquad \ , \qquad
\vec{e}(P) = \rho_R(P) \, \vec{j}(P)
\end{split}
\end{equation*}
\sphinxAtStartPar
essendo la resistività \(\rho_R\), e la conduttanza \(\sigma = \frac{1}{\rho_R}\) le costanti di proporzionalità, caratteristiche del materiale.

\sphinxAtStartPar
In un cavo conduttore, nell’ipotesi di grandezze uniformi sulla sezione \sphinxhyphen{} o riferendosi alle grandezze medie \sphinxhyphen{}, si può integrare la legge in forma locale su un tratto di cavo elementare, di lunghezza \(d \ell\),
\begin{equation*}
\begin{split}\begin{aligned}
 \underbrace{e \, d \ell}_{- d v} \, A & = \rho_R \, \underbrace{j \, A}_{i} \, d \ell \\
 \rightarrow \quad dv & = - \dfrac{\rho_R \, d \ell}{A} \, i = - dR \, i \ , 
\end{aligned}\end{split}
\end{equation*}
\sphinxAtStartPar
avendo introdotto la differenza di potenziale elementare \(d v\) tra gli estremi del tratto di cavo elementare, proporzionale alla corrente che transita nel cavo tramite la \sphinxstylestrong{resistenza elettrica} elementare \(dR\). Queste relazioni che caratterizzano i materiali di Ohm sono le due leggi di Ohm:
\begin{itemize}
\item {} 
\sphinxAtStartPar
\sphinxstylestrong{Prima legge di Ohm.} La differenza di potenziale agli estremi di un cavo di lunghezza elementare è proporzionale alla corrente, tramite la resistenza elettrica elementare,
\begin{equation*}
\begin{split}dv = - dR \, i \ ,\end{split}
\end{equation*}
\item {} 
\sphinxAtStartPar
\sphinxstylestrong{Seconda legge di Ohm.} La resistenza elettrica di un cavo è direttamente proporzionale alla resistività del materiale, alla lunghezza del cavo, e inversamente proporzionale alla sezione del cavo,
\begin{equation*}
\begin{split}dR = \frac{\rho_R \, d\ell}{A} \ .\end{split}
\end{equation*}
\end{itemize}


\subsubsection{Conduzione nei gas}
\label{\detokenize{ch/electromagnetism/electric-current:conduzione-nei-gas}}

\subsubsection{Conduzione nei vuoto?}
\label{\detokenize{ch/electromagnetism/electric-current:conduzione-nei-vuoto}}

\subsubsection{Conduzione nei semiconduttori}
\label{\detokenize{ch/electromagnetism/electric-current:conduzione-nei-semiconduttori}}
\sphinxAtStartPar
cenni all’elettronica: diodi, transistor, …

\sphinxstepscope


\section{Magnetismo ed elettromagnetismo in regime stazionario}
\label{\detokenize{ch/electromagnetism/electromagnetism-steady:magnetismo-ed-elettromagnetismo-in-regime-stazionario}}\label{\detokenize{ch/electromagnetism/electromagnetism-steady:physics-hs-electromagnetism-electromagnetism-steady}}\label{\detokenize{ch/electromagnetism/electromagnetism-steady::doc}}

\subsection{Esperienze elementari su campo magnetico}
\label{\detokenize{ch/electromagnetism/electromagnetism-steady:esperienze-elementari-su-campo-magnetico}}\begin{itemize}
\item {} 
\sphinxAtStartPar
cos’è? come costruire un campo magnetico? o avere multipli di un campo magnetico?

\end{itemize}


\subsection{Esperienza di Faraday}
\label{\detokenize{ch/electromagnetism/electromagnetism-steady:esperienza-di-faraday}}\begin{equation*}
\begin{split}d \vec{F} = - i \, \vec{b} \times d \vec{\ell} \ .\end{split}
\end{equation*}
\sphinxAtStartPar
\sphinxstylestrong{todo} ha senso associarla a Faraday? Nessuno la conosceva prima? Galvani, Volta,… come misuravano la corrente elettrica?


\subsubsection{Il galvanometro}
\label{\detokenize{ch/electromagnetism/electromagnetism-steady:il-galvanometro}}
\sphinxAtStartPar
Il galvanometro è uno strumento utilizzato per la misura della corrente elettrica. Sfrutta l’azione meccanica osservata nell’esperienza di Faraday

\sphinxAtStartPar
Il momento meccanico generato dalla corrente nel cavo elettrico equilibria un momento generato da componenti meccanici “noti”, realizzabili e tarabili con la precisione richiesta.

\sphinxAtStartPar
\sphinxstylestrong{todo} \sphinxstyleemphasis{Serve questo riferimento qui?}
\begin{itemize}
\item {} 
\sphinxAtStartPar
azioni elettro\sphinxhyphen{}meccaniche:…, cenni al motore elettrico in corrente continua? serve accoppiamento \(\vec{e} \leftrightarrow \vec{b}\) di Faraday

\end{itemize}


\subsection{Esperienze di Oersted e Ampere}
\label{\detokenize{ch/electromagnetism/electromagnetism-steady:esperienze-di-oersted-e-ampere}}\begin{itemize}
\item {} 
\sphinxAtStartPar
interazione tra corrente elettrica e campo magnetico, in regime stazionario:
\begin{itemize}
\item {} 
\sphinxAtStartPar
esperienze di Oesrted e Ampére:

\end{itemize}

\end{itemize}


\subsubsection{Legge di Ampére}
\label{\detokenize{ch/electromagnetism/electromagnetism-steady:legge-di-ampere}}
\sphinxAtStartPar
Forza (\sphinxstylestrong{todo} per unità di lunghezza; usare notazione vettoriale per indicare la direzione della forza) scambiata tra due cavi percorsi da corrente elettrica
\begin{equation*}
\begin{split}\frac{F}{L} = \frac{\mu}{2 \pi} \frac{i_1 \, i_2}{d}\end{split}
\end{equation*}

\subsubsection{Legge di Biot\sphinxhyphen{}Savart}
\label{\detokenize{ch/electromagnetism/electromagnetism-steady:legge-di-biot-savart}}
\sphinxAtStartPar
Confrontando la legge di Ampére con l’esperienza di Faraday, è possibile ricavare l’espressione del campo magnetico prodotto da un cavo infinito percorso da corrente elettrica,
\begin{equation*}
\begin{split}b = \frac{\mu}{2 \pi} \frac{i}{d}\end{split}
\end{equation*}
\sphinxAtStartPar
\sphinxstylestrong{todo}
\begin{itemize}
\item {} 
\sphinxAtStartPar
campo magnetico prodotto da un cavo rettilineo infinito

\item {} 
\sphinxAtStartPar
campo magnetico prodotto da un solenoide: lineare infinito, toroidale

\end{itemize}


\subsubsection{Formula generale}
\label{\detokenize{ch/electromagnetism/electromagnetism-steady:formula-generale}}
\sphinxAtStartPar
Contributo elementare
\begin{equation*}
\begin{split}d \vec{b}(\vec{r}_0) = - \frac{\mu}{4 \pi} i(\vec{r}) \frac{ \vec{r}_0 - \vec{r} }{| \vec{r}_0 - \vec{r} |^3} \times d \vec{\ell}(\vec{r})\end{split}
\end{equation*}\begin{equation*}
\begin{split}\vec{b}(\vec{r}_0) = - \frac{\mu}{4 \pi} \int_{\gamma(\vec{r})} i(\vec{r})  \frac{ \vec{r}_0 - \vec{r} }{| \vec{r}_0 - \vec{r} |^3}\times d \vec{\ell}(\vec{r})\end{split}
\end{equation*}

\paragraph{Filo rettilineo infinito}
\label{\detokenize{ch/electromagnetism/electromagnetism-steady:filo-rettilineo-infinito}}\begin{equation*}
\begin{split}z = R \, \tan \theta\end{split}
\end{equation*}\begin{equation*}
\begin{split}dz = R \frac{1}{\cos^2 \theta} \, d \theta\end{split}
\end{equation*}\begin{equation*}
\begin{split}r^2 = R^2 + z^2 = R^2 \left(1+\frac{\sin^2 \theta}{\cos^2 \theta} \right) = R^2 \frac{1}{\cos^2 \theta}\end{split}
\end{equation*}\begin{equation*}
\begin{split}\begin{aligned}
  \vec{b}(\vec{r}_0) & = - \frac{\mu}{4 \pi} i \int_{z=-\infty}^{\infty} \hat{\theta} \frac{r}{r^2} \sin \theta dz = \\
                     & = - \frac{\mu}{4 \pi} i \hat{\theta} \int_{\theta=\pi}^{0} \frac{\cos^2 \theta}{R^2} \sin \theta R \frac{1}{\cos^2 \theta} \, d \theta = \\
                     & =   \frac{\mu}{4 \pi} i \hat{\theta} \int_{\theta=0}^{\pi} \frac{1}{R} \sin \theta \, d \theta = \\
                     & =   \frac{\mu \, i}{2 \pi \, R} \hat{\theta} \ .
\end{aligned}\end{split}
\end{equation*}

\subsubsection{Esempi: spira e solenoide}
\label{\detokenize{ch/electromagnetism/electromagnetism-steady:esempi-spira-e-solenoide}}

\paragraph{Spira circolare}
\label{\detokenize{ch/electromagnetism/electromagnetism-steady:spira-circolare}}
\sphinxAtStartPar
Sfruttando la simmetria cilindrica del problema, è possibile calcolare il campo magnetico \$\$ sull’asse di una spira circolare
\begin{equation*}
\begin{split}\cos \phi = \frac{R}{r} \qquad , \qquad
r^2 = R^2 + z^2\end{split}
\end{equation*}\begin{equation*}
\begin{split}\begin{aligned}
  \vec{b}(\theta) & = 2 \pi R \left( -\frac{\mu}{4 \pi} i \frac{\vec{r}}{r^3} \times \hat{\theta} \cdot \hat{z} \right) \hat{z} = \\
   & = \frac{\mu \, i}{2} \frac{R}{r^2} \cos \phi \hat{z} = \\
   & = \frac{\mu \, i}{2} \frac{R^2}{r^3} \hat{z}  = \\
   & = \frac{\mu \, i}{2} \frac{R^2}{(R^2 + z^2)^{3/2}} \hat{z}  =
       \frac{\mu \, i}{2 \, R} \frac{1}{\left(1 + \left(\frac{z}{R}\right)^2 \right)^{3/2}} \hat{z} 
\end{aligned}\end{split}
\end{equation*}

\paragraph{Solenoide rettilineo}
\label{\detokenize{ch/electromagnetism/electromagnetism-steady:solenoide-rettilineo}}
\sphinxAtStartPar
Applicando la legge di Ampére,
\begin{equation*}
\begin{split} N \, i = \Gamma_{\gamma}(\vec{h}) = \ell \, h = \ell \, \frac{b}{\mu}\end{split}
\end{equation*}\begin{equation*}
\begin{split}b = \mu \frac{N}{\ell} \, i\end{split}
\end{equation*}
\sphinxAtStartPar
Il flusso del campo magnetico (uniforme) vale quindi
\begin{equation*}
\begin{split}\phi = b \, A = \mu \frac{N \, A}{\ell} \, i\end{split}
\end{equation*}

\paragraph{Solenoide toroidale}
\label{\detokenize{ch/electromagnetism/electromagnetism-steady:solenoide-toroidale}}
\sphinxAtStartPar
Applicando la legge di Ampère,
\begin{equation*}
\begin{split}N \, i = \Gamma_{\gamma}(\vec{h}) = r \, 2 \, \pi \, h = r \, 2 \, \pi \frac{b}{\mu}\end{split}
\end{equation*}\begin{equation*}
\begin{split}b(r) = \mu \frac{N}{2 \, \pi \, r } \, i\end{split}
\end{equation*}
\sphinxAtStartPar
Il flusso del campo magnetico attraverso le sezioni del toro vale
\begin{equation*}
\begin{split}\Phi(\vec{b}) = \oint_{S} b(r) \, dS =  \mu \frac{N \, i}{2 \pi}\int_{\rho=0}^{a} \int_{\alpha=0}^{2\pi} \frac{1}{R - \rho \cos \alpha} \rho \, d \rho \, d \alpha  = \end{split}
\end{equation*}
\sphinxAtStartPar
\sphinxstylestrong{todo}




\subsection{Verso le equazioni di Maxwell}
\label{\detokenize{ch/electromagnetism/electromagnetism-steady:verso-le-equazioni-di-maxwell}}

\subsubsection{Legge di Gauss per il flusso del campo magnetico}
\label{\detokenize{ch/electromagnetism/electromagnetism-steady:legge-di-gauss-per-il-flusso-del-campo-magnetico}}\begin{equation*}
\begin{split}\Phi_{\partial V}(\vec{b}) = 0\end{split}
\end{equation*}
\sphinxAtStartPar
\sphinxstylestrong{todo} \sphinxstyleemphasis{interpretazione: inesistenza del monopolo magnetico? linee di campo chiuse?}


\subsubsection{Legge di Ampére}
\label{\detokenize{ch/electromagnetism/electromagnetism-steady:id1}}\begin{equation*}
\begin{split}\oint_{\ell_S} \vec{h} \cdot \hat{t} = \Gamma_{\ell_S}(\vec{h}) = \Phi_{S}(\vec{j}) = i_S \ ,\end{split}
\end{equation*}
\sphinxAtStartPar
essendo \(\ell_S = \partial S\) il contorno \sphinxhyphen{} chiuso \sphinxhyphen{} della superficie \(S\).
\begin{itemize}
\item {} 
\sphinxAtStartPar
Questa equazione è valida \sphinxstylestrong{solo} in un regime elettrostatico: la forma generale dell’equazione di Ampére prevede un termine dipendente dal tempo, che è identicamente nullo nel regime elettrostatico.

\item {} 
\sphinxAtStartPar
Senza questo termine, l’equazione non sarebbe consistente con l’equazione del bilancio della carica elettrica \sphinxstylestrong{todo} (aggiungere riferimento): la correzione di questa inconsistenza da parte di Maxwell è stata l’ultima azione, fondamentale, per ottenere un sistema di equazioni che governano i fenomeni elettromagnetici; la stessa modifica permette anche di riconoscere che i fenomeni EM sono fenomeni ondulatori; il calcolo della misura della velocità di propoagazione delle onde EM, confrontata con le misure disponibili della velocità della luce, permisero di riconoscere la luce come fenomeno EM

\end{itemize}

\sphinxAtStartPar
Per dimostrare l’incongruenza, è sufficiente applicare la legge di Ampére a una superficie che è il contorno di un volume chiuso, e che quindi ha contorno nullo,
\begin{equation*}
\begin{split}S = \partial V  \qquad , \qquad  \partial S = \ell_S = \emptyset\end{split}
\end{equation*}
\sphinxAtStartPar
In questo caso, la legge di Ampére diventa
\begin{equation*}
\begin{split}0 = i_{\partial V} \ ,\end{split}
\end{equation*}
\sphinxAtStartPar
mentre le leggi di conservazione della carica elettrica e la legge di Gauss per il campo elettrico
\begin{equation*}
\begin{split}\begin{aligned}
  \dot{Q}_V & = - i_{\partial V} \\
  \Phi_{\partial V}(\vec{d}) & = Q_V \\
\end{aligned}\end{split}
\end{equation*}
\sphinxAtStartPar
implicano
\begin{equation*}
\begin{split}i_{\partial V} = - \dot{Q}_V = - \dot{\Phi}_{\partial V}(\vec{d}) \ .\end{split}
\end{equation*}
\sphinxAtStartPar
La correzione di Maxwell non è altro che l’aggiunta del termine \(\dot{\Phi}_{\partial V}(\vec{d})\) all’equazione di Ampére per renderla compatibile con le altre equazioni dell’elettromagnetismo. Con questa modifica, l’\sphinxstylestrong{equazione di Ampére\sphinxhyphen{}Maxwell} diventa
\begin{equation*}
\begin{split}\Gamma_{\partial S}(\vec{h}) - \dot{\Phi}_{S}(\vec{d}) = i_S \end{split}
\end{equation*}

\subsection{Modelli microscopici del magnetismo}
\label{\detokenize{ch/electromagnetism/electromagnetism-steady:modelli-microscopici-del-magnetismo}}

\subsection{Moto di una carica elettrica in un campo elettromagnetico}
\label{\detokenize{ch/electromagnetism/electromagnetism-steady:moto-di-una-carica-elettrica-in-un-campo-elettromagnetico}}
\sphinxAtStartPar
Forza di Lorentz
\begin{equation*}
\begin{split}\vec{F}^{Lorentz} = q \left(\vec{e}(P) + \vec{b}(P) \times \vec{v} \right)\end{split}
\end{equation*}
\sphinxAtStartPar
Moto di una carica elettrica in un campo elettromagnetico, nell’ipotesi di effetto nullo su di essa del proprio campo elettrico
\begin{equation*}
\begin{split}m \ddot{ \vec{r} } = \vec{R}^{ext} = q \left( \vec{e}(P) + \vec{b}(P) \times \dot{\vec{r}} \right) + \vec{F}^{\text{non EM}}\end{split}
\end{equation*}\begin{itemize}
\item {} 
\sphinxAtStartPar
\sphinxstylestrong{todo} esempi

\end{itemize}

\sphinxstepscope


\section{Induzione ed elettromagnetismo}
\label{\detokenize{ch/electromagnetism/electromagnetism-general:induzione-ed-elettromagnetismo}}\label{\detokenize{ch/electromagnetism/electromagnetism-general:physics-hs-electromagnetism-electromagnetism-general}}\label{\detokenize{ch/electromagnetism/electromagnetism-general::doc}}

\subsection{Legge di Faraday per l’induzione elettromagnetica}
\label{\detokenize{ch/electromagnetism/electromagnetism-general:legge-di-faraday-per-l-induzione-elettromagnetica}}\begin{itemize}
\item {} 
\sphinxAtStartPar
legge di Faraday: corrente indotta

\item {} 
\sphinxAtStartPar
corrente alternata:
\begin{itemize}
\item {} 
\sphinxAtStartPar
principi e applicazioni:
\begin{itemize}
\item {} 
\sphinxAtStartPar
trasformatori

\item {} 
\sphinxAtStartPar
generatori e motori elettrici

\item {} 
\sphinxAtStartPar
generazione/trasporto/trasformazione/consumo

\end{itemize}

\end{itemize}

\end{itemize}


\subsection{Correzione di Maxwell dell’equazione di Ampére}
\label{\detokenize{ch/electromagnetism/electromagnetism-general:correzione-di-maxwell-dell-equazione-di-ampere}}
\sphinxAtStartPar
 Qui o nella sezione di magnetismo ed elettromagnetismo statico? 
\begin{itemize}
\item {} 
\sphinxAtStartPar
correzione di Maxwell della legge di Ampére, per renderla consistente con l’equazione di bilancio della carica elettrica

\end{itemize}


\subsection{Equazioni di Maxwell dell’elettromagnetismo}
\label{\detokenize{ch/electromagnetism/electromagnetism-general:equazioni-di-maxwell-dell-elettromagnetismo}}\begin{itemize}
\item {} 
\sphinxAtStartPar
le equazioni di Maxwell: le equazioni complete dell’elettromagnetismo

\end{itemize}

\sphinxstepscope


\chapter{Principi di elettrotecnica}
\label{\detokenize{ch/electromagnetism/electrical-engineering:principi-di-elettrotecnica}}\label{\detokenize{ch/electromagnetism/electrical-engineering:physics-hs-electromagnetism-electric-engineering}}\label{\detokenize{ch/electromagnetism/electrical-engineering::doc}}
\sphinxstepscope


\section{Circuiti elettrici}
\label{\detokenize{ch/electromagnetism/circuits-electric:circuiti-elettrici}}\label{\detokenize{ch/electromagnetism/circuits-electric:physics-hs-electromagnetism-circuits-electric}}\label{\detokenize{ch/electromagnetism/circuits-electric::doc}}\begin{itemize}
\item {} 
\sphinxAtStartPar
dalle leggi fisiche alle leggi di Kirchhoff, ipotesi (validità e non\sphinxhyphen{}validità dell’approccio circuitale)

\item {} 
\sphinxAtStartPar
componenti:
\begin{itemize}
\item {} 
\sphinxAtStartPar
resistenze

\item {} 
\sphinxAtStartPar
condensatori

\item {} 
\sphinxAtStartPar
induttori

\item {} 
\sphinxAtStartPar
generatori

\end{itemize}

\item {} 
\sphinxAtStartPar
regimi di funzionamento: in DC, (trascurando gli effetti EM: no campi magnetici esterni, \sphinxstyleemphasis{ogni circuito è una spira}…), e in AC
\begin{itemize}
\item {} 
\sphinxAtStartPar
stazionario
\begin{itemize}
\item {} 
\sphinxAtStartPar
bilancio di energia: “generatori” di energia elettrica, “perdite” nelle resistenze
\begin{itemize}
\item {} 
\sphinxAtStartPar
approfondimenti:

\item {} 
\sphinxAtStartPar
pile  Collegamento ad altre parti: termodinamica? chimica?

\end{itemize}

\end{itemize}

\item {} 
\sphinxAtStartPar
transitorio:
\begin{itemize}
\item {} 
\sphinxAtStartPar
esempio: carica/scarica condensatore

\end{itemize}

\item {} 
\sphinxAtStartPar
armonico, AC:
\begin{itemize}
\item {} 
\sphinxAtStartPar
…

\end{itemize}

\end{itemize}

\end{itemize}

\sphinxstepscope


\section{Circuiti magnetici}
\label{\detokenize{ch/electromagnetism/circuits-magnetic:circuiti-magnetici}}\label{\detokenize{ch/electromagnetism/circuits-magnetic:physics-hs-electromagnetism-circuits-magnetic}}\label{\detokenize{ch/electromagnetism/circuits-magnetic::doc}}\begin{itemize}
\item {} 
\sphinxAtStartPar
dalle leggi fisiche alle leggi di Kirchhoff per i circuiti magnetici, ipotesi (validità e non\sphinxhyphen{}validità dell’approccio circuitale)

\item {} 
\sphinxAtStartPar
esempi:
\begin{itemize}
\item {} 
\sphinxAtStartPar
trasformatori ideali

\end{itemize}

\end{itemize}

\sphinxstepscope


\section{Sistemi elettromeccanici e macchine elettriche}
\label{\detokenize{ch/electromagnetism/electric-machines:sistemi-elettromeccanici-e-macchine-elettriche}}\label{\detokenize{ch/electromagnetism/electric-machines:physics-hs-electromagnetism-electric-machines}}\label{\detokenize{ch/electromagnetism/electric-machines::doc}}
\sphinxstepscope


\chapter{Onde elettromagnetiche}
\label{\detokenize{ch/electromagnetism/em-waves:onde-elettromagnetiche}}\label{\detokenize{ch/electromagnetism/em-waves:physics-hs-electromagnetism-em-waves}}\label{\detokenize{ch/electromagnetism/em-waves::doc}}
\sphinxAtStartPar
Le equazioni di Maxwell prevedono che, sotto opportune condizioni, il campo elettromagnetico possa propagarsi nello spazio come un fenomeno ondulatorio
\begin{itemize}
\item {} 
\sphinxAtStartPar
Dalle equazioni di Maxwell alle equazioni delle onde \sphinxstylestrong{todo} è possibile arrivarci senza passare dalle equazioni in forma differenziale? Magari con qualche analogia meccanica governata dalle equazioni delle onde. Se sì, in maniera sufficiente formale, figata

\item {} 
\sphinxAtStartPar
Esperimenti di Hertz:
\begin{itemize}
\item {} 
\sphinxAtStartPar
onde elettromagnetiche:
\begin{itemize}
\item {} 
\sphinxAtStartPar
esperimento onde EM

\item {} 
\sphinxAtStartPar
\sphinxstylestrong{risonanza}, tipica dei fenomeni ondulatori

\end{itemize}

\item {} 
\sphinxAtStartPar
esperimento fallimentare su raggi catodici; con lo stesso esperimento, Thompson dimostra l’esistena dell’elettrone

\item {} 
\sphinxAtStartPar
effetto fotoelettrico…

\item {} 
\sphinxAtStartPar
luce come onda EM: la velocità di propagazione del campo EM prevista dalle equazioni di Maxwell è vicina alle misure della velocità della luce disponibile in quegli anni. E’ un caso?

\end{itemize}

\end{itemize}

\sphinxstepscope


\part{Fisica moderna}

\sphinxstepscope


\chapter{Fisica del XX secolo}
\label{\detokenize{ch/modern:fisica-del-xx-secolo}}\label{\detokenize{ch/modern:physics-hs-modern}}\label{\detokenize{ch/modern::doc}}






\renewcommand{\indexname}{Index}
\printindex
\end{document}